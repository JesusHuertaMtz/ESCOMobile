\section{Reglas de Negocio}

\subsection{Reglas de Negocio del Sistema}

% RNS1. *******************************************************
\begin{BussinesRule}{EM-RN-S001}{Formato de correo electronico}
	\BRitem[Descripción:] Para ser almacenado y usuado de manera correcta, el usuario debe de introducir un correo electrónico válido, con un formato definido, dicho formato se explica a continuación. El correo electrónico debe ser una cadena de caracteres con la siguiente estructura ordenada:
	1. 1 a 30 caracteres.
	2. Caracter @
	3. Dominio: 'hotmail.com', 'gmail.com', 'outlook.com', 'yahoo.com', 'ipn.mx', 'hotmail.com.mx', 'gmail.com.mx', 'outlook.com.mx', 'yahoo.com.mx'
	\BRitem[Tipo:] Habilitadora.
	\BRitem[Nivel:] Controla la operación.
	\BRitem[Sentencia:] Sea U un usuario (alumno o profesor), C el correo que desea introducir entonces: 
		\begin{center}
			$\forall$ U, U.C tiene que cumplir la siguiente expresión regular: [a-zA-Z]{1,30}@[[[hotmail | gmail | outlook | yahoo][.com][.mx]{0,1}]|[ipn.mx]].
		\end{center}
	\BRitem[Ejemplo positivo:] que cumple la regla.
		\begin{itemize}
			\item Sea $U_{1}$ asdfg@gmail.com  cumple con la regla.
			\item Sea $U_{2}$ escom\_mobile@hotmail.com.mx cumple con la regla.
			\item Sea $U_{3}$ daniel\_perez@ipn.mx cumple con la regla.
		\end{itemize}
	\BRitem[Ejemplo negativo:] que no cumple la regla.
		\begin{itemize}
			\item Sea $U_{1}$ asdf.ipn@jskds.net no cumple con la regla.
			\item Sea $U_{2}$ daniel perez@.com no cumple con la regla.
			\item Sea $U_{3}$ hasdfd.com@unam.mx no cumple con la regla.
		\end{itemize}
\end{BussinesRule}

% RNS2. *******************************************************
\begin{BussinesRule}{EM-RN-S002}{Campos obligatorios}
	\BRitem[Descripción:]Todos los datos pedidos al usuario en el sistema
	que estén marcado con el símbolo * deben ser obligatoriamente intruducidos en el mismo.
	\BRitem[Tipo:] Habilitadora.
	\BRitem[Nivel:] Controla la operación.
	\BRitem[Sentencia:] Sea C un campo requerido obligatoriamente en el sistema y V el valor del campo en el sistema, entonces: 
		\begin{center}
			$\forall$ V en C, V != "" | V != Null | V != 0 | V != 0.0.
		\end{center}
	\BRitem[Ejemplo positivo:] que cumple la regla.
		\begin{itemize}
			\item Sea $V_{1}$ Jesús, el valor del $C_{1}$ Nombre, permite continuar.
			\item Sea $V_{2}$ "", el valor del $C_{1}$ Apellido, no permite continuar.
		\end{itemize}
	\BRitem[Ejemplo negativo:] que no cumple la regla.
		\begin{itemize}
			\item Sea $V_{1}$ Jesús, el valor del $C_{1}$ Nombre, no permite continuar.
			\item Sea $V_{2}$ "", el valor del $C_{1}$ Apellido, permite continuar.
		\end{itemize}
\end{BussinesRule}

% RNS3. *******************************************************
\begin{BussinesRule}{EM-RN-S003}{Caracteres aceptados por el sistema.} 
	\BRitem[Descripción:] Los caracteres que se podrán ingresar por medio del teclado están comprendidos en 
	el rango de dígitos del 0 al 9, letras del alfabeto de $a$ a la $z$ tanto mínusculas como mayúsculas. 
	También se podrán ingresar vocales, tanto mínusculas y mayúsculas, acentuadas.
	Además, se aceptarán los siguientes caracteres especiales:
	!, ", \#, \$, \%, \&, ', (, ), *, +, -, ., /, :, ;, <, =, >, ¿, ¡, ?, @, [, \textbackslash,
	], \^, \_, \{, \}, |, \~.
	\BRitem[Tipo:] Habilitadora.
	\BRitem[Nivel:] Controla.
	\BRitem[Calse:] Integridad.
	\BRitem[Sentencia:] 
	Sea $N$ los dígitos del 0 al 9, $min$ la letras del alfabeto de $a$ a la $z$, $may$ las letras del
	alfabeto de $A$ a la $Z$ y sea $C$ el conjunto de caracteres especiales , sea 
	$CA$ = \{$ N, \:min, \:may, \:C $\} y sea $c$ un caracter introducido por medio del teclado, entonces:
	\newline
	$\forall$ $c$  $\in$ $CA$ $\Rightarrow$ $c$ es un caracter aceptado por el sistema.
	\BRitem[Ejemplo positivo:] que cumple la regla.
		\begin{itemize}
			\item Sea $c_{1}$ w cumple con la regla.
			\item Sea $c_{2}$ Q cumple con la regla.
			\item Sea $c_{3}$ @ cumple con la regla.
			\item Sea $c_{4}$ 2 cumple con la regla.
		\end{itemize}
	\BRitem[Ejemplo negativo:] que no cumple la regla.
		\begin{itemize}
			\item Sea $c_{1}$ $\spadesuit$ no cumple con la regla.
			\item Sea $c_{2}$ $\infty$ no cumple con la regla.
			\item Sea $c_{3}$ $\hat{a}$ no cumple con la regla.
			\item Sea $c_{4}$ $\int$ no cumple con la regla.
		\end{itemize}
\end{BussinesRule}


% RNS4. *******************************************************
\begin{BussinesRule}{EM-RN-S004}{Formato de Contraseña.} 
	\BRitem[Descripción:] La longitud de la contraseña elegida  por el usuario para ingresar al sistema 
	debe tener un mínimo de 8 y un máximo de 30 caracteres. 
	La contraseña debe tener un dígito, una mayúscula, una mínuscula y un caracter especial en su
	formación.
	\BRitem[Tipo:] Habilitadora.
	\BRitem[Nivel:] Controla.
	\BRitem[Clase:] Integridad.
	\BRitem[Sentenia: ] Sea $a$ un entero, sea $b$ una letra mayúscula, sea $c$ una letra mínuscula,
	sea $d$ un caracter especial, sea $C$ el conjunto de todos los caracteres aceptados 
	por el sistema, sea $P$ una subconjunto de caracteres introducidos por el usuario y sea 
	$N$ la suma de  caracteres que contiene $P$, entonces: \newline
	Si para toda $P$ que sea $\subseteq$ $C$ y de longitud $8 < N < 30$ existe en $P$ al menos un $a$, $b$, 
	$c$ y $d$ entonces $P$ es una contraseña aceptada.
	\begin{center}
		$\forall \: P \subseteq C \: donde \: 8 < N < 30 \: \exists a, b, c \:y\: d \in P 
		\Rightarrow  es una contrasenia aceptada$
	\end{center}
	\BRitem[Ejemplo positivo:] que cumple la regla.
		\begin{itemize}
			\item Sea $P_{1}$ == qwerty1Q! es una contraseña aceptada. 
			\item Sea $P_{2}$ == ALV!9000stemenXD es una contraseña aceptada.
			\item Sea $P_{3}$ == 4b(d3fgH es una contraseña acpetada.
		\end{itemize}
	\BRitem[Ejemplo negativo:] que no cumple la regla.
		\begin{itemize}
			\item Sea $P_{1}$ == 123 no es una contraseña aceptada.
			\item Sea $P_{2}$ == abcd123! no es una contraseña aceptada.
			\item Sea $P_{3}$ == 12345678 no en una contraseña aceptada.
		\end{itemize}
\end{BussinesRule}


% RNS5. *******************************************************
\begin{BussinesRule}{EM-RN-S005}{Palabra.} 
	\BRitem[Descripción:] Una palabra es un conjunto de caracteres aceptados por el sistema.
	Veáse \BRref{EM-RN-S003}{Caracteres aceptados por el sistema.}
	Una palabra tiene un número mínimo de 2 caracteres y un máximo de 20 caracteres y no debe
	contener espacios.
	\BRitem[Tipo:] Habilitadora.
	\BRitem[Nivel:] Controla.
	\BRitem[Clase:] Integridad.
	\BRitem[Sentenia: ] Sea $a$ un caracter, sea $W$ una palabra y sea $C$ el conjunto de caracteres
	aceptados por el sistema, entonces:
	\begin{center}
		$\forall$ $a$ $\in$ $W$, $W$ $\subseteq$ $C$ $\Rightarrow$ $W$ es una palabra.
	\end{center}
	\BRitem[Ejemplo positivo:] que cumple la regla.
		\begin{itemize}
			\item Sea $W_{1}$ == Pancho. cumple la regla
			\item Sea $W_{2}$ == 123\$457(!AsW3Exede1\# cumple la regla
			\item Sea $W_{3}$ == \#\$Hola! cumple la regla
			\item Sea $W_{4}$ == it cumple la regla
			\item Sea $W_{5}$ == 123 cumple la regla
		\end{itemize}
	\BRitem[Ejemplo negativo:] que no cumple la regla.
		\begin{itemize}
			\item Sea $W_{1}$ == 1 no cumple la regla.
			\item Sea $W_{2}$ == 123\$457(!AsW3Exede1\#Kl no cumple la regla.
			\item Sea $W_{3}$ == \#\$Hola!$\int$ no cumple la regla.
		\end{itemize}
\end{BussinesRule}


% RNS5. *******************************************************
\begin{BussinesRule}{EM-RN-S006}{Formado del Nombre.} 
	\BRitem[Descripción:] El nombre se compone de una o cuatro palabras donde cada una 
	de ellas es una palabra que excluye caracteres especiales y dígitos en su formación.
	Veáse \BRref{EM-RN-S005}{Palabra.}
	\BRitem[Tipo:] Habilitadora.
	\BRitem[Nivel:] Controla.
	\BRitem[Clase:] Integridad.
	\BRitem[Sentenia:] Sea $a$ un caracter, sea $W$ una palabra y sea $C$ el conjunto de caracteres
	aceptados por el sistema sin incluir dígitos y caracteres especiales, entonces:
	\begin{center}
		$\forall$ $a$ $\in$ $W$, $W$ $\subseteq$ $C$ $\Rightarrow$ $W$ es un nombre aceptado para el 
		profesor.
	\end{center}
	\BRitem[Ejemplo positivo:] que cumple la regla.
		\begin{itemize}
			\item Sea $W_{1}$ == Pancho cumple la regla
			\item Sea $W_{2}$ == López cumple la regla
			\item Sea $W_{3}$ == Bruce Wayne cumple la regla
			\item Sea $W_{4}$ == Hola Jaime Pérez Pérez cumple la regla
			\item Sea $W_{5}$ == osms qwerty sdnklsaadsno cumple la regla
		\end{itemize}
	\BRitem[Ejemplo negativo:] que no cumple la regla.
		\begin{itemize}
			\item Sea $W_{1}$ == Panch1to no cumple la regla.
			\item Sea $W_{2}$ == Jaime@gmail.com no cumple la regla.
			\item Sea $W_{3}$ == 1234 qwerty no cumple la regla.
		\end{itemize}
\end{BussinesRule}

% RNS7. *******************************************************
\begin{BussinesRule}{EM-RN-S007}{Unicidad de elementos.} 
	\BRitem[Descripción:] Los elementos introducidos deben de ser únicos y no pueden repetirse. 
	\BRitem[Tipo:] Habilitadora.
	\BRitem[Nivel:] Controla.
	\BRitem[Clase:] Integridad.
	\BRitem[Sentenia:] Sea $tabla$ una tabla de la base de datos, $campo$ un atributo de la tabla y $valor$ el valor introducido asociado al campo, entonces:
	\begin{center}
		select count(*) from tabla where campo like valor >= 1, no se permite el registro, pues el valor introducido ya se encuentra registrado.
	\end{center}
	\BRitem[Ejemplo positivo:] que cumple la regla.
		Sean $tabla_{1}$ = Empresa, $campo_{1}$ = nombre y $valor_{1}$ = MCCollect, datos a registrar de una empresa
		Y se tiene que select count(*) from Empresa where nombre like "MCCollect" = 1, no permite el regitro. 

	\BRitem[Ejemplo negativo:] que no cumple la regla.
		Sean $tabla_{2}$ = Empresa, $campo_{2}$ = nombre y $valor_{2}$ = MCCollect, datos a registrar de una empresa
		Y se tiene que select count(*) from Empresa where nombre like "MCCollect" = 1, permite el regitro. 
\end{BussinesRule}


\begin{BussinesRule}{EM-RN-S008}{Formato de Horario}
	\BRitem[Descripción:] Para ser usado de manera fácil y correcta, el sistema obtiene el horario de la oferta de trabajo, esta se analiza, si viene como Medio Tiempo , tiempo Completo o Sin definir, se regresará esa misma cadena, en caso de que sea Definido, se tomara  los días, si estoy coinciden el horario se formara una cadena con estos días que coinciden y el respectivo horario, así mismo en caso de que no coincidan se colocara el día respectivo y su horario. El horario será una cadena de caracteres con la siguiente estructura:

1.Día o días.

2.Horario.

3. En caso de que no se seleccione Definido, queda como Indefinido, Medio Tiempo o Tiempo completo.
	\BRitem[Tipo:] Habilitadora.
	\BRitem[Nivel:] Controla la operación.
	\BRitem[Sentencia:] Sea H una cadena de horario  entonces: 
	\begin{center}
		$\forall$ H, h tiene que cumplir la siguiente expresión regular: [L|M|M|J|V|S|D]{00:00-24:00}{00:00-24:00} ||[Medio Timpo|Tiempo Completo|Sin Definir]
	\end{center}
	\BRitem[Ejemplo positivo:] que cumple la regla.
		\begin{itemize}
			\item Sea $H_{1}$ Medio Tiempo cumple la regla.
			\item Sea $U_{2}$ L - V 10:30:00 18:30:00 hrs cumple con la regla.
			\item Sea $U_{3}$ L - J 09:00:00 18:00:00 hrs y V 09:00:00 15:00:00 cumple con la regla.
		\end{itemize}

\end{BussinesRule}	

\subsection{Reglas de Negocio del Sistema}

% RNN1. *******************************************************
\begin{BussinesRule}{EM-RN-N001}{Empresa Registrada.} 
	\BRitem[Descripción:] La empresa ingresada debe de estar registrada en el sistema. 
	\BRitem[Tipo:] Habilitadora.
	\BRitem[Nivel:] Controla.
	\BRitem[Clase:] Integridad.
	\BRitem[Sentenia: ] Sea E una empresa , entonces:
	\begin{center}
		$\forall \: E  \mid E.registrada == true \Rightarrow empresa esta registrada.$
	\end{center}
	\BRitem[Ejemplo positivo:] que cumple la regla.
		\begin{itemize}
			\item Sea $E_{1}$.registrada == true esta registrada.
			\item Sea $E_{2}$.registrada == true esta registrada
		\end{itemize}
	\BRitem[Ejemplo negativo:] que no cumple la regla.
		\begin{itemize}
			\item Sea $E_{1}$.registrada == false false no esta registrada. 
			\item Sea $E_{2}$.registrada == true false no esta registrada.
		\end{itemize}
\end{BussinesRule}

% RNS2. *******************************************************
\begin{BussinesRule}{EM-RN-N002}{Total de Ofertas de Trabajo publicadas} 
	\BRitem[Descripción:] Para conocimiento del administrador es necesario saber el número de ofertas de trabajo que se han publicado en el sistema, si no hay ofertas de trabajo, establece el número de ofertas de trabajo publicadas igual a 0. 
	\BRitem[Tipo:] Habilitadora.
	\BRitem[Nivel:] Controla.
	\BRitem[Clase:] Integridad.
	\BRitem[Sentenia: ] Sea $O$ = {$o_{1}$, $o_{2}$, $o_{3}$, ... , $o_{n}$} el conjunto de ofertas de trabajo publicadas en el sistema por un administrador cada una con un identificador único $idOferta_{n}$.
	\begin{center}
		$\forall O \in O$, $O=$ ($ \sum_{i=O_1} $) .
	\end{center}
\end{BussinesRule}
	
% RNN3. *******************************************************
\begin{BussinesRule}{EM-RN-S011}{Horario de una ofreta de trabajo}
	\BRitem[Descripción:] 
	Una oferta de trabajo puede o no tener un horario definido. En el caso de no ser definido se cuenta con las opciones de Medio Tiempo, Tiempo Completo o Rango. Para los casos, se mostrará la selección de dias y podra introducir el horario de cada dia, los días que no se seleccionen no se tomarán en cuenta y se asignará null. En caso de seleccionar Sin definir no se mostrará ningun submenú.
	\BRitem[Tipo:] Habilitadora.
	\BRitem[Nivel:] Controla la operación.
	\BRitem[Sentencia: ] Sea $T$  el tipo de horario con un identificador único $TipoHorario$.
	\begin{center}
		$\forall \: T  \mid T.seleccion == Medio Tiempo | T.seleccion == Tiempo Completo | T.seleccion == Definido  \Rightarrow se muestra seleccion de dias.$
	\end{center}
\end{BussinesRule}

% RNN4. *******************************************************
\begin{BussinesRule}{EM-RN-N004}{Calificación Promedio de Profesor.} 
	\BRitem[Descripción:] Cada profesor tiene una calificación promedio de desempeño académico, resultado de promediar todas y cada una de las calificaciones registradas por los alumnos.
	\BRitem[Tipo:Habilitadora.] 
	\BRitem[Nivel: Controla la operación] 
	\BRitem[Clase:Integridad.] 
	\BRitem[Sentenia: ] Sea $P$ un profesor, $p$ el promedio del mismo y $C =$ \{ $c_{1},c_{2},c_{3},...,c_{n}$  \} el conjunto de calificaciones registradas por los alumnos, entonces:
	\begin{center}
		$\forall p \in P$, $P=$ ($ \sum_{i=c_1}^{n}c_{i} $) / n.
	\end{center}
	\BRitem[Ejemplo positivo:] que cumple la regla.
		Sea $P_{1}$ un profesor de la ESCOM y $C_{1} =$ \{ $c_{11},c_{12},c_{13},...,c_{1n}$ el conjunto de calificaciones otorgadas por los alumnos al profesor, que además son distintas de 0, entonces:
		\begin{itemize}
			\item ($ \sum_{i=c_1}^{n}c_{i} $) / n != 0, se calculó correctamente el promedio.
			\item ($ \sum_{i=c_1}^{n}c_{i} $) / n = 0, no se calculó correctamente el promedio.
		\end{itemize}
	\BRitem[Ejemplo negativo:] que no cumple la regla.
		Sea $P_{2}$ un profesor de la ESCOM y $C_{2} =$ \{ $c_{21},c_{22},c_{23},...,c_{2n}$ el conjunto de calificaciones otorgadas por los alumnos al profesor, que además son iguales a de 0 o el conjuto es inexistente, entonces:
		\begin{itemize}
			\item ($ \sum_{i=c_1}^{n}c_{i} $) / n != 0, no se calculó correctamente el promedio.
			\item ($ \sum_{i=c_1}^{n}c_{i} $) / n = 0, se calculó correctamente el promedio.
		\end{itemize}
\end{BussinesRule}

% RNN5. *******************************************************
\begin{BussinesRule}{EM-RN-N005}{Boleta o número de empleado}
	\BRitem[Descripción:]Todas los números de boleta o número de empleado tienen que ser aprobadas con el sistema de información de la escuela, así, se sabría si el número de boleta / número de empleados introducidos pertenecen realmente a algun miembro activo de la comunidad de ESCOM, de donde, se permite o no el acceso al sistema.
	\BRitem[Tipo:] Habilitadora.
	\BRitem[Nivel:] Controla la operación.
	\BRitem[Sentenia: ] Sea N una boleta o un número de empleado perteneciente a un alumno A o un profesor P que desee registrarse, entonces:
	\begin{center}
		$\forall \: A|P \in  ESCOM \mid A.verificar(N) == true | P.verificar(N) == True \Rightarrow usuario pertenece a ESCOM.$
	\end{center}
	
	\BRitem[Ejemplo positivo:] que cumple la regla.
		Sea $A_{1}$ un alumno de la ESCOM $P_{2}$ un profesor de la ESCOM
		\begin{itemize}
			\item $A_{1}$.verificar(N) == true, se le permite registrarse.
			\item $P_{1}$.verificar(N) == false, no se permite registrarse.
		\end{itemize}
	\BRitem[Ejemplo negativo:] que no cumple la regla.
		Sea $A_{2}$ un alumno de la ESCOM $P_{2}$ un profesor de la ESCOM
		\begin{itemize}
			\item $A_{2}$.verificar(N) == true, no se le permite registrarse.
			\item $P_{2}$.verificar(N) == false se permite registrarse.
		\end{itemize}
\end{BussinesRule}

% RNN6. *******************************************************
\begin{BussinesRule}{EM-RN-N006}{Eventos Registrados}
	\BRitem[Descripción:]Un evento de en el sistema permite a los alumnos e invitados consultar las diferentes actividades que se realizarán en la ESCOM, se mostrará un icono y el nombre del evento a realizarse, y el podrá consultar mas información del evento que desee. 
	\BRitem[Tipo:] Habilitadora.
	\BRitem[Nivel:] Controla la operación.
	\BRitem[Sentencia: ] Sea $E$ = {$e_{1}$, $e_{2}$, $e_{3}$, ... , $e_{n}$} el conjunto de eventos registrados en el sistema, cada uno con un identificador $idEvento_{n}$  
	\begin{center}
		$(select count(*) from E) > 0$, hay eventos.
	\end{center}
	
	\BRitem[Ejemplo positivo:] que cumple la regla.
		Sea $E1$ = {$e1_{1}$, $e1_{2}$, $e1_{3}$, ... , $e1_{n}$} el conjunto de Eventos con un identificador propio $idEvento1_{n}$, entonces 
		\begin{itemize}
			\item $(select count(*) from E) = 5$, hay propuestas de trabajo registradas 
			
		\end{itemize}
	\BRitem[Ejemplo negativo:] que no cumple la regla.
		Sea $E2$ = {$e2_{1}$, $e2_{2}$, $e2_{3}$, ... , $e2_{n}$} el conjunto de Eventos con un identificador propio $idEvento2_{n}$, entonces 
		\begin{itemize}
			\item $(select count(*) from E) = 0$, no hay propuestas de trabajo registradas 
			
		\end{itemize}
\end{BussinesRule}

% RNN7. *******************************************************
\begin{BussinesRule}{EM-RN-N007}{Propuestas de trabajo registradas}
	\BRitem[Descripción:] Una propuesta de trabajo en el sistema permite a los alumnos consultar diversas oportunidades a las que pueden aplicar laboralmente, por medio de éstas, las empresas ofertan diferentes oportunidades de empleo para informar más fácilmente el perfil de personal y requerimientos que necesitan, así como lo que ofrecen, como sueldo, horas de trabajo, etc. Las información es registrada vía web en el sistema por alguien designado para la tarea específicamente. 
	\BRitem[Tipo:] Habilitadora.
	\BRitem[Nivel:] Controla la operación.
	\BRitem[Sentencia: ] Sea $P$ = {$p_{1}$, $p_{2}$, $p_{3}$, ... , $p_{n}$} el conjunto de propuestas registradas en el sistema, cada una con un identificador $idPropuesta_{n}$  
	\begin{center}
		$(select count(*) from P) > 0$, hay propuestas
	\end{center}
	
	\BRitem[Ejemplo positivo:] que cumple la regla.
		Sea $P1$ = {$p1_{1}$, $p1_{2}$, $p1_{3}$, ... , $p1_{n}$} el conjunto de propuestas con un identificador propio $idPropuesta1_{n}$, entonces 
		\begin{itemize}
			\item $(select count(*) from P) = 5$, hay propuestas de trabajo registradas 
			\item $(select count(*) from P) = 0$, no hay propuestas de trabajo registradas 
		\end{itemize}
	\BRitem[Ejemplo negativo:] que no cumple la regla.
		Sea $P2$ = {$p2_{1}$, $p2_{2}$, $p2_{3}$, ... , $p2_{n}$} el conjunto de propuestas con un identificador propio $idPropuesta2_{n}$, entonces 
		\begin{itemize}
			\item $(select count(*) from P) = 5$, no hay propuestas de trabajo registradas 
			\item $(select count(*) from P) = 0$, hay propuestas de trabajo registradas 
		\end{itemize}
\end{BussinesRule}

% RNN8. *******************************************************
\begin{BussinesRule}{EM-RN-N008}{Citas agendadas}
	\BRitem[Descripción:] Una cita agendada es el convenio por parte del alumno y un profesor para reunirse
	en una fecha y hora determinados por ambos. Siendo el alumno el que proponga estos datos, tomando
	en cuenta el horario del profesor. El profesor aceptará o rechazará dicha cita.
	El lugar en donde se realizará la cita es en uno de los salones de la ESCOM.
	Una cita permite el encuentro de alumnos con profesores, fuera del salón de clase, para tratar temas
	académicos relacionados, o no, con lo visto en clase.
	\BRitem[Tipo:] Habilitadora.
	\BRitem[Nivel:] Controla la operación.
	\BRitem[Sentencia: ] Sea $C$ = {$c_{1}$, $c_{2}$, $c_{3}$, ... , $c_{n}$} el conjunto de citas 
	registradas en el sistema por un alumno y aceptadas por uno o más profesores, cada una con un 
	identificador único $idCita_{n}$.
	\begin{center}
		(SELECT COUNT(*) FROM $C$) $>$ $0$, existe al menos una cita agendada.
	\end{center}
	
	\BRitem[Ejemplo positivo:] que cumple la regla.
		Sea $C1$ = {$c1_{1}$, $c1_{2}$, $c1_{3}$, ... , $c1_{n}$} las citas agendadas por un alumno y
		aceptadas por un profesor y con identificador único $idCita1_{n}$, entonces:
		\begin{itemize}
			\item (SELECT COUNT(*) FROM $C1$) = $100$, existen citas agendadas.
			\item (SELECT COUNT(*) FROM $C1$) = $0$, no existen citas agendadas.
		\end{itemize}
	\BRitem[Ejemplo negativo:] que no cumple la regla.
		Sea $C2$ = {$c2_{1}$, $c2_{2}$, $c2_{3}$, ... , $c2_{n}$} las citas agendadas por un alumno y
		aceptadas por un profesor y con identificador único $idCita2_{n}$, entonces:
		\begin{itemize}
			\item (SELECT COUNT(*) FROM $C2$) = $100$, no existen citas agendadas.
			\item (SELECT COUNT(*) FROM $C2$) = $0$, existen citas agendadas.
		\end{itemize}
\end{BussinesRule}

% RNN9. *******************************************************
\begin{BussinesRule}{EM-RN-N009}{Gráfica de lineas}
	\BRitem[Descripción:] 
	Una oferta de trabajo para los alumnos de ESCOM, tiene cierto período de vigencia, para el administrador es importante obtener datos sobre estas ofertas.
	Se mostrará una gráfica de lineas con los datos de ofertas de trabajo por meses, se realizará una suma para saber cuantas ofertas por mes se han publicado.
	
	\BRitem[Tipo:] Habilitadora.
	\BRitem[Nivel:] Controla la operación.
	\BRitem[Sentencia: ] Sea $O$ = {$o_{1}$, $o_{2}$, $o_{3}$, ... , $o_{n}$} el conjunto de ofertas de trabajo por un mes registradas en el sistema por un administrador cada una con un 
	identificador único $idOferta_{n}$.
	\begin{center}
		$\forall O \in O$, $O=$ {($ \sum_{i=O_1} $)} .
	\end{center}
\end{BussinesRule}

% RNN9. *******************************************************
\begin{BussinesRule}{EM-RN-N009}{Horario citas}
	\BRitem[Descripción:] Una cita agendada es el convenio por parte del alumno y un profesor para reunirse
	en una fecha y hora determinados por ambos. Siendo el alumno el que proponga estos datos, tomando
	en cuenta el horario del profesor. El profesor aceptará o rechazará dicha cita.
	El lugar en donde se realizará la cita es en uno de los salones o cubćiculo del profesor de la ESCOM.
	Una cita permite el encuentro de alumnos con profesores, fuera del salón de clase, para tratar temas
	académicos relacionados, o no, con lo visto en clase.
	\BRitem[Tipo:] Habilitadora.
	\BRitem[Nivel:] Controla la operación.
	\BRitem[Sentencia: ] Sea $F$ la fecha de la cita , $H$ el horario de la cita propuesta por un alumno , $HO$ el horario del profesor, $HC$ horas de clase del profesor , $C$ = {$c1_{1}$, $c1_{2}$, $c1_{3}$, ... , $c1_{n}$} el conjunto de citas agendadas y $DIH$ los dias inhabiles.
	\begin{center}
		$\forall \: C \mid F \&\& H =! F \&\& H. in C (N) == true \&\& (F|H \in HO \&\& H=! HC \&\& F =! DIH ) fecha y hora validos.$

	\end{center}
	\BRitem[Ejemplo positivo:] que cumple la regla.
		Sea $C1$ = {$c1_{1}$, $c1_{2}$, $c1_{3}$, ... , $c1_{n}$} las citas agendadas por un alumno y
		aceptadas por un profesor y con identificador único $idCita1_{n}$ y  entonces $F$ la fecha de la cita y $H$ el horario de la cita propuesta por un alumno , $HO$ el horario del profesor, $HC$ horas de clase del profesor y $DIH$ los dias inhabiles. :
		\begin{itemize}
			\item  F \&\& H =! F \&\& H. in C (N) == true \&\& (
		F|H pertenece HO \&\& H=! HC \&\& F =! DIH ) fecha y hora validos
		\end{itemize}
\end{BussinesRule}

% RNN10. *******************************************************
\begin{BussinesRule}{EM-RN-N010}{Inicio de Sesión}
	\BRitem[Descripción:] Para poder iniciar sesión el usuario deberá estar registrado en el sistema.
	\BRitem[Tipo: ] Habilitadora.
	\BRitem[Nivel: ] Estricta.
	\BRitem[Sentenia: ] Sea U un usuario que desee iniciar sesión, entonces:
	\begin{center}
		$\forall \: U \in  ESCOM \mid U.estadoInscripcion == true \Rightarrow usuario esta inscrito.$
	\end{center}
	\BRitem[Ejemplo positivo:] que cumple la regla.
		\begin{itemize}
			\item Sea $U_{1}$.estadoInscripcion == true se le permite iniciar sesión.
			\item Sea $U_{2}$.estadoInscripcion == false no se permite iniciar sesión.
		\end{itemize}
	\BRitem[Ejemplo negativo:] que no cumple la regla.
		\begin{itemize}
			\item Sea $U_{1}$.estadoInscripcion == false se le permite iniciar sesión.
			\item Sea $U_{2}$.estadoInscripcion == true no se permite iniciar sesión.
		\end{itemize}
\end{BussinesRule}

% RNN11. *******************************************************
\begin{BussinesRule}{EM-RN-N011}{Formato de fotografía}
	\BRitem[Descripción:] Para poder subir alguna fotografía en el sistema, ésta debe
	cumplir con las siguientes caracteristicas: 
		\begin{itemize}
			\item Tener un formato JPG. 
			\item Tener un tamaño menos o igual a 5Mb.
		\end{itemize}
	\BRitem[Tipo: ] Habilitadora.
	\BRitem[Nivel: ] Estricta.
	\BRitem[Sentenia: ] Sea U un usuario y F una 
	fotografía que desea subir al sistema, entonces:
	\begin{center}
		$\forall \: U \in ESCOM \mid F.tamano == 5Mb \&\& (F.formato == PNG || F.formato == png) \Rightarrow$ la fotografía cumple el formato.
	\end{center}
	\BRitem[Ejemplo positivo:] que cumple la regla.
		\begin{itemize}
			\item Sea $F_{1}.tamano = 3Mb ^ F_{1}.formato == PNG$, se permite
			subir la fotografía.
			\item Sea $F_{5}.tamano = 6Mb ^ F_{1}.formato == PNG$, no se permite
			subir la fotografía.
		\end{itemize}
	\BRitem[Ejemplo negativo:] que no cumple la regla.
		\begin{itemize}
			\item Sea $F_{1}.tamano = 4Mb ^ F_{1}.formato == png$, no se permite
			subir la fotografía.
			\item Sea $F_{5}.tamano = 7Mb ^ F_{1}.formato == png$, se permite
			subir la fotografía.
		\end{itemize}

\end{BussinesRule}



% RNN12. *******************************************************
\begin{BussinesRule}{EM-RN-N012}{Tiempo máximo para cancelar cita}
	\BRitem[Descripción:] Para poder cancelar una cita agendada en necesario contar con un límite de tiempo para hacerlo, pues por respeto a los alumnos y profeores y a sus tiempos, se establece que una cita se podrá cancelar máximo dos horas antes de que ésta comience, notificando al otro agente acerca de la cancelación de la misma. Si la cita está a menos de dos hora de realizarce, no puede ser cancelada.
	\BRitem[Tipo: ] Habilitadora.
	\BRitem[Nivel: ] Estricta.
	\BRitem[Sentenia: ] Sea A un alumno, P un profesor, C una cita previamente agendada por A con P y S el servidor de ESCOMobile, entonces:
	\begin{center}
		$\forall \: A \: \&\& P \in ESCOMobile \mid  S.fechaActaul \: <= \: C.fechaAgendada \: \&\& \: S.tiempoActual \: - \: C.tiempoAgendado \: >= \: 2 \: Horas \: \Rightarrow$ se puede cancelar la cita.
	\end{center}
	\BRitem[Ejemplo positivo:] que cumple la regla.
		\begin{itemize}
			\item Sea $S_{1}.fechaActual \: = \: 10 \: SEP \: 2018, \: S_{1}.tiempoActual \: = \: 15:00:00, \: C_{1}.fechaAgendada \: = \: 10 \: SEP \: 2018, \: C_{1}.tiempoAgendado \: = \: 12:00:00$, se permite
			cancelar la cita.
			\item Sea $S_{1}.fechaActual \: = \: 10 \: SEP \: 2018, \: S_{1}.tiempoActual \: = \: 15:00:00, \: C_{1}.fechaAgendada \: = \: 10 \: SEP \: 2018, \: C_{1}.tiempoAgendado \: = \: 14:00:00$, no se permite
			cancelar la cita.
		\end{itemize}
	\BRitem[Ejemplo negativo:] que no cumple la regla.
		\begin{itemize}
			\item Sea $S_{1}.fechaActual \: = \: 10 \: SEP \: 2018, \: S_{1}.tiempoActual \: = \: 15:00:00, \: C_{1}.fechaAgendada \: = \: 10 \: SEP \: 2018, \: C_{1}.tiempoAgendado \: = \: 12:00:00$, no se permite
			cancelar la cita.
			\item Sea $S_{1}.fechaActual \: = \: 10 \: SEP \: 2018, \: S_{1}.tiempoActual \: = \: 15:00:00, \: C_{1}.fechaAgendada \: = \: 10 \: SEP \: 2018, \: C_{1}.tiempoAgendado \: = \: 14:00:00$, se permite
			cancelar la cita.
		\end{itemize}

\end{BussinesRule}