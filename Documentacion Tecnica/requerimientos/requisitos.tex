\section{Requerimientos del sistema}

\begin{requisitos}{Plataforma móvil.}
	\RFitem{Descripción}{La aplicación se ejecutará sobre la plataforma móvil Android.
	De la versión Android 5.0 (Lollipop) a la versión Android 8.0 (Oreo).}
	\RFitem{Propósito}{Para que una gran parte de la comunidad de
	ESCOM disponga de ESCOMobile en sus dispositivos con S.O. Android.}
	\RFitem{Usuario}{Alumno, Profesor, Invitado.}
	\RFitem{Tipo}{Móvil.}
\end{requisitos}

\begin{requisitos}{Lenguaje de programación.}
	\RFitem{Descripción}{La aplicación ESCOMobile se escribirá en el lenguaje de programación 
	Kotlin versión 1.2.}
	\RFitem{Propósito}{Estar a la vanguardia tecnológica.
	Optimizar el código para realizar mismas funciones que Java.
	Se puede mezclar con código de Java sin afectar la funcionalidad.
	Es seguro frente a NullPointerException.}
	\RFitem{Usuario}{N/A.}
	\RFitem{Tipo}{Móvil.}
\end{requisitos}

\begin{requisitos}{Mapa de ESCOM.} 
	\RFitem{Descripción}{La aplicación tendrá un módulo para consultar el mapa de la ESCOM. Se 
	mostrará una vista aérea de los pisos de los edificios.}
	\RFitem{Propósito}{Dar una vista general de los edificios que conforman al plantel y orientar a
	alumnos, a ubicar donde se encuentran ubicados los salones, cubículos, clubes, biblioteca 
	y áreas administrativas en ESCOM.}
	\RFitem{Usuario}{Alumno, Profesor, Invitado.} 
	\RFitem{Tipo}{Móvil.} 
\end{requisitos}

\begin{requisitos}{Presentación del mapa de ESCOM.} 
	\RFitem{Descripción}{El mapa de ESCOM que mostrará la aplicación es en 2D, la vista será aérea 
	y mostrará texto que describe cada área.}
	\RFitem{Propósito}{Hacer la navegación en el mapa sencilla para el actor.} 
	\RFitem{Usuario}{Alumno, Profesor, Invitado.}
	\RFitem{Tipo}{Móvil.} 
\end{requisitos}
 
\begin{requisitos}{Vistas del mapa de ESCOM.} 
	\RFitem{Descripción}{El mapa mostrará una vista de la planta baja, primer piso y segundo piso 
	del plantel. Solo se puede mostrar una vista a la vez.} 
	\RFitem{Propósito}{Saber en qué piso se encuentran ubicados los laboratorios, salones, cubículos, 
	etc. Para orientar al actor.}
	\RFitem{Usuario}{Alumno, profesor, Invitado.}
	\RFitem{Tipo}{Móvil.} 
\end{requisitos}

\begin{requisitos}{Vistas de áreas dentro del mapa de ESCOM.}
	\RFitem{Descripción}{Se mostrará en el mapa textos descriptivos, como el nombre del salón, de 
	cada área, divisiones de los salones y áreas administrativas.}
	\RFitem{Propósito}{Ayudar a ubicar con mayor facilidad el lugar que busca el actor.} 
	\RFitem{Usuario}{Alumno, Profesor, Invitado.}
	\RFitem{Tipo}{Móvil.} 
\end{requisitos}	

\begin{requisitos}{Perfil o cuenta del profesor.}
	\RFitem{Descripción}{Se tendrá un apartado para configurar el perfil del profesor. Es en este 
	apartado podrá modificar su correo electrónico, fotografía, etc. Y consultar sus 
	citas agendadas.} 
	\RFitem{Propósito}{El profesor podrá publicar la información necesaria para que los alumnos lo
	contacten.} 
	\RFitem{Usuario}{Profesor.}
	\RFitem{Tipo}{Móvil.} 
\end{requisitos}

\begin{requisitos}{Registro del Profesor.} 
	\RFitem{Descripción}{La aplicación permitirá a los profesores registrarse y poder acceder a su perfil.}
	\RFitem{Propósito}{\begin{itemize}
		\item Los alumnos podrán saber en qué horario pueden acudir con el profesor.
		\item Ubicar de manera única a los profesores de ESCOM.
	\end{itemize}}
	\RFitem{Usuario}{Alumno, Profesor.} 
	\RFitem{Tipo}{Móvil.} 
\end{requisitos}

\begin{requisitos}{Información para el registro del profesor.}
	\RFitem{Descripción}{Los campos necesarios para el registro del profesor son los siguientes: 
		\begin{itemize}
			\item Nombre completo.
			\item Numero de empleado.
			\item Correo electrónico.
			\item Contraseña.
		\end{itemize}} 
	\RFitem{Propósito}{Identificar a los profesores de ESCOM.} 
	\RFitem{Usuario}{Profesor.} 
	\RFitem{Tipo}{Móvil.}
\end{requisitos}

\begin{requisitos}{Información editable del perfil del profesor.} 
	\RFitem{Descripción}{La información que puede editar el profesor en su perfil es la siguiente:
		\begin{itemize}
			\item Correo electrónico.
			\item Contraseña.
			\item Fotografía (opcional).
		\end{itemize}} 
	\RFitem{Propósito}{Mantener actualizada la información del profesor.} 
	\RFitem{Usuario}{Profesor.} 
	\RFitem{Tipo}{Móvil.} 
\end{requisitos}

\begin{requisitos}{Información para el registro de alumnos.} 
	\RFitem{Descripción}{La información necesaria para poder registrar a un alumno es la siguiente:
		\begin{itemize}
			\item Nombre completo.
			\item Número de boleta.
			\item Correo electrónico.
			\item Contraseña.
		\end{itemize}}
	\RFitem{Propósito}{Saber si el alumno es parte de la comunidad de ESCOM.}
	\RFitem{Usuario}{Alumno.}
	\RFitem{Tipo}{Móvil.} 
\end{requisitos}

\begin{requisitos}{Perfil del alumno.} 
	\RFitem{Descripción}{El alumno contará con un perfil en el que se mostrará su información y donde 
	podrá consultar sus citas agendadas.}
	\RFitem{Propósito}{Mostrar los datos del alumno registrado.} 
	\RFitem{Usuario}{Alumno.}
	\RFitem{Tipo}{Móvil.} 
\end{requisitos}

\begin{requisitos}{Acciones del Alumno.}
	\RFitem{Descripción}{El alumno puede Agregar, editar o eliminar información de su perfil.}
	\RFitem{Propósito}{El alumno podrá personalizar su perfil.}
	\RFitem{Usuario}{Alumno.}
	\RFitem{Tipo}{Móvil.} 
\end{requisitos}

\begin{requisitos}{Citas.}
	\RFitem{Descripción}{La aplicación permite a los alumnos gestionar sus citas con los profesores
	(registrar nuevas citas, editar, eliminar y consultar las ya existentes).}
	\RFitem{Propósito}{El alumno no pierda tiempo al buscar a un profesor.}
	\RFitem{Usuario}{Alumno.}
	\RFitem{Tipo}{Móvil.} 
\end{requisitos}

\begin{requisitos}{Gestión citas profesores.}
	\RFitem{Descripción}{Los profesores gestionan las citas registradas con ellos (aceptan o rechazan la
	cita según el propósito y urgencia de la misma), (Consultar, evaluar). } 
	\RFitem{Propósito}{El profesor pueda controlar sus citas y explicar motivos del cual no puede.}
	\RFitem{Usuario}{Profesor.}
	\RFitem{Tipo}{Móvil.} 
\end{requisitos}

\begin{requisitos}{Estadísticas.} 
	\RFitem{Descripción}{La app genera estadísticas con información importante de los profesores, como
	el número de citas que tiene, las que ha aceptado o rechazado, etc.}
	\RFitem{Propósito}{Para brindar al alumno una mejor visión sobre el comportamiento del profesor ante las citas.} 
	\RFitem{Usuario}{Alumno.} 
	\RFitem{Tipo}{Móvil.} 
\end{requisitos}

\begin{requisitos}{Editar datos del alumno. } 
	\RFitem{Descripción}{La aplicación permite al usuario editar ciertos datos o intereses dentro de su
	perfil.} 
	\RFitem{Propósito}{El usuario mantiene al día (actualizados) su información e intereses.} 
	\RFitem{Usuario}{Alumno.}
	\RFitem{Tipo}{Móvil.} 
\end{requisitos}

\begin{requisitos}{Eliminar alumnos.} 
	\RFitem{Descripción}{La aplicación permite al usuario dar de baja (eliminar) su cuenta previamente
	registrada, eliminado así su perfil e información.} 
	\RFitem{Propósito}{El usuario elimine una cuenta que no desea utilizar más. } 
	\RFitem{Usuario}{Alumno.}
	\RFitem{Tipo}{Móvil.} 
\end{requisitos}

\begin{requisitos}{Consulta de perfiles. } 
	\RFitem{Descripción}{La aplicación al usuario consultar los perfiles únicamente de los profesores, así
	como cierta información pública como sus horarios, fotos, descripción, estadísticas, etc.} 
	\RFitem{Propósito}{El usuario conoce información importante sobre los profesores.} 
	\RFitem{Usuario}{Alumno, Profesor.} 
	\RFitem{Tipo}{Móvil.} 
\end{requisitos}

\begin{requisitos}{Comentarios en perfiles.} 
	\RFitem{Descripción}{La aplicación permite al usuario hacer comentarios en los perfiles de los
	profesores. } 
	\RFitem{Propósito}{Dar al profesor una idea de lo que los alumnos piensan acerca de él. } 
	\RFitem{Usuario}{Alumno.} 
	\RFitem{Tipo}{Móvil.} 
\end{requisitos}

\begin{requisitos}{Usuarios no registrados.} 
	\RFitem{Descripción}{El sistema muestra a los usuarios que no cuenten con una cuenta activa (pues no se
	han registrado) el mapa de la ESCOM, así como la distribución de la misma.} 
	\RFitem{Propósito}{Extender la información y el servicio del mapa a quienes no desean o puedan
	registrar una cuenta.} 
	\RFitem{Usuario}{Invitado.}
	\RFitem{Tipo}{Móvil.} 
\end{requisitos}

\begin{requisitos}{Concepto de GUIs.} 
	\RFitem{Descripción}{El sistema ofrece al usuario interfaces gráficas basadas en Material Design de
	Google.} 
	\RFitem{Propósito}{Ofrecer interfaces frescas y que se puedan adaptar a diferentes tamaños y
	orientaciones de pantalla de los dispositivos.} 
	\RFitem{Usuario}{Alumno, Profesor, Visitante.} 
	\RFitem{Tipo}{Web / móvil.} 
\end{requisitos}

\begin{requisitos}{Estilo de GUIs.} 
	\RFitem{Descripción}{El sistema ofrece al usuario interfaces gráficas minimalistas y con el contenido
	necesario.}
	\RFitem{Propósito}{El usuario siente empatía con el sistema. } 
	\RFitem{Usuario}{Alumno, Profesor, Visitante.}
	\RFitem{Tipo}{Web / móvil.} 
\end{requisitos}

\begin{requisitos}{Diseño de GUIs.} 
	\RFitem{Descripción}{El sistema ofrece al usuario interfaces gráficas visualmente atractivas y
	llamativas usando la psicología del color y formas.} 
	\RFitem{Propósito}{El usuario siente aprende a usar fácil y rápidamente la app.} 
	\RFitem{Usuario}{Alumno, Profesor, Visitante.} 
	\RFitem{Tipo}{Web / móvil.} 
\end{requisitos}

\begin{requisitos}{App sin conexión.} 
	\RFitem{Descripción}{El sistema ofrece un modo “sin conexión”, mismo que permite al usuario
	visualizar ciertos servicios de la app (por medio del caché) sin necesidad de tener una conexión a
	Internet.} 
	\RFitem{Propósito}{El usuario continúa usando servicios de la app incluso sin conexión.} 
	\RFitem{Usuario}{Alumno, Profesor, Visitante} 
	\RFitem{Tipo}{Móvil.} 
\end{requisitos}

\begin{requisitos}{Cancelación de citas.} 
	\RFitem{Descripción}{El usuario puede cancelar una cita previamente registrada, siempre y cuando
	falte determinado tiempo para la cita; ofreciendo además una razón en un apartado que la app 
	también muestra.} 	
	\RFitem{Propósito}{El alumno y el profesor puedan realizar otras actividades cuando sea necesario
	cancelar.} 
	\RFitem{Usuario}{Alumno, Profesor.}
	\RFitem{Tipo}{Móvil.} 
\end{requisitos}

\begin{requisitos}{Campos obligatorios.} 
	\RFitem{Descripción}{El sistema valida que no se omitan los campos marcados como obligatorios.} 
	\RFitem{Propósito}{No se omita información necesaria e importante para el sistema.} 
	\RFitem{Usuario}{Alumno, Profesor.} 
	\RFitem{Tipo}{Web / móvil.} 
\end{requisitos}

\begin{requisitos}{Información correcta.}
	\RFitem{Descripción}{El sistema valida que los datos introducidos en cada campo cumplan con un
	formato determinado.} 
	\RFitem{Propósito}{Los datos proporcionados tengan el formato adecuado para su uso correcto.} 
	\RFitem{Usuario}{Alumno, profesor.} 
	\RFitem{Tipo}{Web / móvil.} 
\end{requisitos}

\begin{requisitos}{Consulta de la Bolsa de Trabajo}
	\RFitem{Descripción}{El usuario puede consultar las ofertas de trabajo publicadas para la Escuela
	Superior de Cómputo desde la app.} 
	\RFitem{Propósito}{Que el usuario esté informado sobre las ofertas de trabajo vigentes.} 
	\RFitem{Usuario}{Alumno.} 
	\RFitem{Tipo}{Móvil.} 
\end{requisitos}

\begin{requisitos}{Consulta de la Bolsa de Trabajo}
	\RFitem{Descripción}{El usuario puede consultar las ofertas de trabajo publicadas para la Escuela
	Superior de Cómputo desde la app.} 
	\RFitem{Propósito}{Que el usuario esté informado sobre las ofertas de trabajo vigentes.} 
	\RFitem{Usuario}{Alumno.} 
	\RFitem{Tipo}{Móvil.} 
\end{requisitos}

\begin{requisitos}{ESCOMobile Bolsa}
	\RFitem{Descripción}{El sistema cuenta con un módulo web dedicado a la bolsa de trabajo para la ESCOM.} 
	\RFitem{Propósito}{Tener control sobre las empresas y ofertas de trabajo registradas en ESCOMobile.} 
	\RFitem{Usuario}{Encargado de departamento de extensión y servicios educativos.} 
	\RFitem{Tipo}{Web.} 
\end{requisitos}

\begin{requisitos}{Registro de empresas}
	\RFitem{Descripción}{El usuario puede registrar nuevas empresas que ofertan empleo para la ESCOM.} 
	\RFitem{Propósito}{Agregar nuevas empresas para ofertar empleos.} 
	\RFitem{Usuario}{Encargado de departamento de extensión y servicios educativos.} 
	\RFitem{Tipo}{Web.} 
\end{requisitos}

\begin{requisitos}{Eliminar empresas}
	\RFitem{Descripción}{El usuario puede eliminar empresas previamente registradas y su información asociada.} 
	\RFitem{Propósito}{Mantener siempre actualizada la información de las empresas.} 
	\RFitem{Usuario}{Encargado de departamento de extensión y servicios educativos.} 
	\RFitem{Tipo}{Web.} 
\end{requisitos}

\begin{requisitos}{Editar empresas}
	\RFitem{Descripción}{El usuario puede editar empresas previamente registradas y su información asociada.} 
	\RFitem{Propósito}{Mantener siempre actualizada la información de las empresas.} 
	\RFitem{Usuario}{Encargado de departamento de extensión y servicios educativos.} 
	\RFitem{Tipo}{Web.} 
\end{requisitos}

\begin{requisitos}{Registro de ofertas de trabajo}
	\RFitem{Descripción}{El usuario puede registrar nuevas ofertas de trabajo para la ESCOM.} 
	\RFitem{Propósito}{Agregar nuevas ofertas de trabajo para consulta de los alumnos.} 
	\RFitem{Usuario}{Encargado de departamento de extensión y servicios educativos.} 
	\RFitem{Tipo}{Web.} 
\end{requisitos}

\begin{requisitos}{Eliminar ofertas de trabajo}
	\RFitem{Descripción}{El usuario puede eliminar ofertas de trabajo previamente registradas y su información asociada.} 
	\RFitem{Propósito}{Mantener siempre actualizada la información de las ofertas de trabajo.} 
	\RFitem{Usuario}{Encargado de departamento de extensión y servicios educativos.} 
	\RFitem{Tipo}{Web.} 
\end{requisitos}

\begin{requisitos}{Editar ofertas de trabajo}
	\RFitem{Descripción}{El usuario puede editar ofertas de trabajo previamente registradas y su información asociada.} 
	\RFitem{Propósito}{Mantener siempre actualizada la información de las ofertas de trabajo.} 
	\RFitem{Usuario}{Encargado de departamento de extensión y servicios educativos.} 
	\RFitem{Tipo}{Web.} 
\end{requisitos}