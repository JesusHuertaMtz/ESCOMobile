\noindent 
En este apartado de describen las tecnologías necesarias para el desarrollo de la aplicación móvil ESCOMobile así como de su complemento ESCOMobile Bolsa, las razones por las cuales son necesarias y una breve comparación con tecnologías similares. 
\noindent
Dichas tecnologías se detallan a continuación.

\begin{requisitos}{Plataforma de la aplicación.}
	\RFitem{Descripción}{La aplicación estará disponible para smartphones Android.}
	\RFitem{Propósito}{Que la mayor parte de la población escolar pueda tener acceso a la aplicación y a los servicios que ésta proveé.}
	\RFitem{Versión}{Android 5.O Lollipop en adelante.}
\end{requisitos}

\begin{requisitos}{Lenguaje de programación.}
	\RFitem{Descripción}{Para la programación de la aplicación se utiliza principalmente el lenguaje Kotlin.}
	\RFitem{Propósito}{Lograr una fácil integración e interacción de la aplicación con la plataforma android, además de estar actualizada con la tendencia de desarrollo Android.}
	\RFitem{Versión}{1.2}
\end{requisitos}

\begin{requisitos}{Entorno de desarrollo integrado.}
	\RFitem{Descripción}{Se hace uso de Android Studio como entorno de desarrollo.}
	\RFitem{Propósito}{Tener al alcance una integración completa entre plataforma y lenguaje.}
	\RFitem{Versión}{}
\end{requisitos}

\begin{requisitos}{Sistema Gestor de Base de Datos.}
	\RFitem{Descripción}{Para este caso se hace uso de MySQL.}
	\RFitem{Propósito}{Fácil interacción con la infraestructura disponible en la ESCOM.}
	\RFitem{Versión}{}
\end{requisitos}

\begin{requisitos}{Servidor}
	\RFitem{Descripción}{El servidor disponible para ESCOMobile es Apache.}
	\RFitem{Propósito}{Fácil interacción con la infraestructura disponible en la ESCOM.}
	\RFitem{Versión}{}
\end{requisitos}

\begin{requisitos}{Lenguaje para el servidor.}
	\RFitem{Descripción}{Para el lenguaje del servidor se ha decidido usar PHP.}
	\RFitem{Propósito}{Fácil interacción con la infraestructura disponible en la ESCOM.}
	\RFitem{Versión}{}
\end{requisitos}

\begin{requisitos}{Lenguaje para desarrollo web.}
	\RFitem{Descripción}{Para la estructura web del proyecto se optó por usar HTML.}
	\RFitem{Propósito}{Acceder y hacer uso de todas las ventajas que HTML brinda en su versión actual.}
	\RFitem{Versión}{5}
\end{requisitos}

\begin{requisitos}{Bocetos de pantlallas.}
	\RFitem{Descripción}{Para el boceto de las interfaces gráficas de usuario se utiliza balsamiq mockups, pues propone una completa gama de esquemas, botones y demás interfaces para bosquejar pantallas móviles de una manera rápida y sencilla.}
	\RFitem{Propósito}{Realizar bocetos realistas, muy cercanos a las interfaces de la aplicación final.}
	\RFitem{Versión}{}
\end{requisitos}

\begin{requisitos}{Diagramas UML.}
	\RFitem{Descripción}{Para el desarrollo de los diagramas UML propuestos para la realización del proyecto de hace uso de la herramienta Visual Paradigm.}
	\RFitem{Propósito}{Esquematizar de manera clara la estructura y comportamiento del sistema por medio de diagramas UML.}
	\RFitem{Versión}{}
\end{requisitos}

\begin{requisitos}{Documentación.}
	\RFitem{Descripción}{Para la realización de los diferentes documentos técnicos se utiliza TexMaker y MikTek.}
	\RFitem{Propósito}{Lograr un documento estructurado, legible y fácil de corregir, modificar y extender.}
	\RFitem{Versión}{}
\end{requisitos}
