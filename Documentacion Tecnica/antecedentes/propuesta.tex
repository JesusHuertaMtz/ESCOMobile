\section{Propuesta de solución}

\noindent
Se propone implementar un sistema móvil para dispositivos Android que ayude a profesores y alumnos a interactuar y llevar una mejor 
comunicación, todo por medio de la difusión de información acerca de profesores, sus horarios, unidades de aprendizaje que imparte, estadísticas, comentarios, etc; así como la posibilidad de generar citas para apoyar el correcto aprendizaje de los 
alumnos o para atender situaciones académicas que se lleguen a presentar a lo largo de los cursos.
\newline
\newline
Se trata de una aplicación en la cual el usuario (alumnos y profesores) puedan gestionar estos tiempos y horarios dedicados al aprendizaje y a la atención de los alumnos por parte del profesor. Se cuenta con funcionalidades o perspectivas diferentes para cada uno usuarios, como
el hecho de generar solicitudes de citas para los alumnos y el aceptar o rechazarlas por los profesores, o la posibilidad que se presenta a los alumnos de mantenerse al día con la información referente a la escuela, nos referimos concretamente al mapa de ESCOM mostrado en la app, el cual ilustra la distribución de salones, cubículos y demás áreas dentro del plantel, contando también con la búsqueda de profesores de interés, ubicarlos en el mapa, consultar sus perfiles, horarios, estadísticas, comentarios o bien, agendar citas, mismas que se espera solucionen ciertos problemas y/o dudas que puedan tener los alumnos en las unidades de aprendizaje. Finalmente se cuenta también con un módulo dedicado a la difusión de la bolsa de trabajo en la Superior de Cómputo, donde se puede registrar empresas quienes ofertan empleos, mismos que son publicados en las redes sociales asociadas y en la propia app móvil para consulta directa de los alumnos.
\newline
\newline
Con lo anterior se pretende apoyar a los estudiantes a reforzar su conocimiento y aprendizaje por medio de las asesorías y comunicación más cercana con sus profesores. La aplicación móvil lleva por nombre ESCOMobile.