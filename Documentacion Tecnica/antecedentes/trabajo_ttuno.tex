\newline
\section{Trabajo realizado en el periodo de Trabajo Terminal I}

\noindent
En este capítulo se muestran las iteraciones que se realizaron durante los primeros 4 meses del año 2018, periodo en el cual se llevó a cabo el desarrollo de trabajo terminal I, siguiendo la metodología Scrum, por lo que el trabajo se dividió en sprints (iteraciones). Con cada iteración se trabajaban nuevos aspectos de la aplicación, concentrándonos principalmente en el análisis y en el diseño de la propia, pues son éstas las bases para obtener los mejores resultados, trabajando, además, en cada iteración sobre correcciones de las anteriores, considerando errores y aspectos a mejorar, para en efecto llegar a los antes mencionados. Sin embargo, no solo se trabajó con el análisis y el diseño, poniendo atención también en el desarrollo, pues se fijan los pilares que, junto al análisis, pronto nos llevarían a la implementación directa de ESCOMobile. Así, lo anterior se desglosa a continuación: 
\newline
\newline
En la primera iteración se obtuvo un primer acercamiento al desarrollo de aplicaciones en Android y la forma en que éstas trabajan, pues se realizó una investigación acerca de los aspectos fundamentales las mismas, con el fin de comprender los conceptos del desarrollo de aplicaciones en esta plataforma móvil y poder llevarlos a la práctica. Es aquí donde se realizan pequeñas apps independientes de prueba con temas fuera del concepto de ESCOMobile, pero que en un futuro sería de gran utilidad para el desarrollo de la misma. En resumen, en esta iteración se realizó: 
\begin{itemize}
	\item Investigación teórica sobre el funcionamiento de Android, sus conceptos y aplicaciones.
	\item Desarrollo de pequeñas aplicaciones de prueba con temas ''aislados'' para reafirmar lo aprendido.
\end{itemize}

\noindent
En la siguiente iteración se comienza a organizar la realización de la app en cuanto a análisis y diseño nos referimos, se hizo un análisis de lo que se va a realizar con ESCOMobile, realizando encuestas, y preguntando opiniones para formular una problemática y encontrar la forma óptima de atacarla y cumplir los objetivos. De ese análisis se obtuvieron los requisitos funcionales y no funcionales del sistema, un primer acercamiento a los posibles casos de uso a contemplar, así como la identificación de los actores a interactuar con la app. Concretamente, en esta iteración se llevó a cabo: 
\begin{itemize}
	\item Encuesta a alumnos de ESCOM sobre su experiencia en la misma.
	\item Detección de requerimientos funcionales y no funcionales.
\end{itemize}

\noindent
A lo largo de esta tercera iteración se comienza el análisis y diseño de la aplicación, proponiendo un diseño limpio pensado para smartphones android. Se establecen los dos primeros módulos de la app, tal cual se pensó en ese momento: acceso y mapa, teniendo así la introducción del que fuera el centro del futuro prototipo, el mapa de la ESCOM. Diseñando entonces el mismo, gracias a un bosquejo de la Superior de cómputo, fotografías, nuestra experiencia en la misma y el apoyo de una arquitecta quien se encargó de digitalizar ideas y bocetos. Se comienza redacción de documento técnico de la app, con casos de uso de los módulos previamente mencionados, mensajes, reglas de negocio y maquetado. Todo ello con LaTeX y Balsamiq Mockups. El avance que se tuvo en esta iteración fue: 
\begin{itemize}
	\item Bosquejo para implementar mapa de ESCOM.
	\item Digitalización de bocetos del mapa de ESCOM, para la primera planta de la misma.
	\item Detección de los dos primeros módulos del sistema (acceso y mapa) y sus casos de uso.
	\item Diseño de pantallas (maquetas) para los casos de uso de los módulos acceso y mapa.
	\item Redacción de primeras reglas de negocio y mensajes para los anteriores.
	\item Redacción de los casos de uso. 
\end{itemize}

\noindent
Para la cuarta iteración se corrigen observaciones de la iteración anterior, para las maquetas, reglas de negocio, mensajes y casos de uso en general. Se comienza con la construcción de la base de datos del sistema, tras previo análisis de la misma. Se realiza modularización completa del sistema obteniendo del mismo 10 módulos (acceso, mapa, alumno, profesor, administración, cita, evento, web evento, web club, web bolsa) y los casos de uso que a éstos pertenecen, para los cuales se diseñan pantallas (maquetas). El trabajo que se realizó durante el presente fue: 
\begin{itemize}
	\item Correcciones a las observaciones realizadas en la iteración anterior. 
	\item Diseño de pantallas (maquetas) para los casos de uso de los 8 módulos restantes.
	\item Redacción de primeras reglas de negocio y mensajes para los módulos de alumno y profesor.
	\item Redacción de los casos de uso para los módulos de alumno y profesor.
\end{itemize}

\noindent
En la presente iteración, además de trabajar de nueva cuenta en correcciones realizadas durante la iteración anterior, nos concentramos en la redacción de casos de uso para los módulos de administración, cita, evento y web evento, así como las reglas de negocio y mensajes usados en éstos. Así, el trabajo realizado en esta iteración fue:
\begin{itemize}
	\item Correcciones a las observaciones realizadas en la iteración anterior. 
	\item Redacción de reglas de negocio y mensajes para los módulos administración, cita, evento y web evento.
	\item Redacción de casos de uso para los antes mencionados.
\end{itemize}

\noindent
Durante la sexta iteración, se corrigen observaciones anteriores y se comienza con la redacción de la documentación técnica del sistema que contiene casos de uso (previamente escritos), introducción, justificación, objetivos, marco teórico, estado del arte, análisis de riesgos, descripción de pantallas, etc. Se redactan además casos de uso para los módulos restantes (web club y web bolsa). Se plantea idea para prototipo a entregar en la presentación de TT1. 
\begin{itemize}
	\item Correcciones a las observaciones realizadas en la iteración anterior. 
	\item Redacción de documento técnico con los puntos mencionados.
	\item Redacción de reglas de negocio y mensajes para los módulos de web club y web bolsa.
	\item Redacción de casos de uso para los módulos descritos. 
	\item Planeación de prototipo para TT1. 
\end{itemize}

\noindent
A lo largo de esta séptima iteración se pone especial énfasis en la corrección del documento técnico, los casos de uso y capítulos que contiene. Se trabaja con el prototipo a entregar, el cual se decidió que tendría los dos primeros módulos del sistema, acceso y mapa, siendo este último la versión para el actor ''invitado'', es decir, que solo permite consulta del mapa propiamente dicho, sin búsquedas o añadidos. Se comienza redacción de reporte final. 
\begin{itemize}
	\item Correcciones al documento técnico del sistema.
	\item Desarrollo del prototipo para entrega en TT1.
	\item Redacción de reporte final. 
\end{itemize}

\noindent
En la octava y última iteración se corrigen ambos documentos y prototipo. Se realizan diapositivas a presentar y se exponen previamente ante directores y una sinodal (Rocío Palacios). Se prepara todo para la presentación oficial.

\subsection{Resultados obtenidos durante el Trabajo Terminal I}

\noindent
Los resultados obtenidos durante el periodo de desarrollo de trabajo terminal 1 son principalmente dos, mismos que se mencionan a continuación:
\newline
\newline
\textbf{Documentación}: Como bien se puede observar en los párrafos anteriores, el análisis y el diseño fueron pieza clave durante el tiempo trascurrido en el periodo de TT1, teniendo así un principal enfoque en los casos de uso y la documentación que éstos conllevan. Así bien, se generan dos principales documentos. El primero de ellos, la documentación técnica del sistema, contiene aquellos aspectos importantes para el mismo, como introducción, justificación, problemática, propuesta de solución, objetivos a alcanzar, diagramas UML, arquitectura de la aplicación, casos de uso, modelado y reglas del negocio, entre otros. El segundo documento, es un reporte que contiene el trabajo que se realizó a lo largo del TT1, cómo se realizó y lo que se consiguió con el antes mencionado. 
\newline
\newline
\textbf{Prototipo de aplicación}: Se desarrolla un prototipo de aplicación basado en los dos primeros módulos de ESCOMobile, los cuales son acceso y mapa. Teniendo en de este último solo el acceso a invitados, es decir, sin permitir realizar búsquedas. Las acciones concretas que el prototipo permite son: registro de un usuario, login para el mismo y consulta de mapa para invitados.