\noindent
El Instituto Politécnico Nacional (IPN) es la institución educativa rectora de la educación tecnológica pública en México en los niveles medio superior, superior y posgrado. Tiene la misión de formar integralmente capital humano capaz de ejercer el liderazgo en los ámbitos de su competencia, con una visión global, para contribuir al desarrollo social y económico de México. 
El Instituto se visualiza como una institución de vanguardia, incluyente, transparente y eficiente que contribuye al desarrollo global, a través de sus funciones sustantivas, con calidad ética y compromiso social. 
A través de su historia, el Politécnico se ha caracterizado por ser una Institución que ha evolucionado de acuerdo a las necesidades y realidades del país, reflejando en su imagen sus orígenes y razón de ser, lo que permite su fácil identificación por las personas y llegando a ser coloquialmente conocido como “el Politécnico” o “el Poli” \cite{IPN}.  

\noindent
\newline
Es el IPN el alma mater de diferentes instituciones y escuelas públicas en México, tal es el caso de la Escuela Superior de Cómputo, escuela donde se procura que la formación de los estudiantes sea integral, pues nos solo se imparten materias referentes a la formación orientada a su carrera impartida (ingeniería en Sistemas Computacionales), sino que contempla diferentes materias enfocadas a desarrollar diferentes aspectos y habilidades que los alumnos pueden poseer, proponiendo además, la posibilidad de participar en clubes y equipos deportivos y culturales. Así, es claro que la ESCOM se preocupa por lograr en sus alumnos una educación integral y de calidad. Sin embargo, ésto se ve opacado en numerosas ocasiones, pues a causa de la desorganización o mala comunicación entre los integrantes de la comunidad de la ESCOM, no se cumplen completamente el tener esta educación integral de la que se habla, siendo esto un problema. Pues en la ESCOM, además, la población tiende a ser individualista y aislada, provocando así barreras de comunicación y progreso. Dentro del plantel, las diferentes maneras de difusión de información pueden no ser las óptimas, pues no se alcanza a distribuir de manera correcta a todos los integrantes de la comunidad, por poner un ejemplo, la localización de los profesores que en el plantel imparten clases, así como la información referente a sus horarios, puede resultar pobre o poco clara; siendo esto un problema, nos debemos enfocar en comprenderlo y aplicar soluciones para que no se presente más. 

\noindent 
\newline
Es por ello que se propone una solución que permita, entre otras cosas, conocer la información presente en la ESCOM, para que alumnos y profesores puedan consultarla y conocer lo que en ella se describa. 

% Mercado meta
\section{Mercado meta}

\noindent
La aplicación ESCOMobile está dirigida principalmente a los alumnos de la ESCOM, aunque son los profesores parte importante de la misma. Así, es importante establecer que, a pesar de que la aplicación podrá ser utilizada por personas ajenas a la institución previamente mencionada, el entorno en el que el usuario final se desarrolla es de suma importancia para la correcta comprensión del problema y de la solución propuesta con ESCOMobile. Así bien, debemos conocer y familiarizarnos con el entorno de la aplicación y los usuarios a los que está destinada. 
