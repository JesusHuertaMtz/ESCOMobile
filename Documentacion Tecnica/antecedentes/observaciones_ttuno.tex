\newline
\section{Observaciones realizadas en la presentación de Trabajo Terminal I}

\noindent
Durante la presentación de trabajo terminal I, antes mencionada, se expusieron ante los directores, sinodales (solo dos presentes de los tres asignados, siendo éstos la profesora Rocío Palacios y el profesor Asunción Enriquez) y el resto de asistentes los resultados del trabajo realizado a lo largo del periodo de TT1, teniendo por parte del jurado calificador preguntas que sirvieron para aclarar dudas acerca del sistema o bien, para plasmar puntos a corregir en aspectos concretos de ESCOMobile. Dichas observaciones se desglosan a continuación, organizadas por cada uno de los participantes del jurado.  
\newline
\newline
\textbf{Vélez Saldaña Ulises, Director}: Las observaciones sobre el trabajo por parte del Director Ulises fueron mínimas, centrándose en la forma en que se iba a obtener calidad en el sistema gracias al hecho de definir métricas ante las mismas, además de solicitar la entrega de más prototipos a lo largo del trabajo terminar II, así como el análisis y diseño ya terminado y corregido. Por otro lado, el profesor resaltó el trabajo realizado, haciendo hincapié en el análisis, diseño, arquitectura y prototipo obtenidos.
\newline
\newline
\textbf{Luna Benoso Benjamín, Director}: Por parte del profesor Benjamín, las observaciones fueron mayormente positivas, pues, al igual que el profesor Ulises, hizo notar de buena forma el trabajo realizado durante el semestre de TT1. Sin embargo, indicó que quedó un punto sin abordar que, a su punto de vista, era de suma importancia, éste es: el precio de la aplicación y la forma en que se va a distribuir. Tomando conciencia al respecto se tomó como un punto a resolver, pues concordamos con la opinión del profesor al ser un punto importante. 
\newline
\newline
\textbf{Palacios Solano Rocío, Sinodal}: Las observaciones realizadas por parte de la profesora Rocío fueron de igual manera bien recibidas, útiles y concretas. La sinodal se centró en el uso que los profesores le darían a la aplicación y el rol que éstos jugarían en la misma. Pues no todos los profesores de la ESCOM tienen el mismo perfil, es decir, unos son profesores con base, otros son invitados de otras escuelas, entre otros. De igual forma hizo notar ciertos problemas de redacción y ortografía en los documentos entregados. 
\newline
\newline
\textbf{Enríquez Zárate José Asunción, Sinodal}: Finalmente, tenemos las observaciones del sinodal Asunción, que fueron rígidas y numerosas. El profesor hizo notar que la redacción y la ortografía con las que se redactaron los documentos fueron pobres, habiendo en ellos bastantes errores ortográficos, mismos que combinados con una mala redacción y organización resultarían en un documento con más de un punto a mejorar. Por otro lado, el profesor hizo notar su interés con el módulo de web bolsa, mismo que, según él, carecía de planeación, pidiendo se pusiera mayor atención en el antes mencionado. Por último, el profesor realizó observaciones generales sobre el sistema y el prototipo, pidiendo además más entregas con más contenido al respecto.
\newline
\newline
Así, en ausencia del tercer sinodal (Macario Hernández), las preguntas cesaron, las observaciones se dieron y el trabajo terminal I se dio por concluido. Se acordó se trabajaría en las observaciones realizadas y se presentarían en TT2 junto al resto del sistema y el trabajo prometido en la sección de trabajo a futuro, mismo que es parte del trabajo realizado en el actual periodo de Trabajo Terminal 2.