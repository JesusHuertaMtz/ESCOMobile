%%%%%%%%%%%%%%%%%%%%%%%%%%%%%%%%%%%%%%%%%%%%%%%%%%%%%%%%%%%%%%%%%%%%%%%%%%%%%%%%%	
%%%%%%%%%%%%%%%%%%%%%%%%%%%%%%%%%%%%%%%%%%%%%%%%%%%%%%%%%%%%%%%%%%%%%%%%%%%%%%%%%

						%%		EM-PROFESOR	  %%

%%%%%%%%%%%%%%%%%%%%%%%%%%%%%%%%%%%%%%%%%%%%%%%%%%%%%%%%%%%%%%%%%%%%%%%%%%%%%%%%%	
%%%%%%%%%%%%%%%%%%%%%%%%%%%%%%%%%%%%%%%%%%%%%%%%%%%%%%%%%%%%%%%%%%%%%%%%%%%%%%%%%
%%%%%%%%%%%%%% MODIFICAR UI DEL PROFESOR. ESTO SOLO ES LA PLANTILLA
\pagebreak
\subsection{EM-Profesor-UI1 Consultar perfil Propio.}

\subsubsection{Objetivo}
	\noindent
	Consultar su información registrada dentro del sistema por medio de su perfil, esta información es su nombre, fotografía, academia, cubículo y próxima cita. También sirve de acceso para buscar a profesores de ESCOM, ver el mapa de la ESCOM o modificar la información de su cuenta. También sirve de acceso para consultar su horario y estadísticas de desempeño docente, o bien para consultar todas sus citas o editar su perfil.  

\subsubsection{Diseño}
	\noindent
	La pantalla se divide en tres secciones principales. La primera muestra información básica del profesor, como lo es su foto y su nombre. La segunda sección está dedicada a sus citas, permitiendo ver la próxima cita agendada que tiene y un botón para consultar todas las citas asociadas a su cuenta; fimalmente, en la tercera sección se muestran tres botones los cuales son \IUbutton{Horario}, \IUbutton{Estadística y comentarios} y \IUbutton{Editar Perfil}, que le permitirán acceder a diferetes acciones. 
 
\pagebreak
\IUfig[.5]{gui/Maquetas/Profesor/EM_Profesor_UI1_Consultar_Perfil_Propio.png}{EM-Profesor-UI1}{Consultar Perfil Propio}

\subsubsection{Salidas}
	Se muestra la siguiente información del profesor:
	\begin{itemize}
		\item Nombre.
		\item Fotografía.
		\item Academia a la que pertenece,
		\item Cúbiculo.
		\item Próxima cita.
	\end{itemize}

\subsubsection{Entradas}
	\noindent
	Ninguna

\subsubsection{Comandos}
	\begin{itemize}
		\item \IUbutton{Estadísticas y comentario}: Muestra pantalla con estadísticas del profesor.
		\item \IUbutton{Comentarios}: Muestra mapa con los comentarios del profesor.
		\item \IUbutton{Editar perfil}: Muestra la pantalla para editar perfil.
	\end{itemize}

\subsubsection{Mensajes}
	\noindent
	Ninguno.

%%%%%%%%%%%%%%%%%%%%%%%%%%%%%%%%%%%%%%%%%%%%%%%%%%%%%%%%%%%%%%%%%%%%%%%%%%%%%%%%%	
%%%%%%%%%%%%%%%%%%%%%%%%%%%%%%%%%%%%%%%%%%%%%%%%%%%%%%%%%%%%%%%%%%%%%%%%%%%%%%%%%

\subsection{EM-Profesor-UI1.1 Consultar Horario Propio.}

\subsubsection{Objetivo}
	\noindent
	El actor podrá consultar su horario de clases de ESCOM, ésto es: las materias que imparte a lo largo de la semana, así como la hora, grupo y salón en las que éstas se ubican.


\subsubsection{Diseño}
	\noindent
	La pantalla muestra primeramente la información básica del profesor, como su foto y su nombre; debajo se muestran en forma de tarjetas desplegables los días de la semana, de lunes a viernes, con la información de las materias impartidas por el profesor, como el nombre de la materia, grupo, salón y hora. Es importante mencionar que las tarjetas correspondientes a los días en donde los profesores no tienen clase se mostrarán vacías.

\pagebreak
\IUfig[.5]{gui/Maquetas/Profesor/EM_Profesor_UI1_1_Consultar_Horario_Propio.png}{EM-Profesor-UI1-1}{Consultar Horario Propio}

\subsubsection{Salidas}
	Se muestra la siguiente información del profesor:
	\begin{itemize}
		\item Nombre.
		\item Fotografía. 
	\end{itemize}
	E información de la materia:
	\begin{itemize}
		\item Nombre.
		\item Hora, grupo y salón en que se imparte. 
	\end{itemize}

\subsubsection{Entradas}
	\noindent
	Ninguna.

\subsubsection{Comandos}
\begin{itemize}
	\item Botón de regreso: Para regresar a la pantalla anterior.
\end{itemize}

\subsubsection{Mensajes}
	\noindent 
	\MSGref{MSG3}{Elementos No Disponibles}

%%%%%%%%%%%%%%%%%%%%%%%%%%%%%%%%%%%%%%%%%%%%%%%%%%%%%%%%%%%%%%%%%%%%%%%%

\subsection{EM-Profesor-UI1.2 Consultar estadísticas Asignadas.}

\subsubsection{Objetivo}
	\noindent
	Permitir consultar sus estadísticas en el sistema, las estadísticas que se muestran son: el promedio de la calificación del profesor, así como una gráfica que detalla el comportamiento del profesor con respecto a sus citas, es decir, del total de citas registradas para el profesor, cuántas de ellas han sido aceptadas, cuántas canceladas y cuántas se encuentran pendientes de ser aceptadas o rechazadas. 

\subsubsection{Diseño}
	\noindent
	La pantalla se divide en tres grandes partes. La primera de ellas muestra las opciones directas que se pueden realizar, éstas son: regresar a la pantalla anterior, y cambiar entre las consultas de estadísticas y comentarios. La segunda sección nos presenta la información básica del profesor, esto es, su nombre, fotografía (en el caso de haber) y su promedio de desempeño asignado (calificación) por los alumnos. Finalmente, en el tercer apartado, tenemos una gráfica que detalla el comportamiento del profesor ante sus citas, los apartados contenidos en ella son: citas aceptadas, citas canceladas y solicitudes pendientes. 

\pagebreak
\IUfig[.5]{gui/Maquetas/Profesor/EM_Profesor_UI1_2_Consultar_Estadisticas_Asignadas.png}{EM-Profesor-UI1-2}{Consultar estadísticas Asignadas}

\subsubsection{Salidas}
	Se muestra la siguiente información del profesor:
	\begin{itemize} 
		\item Nombre.
		\item Fotografía.
		\item Puntuación del profesor.
		\item Gráfica con las citas Aceptadas, Citas canceladas y solicitudes pendientes.
	\end{itemize}

\subsubsection{Entradas}
	\noindent
	Ninguna.

\subsubsection{Comandos}
\begin{itemize}
	\item Botón de regreso: Para regresar a la pantalla anterior.
	\item Botón \IUbutton{Comentarios}: para consultar los comentarios asignados por los alu
\end{itemize}

\subsubsection{Mensajes}
	\noindent
	No aplica.

%%%%%%%%%%%%%%%%%%%%%%%%%%%%%%%%%%%%%%%%%%%%%%%%%%%%%%%%%%%%%%%%%%%%%%%%%%%%%%%%

\subsection{EM-Profesor-UI1.2.1 Consultar comentarios Asignados}

\subsubsection{Objetivo}
	\noindent
	Consultar los diferentes comentarios que los alumnos le han compartido a algún profesor. Así que puede conocer las opiniones de sus alumnos, analizarlas y plantearse nuevas y mejores formas para compartir su conocimiento y experiencia en el salón de calses. Además, es gracias a los comentarios que la comunidad de ESCOM tiene mayor conocimiento de los maestros del plantel, de su forma de trabajo y diversas actitudes que tiene para con sus grupos y alumnos.

\subsubsection{Diseño}
	\noindent
	La pantalla muestra el nombre y fotografía del profesor,así como una lista con los comentarios que le han compartido los alumnos (incuyendo el nombre de los mismos).
	Además muestra la opción para acceder a la consulta de las estadísticas y el botón de regreso.

\pagebreak
\IUfig[.5]{gui/Maquetas/Profesor/EM_Profesor_UI1_2_1_Consultar_Comentarios_Asignados.png}{EM-Profesor-UI1-2-1}{Consultar comentarios asignados}

\subsubsection{Salidas}
	\begin{itemize}
		\item Nombre y fotografía del profesor.
		\item Lista con los comentarios asignados y nombre de los autores.
	\end{itemize}

\subsubsection{Entradas}
	\noindent
	Ninguna.

\subsubsection{Comandos}
	\begin{itemize}
		\item Opción de estadísticas: para consultar las estadísticas del profesor.
		\item Botón de regreso: Para regresar a la pantalla anterior.
	\end{itemize}

\subsubsection{Mensajes}
	\noindent
	\MSGref{MSG3}{Elementos No Disponibles}.

%%%%%%%%%%%%%%%%%%%%%%%%%%%%%%%%%%%%%%%%%%%%%%%%%%%

\subsection{EM-Profesor-UI1.3 Modificar Perfil del Profesor.}

\subsubsection{Objetivo}
	\noindent
	Permitir modificar la información del profesor registrada en el sistema. La información que se encuentra disponible para edición es el correo electrónico, la contraseña, su cubículo y la fotografía, siendo los dos primeros ingresados cuando se realizó el registro. La información antes referida puede ser modificada cuantas veces sea requerido por el actor dentro de la app.

\subsubsection{Diseño}
	\noindent
	La pantalla muestra el nombre y foto del profesor, además de los campos de información que puede cambiar o actualizar, éstos son:
	\begin{itemize} 
		\item Cubículo: para actualizar el cubículo.
		\item Correo: para actualizar el correo electrónico.
		\item Contraseña: para actualizar la contraseña.
		\item Repetir contraseña: escribe nuevamente la contraseña para asegurar que no haya errores con ella por parte del usuario.
	\end{itemize} 

\pagebreak
\IUfig[.5]{gui/Maquetas/Profesor/EM_Profesor_UI1_3_Modificar_Perfil_del_Profesor.png}{EM-Profesor-UI1-3}{Modificar Perfil del Profesor}

\subsubsection{Salidas}
	\noindent
	Se muestra la siguiente información del Profesor:
	\begin{itemize}
		\item Nombre.
		\item Fotografía (original).
		\item Correo electrónico (original).
		\item Contraseña (original).
		\item Cubículo (original).
	\end{itemize}

\subsubsection{Entradas}
	\noindent
	Se introduce la información que el profesor desea modificar, ésta puede ser:
	\begin{itemize}
		\item Correo electrónico (nuevo).
		\item Contraseña (nueva).
		\item Duplicado de contraseña (nueva). 
		\item Cubículo (nueva).
		\item Fotografía (nueva). 
	\end{itemize}

\subsubsection{Comandos}
	\begin{itemize}
		\item Botón \IUbutton{Aceptar}: Para aceptar la edición de los datos del profesor.
		\item Botón ''regresar'': pare regresar a la pantalla anterior.
	\end{itemize}

\subsubsection{Mensajes}
	\begin{Citemize}
		\item \MSGref{MSG1}{Operación Exitosa}.
	\end{Citemize}
%%%%%%%%%%%%%%%%%%%%%%%%%%%%%%%%%%%%%%%%

\subsection{EM-Profesor-UI2 Consultar perfil de otro profesor.}

\subsubsection{Objetivo}
	\noindent
	El actor podrá consultar la información de algún profesor de su interés por medio de su perfil, ésta información es su nombre, fotografía, academia, así también, sirve de acceso para consultar su horario y estadísticas de desempeño docente y ubicarlo en el mapa de ESCOM.


\subsubsection{Diseño}
	\noindent
	La pantalla muestra primeramente la información básica del profesor a consultar, como su foto, su nombre, academia, cubículo y calificación promedio otorgada por los alumnos que lo han calificado; se muestran también tres botones los cuales son \IUbutton{Horario}, \IUbutton{Estadísticas}, \IUbutton{Ubicar en el mapa},  y el botón de regreso.

\pagebreak
\IUfig[.5]{gui/Maquetas/Profesor/EM_Profesor_UI2_Consultar_Perfil_de_Otro_Profesor.png}{EM-Profesor-UI2}{Consultar Perfil de otro Profesor}

\subsubsection{Salidas}
	\noindent
	Se muestra la siguiente información del profesor:
	\begin{itemize}
		\item Nombre.
		\item Academia a la que pertenece.
		\item Cubículo.
		\item Calificación promedio.
		\item Fotografía. 
	\end{itemize}

\subsubsection{Entradas}
	\noindent
	Ninguna.

\subsubsection{Comandos}
\begin{itemize}
	\item \IUbutton{Horario}: Para visualizar el horario del profesor consultado.
	\item \IUbutton{Estadísticas}: Para conocer las estadísticas del profesor de interés. 
	\item \IUbutton{Ubicar en el mapa}: Para ubicar en el mapa de ESCOM el cubículo del profesor.
	\item \IUbutton{Regresar}: Para regresar a la pantalla anterior.
\end{itemize}

\subsubsection{Mensajes}
	\noindent
	Ninguno.

