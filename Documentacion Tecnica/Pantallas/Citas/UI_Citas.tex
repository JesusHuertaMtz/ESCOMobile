%%%%%%%%%%%%%%%%%%%%%%%%%%%%%%%%%%%%%%%%%%%%%%%%%%%%%%%%%%%%%%%%%%%%%%%%%%%%%%%%%	
%%%%%%%%%%%%%%%%%%%%%%%%%%%%%%%%%%%%%%%%%%%%%%%%%%%%%%%%%%%%%%%%%%%%%%%%%%%%%%%%%

						%%		EM-CITAS	  %%

%%%%%%%%%%%%%%%%%%%%%%%%%%%%%%%%%%%%%%%%%%%%%%%%%%%%%%%%%%%%%%%%%%%%%%%%%%%%%%%%%	
%%%%%%%%%%%%%%%%%%%%%%%%%%%%%%%%%%%%%%%%%%%%%%%%%%%%%%%%%%%%%%%%%%%%%%%%%%%%%%%%%
\pagebreak
\subsection{EM-Citas-UI1 Consultar citas Agendadas}

\subsubsection{Objetivo}
	\noindent
	Mostrar un listado de todas las citas agendadas ordenadas por fecha y hora, comenzando por la cita
	próxima a realizarse. De esta manera se podrá consultar el día, hora y asunto de la cita.

\subsubsection{Diseño}
	\noindent
	El listado de las citas agendadas muestra un breve resumen del asunto por el cual se agendó la cita.
	Al presionar sobre un elemento de la lista se desplegará información detallada de 
	la cita. Además, cada elemento de la lista contendrá un menú de opciones donde se podrán cancelar
	o eliminar las citas.

\IUfig[.5]{gui/Maquetas/Citas/EM_Citas_UI1_Consultar_Citas_Agendadas}{EM-Citas-UI1}{Consultar citas agendadas}

\subsubsection{Salidas}
	\begin{itemize}
		\item Fecha en la que se agendó la cita.
		\item Hora en la que comenzará la cita.
			\item Alumno que solicitó la cita.
		\item Asunto por el cual se agendó la cita.
	\end{itemize}

\subsubsection{Entradas}

\begin{itemize}
	\item Ninguna.
\end{itemize}

\subsubsection{Comandos}
\begin{itemize}
	\item \IUbutton{ Por Confirmar } Muestra la pantalla de citas por confirmar.
	\item \IUbutton{ Citas pasadas } Muestra la pantalla del historial de citas.
	\item \UCicono{plus} Muestra pantalla para solicitar una cita.
	\item \UCicono{vdots} Muestra menú para cancelar o eliminar la cita.
	\item \UCicono{arrowL} Regresa a la pantalla anterior.
\end{itemize}

\subsubsection{Mensajes}
\begin{Citemize}
	\item Ninguno.
\end{Citemize}

%%%%%%%%%%%%%%%%%%%%%%%%%%%%%%%%%%%%%%%%%%%%%%%%%%%%%%%%%%%%%%%%%%%%%%%%%%%%%%%%%
\pagebreak

\subsection{EM-Citas-UI1.2 Consultar citas por Confirmar}

\subsubsection{Objetivo}
	\noindent
	Mostrar un listado de todas las citas por confirmar ordenadas por fecha y hora, comenzando por la primera cita solicitada. De esta manera se podrá consultar el día, hora y asunto de la cita y aceptar o rechazar las citas .

\subsubsection{Diseño}
	\noindent
	El listado de las citas por confirmar muestra un breve resumen del asunto por el cual se agendó la cita.
	Al presionar sobre un elemento de la lista se desplegará información detallada de 
	la cita solicitada. Además, cada elemento de la lista contendrá un menú de opciones donde se podrá aceptar o cancelar las citas.

\IUfig[.5]{gui/Maquetas/Citas/EM_Citas_UI1_2_Consultar_Citas_por_Confirmar}{EM-Citas-UI1-2}{Consultar citas por Confirmar}

\subsubsection{Salidas}
	\begin{itemize}
		\item Fecha en la que se agendó la cita.
		\item Hora en la que comenzará la cita.
			\item Alumno que solicitó la cita.
		\item Asunto por el cual se agendó la cita.
	\end{itemize}

\subsubsection{Entradas}

\begin{itemize}
	\item Ninguna.
\end{itemize}

\subsubsection{Comandos}
\begin{itemize}
	\item \IUbutton{ Agendadas } Muestra la pantalla de citas agendadas.
	\item \IUbutton{ Citas pasadas } Muestra la pantalla del historial de citas.
	\item \UCicono{vdots} Muestra menú para cancelar o eliminar la cita.
	\item \UCicono{arrowL} Regresa a la pantalla anterior.
\end{itemize}

\subsubsection{Mensajes}
\begin{Citemize}
	\item Ninguno.
\end{Citemize}

%%%%%%%%%%%%%%%%%%%%%%%%%%%%%%%%%%%%%%%%%%%%%%%%%%%%%%%%%%%%%%%%%%%%%%%%%%%%%%%%%
\pagebreak
%%%%%%%%%%%%%%%%%%%%%%%%%%%%%%%%%%%%%%%%%%%%%%%%%%%%%%%%%%%%%%%%%%%%%%%%%%%%%%%%%


\subsection{EM-Citas-UI1.3 Consultar citas Pasadas}

\subsubsection{Objetivo}
	\noindent
	Mostrar un listado de todas las citas pasadas ordenadas por fecha y hora. De ésta manera se podrá consultar el día, hora y asunto de las citas pasadas y eliminarlas .

\subsubsection{Diseño}
	\noindent
	El listado de las citas pasadas muestra un breve resumen del asunto por el cual se agendó la cita.
	Al presionar sobre un elemento de la lista se desplegará información detallada de 
	la cita . Además, cada elemento de la lista contendrá un botón de eliminar el registro de esa cita.

\IUfig[.5]{gui/Maquetas/Citas/EM_Citas_UI1_3_Consultar_Citas_Pasadas}{EM-Citas-UI1-3}{Consultar citas pasadas}

\subsubsection{Salidas}
	\begin{itemize}
		\item Fecha en la que se agendó la cita.
		\item Hora en la que se realizó la cita.
		\item Alumno que solicitó la cita.
		\item Asunto por el cual se agendó la cita.
	\end{itemize}

\subsubsection{Entradas}

\begin{itemize}
	\item Ninguna.
\end{itemize}

\subsubsection{Comandos}
\begin{itemize}
	\item \IUbutton{ Agendadas } Muestra la pantalla de citas agendadas.
	\item \IUbutton{ Citas por Confirmar } Muestra la pantalla de las citas por confirmar.
	\item Borrar: Borra la cita.
	\item \UCicono{arrowL} Regresa a la pantalla anterior.
\end{itemize}

\subsubsection{Mensajes}
\begin{Citemize}
	\item \MSGref{MSG1}{Operación Exitosa}
\end{Citemize}

%%%%%%%%%%%%%%%%%%%%%%%%%%%%%%%%%%%%%%%%%%%%%%%%%%%%%%%%%%%%%%%%%%%%%%%%%%%%%%%%%

\pagebreak
\subsection{EM-Cita-UI2 Agendar una cita }

\subsubsection{Objetivo}
	\noindent
	Mostrar los campos: Profesor a agendar, fecha, hora, tipo de cita y
	motivo de la cita, para que el actor pueda generar una solicitud de cita con un profesor.

\subsubsection{Diseño}
	\noindent
	Se mostrarán los campos: nombre del profesor, fecha, hora, tipo de cita, salón, agregar recordatorio y
	motivo de la cita.
	 Al presionar sobre la etiqueta de fecha de la cita se mostrará un pop up con un
	calendario mostrando la fecha actual. Al presionar sobre la etiqueta de hora se mostrará un
	pop up que indique la hora actual y pueda seleccionar una hora diferente.
	Al presionar sobre la etiqueta tipo de cita se mostrará un pop up con una lista de opciones, de las
	cuales solo podrá seleccionar una. Al presionar sobre la etiqueta Agregar notificación se mostrará
	un pop up con opciones para programar el recordatorio.

\IUfig[.5]{gui/Maquetas/Citas/EM_Citas_UI2_Agendar_una_Cita}{EM-Cita-UI2}{Agendar una cita}

\subsubsection{Salidas}
	\begin{itemize}
		\item Caledario.
		\item Reloj.
		\item Pop up con opciones para agregar un recordatorio.
		\item Pop up con las opciones para seleccionar el tipo de cita.
		\item Pop up indicando que la operación se realizó con éxito.
	\end{itemize}

\subsubsection{Entradas}

\begin{itemize}
	
	\item Fecha en la que se propone se lleve a cabo la reunión.
	\item Propuesta de la hora de inicio de la cita.
	\item Motivo por el cual se realiza la cita.
\end{itemize}

\subsubsection{Comandos}
\begin{itemize}
	\item \IUbutton{Profesor a agendar}:Muestra la pantalla Seleccionar Profesor para Cita.
	\item \IUbutton{ \UCicono{reloj} Fecha de la cita }: Muestra el pop up para seleccionar la fecha.
	\item \IUbutton{ \UCicono{reloj} Hora de la cita }: Muestra el pop up para seleccionar la hora de 
	la cita.
	\item \IUbutton{ \UCicono{trian_invert} Tipo de cita }: Muestra un listado de opciones.
\end{itemize}

\subsubsection{Mensajes}
\begin{Citemize}
	\item \MSGref{MSG1}{Operación Exitosa}
	\item \MSGref{MSG12}{Traslape de hora de cita con hora de clase}
\end{Citemize}

%%%%%%%%%%%%%%%%%%%%%%%%%%%%%%%%%%%%%%%%%%%%%%%%%%%%%%%%%%%%%%%%%%%%%%%%%%%%%%%%%
%%%%%%%%%%%%%%%%%%%%%%%%%%%%%%%%%%%%%%%%%%%%%%%%%%%%%%%%%%%%%%%%%%%%%%%%%%%%%%%%%
\pagebreak

\subsection{EM-Cita-UI2.1 Seleccionar Profesor para Cita }

\subsubsection{Objetivo}
	\noindent
	Mostrar una lista de profesores para que el alumno seleccione o busque el profesor al cual decida solicitar una cita.

\subsubsection{Diseño}
	\noindent
	La pantalla se muestra en dos secciones. La primera sección muestra una barra de búsqueda para profesores de ESCOM y el botón para regresar. La segunda sección muestra una lista con los nombres de los profesores de la ESCOM registrados en el sistema, o bien, los coincidentes con la búsqueda ordenados alfabéticamente. 

\IUfig[.5]{gui/Maquetas/Citas/EM_Citas_UI2_1_Seleccionar_Profesor_para_Cita}{EM-Cita-UI2-1}{Seleccionar Profesor}

\subsubsection{Salidas}
	\begin{itemize}
		\item Lista que muestra alfabéticamente los nombres de todos los profesores registrados en ESCOM, o bien, los coincidentes con la búsqueda. 
	\end{itemize}

\subsubsection{Entradas}

\begin{itemize}
		\item Nombre del profesor que se requiere solicitar cita.  
	\end{itemize}

\subsubsection{Comandos}
\begin{itemize}
		\item Botón de regreso: Para regresar a la pantalla anterior. 
	\end{itemize}

\subsubsection{Mensajes}
Ninguno.

%%%%%%%%%%%%%%%%%%%%%%%%%%%%%%%%%%%%%%%%%%%%%%%%%%%%%%%%%%%%%%%%%%%%%%%%%%%%%%%%%