%%%%%%%%%%%%%%%%%%%%%%%%%%%%%%%%%%%%%%%%%%%%%%%%%%%%%%%%%%%%%%%%%%%%%%%%%%%%%%%%%	
%%%%%%%%%%%%%%%%%%%%%%%%%%%%%%%%%%%%%%%%%%%%%%%%%%%%%%%%%%%%%%%%%%%%%%%%%%%%%%%%%

						%%		EM-CITAS	  %%

%%%%%%%%%%%%%%%%%%%%%%%%%%%%%%%%%%%%%%%%%%%%%%%%%%%%%%%%%%%%%%%%%%%%%%%%%%%%%%%%%	
%%%%%%%%%%%%%%%%%%%%%%%%%%%%%%%%%%%%%%%%%%%%%%%%%%%%%%%%%%%%%%%%%%%%%%%%%%%%%%%%%
\pagebreak
\subsection{EM-Citas-UI1 Consultar citas Agendadas}

\subsubsection{Objetivo}
	\noindent
	Consultar de manera ordenada todas las citas en estado de agendadas que un alumno o un profesor tiene asociadas a su cuenta, esto es, los próximas citas ya aceptadas para realizarse.

\subsubsection{Diseño}
	\noindent
	La pantalla muestra dos principales secciones. La primera sección, de encabezado, nos permite navegar entre los diferentes tipos de citas (según el estado de las mismas) y las consultas de las las referidas, siendo éstas ''agendadas'', ''por confirmar'', ''pasadas'' y ''canceladas'', asimismo, nos presenta un botón para regresar a la pantalla anterior.
	\newline
	La segunda sección enlista ordenadas y agrupadas por día las citas que el profesor o el alumno tienen programadas, detallando el tipo de cita, nombre del alumno quien la solicitó y un breve motivo de la misma. 
	\newline
	Finalmente, se nos permite también solicitar agregar una nueva cita (función solo disponible para los alumnos), por medio de botón \IUbutton{ + }.

\pagebreak
\IUfig[.5]{gui/Maquetas/Citas/EM_Citas_UI1_Consultar_Citas_Agendadas}{EM-Citas-UI1}{Consultar citas agendadas}

\subsubsection{Salidas}
	\noindent
	Se muestra la siguiente información acerca de las citas agendadas:
		\begin{itemize}
			\item Fecha y hora en las cuales la cita está programada
			\item Tipo de cita (Asesoría, revisión de proyecto, revisión de TT, entrega de tarea, entrega de proyecto, revisión de protocolo, otro).
			\item Nombre del alumno quien solicitó la cita.
			\item Descripción (motivo) por la cual el alumno solicita la cita.
		\end{itemize}

\subsubsection{Entradas}
	\noindent
	Ninguna.

\subsubsection{Comandos}
	\begin{itemize}
		\item Botón \IUbutton{ POR CONFIRMAR }: Muestra la pantalla con las citas por confirmar.
		\item Botón \IUbutton{ PASADAS }: Muestra la pantalla con todas las citas pasadas ya atendidas.
		\item Botón \IUbutton{ CANCELADAS }: Muestra la pantalla con las citas que el alumno o el profesor cancelaron antes de realizarse.
		\item Botón \IUbutton{ + }: Muestra pantalla para solicitar agendar una nueva cita.
		\item Icono \UCicono{tache}: Para cancelar una cita agendada.
		\item Botón \UCicono{arrowL}: Regresa a la pantalla anterior.
	\end{itemize}

\subsubsection{Mensajes}
	\begin{Citemize}
		\item \MSGref{MSG3}{Elementos No Disponibles}
	\end{Citemize}

%%%%%%%%%%%%%%%%%%%%%%%%%%%%%%%%%%%%%%%%%%%%%%%%%%%%%%%%%%%%%%%%%%%%%%%%%%%%%%%%%
\pagebreak

\subsection{EM-Citas-UI1.2 Consultar citas por Confirmar.}

\subsubsection{Objetivo}
	\noindent
	Mostrar para consulta, de manera ordenada, todas las citas en estado de ''por confirmar'' que tiene asociadas a su cuenta, esto es, las solicitudes de citas que ha realizado y que aún no han sido aceptadas o rechazadas. 

\subsubsection{Diseño}
	\noindent
	La pantalla muestra dos principales secciones. La primera sección, de encabezado, nos permite navegar entre los diferentes tipos de citas (según el estado de las mismas) y las consultas de las las referidas, siendo éstas ''agendadas'', ''por confirmar'', ''pasadas'' y ''canceladas'', asimismo, nos presenta un botón para regresar a la pantalla anterior.
	\newline
	La segunda sección enlista ordenadas y agrupadas por día las solicitudes de citas que el profesor o el alumno tienen asociadas, detallando el tipo de cita, nombre del alumno quien la solicitó y un breve motivo de la misma. Mostrando además por solicitud de cita la opción de cancelar (rechazar) o bien, aceptar la solicitud de cita, estando disponible la última opción solo para los profesores. 

\pagebreak
\IUfig[.5]{gui/Maquetas/Citas/EM_Citas_UI1_2_Consultar_Citas_por_Confirmar}{EM-Citas-UI1-2}{Consultar citas por Confirmar}

\subsubsection{Salidas}
	\noindent
	Se muestra la siguiente información acerca de las solicitudes de citas:
		\begin{itemize}
			\item Fecha y hora en las cuales la cita está se desea realizar.
			\item Tipo de cita (Asesoría, revisión de proyecto, revisión de TT, entrega de tarea, entrega de proyecto, revisión de protocolo, otro).
			\item Nombre del alumno quien solicitó la cita.
			\item Descripción (motivo) por la cual el alumno solicita la cita.
		\end{itemize}

\subsubsection{Entradas}
	\noindent
	Ninguna.

\subsubsection{Comandos}
	\begin{itemize}
		\item Botón \IUbutton{ AGENDADAS }: Muestra la pantalla con las cita agendadas.
		\item Botón \IUbutton{ PASADAS }: Muestra la pantalla con todas las citas pasadas ya atendidas.
		\item Botón \IUbutton{ CANCELADAS }: Muestra la pantalla con las citas que el alumno o el profesor cancelaron antes de realizarse.
		\item Icono \UCicono{tache}: Para cancelar una solicitud de cita.
		\item Icono \UCicono{paloma}: Para aceptar una solicitud una cita agendada.
		\item Botón \UCicono{arrowL}: Regresa a la pantalla anterior.
	\end{itemize}

\subsubsection{Mensajes}
	\begin{Citemize}
		\item \MSGref{MSG3}{Elementos No Disponibles}
	\end{Citemize}

%%%%%%%%%%%%%%%%%%%%%%%%%%%%%%%%%%%%%%%%%%%%%%%%%%%%%%%%%%%%%%%%%%%%%%%%%%%%%%%%%
\pagebreak

\subsection{EM-Citas-UI1.3 Consultar citas Pasadas.}

\subsubsection{Objetivo}
	\noindent
	Mostrar un listado ordenado con todas las citas en estado de ''pasada'' que tiene asociadas a su cuenta, esto es, las citas aceptadas que ya se realizaron, pues fue gracias a esas citas de que los profesores puedieron brindar su apoyo los estudiantes de ESCOM para mejorar su aprendizaje.

\subsubsection{Diseño}
	\noindent
	La pantalla muestra dos principales secciones. La primera sección, de encabezado, nos permite navegar entre los diferentes tipos de citas (según el estado de las mismas) y las consultas de las las referidas, siendo éstas ''agendadas'', ''por confirmar'', ''pasadas'' y ''canceladas'', asimismo, nos presenta un botón para regresar a la pantalla anterior.
	\newline
	La segunda sección enlista ordenadas y agrupadas por día las citas ''pasadas'' que el profesor o el alumno tuvieron, detallando el tipo de cita, nombre del alumno quien la solicitó y un breve motivo de la mismas. Además, se tiene por cita el icono \UCicono{eliminar}, con el cual se podrán eliminar del histórico las citas pasadas que así el actor requiera.

\pagebreak
\IUfig[.5]{gui/Maquetas/Citas/EM_Citas_UI1_3_Consultar_Citas_Pasadas}{EM-Citas-UI1-3}{Consultar citas pasadas}

\subsubsection{Salidas}
	\noindent
	Se muestra la siguiente información acerca de las citas pasadas:
		\begin{itemize}
			\item Fecha y hora en las cuales la cita transcurrió.
			\item Tipo de cita (Asesoría, revisión de proyecto, revisión de TT, entrega de tarea, entrega de proyecto, revisión de protocolo, otro).
			\item Nombre del alumno quien solicitó la cita.
			\item Descripción (motivo) por la cual el alumno solicitó la cita.
		\end{itemize}

\subsubsection{Entradas}
	\noindent
	Ninguna.

\subsubsection{Comandos}
	\begin{itemize}
		\item Botón \IUbutton{ AGENDADAS }: Muestra la pantalla con todas las citas agendadas próximas a realizarse.
		\item Botón \IUbutton{ POR CONFIRMAR }: Muestra la pantalla con las citas por confirmar.
		\item Botón \IUbutton{ CANCELADAS }: Muestra la pantalla con las citas que el alumno o el profesor cancelaron antes de realizarse.
		\item Icono \UCicono{eliminar}: Para eliminar del histórico una cita pasada.
		\item Botón \UCicono{arrowL}: Regresa a la pantalla anterior.
	\end{itemize}

\subsubsection{Mensajes}
	\begin{Citemize}
		\item \MSGref{MSG3}{Elementos No Disponibles}
	\end{Citemize}


%%%%%%%%%%%%%%%%%%%%%%%%%%%%%%%%%%%%%%%%%%%%%%%%%%%%%%%%%%%%%%%%%%%%%%%%%%%%%%%%%
\pagebreak

\subsection{EM-Citas-UI1.4 Consultar citas Canceladas.}

\subsubsection{Objetivo}
	\noindent
	Mostrar de manera ordenada todas las citas en estado de ''cancelada'' que tiene asociadas a su cuenta, esto es, las citas o solicitudes de citas que no pudieron realizarse, pues fueron canceladas.

\subsubsection{Diseño}
	\noindent
	La pantalla muestra dos principales secciones. La primera sección, de encabezado, nos permite navegar entre los diferentes tipos de citas (según el estado de las mismas) y las consultas de las las referidas, siendo éstas ''agendadas'', ''por confirmar'', ''pasadas'' y ''canceladas'', asimismo, nos presenta un botón para regresar a la pantalla anterior.
	\newline
	La segunda sección enlista ordenadas y agrupadas por día las citas ''canceladas'' en las que el alumno o el profesor pudieron participar, pero que no llegaron a hacerse realidad, detallando el tipo de cita, nombre del alumno quien la solicitó y un breve motivo de la mismas. Además, se tiene por cita el icono \UCicono{eliminar}, con el cual se podrán eliminar del histórico las citas canceladas que así el actor requiera.

\pagebreak
\IUfig[.5]{gui/Maquetas/Citas/EM_Citas_UI1_4_Consultar_Citas_Canceladas}{EM-Citas-UI1-4}{Consultar citas canceladas}

\subsubsection{Salidas}
	\noindent
	Se muestra la siguiente información acerca de las citas canceladas:
		\begin{itemize}
			\item Fecha y hora en las cuales la cita se llevaría a cabo.
			\item Tipo de cita (Asesoría, revisión de proyecto, revisión de TT, entrega de tarea, entrega de proyecto, revisión de protocolo, otro).
			\item Nombre del alumno quien solicitó la cita.
			\item Descripción (motivo) por la cual el alumno solicitó la cita.
		\end{itemize}

\subsubsection{Entradas}
	\noindent
	Ninguna.

\subsubsection{Comandos}
	\begin{itemize}
		\item Botón \IUbutton{ AGENDADAS }: Muestra la pantalla con todas las citas agendadas próximas a realizarse.
		\item Botón \IUbutton{ POR CONFIRMAR }: Muestra la pantalla con las citas por confirmar.
		\item Botón \IUbutton{ PASADAS }: Muestra la pantalla con las citas que ya transcurrieron.
		\item Icono \UCicono{eliminar}: Para eliminar del histórico una cita cancelada.
		\item Botón \UCicono{arrowL}: Regresa a la pantalla anterior.
	\end{itemize}

\subsubsection{Mensajes}
	\begin{Citemize}
		\item \MSGref{MSG3}{Elementos No Disponibles}
	\end{Citemize}

%%%%%%%%%%%%%%%%%%%%%%%%%%%%%%%%%%%%%%%%%%%%%%%%%%%%%%%%%%%%%%%%%%%%%%%%%%%%%%%%%

\pagebreak
\subsection{EM-Cita-UI2 Agendar una cita }

\subsubsection{Objetivo}
	\noindent
	Permitir al actir solicitar agendar una nueva cita con un profesor de ESCOM, pues requiere del apoyo de éste para solventar alguna situación académica surgida durante el curso, como lo puede ser dudas de algún tema, asesorías extra clase, tutorías escolares, presentaciones de proyectos, revisiones de los mismos, entregas de TT, entre otros.

\subsubsection{Diseño}
	\noindent
	La pantalla muestra un formulario con los campos necesarios para solicitar agendar una cita con algún profesor de ESCOM. En cada uno de los campos se despliega una opción para introducir adecuadamente la información. Los campos requeridos y la forma en introducir correctamente la información se muestra a continuación: 
	\begin{itemize}
		\item Campo ''Profesor a agendar'': Al seleccionar esta opción se redirige a una pantalla que contiene la lista de los profesores disponibles para agendar citas.
		\item Campo ''Fecha de la cita'': Al presionar sobre éste, se muestra un calendario para seleccionar el día en que se desea realizar la cita.
		\item Campo ''Hora de la cita'': Muestra un reloj en donde se puede elegir el horario preferido para la cita.	
		\item Campo ''Tipo de cita'': Despliega una lista con las diferentes opciones disponibles (Asesoría, revisión de proyecto, revisión de TT, entrega de tarea, entrega de proyecto, revisión de protocolo, otro) para elegir solamente una. 
		\item Campo ''Motivo de la cita'': Permite exponer los motivos y/o razones por las cuales se desea llevar a cabo la cita.
	\end{itemize}
	Se muestran además una leyenda con instrucciones rápidas para agendar la cita, así como los botones \IUbutton{Aceptar}, \UCicono{arrowL}, y el icono \UCicono{pregunta}.

\pagebreak
\IUfig[.8]{gui/Maquetas/Citas/EM_Citas_UI2_Agendar_una_Cita}{EM-Cita-UI2}{Agendar una cita}

\subsubsection{Salidas}
	\noindent
	Ninguna.

\subsubsection{Entradas}
	Se requiere la siguiente información acerca de la cita a agendar:
	\begin{itemize}
		\item Nombre del profesor con el cual se desea realizar la cita (seleccionado de una lista).
		\item Fecha y hora propuestas de realización.
		\item Tipo de cita (Asesoría, revisión de proyecto, revisión de TT, entrega de tarea, entrega de proyecto, revisión de protocolo, otro).
		\item Descripción (motivo) por la cual el alumno solicita la cita.
	\end{itemize}

\subsubsection{Comandos}
	\begin{itemize}
		\item Campo ''Profesor a agendar'': Al seleccionar esta opción se redirige a una pantalla que contiene la lista de los profesores disponibles para agendar citas.
		\item Campo ''Fecha de la cita'': Al presionar sobre éste, se muestra un calendario para seleccionar el día en que se desea realizar la cita.
		\item Campo ''Hora de la cita'': Muestra un reloj en donde se puede elegir el horario preferido para la cita.	
		\item Campo ''Tipo de cita'': Despliega una lista con las diferentes opciones disponibles (Asesoría, revisión de proyecto, revisión de TT, entrega de tarea, entrega de proyecto, revisión de protocolo, otro) para elegir solamente una. 
		\item Campo ''Motivo de la cita'': Permite exponer los motivos y/o razones por las cuales se desea llevar a cabo la cita.	
		\item Botón \IUbutton{Aceptar}: Para solicitar agendar la cita.
		\item Icono \UCicono{pregunta}: Para consultar un apartado de ayuda sobre cómo agendar una cita.
		\item Botón \UCicono{arrowL}: Regresa a la pantalla anterior.
	\end{itemize}

\subsubsection{Mensajes}
	\begin{itemize}
		\item \MSGref{MSG1}{Operación Exitosa}.
		\item \MSGref{MSG2}{Operación Fallida}.
		\item \MSGref{MSG5}{Falta dato obligatorio}.
		\item \MSGref{MSG26}{Ayuda sobre agendar citas}.
		\item \MSGref{MSG27}{Fecha u hora de cita no disponibles}.
		\item \MSGref{MSG28}{Tamaño de mensaje superado}
	\end{itemize}
