%%%%%%%%%%%%%%%%%%%%%%%%%%%%%%%%%%%%%%%%%%%%%%%%%%%%%%%%%%%%%%%%%%%%%%%%%%%%%%%%%	
%%%%%%%%%%%%%%%%%%%%%%%%%%%%%%%%%%%%%%%%%%%%%%%%%%%%%%%%%%%%%%%%%%%%%%%%%%%%%%%%%

						%%		EM-WebBolsa	  %%

%%%%%%%%%%%%%%%%%%%%%%%%%%%%%%%%%%%%%%%%%%%%%%%%%%%%%%%%%%%%%%%%%%%%%%%%%%%%%%%%%	
%%%%%%%%%%%%%%%%%%%%%%%%%%%%%%%%%%%%%%%%%%%%%%%%%%%%%%%%%%%%%%%%%%%%%%%%%%%%%%%%%
%%%%%%%%%%%%%% MODIFICAR UI DEL PROFESOR. ESTO SOLO ES LA PLANTILLA


\subsection{EM-WebBolsa-UI Login de ESCOMobile Bolsa.}

\subsubsection{Objetivo}
	\noindent
	Proporcionar al actor acceso a la página web ESCOMobile, para realizar diferentes acciones dentro del sistema gracias a su cuenta de Facebook.

\subsubsection{Diseño}
	\noindent
	La pantalla muestra un mensaje de bienvenida al sistema, informando al actor que debe acceder al mismo por medio de su cuenta de Facebook a través del botón \IUbutton{inicia sesión con Facebook}.

\pagebreak	
\IUfig[.5]{gui/Maquetas/BolsaWeb/BolsaWeb_Login.png}{EM-BolsaWeb-Login}{ESCOMobile Bolsa.}

\subsubsection{Salidas}
	\noindent
	Ninguna.

\subsubsection{Entradas}
	\noindent
	Cuenta de Facebook (inicio de sesión a través de Facebook).

\subsubsection{Comandos}
	\begin{itemize}
		\item Botón \IUbutton{Inicia sesión con Facebook}: Para ingresar al sistema con una cuenta de Facebook.
	\end{itemize}

%%%%%%%%%%%%%%%%%%%%%%%%%%%%%%%%%%%%%%%%%%%%%%%%%%%%%%%%%%%%%%%%%%%%%%%%%%%%%%%%%	

\pagebreak

\subsection{EM-BolsaWeb-UI1 Gestionar Ofertas de Trabajo.}

\subsubsection{Objetivo}
	\noindent
	Proporcionar al actor acceso a la gestión y consulta de las ofertas de trabajo brindadas por las empresas registradas en la bolsa de trabajo de ESCOMobile, así como la información que a éstas se asocian para consulta a los alumnos de ESCOM e interesados en aplicar a alguno de los puestos que se ofertan. 

\subsubsection{Diseño}
	\noindent
	La pantalla muestra la información de las ofertas de trabajo organizada en diferentes secciones. En la primera sección, del lado izquierdo, se encuentra la opción para cambiar entre la gestión de empresas y la gestión de ofertas de trabajo, por medio de los botones \IUbutton{Ofertas laborales} y \IUbutton{Empresas}. La segunda sección, central superior, muestra el logo de ESCOMobile Bolsa,  la foto y nombre del actor propietario de la cuenta de Faecbook, el número de ofertas de trabajo publicadas en el sistema y una gráfica de con la información de las ofertas de trabajo registradas a lo largo del semestre escolar. Finalmente, la sección central (del cuerpo), muestran dos botones más que nos dan acceso a las ofertas publicadas y sin publicadas, así como el botón \IUbutton{Añadir oferta laboral} para agregar una nueva oferta y una tabla con la información de las empresas consultadas (según sea el caso). La información mostrada es:
	\begin{itemize}
		\item Número de oferta de trabajo.
		\item Nombre de la empresa.
		\item Horario.
		\item Vacante.
		\item Sueldo.
		\item Contacto.
		\item Fecha Publicación.
	\end{itemize}	
	Mostrando además, en la tabla, diferentes acciones (editar y eliminar) para cada una de las ofertas registradas.

\pagebreak
\IUfig[.8]{gui/Maquetas/BolsaWeb/EM_BolsaWeb_IU1_Gestionar_Ofertas_de_Trabajo.png}{EM-BolsaWeb-UI1}{Gestionar Ofertas de Trabajo}

\subsubsection{Salidas}
\begin{itemize}
	\item Número de ofertas de trabajo publicadas.
	\item Gráfica de línea mostrando el número de ofertas de los últimos 6 meses.
	\item Tabla que muestra la siguiente información de las ofertas de trabajo:
	\begin{itemize}
 		\item Número de oferta.
		\item Nombre de la empresa.
		\item Horario.
		\item Vacante.
		\item Sueldo.
		\item Contacto.
		\item Fecha Publicación.
	\end{itemize}	
\end{itemize}

\subsubsection{Entradas}
	\noindent
	Ninguna.

\subsubsection{Comandos}
	\begin{itemize}
		\item Botón \IUbutton{Empresas}: Para gestionar las empresas registradas dentro del sistema.
		\item Botón \IUbutton{Ofertas sin publicar}: Para consultar las ofertas de trabajo que no han sido publicadas.
		\item Botón \IUbutton{Ofertas piblicadas}: Para consultar las ofertas de trabajo publicadas.
		\item Botón \IUbutton{Añadir oferta}: Para registrar una nueva oferta de trabajo en el sistema.  
		\item Icono \UCicono{lapiz.png}: Para editar alguna de las empresas registradas.
		\item Icono \UCicono{eliminar.png}: Para eliminar alguna de las empresas registradas.
		\item Botón \IUbutton{Publicar boletín}: Para publicar el boletín de las ofertas de trabajo en la cuenta de facebook asociada.
	\end{itemize}
%%%%%%%%%%%%%%%%%%%%%%%%%%%%%%%%%%%%%%%%%%%%%%%%%%%%%%%%%%%%%%%%%%%%%%%%%%%%%%%%%	

	
\subsection{EM-WebBolsa-UI1.1 Registar nueva oferta}

\subsubsection{Objetivo}
	\noindent
	El actor podrá registrar una Oferta de trabajo llenando un formulario.

\subsubsection{Diseño}
	Mostrará un formulario el cual contiene los campos necesarios para registrar una nueva oferta de trabajo. 
	Los campos requeridos sobre la oferta de trabajo son los siguientes: 
	\begin{itemize}
		\item Nombre de la Empresa.
		\item Tipo Horario.
		\item Horario.
		\item Nombre del Puesto.
		\item Número de vacantes.
		\item Requisitos.
		\item Idiomas.
		\item Perfil.
		\item Sueldo.
		\item Prestaciones. 
	\end{itemize}

\IUfig[.5]{gui/Maquetas/BolsaWeb/EM_BolsaWeb_UI1_1_Registrar_Nueva_Oferta.png}{EM-WebBolsa-IU1.1}{Registrar nueva oferta}

\subsubsection{Salidas}
	Ninguna.

\subsubsection{Entradas}
	\begin{itemize}
		\item Nombre de la Empresa.
		\item Tipo Horario.
		\item Horario.
		\item Nombre del Puesto.
		\item Número de vacantes.
		\item Requisitos.
		\item Idiomas.
		\item Perfil.
		\item Sueldo.
		\item Prestaciones. 
	\end{itemize}

\subsubsection{Comandos}
\begin{itemize}
	\item \IUbutton{Registrar Oferta}: Completa el registro.  
	\item \IUbutton{Cancelar}: Cancela el registro.  
\end{itemize}



%%%%%%%%%%%	%%%%%%%%%%%	%%%%%%%%%%%	%%%%%%%%%%%	%%%%%%%%%%%	%%%%%%%%%%%	
\subsection{EM-WebBolsa-U11.2 Modificar oferta}

\subsubsection{Objetivo}
	\noindent
	El actor podrá modificar los datos introducidos en el formulario de Añadir oferta, solo se podrán modificar las ofertas aún no publicadas. 

\subsubsection{Diseño}
	\noindent
	Mostrará un formulario el cual contiene los campos necesarios para modificar una oferta de trabajo. 
	Los campos mostrados con los datos sobre la oferta de trabajo son los siguientes: 
	\begin{itemize}
		\item Nombre de la Empresa.
		\item Tipo Horario.
		\item Horario.
		\item Nombre del Puesto.
		\item Número de vacantes.
		\item Requisitos.
		\item Idiomas.
		\item Perfil.
		\item Sueldo.
		\item Prestaciones. 
	\end{itemize}

\IUfig[.5]{gui/Maquetas/BolsaWeb/EM_BolsaWeb_UI1_2_Editar_Oferta_de_Trabajo.png}{EM-WebBolsa-UI1.2}{Modificar oferta}

\subsubsection{Salidas}
	\noindent
	Ninguna.
	
\subsubsection{Entradas}
	\begin{itemize}
		\item Nombre de la Empresa.
		\item Tipo Horario.
		\item Horario.
		\item Nombre del Puesto.
		\item Número de vacantes.
		\item Requisitos.
		\item Idiomas.
		\item Perfil.
		\item Sueldo.
		\item Prestaciones. 
	\end{itemize}

\subsubsection{Comandos}
\begin{itemize}
	\item \IUbutton{Editar Oferta}: Se modifica la oferta de trabajo.  
	\item \IUbutton{Cancelar}: Cancelar.  
\end{itemize}
%%%%%%%%%%%%%%%%%%%%%%%%%%%%%%%%%%%%%%%%%%%%%%%%%%%%%%%%%%%%%%%%%%%%%%
%%%%%%%%%%%%%%%%%%%%%%%%%%%%%%%%%%%%%%%%%%%%%%%%%%%%%%%%%%%%%%%%%%%%%%%%%%%%%%%%%	

\subsection{EM-BolsaWeb-UI3 Publicación de Boletín de Ofertas de Trabajo}

\subsubsection{Objetivo}
	\noindent
	Proporcionar al actor una imagen con las ofertas de trabajo que desea publicar.
\subsubsection{Diseño}
	\noindent
	La pantalla muestra una vista previa de la imagen de las ofertas de trabajo que serán publicadas en la página de Facebook.

	Mostrando además los botones \IUbutton{Aceptar} y \IUbutton{Cancelar}.

\IUfig[.5]{gui/Maquetas/BolsaWeb/EM_BolsaWeb_UI3_Publicar_Boletin_Ofertas_de_Trabajo.png}{EM-BolsaWeb-UI3}{Publicar Boletín}


\subsubsection{Salidas}
	\begin{itemize}
	\item Publicación de imagen de ofertas de trabajo.
	\item Actualización de tablas ofertas sin publicar y ofertas publicadas.
	\end{itemize}
\subsubsection{Entradas}
	\begin{itemize}
	\item Ofertas de trabajo.
	\end{itemize}
\subsubsection{Comandos}
 	\begin{itemize}
		\item Botón \IUbutton{Aceptar}: Para subir la imagen con las ofertas de trabajo.  
		\item Botón \IUbutton{Cancelar}: Para cancelar publicación de Ofertas.

	\end{itemize}









%%%%%%%%%%%%%%%%%%%%%%%%%%%%%%%%%%%%%%%%%%%%%%%%%%%%%%%%%%%%%%%%%%%%%%%%%%%%%%%%%
					
					% 			PARA EMPRESAS 			%

%%%%%%%%%%%%%%%%%%%%%%%%%%%%%%%%%%%%%%%%%%%%%%%%%%%%%%%%%%%%%%%%%%%%%%%%%%%%%%%%%


%%%%%%%%%%%%%%%%%%%%%%%%%%%%%%%%%%%%%%%%%%%%%%%%%%%%%%%%%%%%%%%%%%%%%%%%%%%%%%%%%	
\subsection{EM-BolsaWeb-UI4 Gestionar Empresas}

\subsubsection{Objetivo}
	\noindent
	Proporcionar al actor acceso a diferentes acciones sobre las empresas de la ESCOMobile Bolsa, como lo son, registro, edición, consulta y eliminación de empresas en el sistema. 

\subsubsection{Diseño}
	\noindent
	La pantalla se muestra en diferentes secciones. En la parte izquierda las opciones para cambiar entre las gestiones de empresas y ofertas de trabajo, en la parte superior el logo de ''ESCOMobile bolsa'', finalmente en el cuerpo de la pantalla se muestran el número de empresas registradas, un botón para añadir ofertas y una tabla que contiene todas las empresas registradas en el sistema y los contactos a éstas asociados, desplegando la siguiente información:
	\begin{itemize}
		\item Nombre de la Empresa.
		\item Giro.
		\item Contacto.
		\item RFC.
		\item Número de ofertas publicadas.
		\item Número de ofertas visualizadas. 
	\end{itemize}
	Mostrando además en la tabla diferentes acciones (editar y eliminar) para cada una de las empresas registradas.

\IUfig[.5]{gui/Maquetas/BolsaWeb/EM_BolsaWeb_IU4_Gestionar_Empresas.png}{EM-BolsaWeb-UI4}{Gestionar Empresas}

\subsubsection{Salidas}
	\begin{itemize}
		\item Número de empresas registradas.
		\item Tabla que muestra la siguiente información de las empresas registradas
		\begin{itemize}
			\item Nombre de la Empresa.
			\item Giro.
			\item Contacto.
			\item RFC.
			\item Número de ofertas publicadas.
			\item Número de ofertas visualizadas. 
		\end{itemize}
	\end{itemize}

\subsubsection{Entradas}
	\noindent
	Ninguna.

\subsubsection{Comandos}
 	\begin{itemize}
		\item Botón \IUbutton{Añadir empresa}: Para registrar una nueva empresa en el sistema.  
		\item Ícono \UCicono{lapiz.png}: Para editar alguna de las empresas registradas.
		\item Ícono \UCicono{eliminar.png}: Para eliminar alguna de las empresas registradas.
	\end{itemize}




%%%%%%%%%%%%%%%%%%%%%%%%%%%%%%%%%%%%%%%%%%%%%%%%%%%%%%%%%%%%%%%%%%%%%%%%%%%%%%%%%	
\subsection{EM-BolsaWeb-UI4.1 Añadir nueva Empresa}

\subsubsection{Objetivo}
	\noindent
	Proporcionar al actor un mecanismo para registrar una nueva empresa que proponga ofertas de trabajo.
\subsubsection{Diseño}
	\noindent
	La pantalla muestra un formulario para registrar una nueva empresa en el sistema ESCOMobile Bolsa, dicho formulario contiene los diferentes campos requeridos para realizar el registro de la empresa, estando organizado en dos grandes secciones, la primera para introducir los datos de la empresa propiamente dicha y la segunda para el contacto a quien los interesados podrán dirigirse una vez registrada la empresa y publicadas las ofertas de trabajo, mostrando además en las últimas, los campos que son obligatorios para completar el regitro. Así, se tienen los siguientes campos para el registro de la empresa:
	\begin{itemize}
		\item Información de la empresa:
		\begin{itemize}
			\item Nombre.
			\item RFC.
			\item Giro.
			\item Logo.
		\end{itemize}
		\item Contacto de la empresa:
		\begin{itemize}
			\item Nombre del encargado de recursos humanos.
			\item Tipo de medio de contacto. 
			\item Medio de contacto. 
		\end{itemize}
	\end{itemize}
	Mostrando además los botones \IUbutton{Agregar logo}, \IUbutton{+}, \IUbutton{Registrar Empresa} y \IUbutton{Cancelar}.

\IUfig[.5]{gui/Maquetas/BolsaWeb/EM_BolsaWeb_UI4_1_Anadir_nueva_Empresa.png}{EM-BolsaWeb-UI4.1}{Anadir nueva Empresa}


\subsubsection{Salidas}
	\noindent
	Ninguna.

\subsubsection{Entradas}
	\begin{itemize}
		\item Información de la empresa:
		\begin{itemize}
			\item Nombre.
			\item RFC.
			\item Giro.
			\item Logo.
		\end{itemize}
		\item Contacto de la empresa:
		\begin{itemize}
			\item Nombre del encargado de recursos humanos.
			\item Tipo de medio de contacto. 
			\item Medio de contacto. 
		\end{itemize}
	\end{itemize}

\subsubsection{Comandos}
 	\begin{itemize}
		\item Botón \IUbutton{Agregar logo}: Para subir una imagen con el logo de la empresa, no es obligatorio.  
		\item Botón \IUbutton{+}: Para agregar uno o más contactos al encargado de recursos humanos de la empresa.
		\item Botón \IUbutton{Registrar Empresa}: Para registrar una empresa con los datos proporcionados en el formulario.
		\item Botón \IUbutton{Cancelar}: Para cancelar el registro de la empresa.
	\end{itemize}



%%%%%%%%%%%%%%%%%%%%%%%%%%%%%%%%%%%%%%%%%%%%%%%%%%%%%%%%%%%%%%%%%%%%%%%%%%%%%%%%%

\subsection{EM-BolsaWeb-UI4.3 Editar información de Empresa}

\subsubsection{Objetivo}
	\noindent
	Proporcionar al actor un mecanismo para editar la información de una empresa que proponga ofertas de trabajo.
\subsubsection{Diseño}
	\noindent
	La pantalla muestra un formulario para editar la información de una empresa previamente registrada en el sistema ESCOMobile Bolsa, dicho formulario contiene los diferentes campos requeridos para realizar el registro de la empresa, estando organizado en dos grandes secciones, la primera con los datos de la empresa propiamente dicha y la segunda con los datos del conctacto a quien los interesados podrán dirigirse una vez registrada la empresa y publicadas las ofertas de trabajo, mostrando además en las últimas, los campos que son obligatorios para completar la edición. Así, se tienen los siguientes campos para la edición de la empresa:
	\begin{itemize}
		\item Información de la empresa:
		\begin{itemize}
			\item Nombre.
			\item RFC.
			\item Giro.
			\item Logo.
		\end{itemize}
		\item Contacto de la empresa:
		\begin{itemize}
			\item Nombre del encargado de recursos humanos.
			\item Tipo de medio de contacto. 
			\item Medio de contacto. 
		\end{itemize}
	\end{itemize}
	Mostrando además los botones \IUbutton{Agregar logo}, \IUbutton{+}, \IUbutton{Aceptar} y \IUbutton{Cancelar}.

\IUfig[.5]{gui/Maquetas/BolsaWeb/EM_BolsaWeb_UI4_3_Editar_Informacion_de_la_Empresa.png}{EM-BolsaWeb-UI4.3}{Editar información de Empresa}

\subsubsection{Salidas}
	\noindent
	Ninguna.

\subsubsection{Entradas}
	\begin{itemize}
		\item Información de la empresa:
		\begin{itemize}
			\item Nombre.
			\item RFC.
			\item Giro.
			\item Logo.
		\end{itemize}
		\item Contacto de la empresa:
		\begin{itemize}
			\item Nombre del encargado de recursos humanos.
			\item Tipo de medio de contacto. 
			\item Medio de contacto. 
		\end{itemize}

	\end{itemize}

\subsubsection{Comandos}
 	\begin{itemize}
 	\item Botón \IUbutton{Agregar logo}: Para subir una imagen con el logo de la empresa, no es obligatorio.  
		\item Botón \IUbutton{+}: Para agregar uno o más contactos al encargado de recursos humanos de la empresa.
		\item Botón \IUbutton{Aceptar}: Para editar la información de una empresa con los datos proporcionados en el formulario.
		\item Botón \IUbutton{Cancelar}: Para cancelar la edición de la empresa.

	\end{itemize}
	
