%%%%%%%%%%%%%%%%%%%%%%%%%%%%%%%%%%%%%%%%%%%%%%%%%%%%%%%%%%%%%%%%%%%%%%%%%%%%%%%%%	
%%%%%%%%%%%%%%%%%%%%%%%%%%%%%%%%%%%%%%%%%%%%%%%%%%%%%%%%%%%%%%%%%%%%%%%%%%%%%%%%%

\subsection{EM-Mapa-UI1 Consultar Mapa de ESCOM.}

\subsubsection{Objetivo}
	\noindent
	Mostrar al actor la distribución de las diferentes áreas dentro de la ESCOM como lo son lon salones,
	cubículos, academias, etc. en los diferentes edificos y plantas contenidos en la misma institución. 
	Lo anterior se logra por medio de un mapa 2D, en donde se muestran organizados las diferentes áreas
	tal cual se encuentran en la Superior de Cómputo, pues de esta forma se facilitan, entre otras cosas,
	el encontrar una determinada área académica, localizar la estancia de algún profesor en específico,
	conocer los salones en los cuales se imparten clases dado un grupo, etc..

\subsubsection{Diseño}
	\noindent
	La pantalla se divide en dos secciones. La primera sección muestra una barra de búsqueda que permite 
	realizar búsquedas de profesores, grupos, salones y academias. La segunda sección nos presenta el
	mapa de la ESCOM, y la distribución de sus áreas académicas contenidos en cada uno de sus edificios y
	plantas, como lo son los salones, cubículos, académias, entre otros. 

\pagebreak
\IUfig[.5]{gui/Maquetas/Mapa/EM_Mapa_UI1_Consultar_Mapa_de_ESCOM}{EM-Mapa-UI1}{Consultar Mapa de ESCOM}

\subsubsection{Salidas}
	\begin{itemize}
		\item Areas académicas de ESCOM dibujadas en el mapa de la misma.
	\end{itemize}

\subsubsection{Entradas}
	\noindent
	Ninguna. 

\subsubsection{Comandos}
	\begin{itemize}
		\item \IUbutton{ PB }: Dibuja en el mapa los salones ubicados en la planta baja.
		\item \IUbutton{ P1 }: Dibuja en el mapa los salones ubicados en el primer piso.
		\item \IUbutton{ P2 }: Dibuja en el mapa los salones ubicados en el segundo piso.
		\item \IUbutton{  +  }: Acerca la vista del mapa.
		\item \IUbutton{  -  }: Aleja la vista del mapa.
	\end{itemize}

\subsubsection{Mensajes}
	\begin{Citemize}
		\item \MSGref{MSG3}{Elementos No Disponibles}
	\end{Citemize}

%%%%%%%%%%%%%%%%%%%%%%%%%%%%%%%%%%%%%%%%%%%%%%%%%%%%%%%%%%%%%%%%%%%%%%%%%%%%%%%%%	
%%%%%%%%%%%%%%%%%%%%%%%%%%%%%%%%%%%%%%%%%%%%%%%%%%%%%%%%%%%%%%%%%%%%%%%%%%%%%%%%%

%%%%%%%%%%%%%%%%%%%%%%%%%%%%%%%%%%%%%%%%%%%%%%%%%%%%%%%%%%%%%%%%%%%%%%%%%%%%%%%%%	
%%%%%%%%%%%%%%%%%%%%%%%%%%%%%%%%%%%%%%%%%%%%%%%%%%%%%%%%%%%%%%%%%%%%%%%%%%%%%%%%%
\subsection{EM-Mapa-UI2 Realizar Búsqueda sobre el mapa A.}

\subsubsection{Objetivo}
	\noindent
	Proporcionar un mecanismo para realizar una búsqueda de cubículos de profesores, salones, grupos 
	o academmias de su interés dentro del mapa de la ESCOM por medio de ESCOMobile.

\subsubsection{Diseño}
	\noindent
	Esta pantalla consta de una barra de búsqueda y cuatro íconos para filtrar, cada uno de ellos representa un tipo
	de búsqueda sobre el mapa, esto es, para buscar cubículos de profesores, grupos, salones o academias.
	Se cuenta además con un apartado donde se muestras los resultados coincidentes con la búsqueda.

\pagebreak
\IUfig[.5]{gui/Maquetas/Mapa/EM_Mapa_UI2_Realizar_Busqueda_Sobre_el_Mapa_A}{EM-Mapa-UI2-A}{Realizar búsqueda sobre el mapa}

\subsubsection{Salidas}
	\begin{itemize}
		\item Lista con las coincidencias de la búsqueda con los cubículos de profesores, grupos, salones o academias.
	\end{itemize}

\subsubsection{Entradas}	
\begin{itemize}
	\item El nombre del profesor del cual se quiere consultar el cubículo, el grupo que se desea buscar, el salón que se quiera consultar o la academia de la cual se requeira saber su obicación.
\end{itemize}

\subsubsection{Comandos}
\begin{itemize}
	\item Icono \UCicono{lupa}: Realiza la búsqueda de coincidencias del texto contenido en la barra de búsqueda
	con los datos almacenados de salones, profesores, grupos y academias.
	\item Icono ''Profesor'': Filtra los resultados, mostrado solo aquellos que sean cubículos de profesores.
	\item Icono ''Salón'': Filtra los resultados, mostrado solo aquellos que sean cubículos de Salones.
	\item Icono ''Academia'': Filtra los resultados, mostrado solo aquellos que sean cubículos de Academias.
	\item Icono ''Grupo'': Filtra los resultados, mostrado solo aquellos que sean cubículos de Grupos.
\end{itemize}

\subsubsection{Mensajes}
\begin{Citemize}
	\item \MSGref{MSG15}{Ninguna coincidencia encontrada}.
\end{Citemize}

%%%%%%%%%%%%%%%%%%%%%%%%%%%%%%%%%%%%%%%%%%%%%%%%%%%%%%%%%%%%%%%%%%%%%%%%%%%%%%%%%	
%%%%%%%%%%%%%%%%%%%%%%%%%%%%%%%%%%%%%%%%%%%%%%%%%%%%%%%%%%%%%%%%%%%%%%%%%%%%%%%%%

%%%%%%%%%%%%%%%%%%%%%%%%%%%%%%%%%%%%%%%%%%%%%%%%%%%%%%%%%%%%%%%%%%%%%%%%%%%%%%%%%	
%%%%%%%%%%%%%%%%%%%%%%%%%%%%%%%%%%%%%%%%%%%%%%%%%%%%%%%%%%%%%%%%%%%%%%%%%%%%%%%%%
\pagebreak
\subsection{EM-Mapa-UI2 Realizar Búsqueda sobre el mapa B.}

\subsubsection{Objetivo}
	\noindent
	Mostrar en el mapa de ESCOM la ubicación de agún salón, cubículo, académia o grupo en particular de su interés,
	así como información extra del mismo para consulta.  

\subsubsection{Diseño}
	\noindent
	La pantalla muestra el mapa de la ESCOM con algúna de las áreas (salones, cubículos, etc.) delimitada por un polígono.
	Muestra además, el nombre del área así como infomación extra según sea el caso. Al ubicar el cubículo de un profesor, 
	por ejemplo, muestra también el nombre de éste. 

\pagebreak
\IUfig[.5]{gui/Maquetas/Mapa/EM_Mapa_UI2_Realizar_Busqueda_Sobre_el_Mapa_B}{EM-Mapa-UI2-B}{Realizar búsqueda sobre el mapa}

\subsubsection{Salidas}
	\begin{itemize}
		\item Mapa de ESCOM con el área buscada delimitada.
		\item Información sobre el área buscada.
	\end{itemize}

\subsubsection{Entradas}
	\noindent
	Ninguna.

\subsubsection{Comandos}
\begin{itemize}
		\item \IUbutton{ PB }: Dibuja en el mapa los salones ubicados en la planta baja.
		\item \IUbutton{ P1 }: Dibuja en el mapa los salones ubicados en el primer piso.
		\item \IUbutton{ P2 }: Dibuja en el mapa los salones ubicados en el segundo piso.
		\item \IUbutton{  +  }: Acerca la vista del mapa.
		\item \IUbutton{  -  }: Aleja la vista del mapa.
\end{itemize}

\subsubsection{Mensajes}
	\noindent.
	Ninguno.

%%%%%%%%%%%%%%%%%%%%%%%%%%%%%%%%%%%%%%%%%%%%%%%%%%%%%%%%%%%%%%%%%%%%%%%%%%%%%%%%%	
%%%%%%%%%%%%%%%%%%%%%%%%%%%%%%%%%%%%%%%%%%%%%%%%%%%%%%%%%%%%%%%%%%%%%%%%%%%%%%%%%
