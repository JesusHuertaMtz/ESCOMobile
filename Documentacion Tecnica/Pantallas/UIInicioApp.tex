%%%%%%%%%%%%%%%%%%%%%%%%%%%%%%%%%%%%%%%%%%%%%%%%%%%%%%%%%%%%%%%%%%%%%%%%%%%%%%%%%	
%%%%%%%%%%%%%%%%%%%%%%%%%%%%%%%%%%%%%%%%%%%%%%%%%%%%%%%%%%%%%%%%%%%%%%%%%%%%%%%%%

\subsection{IU1 Pantalla de Inicio.}

\subsubsection{Objetivo}
	Mostrar al actor tres opciones. Registrarse, iniciar sesión o consultar el mapa.

\subsubsection{Diseño}
	Esta pantalla debe aparecer cada vez que inicie la aplicación.

\IUfig[.5]{gui/inicio}{IU1}{Pantalla de Inicio.}

\subsubsection{Salidas}

	Ninguna.

\subsubsection{Entradas}
Presionar sobre un botón la opción requerida.

\subsubsection{Comandos}
\begin{itemize}
	\item \IUbutton{Registrarse}: Manda al actor a la pantalla de registro.
	\item \IUbutton{Iniciar sesión}: Manda al actor a la pantalla para iniciar sesión.
	\item \IUbutton{Continuar sin registrarse}: Manda al actor a la pantalla para consultar el mapa.
\end{itemize}

\subsubsection{Mensajes}
\begin{Citemize}
	\item {\bf MSG01} Ningún mensaje es mostrado.
\end{Citemize}

%%%%%%%%%%%%%%%%%%%%%%%%%%%%%%%%%%%%%%%%%%%%%%%%%%%%%%%%%%%%%%%%%%%%%%%%%%%%%%%%%	
%%%%%%%%%%%%%%%%%%%%%%%%%%%%%%%%%%%%%%%%%%%%%%%%%%%%%%%%%%%%%%%%%%%%%%%%%%%%%%%%%

\subsection{IU2 Pantalla de Inicio de Sesión.}

\subsubsection{Objetivo}
	El actor podrá ingresar sus credenciales para acceder al sistema.

\subsubsection{Diseño}
	Mostrará los campos para ingresar credenciales y recuperación de contraseña.

\IUfig[.5]{gui/iniciar_sesion}{IU2}{Pantalla de Inicio de Sesión.}

\subsubsection{Salidas}

	Pantalla de bienvenida.

\subsubsection{Entradas}

	Boleta o número de usuario.
	Contraseña del actor.

\subsubsection{Comandos}
\begin{itemize}
	\item \IUbutton{Ingresar}: Da acceso al sistema si las credenciales del usuario son correctas.
	\item \IUbutton{¿No tienes cuenta? ¡Registrate!}: Manda al actor a la pantalla de registro.
	\item \IUbutton{Olvidé mi contraseña}: Manda al actor a la pantalla de recuperar contraseña.
\end{itemize}

\subsubsection{Mensajes}
\begin{Citemize}
	\item {\bf MSG02} Las credenciales no son correctas.
\end{Citemize}

%%%%%%%%%%%%%%%%%%%%%%%%%%%%%%%%%%%%%%%%%%%%%%%%%%%%%%%%%%%%%%%%%%%%%%%%%%%%%%%%%	
%%%%%%%%%%%%%%%%%%%%%%%%%%%%%%%%%%%%%%%%%%%%%%%%%%%%%%%%%%%%%%%%%%%%%%%%%%%%%%%%%

\subsection{IU3 Pantalla Resgitrar Nuevo Usuario.}

\subsubsection{Objetivo}
	El actor podrá registrarse en el sistema para acceder a todas las funcionalidades de la aplicación.

\subsubsection{Diseño}
	Mostrará los campos que serán obligatorios para llevar a cabo el registro del nuevo usuario.

\IUfig[.5]{gui/registro}{IU3}{Pantalla Registrar Nuevo Usuario.}

\subsubsection{Salidas}

	Pantalla de bienvenida.

\subsubsection{Entradas}

\begin{itemize}
	\item Boleta / número de empleado.
	\item Nombre.
	\item Apellido Paterno.
	\item Apellido Materno.
	\item Correo electrónico.
	\item Contraseña.
\end{itemize}

\subsubsection{Comandos}
\begin{itemize}
	\item \IUbutton{Registrarse}: Manda al actor a la pantalla de \IUref{IU2}{Pantalla de Inicio de
	 Sesión.}
	\item \IUbutton{¿Ya tienes cuenta? ¡Entra!}: Manda al actor a la pantalla de \IUref{IU2}{Pantalla de
	 Inicio de Sesión.}
\end{itemize}

\subsubsection{Mensajes}
\begin{Citemize}
	\item {\bf MSG03} Las contraseñas no coinciden.
	\item {\bf MSG04} La boleta / número de empleado ya se encuentra registrado.
	\item {\bf MSG05} Debes aceptar los términos y condiciones.
	\item {\bf MSG06} El formato del correo electrónico no es correcto.
\end{Citemize}

%%%%%%%%%%%%%%%%%%%%%%%%%%%%%%%%%%%%%%%%%%%%%%%%%%%%%%%%%%%%%%%%%%%%%%%%%%%%%%%%%	
%%%%%%%%%%%%%%%%%%%%%%%%%%%%%%%%%%%%%%%%%%%%%%%%%%%%%%%%%%%%%%%%%%%%%%%%%%%%%%%%%
%%%%%%%%%%%%%%%%%%%%%%%%%%%%%%%%%%%%%%%%%%%%%%%%%%%%%%%%%%%%%%%%%%%%%%%%%%%%%%%%%	
%%%%%%%%%%%%%%%%%%%%%%%%%%%%%%%%%%%%%%%%%%%%%%%%%%%%%%%%%%%%%%%%%%%%%%%%%%%%%%%%%

\subsection{IU Pantalla Cambiar contraseña.}

\subsubsection{Objetivo}
	El actor podrá cambiar su contraseña cada vez que lo desee.

\subsubsection{Diseño}
	Mostrará los campos que serán obligatorios para llevar a cabo el cambio de contraseña.

\IUfig[.5]{gui/contrasena}{IU3.1}{Cambiar contraseña.}

\subsubsection{Salidas}

	Pantalla de bienvenida.

\subsubsection{Entradas}

\begin{itemize}
	\item Contraseña actual.
	\item Contraseña nueva.
	\item  Repetir contraseña nueva.
\end{itemize}

\subsubsection{Comandos}
\begin{itemize}
	\item Algo
\end{itemize}

\subsubsection{Mensajes}
\begin{Citemize}
	\item {\bf MSG10} Las contraseñas no coinciden.
	
\end{Citemize}

\subsection{IU4 Pantalla Recuperar Contraseña.}

\subsubsection{Objetivo}
	El actor podrá recuperar su contraseña en caso que la haya olvidado o no.

\subsubsection{Diseño}
	Mostrará las intrucciones y campo para ingresar su correo electrónico.

\IUfig[.5]{gui/recuperar_contrasenia}{IU4}{Pantalla Recuperar Contraseña.}

\subsubsection{Salidas}

	Mensaje de confirmación.

\subsubsection{Entradas}

\begin{itemize}
	\item correo electrónico.
\end{itemize}

\subsubsection{Comandos}
\begin{itemize}
	\item \IUbutton{Enviar}: Muestra en pantalla un mensaje para informar al usuario que se envió el método 
	para recuperar su contraseña al correo electrónico que proporcionó. Posteroirmente muestra al usuario
	la pantalla \IUref{IU1}{Pantalla de Inicio.}
\end{itemize}

\subsubsection{Mensajes}
\begin{Citemize}
	\item {\bf MSG07} El formato del correo electrónico no es correcto.
\end{Citemize}



%%%%%%%%%%%%%%%%%%%%%%%%%%%%%%%%%%%%%%%%%%%%%%%%%%%%%%%%%%%%%%%%%%%%%%%%%%%%%%%%%	
%%%%%%%%%%%%%%%%%%%%%%%%%%%%%%%%%%%%%%%%%%%%%%%%%%%%%%%%%%%%%%%%%%%%%%%%%%%%%%%%%



\pagebreak
\subsection{IU5 Pantalla Consultar Salones Por Planta.}

\subsubsection{Objetivo}
	El actor podrá consultar los salones de cada una de las tres plantas de ESCOM.

\subsubsection{Diseño}
	Mostrará los salones y nombre de cada uno.

\IUfig[.5]{gui/mapa}{IU5}{Pantalla Consultar Salones Por Planta.}

\subsubsection{Salidas}

	Los diferentes salones de cada planta.

\subsubsection{Entradas}

\begin{itemize}
	\item
\end{itemize}

\subsubsection{Comandos}
\begin{itemize}
	\item \IUbutton{ PB }: Dibuja en el mapa los salones ubicados en la planta baja.
	\item \IUbutton{ P1 }: Dibuja en el mapa los salones ubicados en el primer piso.
	\item \IUbutton{ P2 }: Dibuja en el mapa los salones ubicados en el segundo piso.
	\item \IUbutton{  +  }: Acerca la vista del mapa.
	\item \IUbutton{  -  }: Aleja la vista del mapa.
\end{itemize}

\subsubsection{Mensajes}
\begin{Citemize}
	\item {\bf MSG08} Acceso a la ubicación.
\end{Citemize}

%%%%%%%%%%%%%%%%%%%%%%%%%%%%%%%%%%%%%%%%%%%%%%%%%%%%%%%%%%%%%%%%%%%%%%%%%%%%%%%%%	
%%%%%%%%%%%%%%%%%%%%%%%%%%%%%%%%%%%%%%%%%%%%%%%%%%%%%%%%%%%%%%%%%%%%%%%%%%%%%%%%%
\pagebreak
\subsection{UI6 Pantalla de inicio del alumno.}

\subsubsection{Objetivo}
	El actor podrá consultar los salones de cada una de las tres plantas de ESCOM. Podrá buscar profesores,
	salones, áreas administrativas.

\subsubsection{Diseño}
	Mostrará los salones, cubículos, áreas administrativas indicando el nombre de cada uno.
	El mapa se mostrará en 2D.
	
\IUfig[.5]{gui/inicio_alumno}{IU6}{Pantalla de inicio del alumno.}

\subsubsection{Salidas}

	Los diferentes salones, cubículos y áreas administrativas de cada planta. Además, algunos cuadros de 
	búsqueda.

\subsubsection{Entradas}

\begin{itemize}
	\item Palabras que se quieren buscar.
\end{itemize}

\subsubsection{Comandos}
\begin{itemize}
	\item \IUbutton{ PB }: Dibuja en el mapa los salones ubicados en la planta baja.
	\item \IUbutton{ P1 }: Dibuja en el mapa los salones ubicados en el primer piso.
	\item \IUbutton{ P2 }: Dibuja en el mapa los salones ubicados en el segundo piso.
	\item \IUbutton{  +  }: Acerca la vista del mapa.
	\item \IUbutton{  -  }: Aleja la vista del mapa.
\end{itemize}

\subsubsection{Mensajes}
\begin{Citemize}
	\item {\bf MSG08} Acceso a la ubicación.
\end{Citemize}

%%%%%%%%%%%%%%%%%%%%%%%%%%%%%%%%%%%%%%%%%%%%%%%%%%%%%%%%%%%%%%%%%%%%%%%%%%%%%%%%%	
%%%%%%%%%%%%%%%%%%%%%%%%%%%%%%%%%%%%%%%%%%%%%%%%%%%%%%%%%%%%%%%%%%%%%%%%%%%%%%%%%
\pagebreak
\subsection{Buscar - UI1- Buscar Edificios y áreas académicas.}

\subsubsection{Objetivo}
	El actor podrá realizar búsquedas de profesores y/o áreas académicas.

\subsubsection{Diseño}
	Mostrará una lista con algunas opciones (profesores o áreas académicas) coicidentes con lo solicitado por el usuario. 

\IUfig[.5]{gui/buscar_IU_1_buscar_edificios_areas}{buscarIU1}{Buscar Edificios y áreas académicas}

\subsubsection{Salidas}

	Lista que muestra el nombre del (los) profesor(es) o área(s) académica(s) coicidente(s) con lo solicitado.

\subsubsection{Entradas}

\begin{itemize}
	\item Nombre del Profesor/área académica que se deseé buscar. 
\end{itemize}

\subsubsection{Comandos}
\begin{itemize}
	\item \IUbutton{Ver más}: Muestra una búsqueda con más resultados.
\end{itemize}

\subsubsection{Mensajes}
\begin{Citemize}
	\item {\bf MSG08} No se econtraron resultados.
\end{Citemize}

%%%%%%%%%%%%%%%%%%%%%%%%%%%%%%%%%%%%%%%%%%%%%%%%%%%%%%%%%%%%%%%%%%%%%%%%%%%%%%%%%	
%%%%%%%%%%%%%%%%%%%%%%%%%%%%%%%%%%%%%%%%%%%%%%%%%%%%%%%%%%%%%%%%%%%%%%%%%%%%%%%%%

\pagebreak
\subsection{ESCOMobile Menú Hamburguer}

\subsubsection{Objetivo}
	El actor podrá gestionar diferentes apartados por medio del menú Hamburger, tal es el caso de las citas, actualizar su información, cerrar seción, eliminar su cuenta, entre otros.

\subsubsection{Diseño}
	Mostrará la infomación asociada al profesor además de las opciones de consultar horario, solicitar cita, calificar y ubicar en el mapa. 

\IUfig[.5]{gui/Acceso/Menu_Hamburguer.png}{EM-Menu1}{Menú Hamburguer}

\subsubsection{Salidas}
	Se muestra la siguiente información del profesor:
	\begin{itemize}
		\item Nombre.
		\item Academia a la que pertenece,
		\item Cúbiculo.
		\item Horaras de entrada y salida. 
	\end{itemize}

\subsubsection{Entradas}

\begin{itemize}
	\item Nombre del Profesor.
\end{itemize}

\subsubsection{Comandos}
\begin{itemize}
	\item \IUbutton{Consultar horario}: Muestra pantalla con horario del profesor.
	\item \IUbutton{Ubicar en el mapa}: Muestra mapa con el cubículo del profesor.
	\item \IUbutton{Solicitar cita}: Muestra formulario de cita a solicitar.
	\item \IUbutton{Calificar profesor}: Muestra formulario de cita a solicitar.
\end{itemize}

\subsubsection{Mensajes}
\begin{Citemize}
	\item Ninguno.
\end{Citemize}

%%%%%%%%%%%%%%%%%%%%%%%%%%%%%%%%%%%%%%%%%%%%%%%%%%%%%%%%%%%%%%%%%%%%%%%%%%%%%%%%%	
%%%%%%%%%%%%%%%%%%%%%%%%%%%%%%%%%%%%%%%%%%%%%%%%%%%%%%%%%%%%%%%%%%%%%%%%%%%%%%%%%

						%%		EM-ALUMNO	  %%

%%%%%%%%%%%%%%%%%%%%%%%%%%%%%%%%%%%%%%%%%%%%%%%%%%%%%%%%%%%%%%%%%%%%%%%%%%%%%%%%%	
%%%%%%%%%%%%%%%%%%%%%%%%%%%%%%%%%%%%%%%%%%%%%%%%%%%%%%%%%%%%%%%%%%%%%%%%%%%%%%%%%

\pagebreak
\subsection{EM-Alumno-UI1 Consultar perfil del profesor}

\subsubsection{Objetivo}
	El actor podrá consultar la información de algún profesor de su interés por medio de su perfil.

\subsubsection{Diseño}
	Mostrará la infomación asociada al profesor además de las opciones de consultar horario, solicitar cita, calificar y ubicar en el mapa. 

\IUfig[.5]{gui/Alumno/EM_Alumno_UI1_Consultar_perfIl_del_profesor.png}{EM-Alumno-UI1}{Consultar Perfil del Profesor}

\subsubsection{Salidas}
	Se muestra la siguiente información del profesor:
	\begin{itemize}
		\item Nombre.
		\item Academia a la que pertenece,
		\item Cúbiculo.
		\item Horaras de entrada y salida. 
	\end{itemize}

\subsubsection{Entradas}

\begin{itemize}
	\item Nombre del Profesor.
\end{itemize}

\subsubsection{Comandos}
\begin{itemize}
	\item \IUbutton{Consultar horario}: Muestra pantalla con horario del profesor.
	\item \IUbutton{Ubicar en el mapa}: Muestra mapa con el cubículo del profesor.
	\item \IUbutton{Solicitar cita}: Muestra formulario de cita a solicitar.
	\item \IUbutton{Calificar profesor}: Muestra formulario de cita a solicitar.
\end{itemize}

\subsubsection{Mensajes}
\begin{Citemize}
	\item Ninguno.
\end{Citemize}

%%%%%%%%%%%%%%%%%%%%%%%%%%%%%%%%%%%%%%%%%%%%%%%%%%%%%%%%%%%%%%%%%%%%%%%%%%%%%%%%%	
%%%%%%%%%%%%%%%%%%%%%%%%%%%%%%%%%%%%%%%%%%%%%%%%%%%%%%%%%%%%%%%%%%%%%%%%%%%%%%%%%

\pagebreak
\subsection{EM-Alumno-UI3 Comentar perfil del profesor}

\subsubsection{Objetivo}
	El actor podrá calificar y comentar el desempeño general del profesor para retroalimentarlo.

\subsubsection{Diseño}
	Mostrará el nombre y foto del profesor, su calificación promedio obtenida y un recuadro para poder comentar. Además de un botón de enviar para persistir la información.

\IUfig[.5]{gui/Alumno/EM_Alumno_UI3_Comentar_perfil_del_profesor.png}{EM-Alumno-UI3}{Comentar Perfil del Profesor}

\subsubsection{Salidas}
	Se muestra la siguiente información del profesor:
	\begin{itemize} 
		\item Nombre.
		\item Fotografía,
		\item Promedio general obtenido.
	\end{itemize}

\subsubsection{Entradas}

\begin{itemize}
	\item Calificación otorgada por el alumno para el profesor.
	\item Comentario del alumno.
\end{itemize}

\subsubsection{Comandos}
\begin{itemize}
	\item \IUbutton{Enviar}: Persiste la información brindada por el alumno.
\end{itemize}

\subsubsection{Mensajes}
\begin{Citemize}
	\item \MSGref{MSG1}{Operación Exitosa}
	\item \MSGref{MSG2}{Operación Fallida}
\end{Citemize}

%%%%%%%%%%%%%%%%%%%%%%%%%%%%%%%%%%%%%%%%%%%%%%%%%%%%%%%%%%%%%%%%%%%%%%%%%%%%%%%%%	
%%%%%%%%%%%%%%%%%%%%%%%%%%%%%%%%%%%%%%%%%%%%%%%%%%%%%%%%%%%%%%%%%%%%%%%%%%%%%%%%%

\pagebreak
\subsection{EM-Alumno-UI4a Modificar información del alumno}

\subsubsection{Objetivo}
	El actor podrá visualizar su información como nombre y fotografía, además de tener la posibilidad de elegir cambiar o actualizar su información en el sistema. 

\subsubsection{Diseño}
	Mostrará el nombre y foto del alumno, además muestra tres iconos que representan los campos de información que puede cambiar o actualizar.
	Los iconos son:
	\begin{itemize} 
		\item Sobre: para actualizar el email.
		\item Candado: para actualizar la contraseña.
		\item Fotografia: para actualizar su fotografía.
	\end{itemize} 
	Además, se tiene un botón \UIbutton{Siguiente} que permite continuar con el proceso de cambio.

\IUfig[.5]{gui/Alumno/EM_Alumno_UI4a_Modificar_informacion_de_alumno.png}{EM-Alumno-UI4a}{Modificar información del alumno}

\subsubsection{Salidas}
	\begin{itemize} 
		\item Ninguna
	\end{itemize}

\subsubsection{Entradas}
\begin{itemize}
	\item Nombre asociado de los elementos a modificar.
\end{itemize}

\subsubsection{Comandos}
\begin{itemize}
	\item \UIbutton{Sobre}: Selecciona modificar el email.
	\item \UIbutton{Candado}: Selecciona modificar la contraseña.
	\item \UIbutton{Contraseña}: Selecciona modificar la fotografía.
	\item \IUbutton{Siguiente}: Permite continuar con el proceso de actualización de información.
\end{itemize}

\subsubsection{Mensajes}
\begin{Citemize}
	\item Ningino.
\end{Citemize}

%%%%%%%%%%%%%%%%%%%%%%%%%%%%%%%%%%%%%%%%%%%%%%%%%%%%%%%%%%%%%%%%%%%%%%%%%%%%%%%%%	
%%%%%%%%%%%%%%%%%%%%%%%%%%%%%%%%%%%%%%%%%%%%%%%%%%%%%%%%%%%%%%%%%%%%%%%%%%%%%%%%%

\pagebreak
\subsection{EM-Alumno-UI4b Modificar información del alumno}

\subsubsection{Objetivo}
	El actor podrá continuar con el proceso de actualización de su informción, proporcionando al sistema los campos previamente elegidos y que ahora serán sus datos actualizados. 

\subsubsection{Diseño}
	Mostrará el nombre del alumno, además muestra los campos correspondientes a la información que el alumno solicitó modificar previamente, para que actualice su infrmación. Además, se tiene un botón \UIbutton{Aceptar} que permite finarlizar el proceso de actualización de información.

\IUfig[.5]{gui/Alumno/EM_Alumno_UI4b_Modificar_informacion_de_alumno.png}{EM-Alumno-UI4b}{Modificar información del alumno}

\subsubsection{Salidas}
	\begin{itemize} 
		\item \MSGref{MSG1}{Operación Exitosa}
	\end{itemize}

\subsubsection{Entradas}
\begin{itemize}
	\item Ninguna.
\end{itemize}

\subsubsection{Comandos}
\begin{itemize}
	\item \IUbutton{Aceptar}: Persiste la información actualizada en el sistema.
\end{itemize}

\subsubsection{Mensajes}
\begin{Citemize}
	\item \MSGref{MSG1}{Operación Exitosa}
\end{Citemize}