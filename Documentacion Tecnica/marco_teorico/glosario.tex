%---------------------------------------------------------
\section{Glosario}

% Introducción. 
\noindent
En este apartado se enlistan las diferentes palabras y términos que a lo largo del documento y en la propia aplicación se utilizan, y una descripción de los mismos, con el objetivo de contextualizar al lector y comprender mejor la aplicación, su estructura, lo que ésta realiza y la interacción que tiene con el usuario final. \\

\begin{description}
	% A.
	\item[Actividad cultural:] Taller y espacio cultural ofrecidos por parte de la ESCOM hacia los alumnos para complementar su formación. 
	\item[Actividad deportiva:] Tiempos, espacios y equipos deportivos ofrecidos por parte de la ESCOM hacia los alumnos para complementar su formación. 
	\item[Actividad extraclase:] Actividades que los alumnos de la ESCOM realizan fuera del aula pero dentro del plantel. Actividades afines a las propias pero que no tienen relación directa con las unidades de aprendizaje o sus contenidos. 
	\item[Activity:] Pantallas o vistas que forman una aplicación en Android.
	\item[Alumno:] Persona que cuenta con un número de boleta y está inscrito en la ESCOM. Además debe estar registrado en el sistema.
	\item[Aplicación / App:] Programa informático diseñado como herramienta para permitir a un usuario realizar uno o diversos tipos de tareas. 
	\item[Aplicación móvil:] Aplicación informática diseñada para ser ejecutada en teléfonos inteligentes, tabletas y otros dispositivos móviles. 
	\item[Área:] Es una superficie acotada. Puede ser un salón, área administrativa, sala de TT, laboratorio, baño, cubículo, etc.
	\item[Asesoría:] Proceso en el que se da asistencia, apoyo mediante la sugerencia, ilustración u opinión con conocimiento por parte de los profesores a los alumnos de la ESCOM.
	
	% B.
	\item[Base de datos:] Conjunto de datos pertenecientes a un mismo contexto y almacenados sistemáticamente para su posterior uso.
	\item[Boleta:] Identificador único de cada alumno dentro del IPN, a usar dentro del sistema, proporcionado por el IPN a los alumnos inscritos.
	\item[Bolsa de trabajo:] Conjunto de ofertas laborales presentadas por diferentes empresas interesadas en llenar una vacante relacionada con la carrera de Ingeniería en Sistemas Computacionales.
	
	% C. 
	\item[Caso de uso:] Descripción de los pasos o las actividades que deberán realizarse para llevar a cabo algún proceso. Los personajes o entidades que participarán en un caso de uso se denominan actores. Secuencia de interacciones que se desarrollarán entre un sistema y sus actores en respuesta a un evento que inicia un actor principal sobre el propio sistema.
	\item[Cita:] Es un acuerdo entre profesor y alumno para reunirse en una fecha y hora específica para solventar alguna situación académica (asesoría, solución de dudas, revisión de protocolo/trabajo terminal, tutoría, revisión prácticas/proyectos) del alumno.
	\item[Cliente:] Aplicación software que consume un servicio remoto en otro ordenador conocido como servidor, normalmente a través de una red de telecomunicaciones.
	\item[Contraseña:] Clave de acceso conformada por caracteres alfanuméricos asociada a una boleta o número de empleado.
	\item[Componente Software:] Elemento de un sistema de software que ofrece un conjunto de servicios, o funcionalidades, a través de interfaces definidas.
	\item[Cubículo:] Lugar físico designado a los profesores para atender situaciones escolares.

	% D.
	\item[Diagrama de casos de uso:] Notación gráfica para representar casos de uso.
	\item[Disponibilidad de profesor:] Se refiere al tiempo libre que el profesor desea dedicar a atender citas con alumnos.
	
	% E.
	\item[Entorno de desarrollo integrado / IDE:] Aplicación informática que proporciona servicios integrales para facilitarle al desarrollador o programador el desarrollo de software.
	\item[Equipo Scrum:] Grupo de personas encargadas de realizar un proyecto haciendo uso de la metodología Scrum. Los integrantes deben tener los conocimientos y habilidades para realizar el trabajo (análisis, diseño, desarrollo, pruebas, documentación, etc).
	\item[Escuela Superior de Cómputo / ESCOM:] Institución pública mexicana de educación superior perteneciente al Instituto Politécnico Nacional.
	\item[Evento:] Es una actividad cultural, deportiva, social, informativa y recreativa que se llevará a cabo en las instalaciones de ESCOM en una determinada fecha y hora o en algún período de fechas determinado.

	%F.
	\item[Fragment:] Representación de un comportamiento o una parte de la interfaz de usuario en una Activity de Android.
	
	% H. 
	\item[Horario:] Son los días de la semana y la hora de inicio y término en el que un profesor imparte una unidad de aprendizaje.
	
	% I.
	\item[Iniciar sesión:] Sección del sistema que auténtica al usuario mediante una boleta o un número de empleado y una contraseña, permitiéndonos identificar su tipo (alumno, profesor) brindándole acceso a su perfil.
	\item[Instituto Politécnico Nacional / IPN:] Institución pública mexicana de investigación y educación en niveles medio superior, superior y posgrado.
	\item[Interfaz Gráfica de Usuario:] Conocida también como GUI (del inglés graphical user interface), es un programa informático que actúa de interfaz de usuario, utilizando un conjunto de imágenes y objetos gráficos para representar la información y acciones disponibles en la interfaz. Su principal uso, consiste en proporcionar un entorno visual sencillo para permitir la comunicación con el sistema operativo de una máquina o computador.
	\item[Iteración:] Acto de repetir un proceso con la intención de alcanzar una meta deseada, objetivo o resultado.
	
	%L.
	\item[Lenguaje de programación:] Lenguaje formal que especifica una serie de instrucciones para que una computadora produzca diversas clases de datos. Los lenguajes de programación pueden usarse para crear programas que pongan en práctica algoritmos específicos que controlen el comportamiento físico y lógico de una computadora.

	% M.
	\item[Mapa:] Representación gráfica de las áreas de la ESCOM.
	\item[Metodología ágil:] Métodos de ingeniería del software basados en el desarrollo iterativo e incremental, donde los requisitos y soluciones evolucionan con el tiempo según la necesidad del proyecto.
	\item[Metodología de desarrollo de software / Metodología:] Marco de trabajo usado para estructurar, planificar y controlar el proceso de desarrollo en sistemas de información.
	\item[Metodología SCUM:] Conjunto de buenas prácticas para trabajar colaborativamente, en equipo, y obtener el mejor resultado posible de un proyecto. 
	
	%N.
	\item[Número de empleado:] Identificador único de cada profesor dentro del IPN, a usar dentro del sistema, proporcionado por el IPN a los profesores contratados.

	% P.
	\item[Perfil:] Conjunto de información que contiene su configuración, preferencias, contraseñas, etc. de los alumnos y profesores dentro de la aplicación.
	\item[Product Owner:] Integrante del equipo Scrum, es quien asegura que el equipo trabaje de forma adecuada desde la perspectiva del negocio. Se mantiene en total cercanía con el cliente del proyecto.
	\item[Profesor:] Persona que imparte clases en ESCOM, cuenta con un número de empleado y está registrado o no en el sistema.
	\item[Programación Orientada a Objetos / POO:] Paradigma de programación que innova la forma de obtener resultados. Los objetos manipulan los datos de entrada para la obtención de datos de salida específicos, donde cada objeto ofrece una funcionalidad especial.

	% R.
	\item[Requisito funcional:] Función del sistema de software o sus componentes. Función es descrita como un conjunto de entradas, comportamientos y salidas.
	\item[Requisito no funcional:] Requisito que especifica criterios que pueden usarse para juzgar la operación de un sistema en lugar de sus comportamientos específicos, ya que éstos corresponden a los requisitos funcionales. 

	% S.
	\item[Salón:] Entorno físico en el cual se llevan a cabo los procesos de enseñanza y aprendizaje.
	\item[Scrum Master:] Integrante del equipo Scrum cuyo trabajo primario es eliminar los obstáculos que impiden que el equipo alcance el objetivo del sprint. No es el líder del equipo (porque ellos se autoorganizan), actúa como una protección entre el equipo y cualquier influencia que le distraiga. 
	\item[Semestre:] Período escolar donde se imparten unidades de aprendizaje.
	\item[Servicio Web:] Tecnología que utiliza un conjunto de protocolos y estándares que sirven para intercambiar datos entre aplicaciones.
	\item[Servidor:] Aplicación software en ejecución capaz de atender las peticiones de un cliente y devolverle una respuesta en concordancia.
	\item[Sistema gestor de base de datos / SGBD:] Conjunto de programas que permiten el almacenamiento, modificación y extracción de la información en una base de datos, además de proporcionar herramientas para añadir, borrar, modificar y analizar los datos. 
	\item[Sistema móvil:] Aplicación informática diseñada para ser ejecutada en teléfonos inteligentes, tabletas y otros dispositivos móviles. 
	\item[Software:] Conjunto de programas y rutinas que permiten a la computadora realizar determinadas tareas.
	\item[Sprint:] Período en el cual se lleva a cabo parte del trabajo final. 

	% T.
	\item[Tarea:] Trabajo que se asigna a los estudiantes de ESCOM por parte de sus profesores, a realizar fuera del aula y de la jornada escolar.
	\item[Tiempo libre:] Se refiere al tiempo en el horario del profesor en el que no está impartiendo clase y se encuentra dentro de la ESCOM.
	
	% U.
	\item[Unidad de aprendizaje:] Curso impartido en la ESCOM y que tiene la intención educativa para que se apliquen y se adquieran conocimientos con el fin de que los alumnos desarrollen competencias como el pensamiento estratégico, el pensamiento creativo, trabajo colaborativo, trabajo participativo, ética, manejo de conflictos, responsabilidad social, comunicación asertiva, actitud emprendedora.
	\item[Usuario:] Conjunto de permisos y de recursos (o dispositivos) a los cuales se tiene acceso. Es decir, un usuario puede ser tanto una persona como una máquina, un programa, etc.
	
	% V.
	\item[Versión Software:] Proceso de asignación de un nombre, código o número único, a un software para indicar su nivel de desarrollo.
	\item[Visitante:] Usuario no registrado en la aplicación ESCOMobile.
\end{description}

