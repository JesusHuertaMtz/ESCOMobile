\noindent
Para el futuro de la app ESCOMobile se han contemplado distintas ideas, evolucionando en imagen, o en módulos, por ejemplo. Sin embargo, nos hemos convencido de que la app, si bien va a evolucionar en los antes mencionados, merece en un futuro próximo tres principales cosas: dar por terminado el sistema con la implementación total de sus ocho módulos y pruebas a cada uno de ellos, mayor estabilidad (con más pruebas en un mayor público) y el hecho de ser publicada. Con ello nos referimos, primeramente, a que se encuentre disponible para un mayor grupo de personas dentro de la ESCOM, donde la interacción de éstas con la app sea real, con situaciones y necesidades reales. Para lograrlo, es necesario entonces que las autoridades del plantel la aprueben y cooperen con información necesaria y relevante para el total funcionamiento del sistema, otro de nuestros objetivos a corto plazo. 
\newline
Así bien, a lo largo de este periodo con la app en pruebas con datos y un público adecuado, con incidencias encontradas y resultas, deseamos que ESCOMobile sea una completa realidad, haciéndolo oficial dentro de la tienda de aplicaciones Android y disponible para los usuarios con Android en general. El trabajo para lograr todo lo anterior, si bien es competencia futura, ya se está realizando, pues, hemos estado en reuniones con el licenciado Andrés Ortigoza, director de la ESCOM, para externar nuestra idea y planes para los siguientes meses, obteniendo de su parte respuestas entusiastas y positivas.
\newline
\newline
Por otro lado, hablando de un futuro a mediano plazo, tenemos contemplados el mantenimiento y actualizaciones del sistema, y no solo a corregir errores detectados nos referimos, sino a estar siempre al pendiente de que nuestra app brinde una apariencia y experiencia agradable y sencilla, para ello se tiene en mente que la vista y el diseño pueden cambiar a lo largo del tiempo, para mantener las líneas de diseño actuales y que ESCOMobile se sienta siempre fresca. 
\newline
Además, es importante para nosotros el mantener actualizada la aplicación, corrigiendo con cada versión errores encontrados y manteniendo la información del plantel al día, pues ha sido esto uno de nuestros objetivos y pilares de la aplicación desde que fue concebida como idea. Con ello nos referimos a tener siempre la más actual información de salones, profesores, academias, cubículos, edificios, etc. en la app y los servicios que ofrece. Otro aspecto interesante que deseamos cubrir durante el futuro a mediano plazo es la llegada de la app a otros dispositivos, como lo son aquellos con sistema operativo iOS, cumpliendo así con el compromiso de ESCOMobile para con los alumnos de la propia ESCOM, pues, aunque son minoría, podrían disfrutar de la app al igual que el resto con dispositivos Android.
\newline
\newline
Finalmente, para el futuro a largo plazo tenemos en la mira la ampliación del sistema, con esto pensamos agregar más módulos y funcionalidad al propio, para atender nuevas necesidades y expandir las posibilidades. 
\newline
Por otra parte, deseamos poder llevar los servicios que ESCOMobile ofrece a otras escuelas y adaptarlos a las mismas. Que las diferentes escuelas del Instituto Politécnico Nacional, cuenten con apps similares, que nos permitan tener entonces los resultados obtenidos para ESCOM en un mayor plano, satisfaciendo así las necesidades de un mayor público y cumpliendo objetivos a en una escala mayor. 