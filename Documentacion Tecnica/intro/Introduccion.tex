%---------------------------------------------------------
\section{Introducción}

% Introducción.
\noindent
Este documento presenta la aplicación ESCOMobile, propuesta por estudiantes de la Escuela Superior de Cómputo (ESCOM) y dirigida a la misma institución. Se plantean los aspectos importantes a considerar en torno a la app, como la problemática que la origina, la solución que se propone, el diseño de ésta como aplicación, la investigación relizada acerca de otras aplicaciones similares, lo requerimientos necesarios para su desarrollo, etcétera. Pues son todos estos factores importantes para que la aplicación funcione correctamente y sea de apoyo para su solución del problema que más adelante se describirá detalladamente. Así, el documento se encuentra estructurado de la siguiente forma:
\newline

% Justificación.
\noindent
Capítulo 1: Aquí se exponen las razones y los problemas por los cuales ESCOMobile surge, los caminos a seguir para conseguir resultados favorables ante las problemáticas y los resultados que se espera tener cuando la aplicación llegue al usuario final. Pues es bien sabido que debemos enfocarnos en ellos, y en cómo tratar con los problemas para así encontrar la mejor solución, con los mejores beneficios y mayores resultados. 
\newline

% Marco teórico. 
\noindent
Capitulo 2: Se establece un marco teórico en donde se describe el entorno en el cual se desarrolla el proyecto y el público al que va dirigida. Se plantean las ideas para la creación de ESCOMobile planteando un análisis sobre las diferentes tecnologías y plataformas necesarios para realizar el proyecto, destacando la importancia de éstos y las concecuencias que en la app reflejan. Así mismo, se enlistan las diferentes palabras y términos que a lo largo del documento y en la propia aplicación se utilizan, con el objetivo de contextualizar al lector para comprender mejor la aplicación, su estructura y lo que realiza.
\newline	

% Estado del arte. 
\noindent
Capítulo 3: Se presentan las aplicaciones ya existentes que realizan tareas similares al proyecto aquí propuesto, el funcionamiento, propósitos y el público al que se dirigen las mismas. Se realizan comparativas para encontrar puntos diferenciales entre una aplicación y otra, y así implementar de la mejor manera en la aplicación las características con mayor importancia y que nos ayudarán a lograr los objetivos propuestos más adelante.
\newline 

% Propuesta del proyecto.
\newpage 
\noindent
Capítulo 4: Se describen concretamente la propuesta de la aplicación, las acciones que ésta puede realizar, los usuarios que tienen una interacción con ella, así como las características y herramientas que en ella se contemplan para su funcionamiento. Se enlistan los objetivos general y específicos que se pretenden alcanzar con la aplicación además, es en este capítulo donde se tiene un primer gran acercamiento con el sistema, pues es aquí donde se analizan las acciones que se requieren y se empiezan a descubrir y organizar los diferentes requerimientos funcionales y no funcionales que harán que ESCOMobile funcione y logre cumplir sus objetivos. 
\newline

% Trabajo drealizado y a futuro. 
\noindent
Finalmente se cuenta con un varios capítulos dedicados al anális, diseño y desarrollo de la aplicación. En él se muestra el trabajo que se ha realizado desde que nació la idea hasta el presente día. Se detallan las diversas tareas realizadas para dar vida a la aplicación, tomando como base todo lo previamente analizado y postulado. Aquí se concentran en forma de iteraciones todas las acciones que se realizaron. Se muestran avances de la aplicación, resultados, cambios y problemas a los que se ha llegado a lo largo de estos meses de trabajo. Por último, se describe todo aquello que falta realizar y que se implementará en un futuro para que ESCOMobile se realice completa y exitosamente.
\newline

