%---------------------------------------------------------
\section{Introducción}

% Introducción.
\noindent
Este documento presenta la aplicación ESCOMobile, propuesta por estudiantes de la Escuela Superior de Cómputo (ESCOM) y dirigida a la misma institución. Se plantean los aspectos importantes a considerar en torno a la app, como la problemática que la origina, la solución que se propone, el diseño de ésta como aplicación, la investigación relizada acerca de otras aplicaciones similares, lo requerimientos necesarios para su desarrollo, etcétera. Pues son todos estos factores importantes para que la aplicación funcione correctamente y sea de apoyo para la solución del problema. Así, el documento se encuentra estructurado de la siguiente forma:
\newline
Así, el presente se encuentra divido por 9 bloques: Introducción, Planteamiento del proyecto, Requisitos de Software, Modelo de negocio, Descripción de los módulos del sistema, Modelo de Casos de uso, Modelado de la interacción, Modelo de mensajes y Modelo de Dominio del problema, mismos que se describen a continuación. 
\newline

% Introducción.
\noindent
\textbf{Introducción}: En la introducción, que en este momento lee, se da una presentación al sistema, a lo que ofrece y la forma en que se trabaja. Se ofrece un primer acercamiento al problema y se da una propuesta de solución al mismo, añadiendo, las razones por las cuales la propuesta mencionada ayuda a la solución del problema, así como las razones por las cuales se decide trabajar con ciertos requerimientos software o hardware para el sistema. Se describen los objetivos a alcanzar a lo largo del desarrollo del sistema y los que éste debe de cumplir. Además, se presenta un listado con palabras técnicas del sistema que se pueden encontrar a lo largo del documento y que pueden resultar confusas para el lector. Así, la lista mencionada sirve para contextualizar a quienes así lo requieran y comprender el documento y su contenido. 
\newline

% Planteamiento del proyecto. 
\noindent
\textbf{Planteamiento del proyecto}: En este apartado se muestra la propuesta que tenemos como solución al problema previamente descrito, y su trasfondo necesario para comprenderla mejor, la implementación que se le dará, y el trabajo necesario para que ésta cumpla con los objetivos definidos. Así bien, es aquí donde se presentan los antecedentes de las tecnologías requeridas y cómo se han utilizado para dar vida a sistemas similares al que nosotros proponemos, así como las diferencias que hay entre los mencionados y nuestra solución. Por otro lado, se describe el trabajo realizado a lo largo de trabajo terminal I, organizado por iteraciones, se menciona los resultados obtenidos en el antes mencionado y se enlistan las observaciones que directores y sinodales hicieron al respecto durante la presentación de TT1. En otro aspecto, se determina un análisis de los posibles riesgos que se puedan presentar durante el desarrollo del sistema, sus prioridades y la forma en que pueden afectar el proyecto. 
\newline	

% Requisitos de software.  
\noindent
\textbf{Requisitos de software}: Para este capítulo se presentan los requisitos necesarios para la implementación del sistema, como lo son las tecnologías y herramientas a usar para su desarrollo, por ejemplo, servidor, sistema operativo, lenguaje de programación, Sistema gestor de bases de datos, etcétera. Así como los requerimientos funcionales y no funcionales contemplados para la aplicación, describiendo éstos y brindando sus propósitos y actores con los cuales interactúan, teniendo así un primer acercamiento a los casos de uso.
\newline 

% Modelo del Negocios. 
\noindent
\textbf{Modelo del Negocios}: En esta sección se presentan las reglas de negocio que componen el sistema ESCOMobile, dividiéndolas en reglas del negocio del negocio y reglas de negocio del sistema y describiendo de cada una de éstas su tipo, nivel, sentencia y ejemplos.
\newline

% Descripción de los módulos del sistema.
\noindent
\textbf{Descripción de los módulos del sistema}: En este apartado se muestran los ocho módulos que componen al sistema, se describe cada uno de ellos, brindando la información acerca de la forma en que implementa e interactúan, así como su propósito y funcionalidad en la app. Mostrando además, por cada uno de ellos, los casos de uso que los componen y sus respectivos diagramas de casos de uso.
\newline

% Modelado de Casos de uso.
\noindent
\textbf{Modelados de casos de uso}: Siendo uno de los capítulos más extensos del presente documento, el modelado de casos de uso desglosa, por módulo, cada uno de los casos de uso propuestos. Mostrando así, por cada uno de ellos, descripción, atributos importantes (Propósito, Actor, Entradas y salidas para el propio, precondiciones y postcondiciones, errores, entre otros), y las trayectorias principales y alternativas que el actor y el sistema siguen para cumplir su cometido (tarea específica).
Son los casos de uso de especial importancia, ya que describen el comportamiento de la app y su interacción con los actores, además, concentra reglas de negocio, mensajes y descripción de pantallas para lograr lo antes mencionado, siendo base también para los guiones de prueba.
\newline

% Modelo de la interacción.
\noindent
\textbf{Modelo de la interacción}: En este capítulo se puede observar el diseño de las pantallas que componen a cada uno de los módulos del sistema, teniendo por cada una de ellas su descripción, diseño, entradas, salidas, comandos (botones, iconos, etc.), así como una imagen con el diseño que la pantalla final (programada) tendrá dentro de la aplicación, contemplando así la distribución de la misma, la información que muestra, los botones, colores y estilo en general para los módulos y la propia aplicación. 
\newline

% Modelo de mesajes.
\noindent
\textbf{Modelo de mensajes}: En este apartado se enlistan los mensajes que se utilizan dentro del sistema ESCOMobile para indicar al usuario el estado de las acciones y actividades que realiza dentro de la app, como lo son mensajes de confirmación, de éxito, fallo, entre otros. Por cada mensaje se muestra redacción, propósito, parámetros y ejemplo.
\newline

% Modelo del dominio del problema.
\noindent
\textbf{Modelo del dominio del problema}: Finalmente, en esta sección del documento, se presentan diferentes diagramas para establecer la estructura de la aplicación, como lo son el diagrama de base de datos, el diagrama de clases y la arquitectura del sistema, así como descripciones de los mismos.
\newline
