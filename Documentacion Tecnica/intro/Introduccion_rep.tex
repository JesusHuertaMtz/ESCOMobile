%---------------------------------------------------------
\section{Introducción}

% Introducción.
\noindent
Este documento presenta una descripción de las actividades realizadas a lo largo de Trabajo Terminal II para el sistema ESCOMobile. Organizado y mostrando en diferentes capítulos, mismos que enfatizan un aspecto importante del proyecto. Se plantean los aspectos importantes a considerar en torno a la app y la forma en que cada una de las tareas en su desarrollo se desempeñan. 
\newline
Así, el presente se encuentra divido por 6 bloques: Introducción, Términos del negocio, Planteamiento del proyecto, Trabajo realizado a lo largo de TT2, Conclusiones y Trabajo a futuro, mismos que se describen a continuación. 
\newline

% Introducción.
\noindent
\textbf{Introducción}: En la introducción, que en este momento lee, se da una presentación al sistema, a lo que ofrece y la forma en que se trabaja. Se ofrece un primer acercamiento al problema y se da una propuesta de solución al mismo, añadiendo, las razones por las cuales la propuesta mencionada ayuda a la solución del problema, así como las razones por las cuales se decide trabajar con ciertos requerimientos software o hardware para el sistema. Se describen los objetivos a alcanzar a lo largo del desarrollo del sistema y los que éste debe de cumplir. 
\newline

% Términos del negocio. 
\noindent
\textbf{Términos del negocio}: En esta sección se presenta un listado con palabras técnicas del sistema que se pueden encontrar a lo largo del documento y que pueden resultar confusas para el lector. Así, la lista mencionada sirve para contextualizar a quienes así lo requieran y comprender el documento y su contenido. 
\newline	

% Planteamiento del proyecto.
\noindent
\textbf{Planteamiento del proyecto}: En este apartado se muestra la propuesta que tenemos como solución al problema previamente descrito, y su trasfondo necesario para comprenderla mejor, la implementación que se le dará, y el trabajo necesario para que ésta cumpla con los objetivos definidos. Así bien, es aquí donde se presentan los antecedentes de las tecnologías requeridas y cómo se han utilizado para dar vida a sistemas similares al que nosotros proponemos, así como las diferencias que hay entre los mencionados y nuestra solución. Por otro lado, se describe el trabajo realizado a lo largo de trabajo terminal I, organizado por iteraciones, se menciona los resultados obtenidos en el antes mencionado y se enlistan las observaciones que directores y sinodales hicieron al respecto durante la presentación de TT1. 
\newline 

% Trabajo realizado durante TT2.
\newpage 
\noindent
\textbf{Trabajo realizado durante TT2}: En el presente capítulo se describe de forma detallada el trabajo realizado durante el periodo correspondiente al trabajo terminal II, organizado y mostrado en iteraciones, explicando en cada una el objetivo y avance obtenido en los diferentes apartados considerados, como el análisis, el diseño o las pruebas realizadas al sistema. Se adjuntan además esquemas y diagramas que permiten una mejor comprensión del trabajo realizado durante el desarrollo del sistema.
\newline

% Conclusiones.
\noindent
\textbf{Conclusiones}: En esta sección se pueden leer las conclusiones a las que se llegaron con la realización de la aplicación, se retoman los objetivos propuestos y se mencionan la forma en que la app los cumple. Se presentan experiencias y enseñanzas obtenidas con la realización del proyecto y se mencionan nuevos objetivos a alcanzar con el resultado obtenido. 
\newline

% Trabajo a futuro.
\noindent
\textbf{Trabajo a futuro}: Finalmente, en este apartado, se menciona la forma en que se pretende dar continuidad a los resultados obtenidos en TT, pues, se describe detalladamente los planes que tenemos para el sistema a corto, mediano y largo plazo.
\newline
