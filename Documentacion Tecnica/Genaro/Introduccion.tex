%---------------------------------------------------------
\section{Introducción}

Los desafíos que enfrenta la educación superior son numerosos y muy variados. Las cambiantes circunstancias del entorno representan oportunidades y amenazas, ante las cuales deben buscarse soluciones creativas, ya que en esta era del conocimiento, el desarrollo de las naciones sólo podrá darse reconociendo el valor estratégico del conocimiento y la información, aspectos en los cuales las instituciones de educación superior deben desempeñar un papel preponderante [3].

En México, las últimas dos décadas, las principales iniciativas para mejorar la calidad de la educación superior se han centrado en la asociación entre evaluación y recursos financieros como principal estrategia para alcanzar los fines propuestos, y tal estrategia se ha puesto en marcha por medio de múltiples y muy variados programas [a].

Si bien la evaluación ha logrado instaurar una necesaria rendición de cuentas de las instituciones educativas de nivel superior, desafortunadamente no ha logrado establecer claramente una mejoría en la calidad de la educación. Los indicadores que se han utilizado para valorar la calidad educativa, aunque importantes institucionalmente (como incrementar el nivel de escolaridad del personal académico, mejorar el tiempo de dedicación, consolidar la infraestructura, etc.), solamente se aproximan, pero no miden la calidad en sí [a].

Entre los diversos factores que puedan afectar la calidad en las escuelas de nivel superior en méxico se encuentra la mala información sobre estas estrategias para el beneficio del alumno, pues, en la mayor parte de los casos, son desconocidos por éstos; o la información con la que se cuenta es poco fiable. Ésto ocasiona que no se aprovechen los programas en su totalidad o bien, que la calidad de nuestras instituciones sea menor. Así es importante tener claros ideas, estratégias y objetivos para lograr tener una educación de calidad.  

La escuela superior de cómputo, en adelante ESCOM,  tiene como visión "Ser la Unidad Académica, líder en la formación de profesionales en ingeniería, tecnología y ciencias, de la computación, [...]." \cite{ESCOM}. Para cumplir su objetivo, la ESCOM,  se apoya de programas como el programa institucional de tutorías (PIT), contando con docentes, en su mayoría, con grado académico de maestría en ciencias[1], se sirve de un departamento de prefectura para dar seguimiento a que se cumplan los horarios de clase. También los docentes dedican horas de asesorías académica, en las cuales se da seguimiento a trabajos terminales, sinodalias, dudas específicas sobre algún tema visto en clase con los alumnos de ESCOM, revisión de proyectos finales, etc. 

Sin embargo, como muchas instituciones de nivel superior, la ESCOM, no está excenta de sufrir problemas que originen una caída en su caldad o el aprovechamiento de sus alumnos. Así, podemos pensar en diferentes causas por las cuales nos enfrentamos a ésto, pero nos enfocamos en una, y es que los alumnos en ESCOM no conocen en su totalidad los programas que la institución ofrece para que el índice de reprobación sea menor y se comprendan los temas vistos en clase, ya sea por falta de confianza, desinterés o apatía por parte del alumno. 
Por otro lado, por motivos personales o laborales, en ocasiones los profesores se ven obligados a desviar parte de su tiempo a actividades extracurriculares dejando sin atender las posibles situaciones de los alumnos que necesiten de su atención (asesorías, revisiones, dudas y demás situaciones extra-clase). Entre las diferentes causas por las que un docente no imparte asesorías destacan: horarios de las distintas materias, cantidad de alumnos asignados en un grupo, etc. [2]. 
Así, podemos percatarnos de las deficiencias que encontramos en algunos alumnos o profesores en la ESCOM, y que juntos causan un gran conflicto que conlleva a pensar que, es más bien la falta de información y comuncación entre alumnos y profesores (dentro y fuera del aula) la que afecta en el aprovechamiento y calidad de la escuela. Pero, lograr dicha comunicación resulta de igual forma complicado debido a las múltiples actividades que un docente lleva a cabo en su día a día, ésto hace que sea una tarea difícil para el alumno poder localizar al docente dentro de la institución y poder hablar con éste acerca de actividades y demás asuntos académicos que puedan influir positivamente en el alumno y su desempeño. A veces encontrar al docente para tratar dichos temas puede llevar horas, e inclusive -en el peor de los casos- días, repercutiendo de manera significativa en los recursos del alumno (tiempo y dinero), hecho que se podría evitar y lograr en un futuro una posible mejora en el aprovechamiento del alumno, así como el interés por el aprendizaje y conocimiento que la "Superior de Cómputo", aulas áreas y profesores pueden aportarle.

Dicho lo anterior, y con los alumnos y su desarrollo académico como prioridad, se propone desarrollar una aplicación móvil para dispositivos Android que permita al alumno un desgloce total de información de la ESCOM, pues se permitirá -por medio de un mapa- conocer y localizar áreas importantes y destacadas del plantel, como edificios, salones, áreas deportivas, académias, etc.; se muestran eventos, actividades culturares, deportivas y recreativas que se puedan realizar y que los alumnos puedan disfrutar, siempre pensando en su formación integral; por otro lado y con especial relevancia, se presenta una interacción entre profesores y alumnos de ESCOM por medio de perfiles y citas, en donde los alumnos podrán consultar el perfil e información académica de los profesores, como su horario de clases, cubículo, entre otros, de modo que el alumno pueda conocer más a sus profesores y, en caso de ser necesario, poder agendar citas con el mismo para tratar asuntos escolares de una manera más sencilla y rápida. 
Con ella se planea un mejor aprovechamiento de los recursos que la escuela ofrece a sus estudiantes, como eventos, cursos, actividades y materias impartidas, con un mayor conocimiento de las áreas y el personal que labora ahí y que trabaja directamente con ellos. Por otro lado, la facilidad de tratar con los profesores asuntos académicos, sin necesidad de desplazarse grandes distancias o gastar recursos monetarios en vano. Todo para, a largo plazo, lograr un mejor aprovechamiento académico en los alumnos que se verá reflejado en su formación, solventando así los problemas descritos anteriormente y causando un impacto mayormente favorable en las situaciones mencionadas. 


%%\begin{description}
%%\end{description}

%---------------------------------------------------------
