\section{Mensajes de ESCOMobile}

%===========================  MSG1 ==================================
\begin{mensaje}{MSG1}{Operación Exitosa}{Informativo}
	\item[Canal:] Sistema.
	\item[Propósito:] Notificar al actor que la operación que se solicitó al sistema se se realizó de
	manera correcta y exitosa. 
	\item[Redacción:] La $<OPERACION>$ se llevó a cabo correctamente.
	\item[Parámetros:] OPERACION: Actividad que el actor debe realizar.
	\item[Ejemplo:] El registro se realizó exitosamente.
	%\item[Referenciado por:] %\refIdElem{DIC-UA-COSIE-CU1.3.1},  \refIdElem{DIC-UA-COSIE-CU1.3.2}%\refIdElem{DIC-A-CU3.3}, \refIdElem{DIC-A-CU3.3}, \refIdElem{DIC-CGC-COSIE-CU1.1},  refIdElem{DIC-CGC-COSIE-CU1.2.1},  \refIdElem{DIC-CGC-DTE-CU1.1}
\end{mensaje}

%============================== MSG2 =================================
\begin{mensaje}{MSG2}{Operación Fallida}{Error}
	\item[Canal:] Sistema.
	\item[Propósito:] Notificar al actor que la operación que se solicitó al sistema no se se pudo 
	realizar correctamente.
	\item[Redacción:] Error, $<ERROR_ENCONTRADO>$.
	\item[Parámetros:] ERROR\_ENCONTRADO: Error que el sistema identificó al intentar realizar 
	la operación.
	\item[Ejemplo:] Error, No hay conexión al sistema.
	%\item[Referenciado por: ] \refIdElem{DIC-UA-COSIE-CU1.3.1}
\end{mensaje}

%===========================  MSG3 ==================================
\begin{mensaje}{MSG3}{Elementos No Disponibles}{Error}
	\item[Canal:] Sistema.
    \item[Propósito:] Notificar al actor que no existen elementos en el sistema para mostrar.
    \item[Redacción:] ¡No hay $<ELEMENTOS>$ para mostrarte!
    \item[Parámetros:] ELEMENTOS: Información que se requiere para concluir un proceso.
    \item[Ejemplo:] ¡No hay profesores para mostrarte!
	%\item[Referenciado por: ] \refIdElem{DIC-UA-COSIE-CU1.3}, \refIdElem{DIC-UA-COSIE-CU1.3.1},  \refIdElem{DIC-UA-COSIE-CU1.3.2}%\refIdElem{DIC-A-CU1}, \refIdElem{DIC-A-CU2}, \refIdElem{DIC-A-CU3}  \refIdElem{DIC-CGC-DPF-CU1.3}
\end{mensaje}

%============================== MSG4 =================================
\begin{mensaje}{MSG4}{Información duplicada}{Error}
	\item[Canal:] Sistema.
    \item[Propósito:] Notificar al actor que alguno de los campos introducidos ya ha sido registrado antes.
    \item[Redacción:] ¡El/la $<ELEMENTO>$ ya se encuentra registrado(a)!
    \item[Parámetros:] ELEMENTO: Parte de la información que está duplicada.
    \item[Ejemplo:] ¡El/la boleta ya se encuentra registrado(a)!
	%\item[Referenciado por: ] 
\end{mensaje}

%============================== MSG5 =================================
\begin{mensaje}{MSG5}{Falta dato obligatorio}{Error}
	\item[Canal:] Sistema.
	\item[Propósito:] Notificar al actor que no introdujo alguno de los datos marcados como obligatorios. 
	\item[Redacción ] ¡Error! El campo $<DATO>$ es un campo obligatorio, favor de ingresarlo.
	\item[Parámetros:] DATO: Información que se requiere para completar la operación.
	\item[Ejemplo:] ¡ERROR! El campo boleta es un campo obligatorio, favor de ingresarlo.
	%\item[Referenciado por: ] \refIdElem{DIC-UA-COSIE-CU1.3.1}  %\refIdElem{DIC-A-CU3.1}, \refIdElem{DIC-CGC-COSIE-CU1.1},  \refIdElem{DIC-CGC-DPF-CU1.3.1}, \refIdElem{DIC-CGC-DTE-CU1.1}, \refIdElem{DIC-CGC-DTE-CU1.2}
\end{mensaje}

%============================== MSG6 =================================
\begin{mensaje}{MSG6}{Formato de campo Incorrecto}{Error}
	\item[Canal:] Sistema.
	\item[Propósito:] Notificar al actor que el dato ingresado en alguno de los campos no 
	cumple con el formato definido.
	\item[Redacción:] ¡Error! El campo $<DATO>$ no cumple con el formato válido definido.
	\item[Parámetros:] DATO: Información que se requiere para completar la operación.
	\item[Ejemplo:] ¡Error! El campo contraseña no cumple con el formato válido definido.
	%\item[Referenciado por: ] \refIdElem{DIC-UA-COSIE-CU1.3.1} %\refIdElem{DIC-A-CU3.1}, \refIdElem{DIC-CGC-DTE-CU1.1}, \refIdElem{DIC-CGC-DTE-CU1.2}
\end{mensaje}

%===========================  MSG7 ==================================
\begin{mensaje}{MSG7}{Eliminar elemento}{Confirmación}
	\item[Canal:] Sistema.
    \item[Propósito:] Solicitar la confirmación del actor para eliminar un elemento.
    \item[Redacción:] ¿Está seguro que quieres elimimar: $<ELEMENTO>$?
    \item[Parámetros:] \item ELEMENTO: Es el elemento que se requiere eliminar.
    \item[Ejemplo:] ¿Está seguro que quieres elimimar: Evento?
	\item[Referenciado por: ] 
\end{mensaje}

%===========================  MSG8  ==================================
\begin{mensaje}{MSG8}{Acceso a la ubicación del actor}{Confirmación}
	\item[Canal:] Sistema.
    \item[Propósito:] Solicitar la confirmación del actor para poder acceder a su ubicación.
    \item[Redacción:] Proporcionada por el API de Google Maps.
    \item[Parámetros:] \item Ningúno.
    \item[Ejemplo:] Para obtener mejores resultados, activa la ubicación del dispositivo.
	\item[Referenciado por: ] 
\end{mensaje}

%=========================MSG9=======================================
\begin{mensaje}{MSG9}{Número de Boleta/Número de empleado no válido}{Error}
\item[Canal:] Sistema.
	\item[Propósito:] Notificar al actor que la boleta o en número de empleado no es válido o es inexistente.
	\item[Redacción:] ¡Error! La Boleta no cumple con el formato válido definido.
	\item[Parámetros:] Boleta o número de empleado: Identificador único de alumno o prefesor perteneciente al IPN.
	\item[Ejemplo:] ¡Error! La boleta no existe.
	\item[Ejemplo:] ¡Error! La boleta no cumple el formato valido.
	\item[Ejemplo:] ¡Error! El número de empleado no existe.
	%\item[Referenciado por: ] \refIdElem{DIC-UA-COSIE-CU1.3.1} %\refIdElem{DIC-A-CU3.1}, \refIdElem{DIC-CGC-DTE-CU1.1}, \refIdElem{DIC-CGC-DTE-CU1.2}
\end{mensaje}
%=========================MSG10=======================================
\begin{mensaje}{MSG10}{Correo Electrónico inexistente}{Error}
\item[Canal:] Sistema.
	\item[Propósito:] Notificar al actor que el correo electŕonico proporcionado no existe en el sistema.
	\item[Redacción:] ¡Error! El correo $<CORREO\_ELECTRONICO>$ es incorrecto o inexistente.
	\item[Parámetros:] CORREO\_ELECTRONICO: correo electrónico proporcionado por el usuario.
	\item[Ejemplo:] ¡Error! El correo mane.13@hotmail.com es incorrecto o inexistente.
	%\item[Referenciado por: ] \refIdElem{DIC-UA-COSIE-CU1.3.1} %\refIdElem{DIC-A-CU3.1}, \refIdElem{DIC-CGC-DTE-CU1.1}, \refIdElem{DIC-CGC-DTE-CU1.2}
\end{mensaje}

%=========================MSG11=======================================
\begin{mensaje}{MSG11}{Eventos inexistentes}{Error}
\item[Canal:] Sistema.
	\item[Propósito:] Notificar al actor que no hay eventos existentes.
	\item[Redacción:] ¡Error! No existen eventos para mostrar.
	\item[Parámetros:] Ninguno.
	\item[Ejemplo:] ¡Error! No existen eventos para mostrar.
	
	%\item[Referenciado por: ] \refIdElem{DIC-UA-COSIE-CU1.3.1} %\refIdElem{DIC-A-CU3.1}, \refIdElem{DIC-CGC-DTE-CU1.1}, \refIdElem{DIC-CGC-DTE-CU1.2}
\end{mensaje}

%===========================  MSG12 ==================================
\begin{mensaje}{MSG12}{Translape o fecha y hora no valida}{Advertencia}
	\item[Canal:] Sistema.
	\item[Propósito:] Notificar al actor que la fecha y hora que solicitó para agendar una cita tiene
	una alta probabilidad de ser rechadaza por el profesor debido a que se traslapa con el horario de
	clases de este último ,ya existe una cita ese dia y horario o es dia inhabil.
	\item[Redacción:] La cita en el horario de $<HORA>$ y fecha de $<FECHA>$, se traslapa o es un dia inhabil.
	\item[Parámetros:] 
		\begin{itemize}
			\item HORA: Hora en la que el actor desea agendar la cita.
			\item FECHA: Fecha en la que el actor desea agendar la cita.
		\end{itemize}
	\item[Ejemplo:] La cita en la fecha 22/03/2018 y el horario de 12:30:00, no se puede agendar por translape o dia inhabil.
	\item[Referenciado por:] %\refIdElem{DIC-UA-COSIE-CU1.3.1},  \refIdElem{DIC-UA-COSIE-CU1.3.2}%\refIdElem{DIC-A-CU3.3}, \refIdElem{DIC-A-CU3.3}, \refIdElem{DIC-CGC-COSIE-CU1.1},  refIdElem{DIC-CGC-COSIE-CU1.2.1},  \refIdElem{DIC-CGC-DTE-CU1.1}
\end{mensaje}

%=========================MSG13=======================================
\begin{mensaje}{MSG13}{Empresa no registrada}{Error}
\item[Canal:] Sistema.
	\item[Propósito:] Notificar al actor que la empresa introducida no esta registrada en el sistema.
	\item[Redacción:] ¡Error! Empresa sin registrar o inexistente.
	\item[Parámetros:] Ninguno.
	\item[Ejemplo:] ¡Error! Empresa sin registrar o inexistente.
	
	%\item[Referenciado por: ] \refIdElem{DIC-UA-COSIE-CU1.3.1} %\refIdElem{DIC-A-CU3.1}, \refIdElem{DIC-CGC-DTE-CU1.1}, \refIdElem{DIC-CGC-DTE-CU1.2}
\end{mensaje}

%===========================  MSG14 ==================================
\begin{mensaje}{MSG14}{Editar elemento}{Confirmación}
	\item[Canal:] Sistema.
    \item[Propósito:] Solicitar la confirmación del actor para editar un elemento.
    \item[Redacción:] ¿Está seguro que quieres editar la información de: $<ELEMENTO>$?
    \item[Parámetros:] \item ELEMENTO: Es el elemento que se requiere editar.
    \item[Ejemplo:] ¿Está seguro que quieres editar la información de: ESCOMobile?
	\item[Referenciado por: ] 
\end{mensaje}

%===========================  MSG15 ==================================
\begin{mensaje}{MSG15}{Ninguna coincidencia encontrada}{Informativo}
	\item[Canal:] Sistema.
    \item[Propósito:] Notificar al actor que no se encontraron coincidencias en el sistema para mostrar.
    \item[Redacción:] No se encontraron coincidencias con $<ELEMENTO>$ en el Sistema.
    \item[Parámetros:] ELEMENTO: Dato introducido por el actor, con el cual se pretende buscar algún tipo de coincidencia en la base de datos del sistema.
    \item[Ejemplo:] No se encontraron coincidencias con Huerta Martínez Jesús en el Sistema.
	%\item[Referenciado por: ] \refIdElem{DIC-UA-COSIE-CU1.3}, \refIdElem{DIC-UA-COSIE-CU1.3.1},  \refIdElem{DIC-UA-COSIE-CU1.3.2}%\refIdElem{DIC-A-CU1}, \refIdElem{DIC-A-CU2}, \refIdElem{DIC-A-CU3}  \refIdElem{DIC-CGC-DPF-CU1.3}
\end{mensaje}

%===========================  MSG16 ==================================
\begin{mensaje}{MSG16}{Contraseñas no coinciden}{Error}
	\item[Canal:] Sistema.
    \item[Propósito:] Notificar al actor que las dos contraseñas introducidas no coinciden una con la otra.
    \item[Redacción:] Las constraseñas introducidas no coinciden.
    \item[Parámetros:] Ninguno.
    \item[Ejemplo:] Las constraseñas introducidas no coinciden.
	%\item[Referenciado por: ] \refIdElem{DIC-UA-COSIE-CU1.3}, \refIdElem{DIC-UA-COSIE-CU1.3.1},  \refIdElem{DIC-UA-COSIE-CU1.3.2}%\refIdElem{DIC-A-CU1}, \refIdElem{DIC-A-CU2}, \refIdElem{DIC-A-CU3}  \refIdElem{DIC-CGC-DPF-CU1.3}
\end{mensaje}

%===========================  MSG17 ==================================
\begin{mensaje}{MSG17}{Cerrar sesión}{Confirmación}
	\item[Canal:] Sistema.
    \item[Propósito:] Solicitar al actor la confirmación sobre cerrar la sesión de su cuenta.
    \item[Redacción:] ¿Estás seguro que deseas cerrar la sesión?
    \item[Parámetros:] Ninguno.
    \item[Ejemplo:] ¿Estás seguro que deseas cerrar la sesión?
	%\item[Referenciado por: ] \refIdElem{DIC-UA-COSIE-CU1.3}, \refIdElem{DIC-UA-COSIE-CU1.3.1},  \refIdElem{DIC-UA-COSIE-CU1.3.2}%\refIdElem{DIC-A-CU1}, \refIdElem{DIC-A-CU2}, \refIdElem{DIC-A-CU3}  \refIdElem{DIC-CGC-DPF-CU1.3}
\end{mensaje}

%===========================  MSG18 ==================================
\begin{mensaje}{MSG18}{Eliminar cuenta}{Confirmación}
	\item[Canal:] Sistema.
    \item[Propósito:] Solicitar al actor la confirmación sobre eliminar su cuenta.
    \item[Redacción:] ¿Estás seguro que deseas eliminar tu cuenta?
    \item[Parámetros:] Ninguno.
    \item[Ejemplo:] ¿Estás seguro que deseas eliminar tu cuenta?
	%\item[Referenciado por: ] \refIdElem{DIC-UA-COSIE-CU1.3}, \refIdElem{DIC-UA-COSIE-CU1.3.1},  \refIdElem{DIC-UA-COSIE-CU1.3.2}%\refIdElem{DIC-A-CU1}, \refIdElem{DIC-A-CU2}, \refIdElem{DIC-A-CU3}  \refIdElem{DIC-CGC-DPF-CU1.3}
\end{mensaje}

%===========================  MSG19 ==================================
\begin{mensaje}{MSG19}{Cancelar cita}{Confirmación}
	\item[Canal:] Sistema.
    \item[Propósito:] Solicitar al actor una confirmación acerca de la cancelación de una cita previamente agendada.
    \item[Redacción:] ¿Estás seguro de que deseas cancelar la cita?
    \item[Parámetros:] Ninguno.
    \item[Ejemplo:] ¿Estás seguro de que deseas cancelar la cita?
	%\item[Referenciado por: ] \refIdElem{DIC-UA-COSIE-CU1.3}, \refIdElem{DIC-UA-COSIE-CU1.3.1},  \refIdElem{DIC-UA-COSIE-CU1.3.2}%\refIdElem{DIC-A-CU1}, \refIdElem{DIC-A-CU2}, \refIdElem{DIC-A-CU3}  \refIdElem{DIC-CGC-DPF-CU1.3}
\end{mensaje}

%===========================  MSG20 ==================================
\begin{mensaje}{MSG20}{No se puede cancelar cita}{Error}
	\item[Canal:] Sistema.
    \item[Propósito:] Informar al actor que la cita seleccionada no se puede Cancelar, pues faltan menos de dos horas para que ésta comience. 
    \item[Redacción:] No es posible cancelar la cita, pues falta menos de dos horas para que ésta comience.
    \item[Parámetros:] Ninguno.
    \item[Ejemplo:] No es posible cancelar la cita, pues falta menos de dos horas para que ésta comience.
	%\item[Referenciado por: ] \refIdElem{DIC-UA-COSIE-CU1.3}, \refIdElem{DIC-UA-COSIE-CU1.3.1},  \refIdElem{DIC-UA-COSIE-CU1.3.2}%\refIdElem{DIC-A-CU1}, \refIdElem{DIC-A-CU2}, \refIdElem{DIC-A-CU3}  \refIdElem{DIC-CGC-DPF-CU1.3}
\end{mensaje}

%===========================  MSG21 ==================================
\begin{mensaje}{MSG21}{Cita agendada cancelada}{Informativo}
	\item[Canal:] Sistema.
    \item[Propósito:] Informar al alumno o profesor que una cita previamente agendada se canceló. 
    \item[Redacción:] Tu cita con $<ACTOR>$ del día $<DIA>$ a las $<HORA>$ hrs, ha sido cancelada. 
    \item[Parámetros:] ACTOR: Profesor o alumno con quien se llevaría a cabo la cita. DIA: Día acordado para realizar la cita. HORA: Hora establecida para la cita.
    \item[Ejemplo:] Tu cita con Hernández Contreras Euler del día 10 NOV 2018 a las 17:00:00 hrs, ha sido cancelada. 
	%\item[Referenciado por: ] \refIdElem{DIC-UA-COSIE-CU1.3}, \refIdElem{DIC-UA-COSIE-CU1.3.1},  \refIdElem{DIC-UA-COSIE-CU1.3.2}%\refIdElem{DIC-A-CU1}, \refIdElem{DIC-A-CU2}, \refIdElem{DIC-A-CU3}  \refIdElem{DIC-CGC-DPF-CU1.3}
\end{mensaje}

%===========================  MSG22 ==================================
\begin{mensaje}{MSG22}{Aceptar solicitud de cita}{Confirmación}
	\item[Canal:] Sistema.
    \item[Propósito:] Solicitar al actor una confirmación acerca de la aceptación de una solicitud de cita previamente realizada.
    \item[Redacción:] ¿Estás seguro de que deseas aceptar la solicitud de cita? 
    \item[Parámetros:] Ninguno.
    \item[Ejemplo:] ¿Estás seguro de que deseas aceptar esta solicitud solicitud de cita? 
	%\item[Referenciado por: ] \refIdElem{DIC-UA-COSIE-CU1.3}, \refIdElem{DIC-UA-COSIE-CU1.3.1},  \refIdElem{DIC-UA-COSIE-CU1.3.2}%\refIdElem{DIC-A-CU1}, \refIdElem{DIC-A-CU2}, \refIdElem{DIC-A-CU3}  \refIdElem{DIC-CGC-DPF-CU1.3}
\end{mensaje}

%===========================  MSG23 ==================================
\begin{mensaje}{MSG23}{No se puede aceptar solicitud de cita}{Informativo}
	\item[Canal:] Sistema.
    \item[Propósito:] Informar al actor que la solicitud de cita seleccionada no se puede Aceptar, pues faltan menos de dos horas para que la hora y fecha propuestas para ésta transcurran.. 
    \item[Redacción:] No es posible aceptar la solicitud de cita, pues faltan menos de dos horas para que la fecha y hora propuestas se cumplan.
    \item[Parámetros:] Ninguno.
    \item[Ejemplo:] No es posible aceptar la solicitud de cita, pues faltan menos de dos horas para que la fecha y hora propuestas se cumplan.
	%\item[Referenciado por: ] \refIdElem{DIC-UA-COSIE-CU1.3}, \refIdElem{DIC-UA-COSIE-CU1.3.1},  \refIdElem{DIC-UA-COSIE-CU1.3.2}%\refIdElem{DIC-A-CU1}, \refIdElem{DIC-A-CU2}, \refIdElem{DIC-A-CU3}  \refIdElem{DIC-CGC-DPF-CU1.3}
\end{mensaje}

%===========================  MSG24 ==================================
\begin{mensaje}{MSG24}{Solicitud de cita aceptada}{Informativo}
	\item[Canal:] Sistema.
    \item[Propósito:] Informar al alumno o profesor que una solicitud de cita previamente realizada se aceptó. 
    \item[Redacción:] Tu solicitud de cita con $<ACTOR>$ del día $<DIA>$ a las $<HORA>$ hrs, ha sido aceptada. 
    \item[Parámetros:] ACTOR: Profesor o alumno con quien se llevaría a cabo la cita. DIA: Día acordado para realizar la cita. HORA: Hora establecida para la cita.
    \item[Ejemplo:] Tu solicitud de cita con Hernández Contreras Euler del día 10 NOV 2018 a las 17:00:00 hrs, ha sido aceptada. 
	%\item[Referenciado por: ] \refIdElem{DIC-UA-COSIE-CU1.3}, \refIdElem{DIC-UA-COSIE-CU1.3.1},  \refIdElem{DIC-UA-COSIE-CU1.3.2}%\refIdElem{DIC-A-CU1}, \refIdElem{DIC-A-CU2}, \refIdElem{DIC-A-CU3}  \refIdElem{DIC-CGC-DPF-CU1.3}
\end{mensaje}

%===========================  MSG25 ==================================
\begin{mensaje}{MSG25}{Solicitud de cita cancelada}{Informativo}
	\item[Canal:] Sistema.
    \item[Propósito:] Informar al alumno o profesor que una solicitud de cita previamente realizada se canceló. 
    \item[Redacción:] Tu solicitud de cita con $<ACTOR>$ del día $<DIA>$ a las $<HORA>$ hrs, ha sido cancelada. 
    \item[Parámetros:] ACTOR: Profesor o alumno con quien se llevaría a cabo la cita. DIA: Día acordado para realizar la cita. HORA: Hora establecida para la cita.
    \item[Ejemplo:] Tu solicitud de cita con Hernández Contreras Euler del día 10 NOV 2018 a las 17:00:00 hrs, ha sido cancelada. 
	%\item[Referenciado por: ] \refIdElem{DIC-UA-COSIE-CU1.3}, \refIdElem{DIC-UA-COSIE-CU1.3.1},  \refIdElem{DIC-UA-COSIE-CU1.3.2}%\refIdElem{DIC-A-CU1}, \refIdElem{DIC-A-CU2}, \refIdElem{DIC-A-CU3}  \refIdElem{DIC-CGC-DPF-CU1.3}
\end{mensaje}

%===========================  MSG26 ==================================
\begin{mensaje}{MSG26}{Ayuda sobre agendar citas}{Informativo}
	\item[Canal:] Sistema.
    \item[Propósito:] Informar e instruir al alumno sobre cómo solicitar agendar una cita. 
    \item[Redacción:] 
		Gracias por usar el sistema de citas de ESCOMobile.
		\newline
		Para comenzar a disfrutar de este beneficio con tus profesores debes saber que:
		\newline
		1) Debes introducir todos los campos desplegados en la pantalla.
		\newline
		2) Para poder agendar una cita con algún profesor, éste debe de estar registrado (tener una cuenta) en el sistema, de lo contrario no podrás solicitar agendar con él. 
		\newline
		3) Los profesores que encontrarás para seleccionar en este apartado son solo aquellos que se encuentran registrados. La lista se actualiza periódicamente para brindarte más posibilidades en caso de haber nuevos registros.  
		\newline
		4) Debes expresar a tu profesor un motivo por el cual deseas obtener la cita, éste será leído por tu profesor y le ayudará a decidir sobre aceptar o no tu cita.
		\newline
		5) Tu nombre puede ser consultado por tu profesor cuando solicitas agendar una cita.
		\newline
		6) La desición de tu profesor sobre la cita es única y personal, puede depender, entre otras cosas, de asuntos personales o de tiempo.
		\newline
		7) El lugar en el que se desempeña la cita es por defecto el cubículo del profesor.
		\newline
		8) Una vez solicitada la cita, no podrás modificar la información introducida.
		\newline
		9) La puede ser cancelada (por ti o el profesor) antes de que ésta sea aprobada o rechazada, incluso una ves aprobada podrá se podrá cancelar, siempre y cuando falten más de dos hora para que llegue el tiempo propuesto.
		\newline
		10) El cualquiera de los casos, a la hora de solicitar agendar una cita, cuando ésta es aceptada, rechazada o cancelada se te notificará al respecto, o bien al profesor en caso de ser tú quien cancela.
		\newline
		11) Puedes solicitar agendar más citas con el mismo u otros profesores registradas, y el que éstas se acepten o rechacen no afecta para nada en el cómo usas la app, a tu información o tu perfil. 
		\newline
		12) El uso que le des al apartado es tu responsabilidad, úsalo con respeto.				  
    \item[Parámetros:] Ninguno.	

	%\item[Referenciado por: ] \refIdElem{DIC-UA-COSIE-CU1.3}, \refIdElem{DIC-UA-COSIE-CU1.3.1},  \refIdElem{DIC-UA-COSIE-CU1.3.2}%\refIdElem{DIC-A-CU1}, \refIdElem{DIC-A-CU2}, \refIdElem{DIC-A-CU3}  \refIdElem{DIC-CGC-DPF-CU1.3}
\end{mensaje}

%===========================  MSG27 ==================================
\begin{mensaje}{MSG27}{Fecha u hora de cita no disponibles}{Error}
	\item[Canal:] Sistema.
    \item[Propósito:] Informar al alumno que la fecha o el horario seleccionados para agendar la cita no están disponibles, pues se trata de un día festivo, fin de semana, alguna hora fuera de clase o una hora traslapada con el horario del profesor.
    \item[Redacción:] No es posible agendar en el día: $<DIA>$ o en la hora: $<HORA>$hrs. 
    \item[Parámetros:] DIA: Día acordado para realizar la cita. HORA: Hora establecida para la cita.
    \item[Ejemplo:] No es posible agendar en el día: 20 OCT 2018 o en la hora: 17:30:00 hrs.
	%\item[Referenciado por: ] \refIdElem{DIC-UA-COSIE-CU1.3}, \refIdElem{DIC-UA-COSIE-CU1.3.1},  \refIdElem{DIC-UA-COSIE-CU1.3.2}%\refIdElem{DIC-A-CU1}, \refIdElem{DIC-A-CU2}, \refIdElem{DIC-A-CU3}  \refIdElem{DIC-CGC-DPF-CU1.3}
\end{mensaje}

%===========================  MSG28 ==================================
\begin{mensaje}{MSG28}{Tamaño de mensaje superado}{Error}
	\item[Canal:] Sistema.
    \item[Propósito:] Informar al actor que el tamaño máxico en un campo de texto a introducir se superó. 
    \item[Redacción:] Tamaño de mensaje demasiado grande, el máximo es de 240 caracteres.
    \item[Parámetros:] Ninguno.
    \item[Ejemplo:] Tamaño de mensaje demasiado grande, el máximo es de 240 caracteres.
	%\item[Referenciado por: ] \refIdElem{DIC-UA-COSIE-CU1.3}, \refIdElem{DIC-UA-COSIE-CU1.3.1},  \refIdElem{DIC-UA-COSIE-CU1.3.2}%\refIdElem{DIC-A-CU1}, \refIdElem{DIC-A-CU2}, \refIdElem{DIC-A-CU3}  \refIdElem{DIC-CGC-DPF-CU1.3}
\end{mensaje}