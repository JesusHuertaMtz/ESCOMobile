% Copie este bloque por cada caso de uso:
%-------------------------------------- COMIENZA descripción del caso de uso.
%\begin{UseCase}[archivo de imágen]{UCX}{Nombre del Caso de uso}{
\begin{UseCase}{EM-Alumno-CU1.1}{Modificar Información.}{
	\noindent
	Este caso de uso permite al actor modificar su información registrada en el 
	sistema, pues algún dato está erróneo o desactualizado. La información que
	se encuentra disponible para edición es el correo electrónico, la contraseña
	y la fotografía, siendo los dos primeros ingresados cuando se realizó el registro. Se debe mencionar que la edición de los anteriores no afecta en
	nada el funcionamiento de la cuenta en la aplicación o las citas previamente
	asociadas a su cuenta. Además, la información antes referida puede ser
	modificada cuantas veces sea requerido por el actor dentro de la app.
	\newline
	}
	\UCitem{Versión}{0.1}
	\UCitem{Actor}{Alumno.}
	\UCitem{Propósito}{Proporcionar al actor un mecanismo que le permita
	actualizar su información registrada en el sistema.}
	\UCitem{Entradas}{
		\begin{itemize}
			\item Correo electrónico (nuevo).
			\item Contraseña (nueva).
			\item Duplicado de contraseña (nueva). 
			\item Fotografía (nueva). 
		\end{itemize}
	}
	\UCitem{Origen}{Pantalla.}
	\UCitem{Salidas}{
		Se muestra la siguiente información del alumno:
		\begin{itemize}
			\item Nombre.
			\item Boleta.
			\item Fotografía (original).
			\item Correo electrónico (original).
			\item Contraseña (original).
			\item \MSGref{MSG1}{Operación Exitosa}.
		\end{itemize}
	}
	\UCitem{Destino}{Pantalla.}
	\UCitem{Precondiciones}{Ninguna.}
	\UCitem{Postcondiciones}{Persiste la información actualizada en el sistema.}
	\UCitem{Errores}{
		\begin{enumerate}[\hspace*{0.5cm} \bfseries{E}1:]
			\item \label{EM-Alumno-CU1-1-E1} Cuando algún campo no comple con el formato valido definido. Muestra el mensaje \MSGref{MSG6}{Formato de campo Incorrecto} y \textbf{continúa en el paso \ref{l_Alumno_CU1_1_E1} de la trayectoria Principal.}

			\item \label{EM-Alumno-CU1-1-E2} Cuando las contraseñas introducidas no coinciden. Muestra el mensaje \MSGref{MSG16}{Contraseñas no coinciden} y \textbf{continúa en el paso \ref{l_Alumno_CU1_1_E1} de la trayectoria Principal.}.
		\end{enumerate}
	}
	\UCitem{Tipo}{Secundario, viene de \UCref{EM-Alumno-CU1}.}
	\UCitem{Autor}{Huerta Matínez Jesús Manuel.}
	\UCitem{Revisor}{Fernández Quiñones Isaac.}
	\UCitem{Estatus}{Corregido.}
\end{UseCase}

\begin{UCtrayectoria}{Principal}

	%Paso 1.
	\UCpaso [\UCactor] Solicita modificar su información registrada en el sistema
	presionando en el icono \UCicono{engrane} de la pantalla \IUref{EM-Alumno-UI1}{Consultar Perfil del Alumno}. 

	%Paso 2.
	\UCpaso Obtiene Nombre, Boleta, Fotografía, Correo electrónico y contraseña
	asociados a la cuenta del alumno.

	%Paso 3.
	\UCpaso Muestra la pantalla \IUref{EM-Alumno-UI1-1}{Modificar información} con la información obtenida.

	%Paso 4.
	\UCpaso[\UCactor] Introduce los campos que desea modificar. \label{l_Alumno_CU1_1_E1}

	%Paso 5.
	\UCpaso [\UCactor] Solicita modificar la información de la cuenta
	presionando el botón \IUbutton{Aceptar}.
	
	%Paso 6.
	\UCpaso Valida que el correo electrónico proporcionado cumpla con el formato correcto, según la regla \BRref{EM-RN-S001}{Formato de correo electronico}. [Error  \ref{EM-Alumno-CU1-1-E1}]

    % Paso 7.
    \UCpaso Valida la contraseña ingresada cumpla con el formato de contraseña defido, según la regla de nogocio \BRref{EM-RN-S004}{Formato de Contraseña}. [Error  \ref{EM-Alumno-CU1-1-E1}]

    % Paso 8.
    \UCpaso Valida que las contraseñas obtenidas de los campos ''Contraseña'' y ''Repite tu contraseña'' coincidan una con la otra. [Error  \ref{EM-Alumno-CU1-1-E2}]

    % Paso 9. 
    \UCpaso Valida que la fotografía seleccionada cumpla con el formato adecuado, según la regla \BRref{EM-RN-N011}{Formato de fotografía}. [Error \ref{EM-Alumno-CU1-1-E1}] \Trayref{A} 

    % Paso 10. 
    \UCpaso Persiste la información introducida del alumno en el sistema. \label{l_EM_Alumno_CU1_1_PersisteInfo}

	% Paso 11. 
    \UCpaso Muestra el mensaje \MSGref{MSG1}{Operación Exitosa} en la pantalla
    \IUref{EM-Alumno-UI1}{Consultar Perfil del Alumno} y actualiza la
    información de la misma. 
     
\end{UCtrayectoria}

\begin{UCtrayectoriaA}{A}{Cuando el actor no desea agregar o modificar
una foto en su perfil.}

	% A1.
	\UCpaso Continúa en el paso \ref{l_EM_Alumno_CU1_1_PersisteInfo} de la trayectoria principal.

\end{UCtrayectoriaA}


%-------------------------------------- TERMINA descripción del caso de uso.
%%%%%%%%%%%%%%%%%%%%%%%%%%%%%%%%%%%%%%