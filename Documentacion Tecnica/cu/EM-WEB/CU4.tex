	% \IUref{IUAdmPS}{Administrar Planta de Selección}
% \IUref{IUModPS}{Modificar Planta de Selección}
% \IUref{IUEliPS}{Eliminar Planta de Selección}


% Copie este bloque por cada caso de uso:
%-------------------------------------- COMIENZA descripción del caso de uso.

%\begin{UseCase}[archivo de imágen]{UCX}{Nombre del Caso de uso}{
	\begin{UseCase}{CU04}{EM-WebBolsa-CU04-Modificar Oferta}{
		En este caso de uso el actor  modifica una oferta de trabajo, seleccionara la empresa, horario,vacante, sueldo y seleccionara el contacto. Para asi dar de alta sus respectivas ofertas de trabajo.}
		\UCitem{Versión}{0.1}
		\UCitem{Actor}{Administrador}
		\UCitem{Propósito}{Modificar y mantener actualizadas las ofertas para los alumnos de la ESCOM.}

		\UCitem{Entradas}{
		\begin{itemize}
		\item Empresa
		\item Horario
		\item Vacante
		\item Sueldo
		\item Contacto
		\end{itemize}
		}

		\UCitem{Origen}{}
		\UCitem{Salidas}{Ninguna}
		\UCitem{Destino}{No aplica}
		\UCitem{Precondiciones}{NInguna. }
		\UCitem{Postcondiciones}{
		Ninguna
		}
		\UCitem{Errores}{
		\begin{enumerate}[\hspace*{0.5cm} \bfseries{E}1:]
\item \label{EM-WebEvento-CU04-E1} Cuando no se introdujeron todos los campos marcados como obligatorios. Muestra el mensaje \MSGref{MSG5}{Falta dato obligatorio} y \textbf{continúa en el paso \ref{faltainfo} de la trayectoria Principal.}
			
		\end{enumerate}	
		}
		\UCitem{Tipo}{Caso de uso primario}
		\UCitem{Observaciones}{	
			
			En el resumen.
			\begin{itemize}
				\item Falta de ortografía: ''Seleccionara'' a ''Seleccionará'' y ''asi'' a ''así''.
				\item Espaciar de manera adecuada después de las comas. 
				\item Verbos en futuro no describen lo que el caso de uso. 
				\item Se deben agregar elementos importantes para comprender el caso, como lo son definiciones puntuales y carácterísticas generales a considerar. YA LO HEMOS HABLADO. 
			\end{itemize}
			
			En el propósito.
			\begin{itemize}
				\item No se recomienda mencionar aspectos ''técnicos'', la base de datos puede ser uno de ellos. ¿Es ese realmente el propósito?
			\end{itemize}

			En las entradas.
			\begin{itemize}
				\item No corresponden a las solicitadas en la pantalla UI3 Vista.
			\end{itemize}

			En el origen.
			\begin{itemize}
				\item ¿Cuál es el origen de las entradas?
			\end{itemize}

			En las precondiciones y postcondiciones. 
			\begin{itemize}
				\item Debe de haber precondiciones y postcondiciones para modificar una oferta, por ejemplo, que haya una OFERTA o EMPRESA previamente REGISTRADA o bien, QUE PERSISTA LA INFORMACIÓN.
			\end{itemize}

			En los errores. 
			\begin{itemize}
				\item Link roto después de ''[...] continúa en el paso...''.
			\end{itemize}

			En la trayectoria principal. 
			\begin{itemize}
				\item Se recomienda describir la acción que el actor realiza en los pasos en donde él participa, y de escribir el verbo ''presiona'' para cuando la acción se realiza a través de un botón. Por ejemplo, en el paso 1: ''[el actor] solicita <acción en infinitivo> una nueva oferta presionando en el botón...'', corregir esos detalles en los actuales pasos 1, 3, 5 y 6.
				\item Se recomieda representar los botones con el comando de botones.
				\item Paso 1: De dónde o de quién presiona el botón.
				\item Pasos 1 y 2: ¿Dónde está el link a la pantalla? 
				\item ¿Podrá? Mejor dicho ''Introduce los campos a modificar'' o algo por el estilo y bien redactado. NO verbos en futuro, porque además les faltan acentos, verbos en presente. 
				\item Paso 4: enriquece la descipción. 
				\item No sé si los botones descritos existan en las interfaces, no las muestras.
				\item ¿No puede cancelar la operación? Lo puedes manejar como una trayectoria alternativa.
			\end{itemize}
		
		
 		}
		\UCitem{Autores}{Pérez García José David.}
		\UCitem{Reviso}{Huerta Martínez Jesús Manuel}
	\end{UseCase}
	\newpage
	
	\begin{UCtrayectoria}{Principal}
	\UCpaso[\UCactor] Pulsa el botón Modificar oferta de la pantalla principal.
	\UCpaso Muestra la pantalla registro de ofertas.
	\UCpaso[\UCactor] podrá modificar la empresa,ingresará horario,vacante, sueldo y seleccionara el contacto.
	\UCpaso[\UCactor] Da click en aceptar.
	\UCpaso Valida que los campos introducidos no estén vacios. [Error \ref{EM-WebEvento-CU04-E1}] \label{Almacenainfo}
	\UCpaso Obtiene la empresa, el horario ingresado, la vacante, sueldo y el contacto.
	\UCpaso Persiste la informacion en el sistema.
	\UCpaso Muestra el mensaje \MSGref{MSG1}{Operación Exitosa}
	
	\end{UCtrayectoria}

%-------------------------------------- TERMINA descripción del caso de uso.