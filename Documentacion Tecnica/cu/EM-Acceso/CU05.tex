	\begin{UseCase}{EM-Acceso-CU5}{Eliminar Cuenta}{
		\noindent
		Este caso de uso permite al actor eliminar su cuenta previamente creada en ESCOMobile, pues no desea
		continuar, permanentemente, con los servicios que ésta ofrece; porque la información ingresada es errónea,
		no deseaba crear una cuenta o bien, por la seguridad de su cuenta. Se debe mencionar que al eliminar
		la cuenta, los datos proporcionados serán eliminados, así como citas y demás información, y ésta no se
		se podrá recuperar ni se podrá reingresar a la cuenta desde el mismo u otro dispositivo.
		\newline
		}
		\UCitem{Versión}{0.1}
		\UCitem{Actor}{Alumno y Profesor.}
		\UCitem{Propósito}{Proporcionar al actor un mecanismo que le permita eliminar su cuenta, sin contar más
		con los servicios de la app y eliminando su información proporcionada permanentemente.}
		\UCitem{Entradas}{Ninguna.}
		\UCitem{Origen}{No aplica.}
		\UCitem{Salidas}{
			\begin{itemize}
				\item \MSGref{MSG18}{Eliminar cuenta}.
			\end{itemize}
		}
		\UCitem{Destino}{Pantalla.}
		\UCitem{Precondiciones}{Tener una sesión iniciada.}
		\UCitem{Postcondiciones}{Elimina toda la información que el actor introdujo para su registro, así
		como la información de citas asociada a su cuenta.}
		\UCitem{Errores}{Ninguno.}
		\UCitem{Tipo}{Caso de uso primario.}
		\UCitem{Observaciones}{Corregido.}
		\UCitem{Autores}{Pérez García José David, Huerta Martínez Jesús Manuel.}
		\UCitem{Reviso}{Huerta Martínez Jesús Manuel.}
	\end{UseCase}
	
	\begin{UCtrayectoria}{Principal.}

		% Paso 1.
		\UCpaso[\UCactor] Presiona el botón \IUbutton{Eliminar Cuenta} de la pantalla \IUref{EM-ESCOMobile-Hamburger}{}.

		% Paso 2.
		\UCpaso Muestra el mensaje \MSGref{MSG18}{Eliminar cuenta}, solicitando la confirmación para eliminar la cuenta.

		% Paso 3.
		\UCpaso [\UCactor] Presiona el botón \IUbutton{Aceptar} del mensaje mostrado. \Trayref{A}

		% Paso 4.
		\UCpaso Elimina la cuenta, la información proporcionada al hacer el registro y la referente a citas asociada a la cuenta, muestra la pantalla \IUref{Pantalla Inicio}{}.

	\end{UCtrayectoria}

\begin{UCtrayectoriaA}{A}{cancela.}

		% A1.
		\UCpaso	Presiona el botón \IUbutton{Cancelar} del mensaje mostrado.

		% A2.
		\UCpaso Muestra Pantalla anterior.

\end{UCtrayectoriaA}


%-------------------------------------- TERMINA descripción del caso de uso.