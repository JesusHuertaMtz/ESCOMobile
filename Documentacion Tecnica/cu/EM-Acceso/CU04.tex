
	\begin{UseCase}{EM-Acceso-CU4}{Cerrar Sesión.}{
		\noindent
		Este caso de uso permite al actor cerrar la sesión previamente iniciada en ESCOMobile, pues no desea
		continuar, temporalmente, con los servicios que ésta ofrece; porque desea iniciar sesión con otro usuario
		y contraseña; o bien, por la seguridad de su cuenta. Se debe mencionar que, al cerrar la sesión, se conservan
		los datos del usuario, así como citas y demás información, y se podrá reingresar a la cuenta desde el mismo
		u otro dispositivo en cualquier momento. 
		\newline
		}
		\UCitem{Versión}{0.2}
		\UCitem{Actor}{Alumno y Profesor.}
		\UCitem{Propósito}{Proporcionar al actor un mecanismo que le permita no contar con los servicios de la app
		sin perder su información dentro de la misma y con la opción de reingresar después.}
		\UCitem{Entradas}{Ninguna.}
		\UCitem{Origen}{No aplica.}
		\UCitem{Salidas}{
			\begin{itemize}
				\item \MSGref{MSG17}{Cerrar sesión}.
			\end{itemize}
		}
		\UCitem{Destino}{Pantalla.}
		\UCitem{Precondiciones}{Tener una sesión iniciada.}
		\UCitem{Postcondiciones}{Ninguna.}
		\UCitem{Errores}{Ninguno.}
		\UCitem{Tipo}{Caso de uso primario.}
		\UCitem{Observaciones}{Corregido.}
		\UCitem{Autores}{Pérez García José David, Huerta Martínez Jesús Manuel.}
		\UCitem{Reviso}{Huerta Martínez Jesús Manuel.}
	\end{UseCase}

	\begin{UCtrayectoria}{Principal.}

		% Paso 1.
		\UCpaso[\UCactor] Presiona el botón \IUbutton{Cerrar sesión} de la pantalla \IUref{EM-ESCOMobile-Hamburger}{}.

		% Paso 2.
		\UCpaso Muestra el mensaje \MSGref{MSG17}{Cerrar sesión}, solicitando la confirmación para cerrar la sesión.

		% Paso 3.
		\UCpaso [\UCactor] Presiona el botón \IUbutton{Aceptar} del mensaje mostrado. \Trayref{A}

		% Paso 4.
		\UCpaso Cierra la sesión y muestra la pantalla \IUref{EM-Acceso-UI2}{Iniciar Sesión}.

	\end{UCtrayectoria}

	\begin{UCtrayectoriaA}{A}{Cuando el actor no desea cerrar su sesión.}

		% A1.
		\UCpaso	Presiona el botón \IUbutton{Cancelar} del mensaje mostrado.

		% A2.
		\UCpaso Muestra Pantalla anterior.

	\end{UCtrayectoriaA}
