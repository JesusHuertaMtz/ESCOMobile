%\begin{UseCase}[archivo de imágen]{UCX}{Nombre del Caso de uso}{
	\begin{UseCase}{EM-Acceso-CU3}{Recuperar Contraseña.}{
		Este caso de uso permite al usuario recuperar la contraseña de su cuenta en el caso de haberla olvidado. Para ello
		es necesario introducir el correo electrónico que se proporcionó a la hora de crear la cuenta, al mismo se enviará
		una nueva contraseña provisional para acceder al sistema, con el objetivo de mantener la seguridad de los datos 
		personales del usuario, pues es éste quien tiene el control de su cuenta de correo electrónico. Dicha contraseña
		temporal se podrá modificar una vez que se ingrese al sistema.
		}
		\UCitem{Versión}{3.0}	
		\UCitem{Actor}{Alumno y Profesor.}
		\UCitem{Propósito}{Proporcionar al actor un mecanismo que le permita establecer una nueva contraseña en caso de
		haber olvidado la primera.}
		\UCitem{Entradas}{
			\begin{itemize}
				\item Correo electrónico.
			\end{itemize}
		}
		\UCitem{Origen}{Teclado.}
		\UCitem{Salidas}{
			\begin{itemize}
				\item \MSGref{MSG1}{Operación Exitosa}.
			\end{itemize}
		}
		\UCitem{Destino}{Pantalla-}
		\UCitem{Precondiciones}{Existir una cuenta asociada al correo electrónico cuya contraseña se desea restablecer.}
		\UCitem{Postcondiciones}{Persiste la nueva contraseña de la cuenta en el sistema.}
		\UCitem{Errores}{
			\begin{enumerate}[\hspace*{0.5cm} \bfseries{E}1:]	
				\item \label{EM-Acceso-CU03-E1} Cuando no se introdujeron todos los campos marcados como obligatorios. Muestra el mensaje \MSGref{MSG5}{Falta dato obligatorio} y \textbf{continúa en el paso \ref{l_Acceso_CU3_E1} de la trayectoria Principal.}
				\item \label{EM-Acceso-CU03-E2} Cuando no se introduce un correo electrónico existente. Muestra el mensaje \MSGref{MSG10}{Correo electrónico inexistente} y \textbf{termina el caso de uso.}
			\end{enumerate}	
		}
		\UCitem{Tipo}{Caso de uso primario}
		\UCitem{Observaciones}{
			Corregido. 
		}
		\UCitem{Autores}{Pérez García José David, Huerta Martínez Jesús Manuel.}
		\UCitem{Reviso}{Huerta Martínez Jesús Manuel.}
	\end{UseCase}

	\begin{UCtrayectoria}{Principal}

		% Paso 1. 
		\UCpaso[\UCactor] Presiona el botón \IUbutton{Olvidé mi contraseña} de la pantalla \IUref{EM-Acceso-UI2}{Iniciar Sesión}.

		% Paso 2.
		\UCpaso Muestra la pantalla \IUref{EM-Acceso-UI3}{Recuperar Contraseña}.

		% Paso 3.
		\UCpaso[\UCactor] Introduce el correo electrónico con el que creó con cuenta en el campo correo electrónico. \label{l_Acceso_CU3_E1}

		% Paso 4. 
		\UCpaso[\UCactor] Presiona botón \IUbutton{Enviar}.

		% Paso 5.
		\UCpaso Valida que los campos introducidos no estén vacíos. [Error \ref{EM-Acceso-CU03-E1}]

		% Paso 6.
		\UCpaso	 Válida que el correo se encuentre dado de alta en el sistema. [Error \ref{EM-Acceso-CU03-E2}] 

		% Paso 7.
		\UCpaso Genera una contraseña aleatoria de acuerdo a la regla de negocio \BRref{EM-RN-S004}{Formato de Contraseña}.

		% Paso 8.
		\UCpaso Envía la contraseña generada al correo electrónico introducido.

		% Paso 9.
	    \UCpaso Muestra el mensaje \MSGref{MSG1}{Operación Exitosa} en la pantalla \IUref{EM-Acceso-UI3}{Recuperar Contraseña}.

	    % Paso 10.
	    \UCpaso Regresa a la pantalla \IUref{EM-Acceso-UI2}{Iniciar Sesión}.
	
	\end{UCtrayectoria}



		
%-------------------------------------- TERMINA descripción del caso de uso.