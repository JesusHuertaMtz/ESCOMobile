% Copie este bloque por cada caso de uso:
%-------------------------------------- COMIENZA descripción del caso de uso.
%\begin{UseCase}[archivo de imágen]{UCX}{Nombre del Caso de uso}{
\begin{UseCase}{EM-Citas-CU1.4.1}{Eliminar citas Canceladas.}
	{
	\noindent
	Este caso de uso permite al actor eliminar de su histórico una cita en estado de ''cancelada'', es decir, una cita que se solicitó a un profesor y que por alguna razón se tuvo que cancelar antes de que s llevara a cabo. El registro que se lleva acerca de las citas canceladas es para brindar más información al actor acerca de su desempeño con los alumnos en este aspecto.
	\newline
	Es importante mencionar que, de ser eliminada una de estas citas, su información se pierde y no se puede recuperar. Sin embargo, ésto solo se ve reflejado en la copia de la cita e información asociada que el actor elimina, conservando una copia más en el perfil del otro agente que participó en la cita. Además, la eliminación de ésta última no afecta en nada las estadísticas generadas acerca de las mismas que las estadísticas de los profesores.
	\newline
	}
	\UCitem{Versión}{0.1}
	\UCitem{Actor}{Alumno, Profesor.}
	\UCitem{Propósito}{Proporcionar al actor un mecanismo que le permita eliminar de su cuenta el registro de una cita cancelada.}
	\UCitem{Entradas}{Ninguna.}
	\UCitem{Origen}{No aplica.}
	\UCitem{Salidas}{
		\begin{itemize}
			\item \MSGref{MSG1}{Operación Exitosa}.
			\item \MSGref{MSG2}{Operación Fallida}.
			\item \MSGref{MSG7}{Eliminar elemento}.
		\end{itemize}
	}
	\UCitem{Destino}{Pantalla.}
	\UCitem{Precondiciones}{Ninguna.}
	\UCitem{Postcondiciones}{Persiste la información en el sistema.}
	\UCitem{Errores}{
		\begin{enumerate}[\hspace*{0.5cm} \bfseries{E}1:]
			\item \label{EM-Cita-CU1-4-1-E1} Cuando no es posible eliminar una oferta cancelada, pues la operación falló, muestra el mensaje \MSGref{MSG2}{Operación Fallida} y \textbf{termina el caso de uso.}
		\end{enumerate}
	}
	\UCitem{Tipo}{Caso de uso secundario, viene de \UCref{EM-Citas-CU1.4}.}
	\UCitem{Observaciones}{}
	\UCitem{Autor}{Huerta Martínez Jesús Manuel.}
	\UCitem{Revisor}{Fernández Quiñones Isaac.}
	\UCitem{Estatus}{Corregido.}
\end{UseCase}

\begin{UCtrayectoria}{Principal}

	% Paso 1.
	\UCpaso [\UCactor] Solicita eliminar del registro una cita cancelada, presionando en el icono \UCicono{eliminar} de alguna de las citas de la pantalla \IUref{EM-Citas-UI1-4}{Consultar citas canceladas}.

	% Paso 2.
	\UCpaso Muestra el mensaje \MSGref{MSG7}{Eliminar elemento}, solicitando la confirmación para eliminar la cita seleccionada. 

	% Paso 3. 
	\UCpaso Presiona el botón \IUbutton{Aceptar} del mensaje anterior. \Trayref{A} 
 
	% Paso 4. 
	\UCpaso Elimina del registro del actor la copia de la cita elegida y su información asociada. 

	% Paso 5.
	\UCpaso Muestra la pantalla \IUref{EM-Citas-UI1-4}{Consultar citas canceladas} con el mensaje \MSGref{MSG1}{Operación Exitosa} y la información de las citas canceladas actualizada. [Error: \ref{EM-Cita-CU1-4-1-E1}.]
	

\end{UCtrayectoria}

\begin{UCtrayectoriaA}{A}{Cuando el actor no desea eliminar la cita cancelada.}

	% A1. 
	\UCpaso Presiona el botón \IUbutton{Cancelar} del mensaje \MSGref{MSG7}{Eliminar elemento}.

	% A2.
	\UCpaso Muestra la pantalla \IUref{EM-Citas-UI1-4}{Consultar citas canceladas} y \textbf{termina el caso de uso.}

\end{UCtrayectoriaA}