% Copie este bloque por cada caso de uso:
%-------------------------------------- COMIENZA descripción del caso de uso.
%\begin{UseCase}[archivo de imágen]{UCX}{Nombre del Caso de uso}{
\begin{UseCase}{EM-alumnoProfesor-CU1.5}{Solicitar Cita.}
	{Este caso de uso permite solicitar a un alumno agendar una cita con un profesor de la ESCOM para tratar temas
	académicos. La fecha, hora y el salón dónde se llevará a cabo la cita son propuestos por el alumno, tomando en cuenta los horarios de clase del profesor.
	La hora y salón pueden ser modificadas por el profesor siempre y cuando explique el motivo del cambio.}
	\UCitem{Versión}{0.1}
	\UCitem{Actor}{Alumno.}
	\UCitem{Propósito}{Proporcionar un mecanismo para que el alumno pueda solicitar una cita al profesor para tratar temas académicos.}
	\UCitem{Entradas}{
		\begin{itemize}
			\item Fecha  para agendar la cita.
			\item Hora  para comenzar la cita.
			\item Salón  para que tenga lugar la cita.
			\item Nombre del profesor con el que se agendará la cita.
			\item Motivo por el cual se requiere agendar la cita.
		\end{itemize}
	}
	\UCitem{Origen}{Teclado.}
	\UCitem{Salidas}{
		\begin{itemize}
			\item Fecha en la que se realizará la cita.
			\item Hora en la que comenzará la cita.
			\item Salón en el que tendrá lugar la cita.
			\item Asunto por el cual se llevará a cabo la cita.
			\item Nombre de la persona con la que se reunirá en la cita.
		\end{itemize}
	}
	\UCitem{Destino}{Pantalla.}
	\UCitem{Precondiciones}{
		\begin{itemize}
			\item El actor debe estar registrado.
			\item El actor debe haber iniciado sesión.
			\item Debe estar registrado el profesor con el que se requiere agendar la cita.
		\end{itemize}
	}
	\UCitem{Postcondiciones}{
		\begin{itemize}
			\item El sistema tendrá un nuevo registro asociado al alumno con la fecha, hora, asunto,
			salón y nombre del profesor con el que se agendará la cita.
			\item El nuevo registro de la cita estará en estado de no aceptado.
			\item El sistema notificará al profesor que tiene una nueva solicitud de cita.
		\end{itemize}			
	}
	\UCitem{Errores}{
	
		\begin{enumerate}[\hspace*{0.5cm} \bfseries{E}1:]	
		\item \label{EM-alumnoProfesor-CU1.5-E1} Cuando no se introducen todos los campos obligatorios, muestra el mensaje \MSGref{MSG5}{Falta dato obligatorio} y \textbf{Continúa en el paso \ref{EM-alumnoProfesor-CU1.5-DatosObligatorios} de la trayectoria principal}.
}
			Continua en el paso \ref{} de la trayectoria principal.
		\end{enumerate}
	}
	\UCitem{Tipo}{Primario}
	\UCitem{Observaciones}{}
	\UCitem{Autor}{José David Pérez García.}
	\UCitem{Revisor}{}
	\UCitem{Estatus}{Sin revisión.}
\end{UseCase}

\begin{UCtrayectoria}{Principal}
	\UCpaso[\UCactor] Solicita una cita presionando el botón \IUbutton{ Solicitar cita  } de la pantalla \IUref{EM-Alumno-UI1}{Consultar Perfil del Profesor}.
	\UCpaso Obtiene el nombre del profesor.
	\UCpaso Muestra pantalla \IUref{EM-Cita-IU09}{Crear cita II} con el nombre del profesor al que se le pedirá la cita y se mostrará un formulario a llenar.\label{EM-alumnoProfesor-CU1.5-DatosObligatorios}
	\UCpaso [\UCactor] Selecciona fecha y hora, en la cual desea la cita.
	\UCpaso  [\UCactor] Selecciona el tipo de cita.
	\UCpaso  [\UCactor] Selecciona el salón en el cual se realizará la cita.
	\UCpaso  [\UCactor] Selecciona si desea agregar una notificación.
	\UCpaso  [\UCactor] Escribe el motivo de la cita.
	\UCpaso  [\UCactor] Presiona el botón \IUbutton{ Agendar cita }.
	\UCpaso Valida que los campos del formulario no estén vacios, de acuedo a la regla de negocio \BRref{EM-RN-S002}{campos obligatorios}. [Error \ref{EM-alumnoProfesor-CU1.5-E1}]  \Trayref{A}
		\UCpaso Obtiene los datos de la cita introducidos en el formulario. 
		\UCpaso Persiste la información en el sistema.
	\UCpaso Muestra el mensaje \MSGref{MSG1}{Operación Exitosa} en la pantalla \IUref{EM-Alumno-UI1}{Consultar Perfil del Profesor}.


\end{UCtrayectoria}
		
%========Trayectorias alternativas

\begin{UCtrayectoriaA}{A}{Cuando el actor desea cancelar la operación.}
	\UCpaso [\UCactor] Presiona el botón \IUbutton{Cancelar} de la pantalla \IUref{EM-Cita-IU09}{Crear cita II}
	\UCpaso Muestra la pantalla \\IUref{EM-Alumno-UI1}{Consultar Perfil del Profesor.
\end{UCtrayectoriaA}

%========

%-------------------------------------- TERMINA descripción del caso de uso.
%%%%%%%%%%%%%%%%%%%%%%%%%%%%%%%%%%%%%%