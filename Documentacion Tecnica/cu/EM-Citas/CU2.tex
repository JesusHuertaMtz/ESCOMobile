% Copie este bloque por cada caso de uso:
%-------------------------------------- COMIENZA descripción del caso de uso.
%\begin{UseCase}[archivo de imágen]{UCX}{Nombre del Caso de uso}{
\begin{UseCase}{EM-Citas-CU2}{Agendar una cita}{
	\noindent
	Este caso de uso permite al actor aolicitar agendar una nueva cita con un profesor de ESCOM, pues requiere del apoyo de éste para solventar alguna situación académica surgida durante el curso, como lo puede ser dudas de algún tema, asesorías extra clase, tutorías escolares, presentaciones de proyectos, revisiones de los mismos, entregas de TT, entre otros.
	\newline
	Para el registro de nuevas citas, además del profesor con quien se desea llevar a cabo y de expresar un propósito académico a atender, se deben especificar por el actor el día y la hora en las que él prefiere que se realice la cita, pues es toda esta información importante para que el profesor pueda determinar si la solicitud es aceptada o no.
	\newline
	Se debe mencionar que el actor puede solicitar agender más de una cita con los profesores de ESCOM (siempre y cuando éstos últimos se encuentres regitrados en el sistema), y que la desición que los académicos tomen sobre la solicitud no afecta en nada los perfiles de los involucrados, o bien, el hecho de poder solicitar nuevamente citas entre los mismos.
	}
	\UCitem{Versión}{0.1}
	\UCitem{Actor}{Alumno, Profesor.}
	\UCitem{Propósito}{Proporcionar al actor un mecanismo que le permita estar al corrinte con las citas que tiene agendadas por medio de la consulta de éstas.}
	\UCitem{Entradas}{Ninguna.}
	\UCitem{Origen}{No aplica.}
	\UCitem{Salidas}{
		Se muestra la siguiente información acerca de las citas agendadas:
		\begin{itemize}
			\item Fecha y hora en las cuales la cita está programada.
			\item Tipo de cita (Asesoría, revisión de proyecto, revisión de TT, entrega de tarea, entrega de proyecto, revisión de protocolo, otro).
			\item Nombre del alumno quien solicitó la cita.
			\item Descripción (motivo) por la cual el alumno solicita la cita.
		\end{itemize}
	}
	\UCitem{Destino}{Pantalla.}
	\UCitem{Precondiciones}{Ninguna.}
	\UCitem{Postcondiciones}{Ninguna.}
	\UCitem{Errores}{
		\begin{enumerate}[\hspace*{0.5cm} \bfseries{E}1:]
			\item \label{EM-Cita-CU1-E1} Cuando no hay citas agendadas para consulta, muestra el mensaje \MSGref{MSG3}{Elementos No Disponibles} y \textbf{termina el caso de uso.}
		\end{enumerate}
	}
	\UCitem{Tipo}{Caso de uso primario.}
	\UCitem{Observaciones}{}
	\UCitem{Autor}{Huerta Martínez Jesús Manuel.}
	\UCitem{Revisor}{Fernández Quiñones Isaac.}
	\UCitem{Estatus}{Corregido.}
\end{UseCase}

\begin{UCtrayectoria}{Principal}

	% Paso 1.
	\UCpaso [\UCactor] Solicita consultar sus citas agendadas seleccionando la opción ''Mis citas'' de la pantalla \IUref{EM-ESCOMobile-Hamburger}{} o bien, presionando el botón \IUbutton{Consultar citas} de las pantallas \IUref{EM-Alumno-UI1}{Consultar Perfil del Alumno} o \IUref{EM-Profesor-UI1}{Consultar Perfil Propio}, según sea el caso.

	% Paso 2. 
	\UCpaso Valida que haya citas en estado ''agendada'' asociadas a la cuenta del actor. [Error: \ref{EM-Cita-CU1-E1}.]

	% Paso 3. 
	\UCpaso Obtiene las siguiente información de las citas ''agendadas'' asociadas a la cuenta del actor cuyas fechas y horas aún no transcurren: fecha y hora en las cuales la cita está programada, tipo de cita (Asesoría, revisión de proyecto, revisión de TT, entrega de tarea, entrega de proyecto, revisión de protocolo, otro), nombre del alumno quien solicitó la cita y descripción (motivo) por la cual el alumno solicita la cita.

	% Paso 4.
	\UCpaso Agrupa las citas obtenidas según la fecha programada para la cita.

	% Paso 5.
	\UCpaso Ordena las citas de cada grupo según la hora en la que se vayan a realizar. 

	% Paso 5.
	\UCpaso Ordena los grupos de citas según la fecha para la cual éstas están programadas.

	% Paso 6. 
	\UCpaso Muestra la pantalla \IUref{EM-Citas-UI1}{Consultar citas agendadas} con las citas agrupadas por fecha y hora de realizción. 

	% Paso 7.
	\UCpaso [\UCactor] Consulta sus citas agendadas. 

\end{UCtrayectoria}
