% Copie este bloque por cada caso de uso:
%-------------------------------------- COMIENZA descripción del caso de uso.
%\begin{UseCase}[archivo de imágen]{UCX}{Nombre del Caso de uso}{
\begin{UseCase}{EM-Cita-CU04}{Consultar citas agendadas.}
	{Permite listar las citas agendadas de manera ordenada comenzando por la cita próxima a realizarse 
	y así sucesivamente. Cada cita mostrará la fecha, hora, salón y el asunto por el cual se realizó.
	Con el fin de recordar a las partes, alumno y profesor, el lugar y la fecha en la que se llevará a cabo 	la cita o para saber si aún se tiene tiempo para agendar otra cita en un día determinado.}
	\UCitem{Versión}{0.1}
	\UCitem{Actor}{Alumno, Profesor.}
	\UCitem{Propósito}{Proporcionar la fecha, hora, salón y asunto de cada cita próxima a realizarse.}
	\UCitem{Entradas}{Ninguna.}
	\UCitem{Origen}{No aplica.}
	\UCitem{Salidas}{
		\begin{itemize}
			\item Fecha en la que se realizará la cita.
			\item Hora en la que comenzará la cita.
			\item Salón en el que tendrá lugar la cita.
			\item asunto por el cual se llevará a cabo la cita.
			\item Nombre de la persona con la que se reunirá en la cita.
		\end{itemize}
	}
	\UCitem{Destino}{Pantalla.}
	\UCitem{Precondiciones}{
		\begin{itemize}
			\item El actor debe estar registrado.
			\item El actor debe haber iniciado sesión.
			\item Debe estar registrado el profesor con el que se agendó la cita.
			\item Deben estar resgitrados en el sistema la información de la cita.
			\item El profesor debe haber aceptado la cita.
		\end{itemize}
	}
	\UCitem{Postcondiciones}{Ninguna. Solo se realizan consultas.}
	\UCitem{Errores}{
		\begin{enumerate}[\hspace*{0.5cm} \bfseries{E}1:]
			\item 
		\end{enumerate}
	}
	\UCitem{Tipo}{Primario}
	\UCitem{Observaciones}{}
%	Revisiones:
%		\begin{itemize}
%
%			\item Resumen: 
%				\begin{itemize} 
%					\item Complementar resumen. Esribir y dividir en párrafos contestando a lo siguiente: Por qué, para qué, efectos que se tienen y aspectos a considerar (términos y circustancias.)
%				\end{itemize}
%
%			\item Propósito: 
%				\begin{itemize} 
%					\item Describe lo mismo que el resumen. El propósito debe de escribir la razón por la cual el caso de uso existe, la razón por la cual el alumno lo necesita, por ejemplo: proporcionar al actor un mecanismo de búsqueda de salones y plantas en la ESCOM. Mejor planteado, claro.
%				\end{itemize}
%
%			\item Entradas: 
%				\begin{itemize} 
%					\item Se refiere a datos que necesita el sistema para transformarlos y crear salidas. Los introduce el usuario. En este caso, serían los pisos, pero esos ya están establecidos en la app, no las introduce el usuario como tal. Es una consulta, no hay entradas. Escribe "ninguna".
%				\end{itemize}
%
%			\item Errores: 
%				\begin{itemize} 
%					\item Revisar errores, éstos se realizan por fallos en las trayectorias, tienen un ID, una redacción y casi siempre se ligan con un mensaje.
%				\end{itemize}
%
%			\item Trayectoria principal: 
%				\begin{itemize} 
%					\item Comenta cada paso con una etiqueta 'Paso 1', 'paso 2', etc.
%					\item No es necesario escribir todo lo que sucede antes. Concéntrate en tu caso, éste comienza en el actual paso 4 .
%					\item Siempre inicia el actor, en este caso puede ser, 'El actor presiona el botón <continuar sin regitrarse> de la pantalla X'. Ese completo y bien redactado sería paso 1.
%					\item no es necesario escribir el actual paso 6 (carga los elemnetos para ejecutar la app).
%					\item Paso 8, en lugar de 'lee' puedes escribir 'obtiene'.
%					\item Pasos 9 y 10. Pueden hacerse 1 solo que describe muy general el proceso. Incluso pueden hacerse 1 con el paso 8.
%					\item Falta manejo de errores en verificaciones y cuando obtiene los datos. 
%				\end{itemize}		
%
%			\item Trayectoria alternativa A: 
%				\begin{itemize} 
%					\item Está descrica muy técnica. 
%					\item Podría manejarse como error. 
%					\item Paso A4, no continúa en el paso 8 del CU, sino en el paso 8 de la trayectoria principal, basta con escribir eso. 
%				\end{itemize}	
%
%		\end{itemize}
%	 }
%	\UCitem{Observaciones (continuación)}{
%		\begin{itemize}
%
%			\item Trayectoria alternativa B: 
%				\begin{itemize} 
%					\item No es como tal una trayectoria, porque eso hace referencia a que se ejecuta únicamente en el paso X de la trayectoria principal, y eso no es verdad, el botón 'retroceso' puede ser oprimido en cualquier momento. No solo en X paso de la trayectoria principal. 
%					\item No es necesario paso B2.
%				\end{itemize}	
%
%			\item Trayectoria alternativa C: 
%				\begin{itemize} 
%					\item La trayectoria C se engloba en el último paso de la trayectoria principal, pues como default el mapa muestra ese piso, de ahí, y como trayectorias alternativas sí tendríamos la D y la E (Con otros nombres, claro).
%				\end{itemize}
%		
%			\item Trayectoria alternativa D: 
%				\begin{itemize} 
%					\item A qué pantalla pertenece el botón 'P1'. 
%				\end{itemize}
%
%			\item Trayectoria alternativa E: 
%				\begin{itemize} 
%					\item A qué pantalla pertenece el botón 'P2'. 
%				\end{itemize}
%
%		\end{itemize}
%	}	
	\UCitem{Autor}{Fernández Quiñones Isaac.}
	\UCitem{Revisor}{}
	\UCitem{Estatus}{Sin revisión.}
\end{UseCase}

\begin{UCtrayectoria}{Principal}
	\UCpaso[\UCactor]Realizó los pasos del Login.
	\UCpaso Muestra la pantalla de inicio de alumno o profesor dependiendo del tipo de cuenta.
	\UCpaso[\UCactor] Solicita deslizando el dedo sobre la pantalla de izquierda a derecha o presionando 
	el icono \UCicono{hamburger} si se encuentra disponible en la parte superior izquierda de la pantalla.
	\UCpaso Muestra la pantalla \IUref{EM-Menu1}{Menú hamburguer}.
	\UCpaso[\UCactor] Presiona la opción \IUbutton{ Citas } del menú.
	\UCpaso Verifica que existan citas agendadas por el actor según \BRref{EM-RN-N005}{Citas agendadas}. \Trayref{A} 
	\UCpaso Obtiene la fecha, hora, salón y asunto de cada una de las citas agendadas por el actor.
	\UCpaso Ordena las citas por fecha y hora, comenzando por la próxima cita a realizarse.
	\UCpaso Crea una lista con las citas previamente ordenadas.
	\UCpaso Muestra la pantalla \IUref{EM-Cita-IU02}{Consultar citas II}
	\UCpaso[\UCactor] Consulta las citas agendadas. %[Error \ref{EM-Cita-CU01-E1}]
\end{UCtrayectoria}
		
%========Trayectorias alternativas

\begin{UCtrayectoriaA}{A}{El actor no tiene ninguna cita agendada.}
	\UCpaso Muestra la pantalla \IUref{EM-Cita-UI01}{Consultar citas I}
	\UCpaso termina el caso de uso.
\end{UCtrayectoriaA}

%========

%-------------------------------------- TERMINA descripción del caso de uso.
%%%%%%%%%%%%%%%%%%%%%%%%%%%%%%%%%%%%%%