% Copie este bloque por cada caso de uso:
%-------------------------------------- COMIENZA descripción del caso de uso.
%\begin{UseCase}[archivo de imágen]{UCX}{Nombre del Caso de uso}{
\begin{UseCase}{EM-Citas-CU1}{Consultar citas Agendadas.}
	{
	\noindent
	Este caso de uso permite al actor consultar de manera ordenada todas las citas en estado de ''agendadas'' que tiene asociadas a su cuenta, esto es, las próximas citas ya aceptadas para realizarse, pues es de esta manera que los profesores pueden brindar su apoyo los estudiantes de ESCOM para mejorar su aprendizaje, ya que se pueden presentar dudas o inquietudes con las clases impartidas en los salones de la institución, o bien, se puede requerir asistencia académica de los profesores para los alumnos en diversos temas, como lo son proyectos, asesorías, trabajos terminales, etc. 
	\newline
	Para que una cita se considere ''agendada'' debe ser solicitada por un alumno para llevarse a cabo en ciertas condiciones y aceptada por el profesor a quien se solicitó la misma. Estas citas pueden ser canceladas (si una situación dada así lo amerita). De ser cancelada una cita, ésta deja de ser una cita ''agendada'' y pasa a ser una cita ''cancelada'', enviando una notificación al alumno o el profesor (según sea el caso) al respecto. Asimismo, se notifica al profesor en caso de que un alumno le solicite una cita y se lo notifica a este último si su petición es aceptada o rechazada. 
	\newline
	}
	\UCitem{Versión}{0.1}
	\UCitem{Actor}{Alumno, Profesor.}
	\UCitem{Propósito}{Proporcionar al actor un mecanismo que le permita estar al corriente con las citas que tiene agendadas por medio de la consulta de éstas.}
	\UCitem{Entradas}{Ninguna.}
	\UCitem{Origen}{No aplica.}
	\UCitem{Salidas}{
		Se muestra la siguiente información acerca de las citas agendadas:
		\begin{itemize}
			\item Fecha y hora en las cuales la cita está programada.
			\item Tipo de cita (Asesoría, revisión de proyecto, revisión de TT, entrega de tarea, entrega de proyecto, revisión de protocolo, otro).
			\item Nombre del alumno quien solicitó la cita.
			\item Descripción (motivo) por la cual el alumno solicita la cita.
		\end{itemize}
	}
	\UCitem{Destino}{Pantalla.}
	\UCitem{Precondiciones}{Ninguna.}
	\UCitem{Postcondiciones}{Ninguna.}
	\UCitem{Errores}{
		\begin{enumerate}[\hspace*{0.5cm} \bfseries{E}1:]
			\item \label{EM-Cita-CU1-E1} Cuando no hay citas agendadas para consulta, muestra el mensaje \MSGref{MSG3}{Elementos No Disponibles} y \textbf{termina el caso de uso.}
		\end{enumerate}
	}
	\UCitem{Tipo}{Caso de uso primario.}
	\UCitem{Observaciones}{}
	\UCitem{Autor}{Huerta Martínez Jesús Manuel.}
	\UCitem{Revisor}{Fernández Quiñones Isaac.}
	\UCitem{Estatus}{Corregido.}
\end{UseCase}

\begin{UCtrayectoria}{Principal}

	% Paso 1.
	\UCpaso [\UCactor] Solicita consultar sus citas agendadas seleccionando la opción ''Mis citas'' de la pantalla \IUref{EM-ESCOMobile-Hamburger}{} o bien, presionando el botón \IUbutton{Consultar citas} de las pantallas \IUref{EM-Alumno-UI1}{Consultar Perfil del Alumno} o \IUref{EM-Profesor-UI1}{Consultar Perfil Propio}, según sea el caso.

	% Paso 2. 
	\UCpaso Valida que haya citas en estado ''agendada'' asociadas a la cuenta del actor. [Error: \ref{EM-Cita-CU1-E1}.]

	% Paso 3. 
	\UCpaso Obtiene la siguiente información de las citas ''agendadas'' asociadas a la cuenta del actor cuyas fechas y horas aún no transcurren: fecha y hora en las cuales la cita está programada, tipo de cita (Asesoría, revisión de proyecto, revisión de TT, entrega de tarea, entrega de proyecto, revisión de protocolo, otro), nombre del alumno quien solicitó la cita y descripción (motivo) por la cual el alumno solicita la cita.

	% Paso 4.
	\UCpaso Agrupa las citas obtenidas según la fecha programada para la cita.

	% Paso 5.
	\UCpaso Ordena las citas de cada grupo según la hora en la que se vayan a realizar. 

	% Paso 5.
	\UCpaso Ordena los grupos de citas según la fecha para la cual éstas están programadas.

	% Paso 6. 
	\UCpaso Muestra la pantalla \IUref{EM-Citas-UI1}{Consultar citas agendadas} con las citas agrupadas por fecha y hora de realización. 

	% Paso 7.
	\UCpaso [\UCactor] Consulta sus citas agendadas. 

\end{UCtrayectoria}
