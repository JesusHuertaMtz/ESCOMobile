% Copie este bloque por cada caso de uso:
%-------------------------------------- COMIENZA descripción del caso de uso.
%\begin{UseCase}[archivo de imágen]{UCX}{Nombre del Caso de uso}{
\begin{UseCase}{EM-BolsaWeb-CU4}{Gestionar Empresas}{
	\noindent
	Este caso de uso permite al actor gestionar las empresas registradas en la bolsa de trabajo de ESCOMobile y la información que a éstas se asocia, dando además un punto de acceso al registro, edición, consulta y eliminación de las mismas. Pues son las empresas quienes ofertarán trabajo a los alumnos de la ESCOM interesados en encontrar empleo.
	\newpage
	}
	\UCitem{Versión}{0.1}
	\UCitem{Actor}{Encargado de Bolsa de Trabajo.}
	\UCitem{Propósito}{Proporcionar al actor acceso al registro, edición, consulta y eliminación de empresas en el sistema.}
	\UCitem{Entradas}{
		Ninguna.
	}
	\UCitem{Origen}{
		No aplica.
	}
	\UCitem{Salidas}{
		\begin{itemize}
			\item Número de empresas registradas.
			\item Tabla que muestra la siguiente información de las empresas registradas
			\begin{itemize}
				\item Nombre de la Empresa.
				\item Giro.
				\item Contacto.
				\item RFC.
				\item Número de ofertas publicadas.
				\item Número de ofertas visualizadas. 
			\end{itemize}
			\end{itemize}
	}
	\UCitem{Destino}{
		Pantalla.
	}
	\UCitem{Precondiciones}{
		Ninguna.
	}
	\UCitem{Postcondiciones}{
		Ninguna.
	}
	\UCitem{Errores}{
		Ninguno. 
	}
	\UCitem{Tipo}{Primario}
	\UCitem{Observaciones}{ Es la primera propuesta de este caso de uso, se espera que se revise para implementar las correciones adecuadas.}
	\UCitem{Autor}{ Huerta Matínez Jesús Manuel.}
	\UCitem{Revisor}{}
	\UCitem{Estatus}{ Sin revisión.}
\end{UseCase}

\begin{UCtrayectoria}{Principal}

	%Paso 1.
	\UCpaso [\UCactor] Solicita gestionar las empresas de ESCOMobile Bolsa dando clic en la opción ''Empresas'' de la pantalla \IUref{EM-BolsaWeb-UI1}{Gestionar Ofertas de Trabajo}.

	%Paso 2.
	\UCpaso Verifica que haya empresas registradas. \Trayref{A} 

	%Paso 3.
	\UCpaso Obtiene nombre, giro y RFC de las empresas registradas, así como el número total de empresas registradas.

	%Paso 4.
	\UCpaso Obtiene el nombre de contacto, los tipos de contacto y los valores de contacto asociados a cada una de las empresas previamente obtenidas.

	%Paso 5.
	\UCpaso Obtiene las ofertas con estado de ''publicadas'' y ''visualizadas'' asociadas a cada una de las empresas previamente obtenidas.

	%Paso 6.
	\UCpaso Muestra la pantalla \IUref{EM-BolsaWeb-UI4}{Gestionar Empresas} con el número de empresas registradas, así como una tabla que contiene: Nombre de la Empresa, Giro, Contacto, RFC, Número de ofertas publicadas, Número de ofertas visualizadas, así como las acciones editar (\UCicono{lapiz.png}) y eliminar (\UCicono{eliminar.png}) habilitadas para las empresas registradas en el sistema. \label{EM-BolsaWeb-CU4-TAA}

	%Paso 7.
	\UCpaso [\UCactor] Consulta las empresas registradas en el sistema y los contactos que a éstas se asocian. 

	%Paso 8. 
	\UCpaso [\UCactor] Gestiona las empresas por medio del botón \IUbutton{Añadir empresa} para registrar una nueva empresa en el sistema o bien, de las acciones \UCicono{lapiz.png} para editar alguna de las empresas registradas y \UCicono{eliminar.png} para eliminar alguna de las empresas registradas.

\end{UCtrayectoria}

%Trayectoria Alternativa A.
\begin{UCtrayectoriaA}{A}{Cuando no hay empresas registradas en el sistema.}
	%Paso A1.
	\UCpaso Establece el número de empresas registradas igual a 0. 

	\UCpaso Regresa al paso \ref{EM-BolsaWeb-CU4-TAA} de la trayectoria principal.
\end{UCtrayectoriaA}



%-------------------------------------- TERMINA descripción del caso de uso.
%%%%%%%%%%%%%%%%%%%%%%%%%%%%%%%%%%%%%%