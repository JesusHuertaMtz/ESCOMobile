% Copie este bloque por cada caso de uso:
%-------------------------------------- COMIENZA descripción del caso de uso.
%\begin{UseCase}[archivo de imágen]{UCX}{Nombre del Caso de uso}{
\begin{UseCase}{EM-BolsaWeb-CU1}{Gestionar Ofertas de Trabajo}{
	\noindent
	Este caso de uso permite al actor gestionar las ofertas de trabajo brindadas por las empresas registradas en la bolsa de trabajo de ESCOMobile, así como la información que a éstas se asocia para consulta a los alumnos de ESCOM e interesados en aplicar a alguno de los puestos que se ofertan. Se cuenta también con un punto de acceso al registro, edición, consulta y eliminación de las ofertas previamente registradas. Pues es así que se logra tener siempre actualizada la información para quienes a ella acceden desde la app móvil de ESCOMobile, siendo entre la bolsa Web y la antes mencionada un sistema en conjunto y complementado. 
	\newline
	}
	\UCitem{Versión}{0.1}
	\UCitem{Actor}{Encargado dpto. de Extensión y servisios eductativos de la ESCOM.}
	\UCitem{Propósito}{Proporcionar al actor acceso al registro, edición, consulta, eliminación y publicación de ofertas de Trabajo de la bolsa de trabajo ESCOMobile en el sistema.}
	\UCitem{Entradas}{Ninguna.}
	\UCitem{Origen}{No aplica.}
	\UCitem{Salidas}{
		\begin{itemize}
			\item Número de ofertas de trabajo publicadas para los alumnos de la ESCOM.
			\item Gráfica de linea mostrando el número de ofertas de los ultimos 6 meses.
			\item Tabla que muestra la siguiente información de las ofertas:
				\begin{itemize}
         		\item Número de Oferta.
				\item Nombre de la empresa.
				\item Horario.
				\item Vacante.
				\item Sueldo.
				\item Contacto.
				\item Fecha Publicación.
				\end{itemize}	
			\end{itemize}
	}
	\UCitem{Destino}{Pantalla.}
	\UCitem{Precondiciones}{Haber inicido sesión en el sistema por medio de Facebook.}
	\UCitem{Postcondiciones}{Ninguna.}
	\UCitem{Errores}{Ninguno. }
	\UCitem{Tipo}{Caso de uso primario.}
	\UCitem{Observaciones}{}
	\UCitem{Autor}{Huerta Martínez Jesús Manuel.}
	\UCitem{Revisor}{Fernández Quiñones Isaac.}
	\UCitem{Estatus}{Corregido.}
\end{UseCase}

\begin{UCtrayectoria}{Principal}

	%Paso 1.
	\UCpaso [\UCactor] Solicita gestionar las ofertas de ESCOMobile Bolsa presionando en la opción ''Ofertas laborales'' de la pantalla \IUref{EM-BolsaWeb-UI1}{Gestionar Ofertas de Trabajo}

	%Paso 2.
	\UCpaso Verifica que existan ofertas laborales en estado para los alumnos de ESCOM registradas. \Trayref{A} 

	%Paso 3.
	\UCpaso Obtiene las ofertas de trabajo con estado de ''Sin publicar'' y ''Publicadas''.

	%Paso 4.
	\UCpaso Calcula el número de ofertas Publicadas según la regla de negocio \BRref{EM-RN-N002}{Total de Ofertas de Trabajo publicadas}.

	%Paso 5.
	\UCpaso Calcula el número de ofertas publicadas en el sistema de cada mes por los ultimos 6 meses según la regla de Negocio.

	%Paso 6.
	\UCpaso Obtiene nombre de las empresas, horario, vacante, sueldo y contacto de las ofertas de trabajo registradas con el estado de "Sin Publicar".
	
	%Paso 7.	
	\UCpaso Realiza una grafica de lineas con la información obtenida de las ofertas de trabajo publicadas en los ultimos 6 meses.
	
	%Paso 8.	
	\UCpaso Muestra la pantalla \IUref{EM-BolsaWeb-UI1}{Gestionar Ofertas de Trabajo} con la gráfica de lineas, en el eje X el mes y en el eje Y el número de ofertas de trabajo con los datos obtenidos. El número de Ofertas de trabajo publicadas en total, así como una tabla que contiene la siguiente información de la oferta de trabajo: Número, Nombre de la Empresa, Horario, Vacante, Sueldo y contacto; además las acciones editar (\UCicono{lapiz.png}) y eliminar (\UCicono{eliminar.png}) habilitadas para las ofertas registradas en el sistema. \label{EM-BolsaWeb-CU1-TAA}

	%Paso 9.
	\UCpaso [\UCactor] Gestiona las ofertas por medio de las siguientes acciones: 
	\begin{itemize}
		\item Botón \IUbutton{Añadir Oferta}: para registrar una nueva oferta de trabajo en el sistema.
		\item Icono \UCicono{lapiz.png}: para editar alguna de las Ofertas de trabajo registradas.
		\item Icono \UCicono{eliminar.png}: para eliminar alguna de las Ofertas de trabajo registradas.
	\end{itemize}

\end{UCtrayectoria}

%Trayectoria Alternativa A.
\begin{UCtrayectoriaA}{A}{Cuando no hay Ofertas de trabajo Publicadas.}
	
	%Paso A1.
	\UCpaso Establece el número de Ofertas de trabajo publicadas igual a 0. 

	%Paso A2.
	\UCpaso Regresa al paso \ref{EM-BolsaWeb-CU1-TAA} de la trayectoria principal.
\end{UCtrayectoriaA}

\subsection{Puntos de extensión}

\UCExtensionPoint{Solicitar Añadir Oferta} {El actor desea añadir una oferta de trabajo} {Paso 8 de la Trayectoria Principal}{referancia a Caso de uso Añadir Oferta de Trabajo}


\UCExtensionPoint{Modificar Oferta}{El actor desea modificar una oferta de trabajo}{Paso 8 la Trayectoria Principal} {referencia a caso de uso Editar Oferta de trabajo}

\UCExtensionPoint{Eliminar Oferta}{El actor desea eliminar una oferta de trabajo}{Paso 8 la Trayectoria Principal} {referencia a caso de uso Eliminar Oferta de Trabajo}

\UCExtensionPoint{Publicar Boletín}{El actor desea publicar las ofertas de trabajo}{Paso 8 la Trayectoria Principal} {referencia a caso de uso Publicar Boletín de Ofertas de Trabajo}
%-------------------------------------- TERMINA descripción del caso de uso.
%%%%%%%%%%%%%%%%%%%%%%%%%%%%%%%%%%%%%%