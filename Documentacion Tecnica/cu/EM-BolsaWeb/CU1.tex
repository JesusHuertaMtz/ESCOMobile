% Copie este bloque por cada caso de uso:
%-------------------------------------- COMIENZA descripción del caso de uso.
%\begin{UseCase}[archivo de imágen]{UCX}{Nombre del Caso de uso}{
\begin{UseCase}{EM-BolsaWeb-CU1}{Gestionar Ofertas de Trabajo}{
Este caso de uso permite al actor subir y editar las ofertas de trabajo para los alumnos de la ESCOM en la bolsa de trabajo de ESCOMobile, proporcionando el acceso al registro, edición, consulta, eliminación y publicación de las ofertas antes mencionadas en el sistema, ya que éste es el medio por el cual dichos alumnos estarán informados de las ofertas.
	}
	\UCitem{Versión}{0.1}
	\UCitem{Actor}{ Encargado de Bolsa de Trabajo.}
	\UCitem{Propósito}{ Proporcionar al actor acceso al registro, edición, consulta , eliminación y publicación de ofertas de Trabajo para los alumnos de ESCOM de la bolsa de trabajo ESCOMobile en el sistema.}
	\UCitem{Entradas}{
		Ninguna.
	}
	\UCitem{Origen}{
		No aplica.
	}
	\UCitem{Salidas}{
		\begin{itemize}
			\item Número de ofertas de trabajo para los alumnos de la ESCOM publicadas.
			\item Gráfica de linea mostrando el número de ofertas de los ultimos 6 meses.
			\item Tabla que muestra la siguiente información de las ofertas sin publicar
				\begin{itemize}
         		\item Numero de Oferta.
				\item Nombre de la empresa.
				\item Horario.
				\item Vacante.
				\item Sueldo.
				\item Contacto.
				\item Fecha Publicación.
				\end{itemize}	
			\end{itemize}
	}
	\UCitem{Destino}{
		Pantalla.
	}
	\UCitem{Precondiciones}{
		Ninguna.
	}
	\UCitem{Postcondiciones}{
		Ninguna.
	}
	\UCitem{Errores}{
		Ninguno. 
	}
	\UCitem{Tipo}{ Primario}
	\UCitem{Observaciones}{ 
		\begin{itemize}
			\item NO identar texto en los párrafos del documento. 
			\item Descripción completa:

					1) Resumen poco comprensible al inicio. 
					2) En el renglón 3, ''Ofertas'' => ''Ofertas de Trabajo'', siempre ser MUY especifico. Ofertas de quién, ¿de mi abuelita? NO, ofertas de trabajo. Este error está dos veces. 
					3) En el tercer renglón, ''Este'' => ''Éste''. Revisa cuándo acentuar esas palabras y cómo. 
			\item Atributos importantes.

					1) Propósito: Ofertas de quién, ¿de mi abuelita?

					2) Salidas: En el primer punto, Ofertas de quién, ¿de mi abuelita?, además, no son ofertas de trabajo registradas, son ofertas de trabajo PUBLICADAS. En la tabla, ¿qué es ''no''? 
			\item Trayectoria principal:

					1) En el paso 2: Ofertas de quién, ¿de mi abuelita? 

					2) Paso 3: ¿Obtiene las ofertas con el estado ''sin publicar'' y en el paso 4 dices que son la de estado ''publicadas''?

					3) Paso 5: Ofertas de quién, ¿de mi abuelita? Faltan datos mostrados en la tabla. Siguen incosistencias entre las ofertas publicadas y no publicadas. Cuándo calculaste la gráfica para poder mostrarla, no salió de la nada. 

					4) Paso 6: En el paso 3 de la TP dices que las ofertas sin publicar se van a trabajar como una trayectoria alternativa (la TA B, que ni siquiera redactaste ¿qué pasó ahí?), y en el paso 6 dices que es por un punto de extensión, es decir, otro caso de uso, o es TA o es PE. En cualquiera de los casos REDACTA.

					5) En general. Ofertas de quién, ¿de mi abuelita? 

					6) Puntos de extensión. Los puntos de extensión se llaman igual a los CU a los cuales hacen referencias. Así no se llaman.
			\item En la descripción de la pantalla WebBolsa-UI3 Vista:

					1) Describir bien los atributos del diseño / salida. Y Ofertas de quién, ¿de mi abuelita? xD
		\end{itemize}
	}
	\UCitem{Autor}{ Pérez García José David.}
	\UCitem{Revisor}{}
	\UCitem{Estatus}{ Sin revisión.}
\end{UseCase}

\begin{UCtrayectoria}{Principal}
	%Paso 1.
	\UCpaso [\UCactor] Solicita gestionar las ofertas de ESCOMobile Bolsa presionando en la opción ''Ofertas laborales'' de la pantalla \IUref{EM-BolsaWeb-UI4}{Gestionar Empresas}.

	%Paso 2.
	\UCpaso  Verifica que existan ofertas laborales para los alumnos de ESCOM registradas. \Trayref{A} 

	%Paso 3.
	\UCpaso Obtiene las ofertas con estado de '' Sin publicar'' y ''Publicadas'' 
	%Paso 4.
	\UCpaso Calcula el número de ofertas Publicadas según la regla de negocio \BRref{EM-RN-S010}{Total de Ofertas de Trabajo publicadas.} .
%Paso 5.
	\UCpaso Calcula el número de ofertas publicadas en el sistema de cada mes por los ultimos 6 meses según la regla de Negocio \BRref{EM-RN-N006}.
	%Paso 6.
	\UCpaso Obtiene nombre de las empresas, horario, vacante, sueldo y contacto de las ofertas de trabajo registradas con el estado de "Sin Publicar".
	
		%Paso 7.
		
		
		\UCpaso Realiza una grafica de lineas con la información obtenida de las ofertas de trabajo publicadas en los ultimos 6 meses.
		
	\UCpaso Muestra la pantalla \IUref{EM-BolsaWeb-UI1}{Gestionar Ofertas de Trabajo} con la gráfica de lineas, en el eje X el mes y en el eje Y el número de ofertas de trabajo con los datos obtenidos. El número de Ofertas de trabajo publicadas en total, así como una tabla que contiene:Selección,Número de oferta, Nombre de la Empresa, Horario, Vacante, Sueldo y contacto, así como las acciones editar (\UCicono{lapiz.png}) y eliminar( \UCicono{eliminar.png}) habilitadas para las ofertas registradas en el sistema. \label{EM-BolsaWeb-CU1-TAA}  \Trayref{B}

%Paso 9.

	\UCpaso Gestiona las ofertas por medio del botón \IUbutton{Añadir Oferta} para registrar una nueva oferta de trabajo para los alumnos de ESCOM en el sistema o bien, de las acciones \IUbutton{Ofertas Publicadas} para visualizar y gestionar las ofertas de trabajo publicadas, \UCicono{lapiz.png} para editar alguna de las Ofertas de trabajo registradas y \UCicono{eliminar.png} para eliminar alguna de las Ofertas de trabajo registradas.

\end{UCtrayectoria}

%Trayectoria Alternativa A.
\begin{UCtrayectoriaA}{A}{Cuando no hay Ofertas Publicadas.}
	%Paso A1.
	\UCpaso Establece el número de Ofertas de trabajo publicadas igual a 0. 

	\UCpaso Regresa al paso \ref{EM-BolsaWeb-CU1-TAA} de la trayectoria principal.
\end{UCtrayectoriaA}

\begin{UCtrayectoriaB}{B}{Consultar ofertas Publicadas.}
	\UCpaso[\UCactor] Presiona el \IUbutton{Ofertas publicadas}.
	\UCpaso Verifica que existan ofertas  de trabajo registradas.
	\UCpaso Busca los nombres de las empresas. 
	\UCpaso Obtiene la lista de ofertas registradas.
	\UCpaso Separa las ofertas  de trabajo publicadas.	
	\UCpaso Muestra en la pantalla  \IUref {EM-BolsaWeb-UI1}{Gestionar Ofertas de Trabajo} una tabla de Ofertas publicadas la cual que tiene los siguiente atributos:
	\begin{itemize}
	\item No.
	\item Nombre de la empresa.
	\item Horario.
	\item Vacante.
	\item Sueldo.
	\item Contacto.
	\item Fecha Publicación.
	\end{itemize}
	\UCpaso[\UCactor] Consulta las ofertas de trabajo obtenidas.
	\UCpaso Fin caso de uso.
\end{UCtrayectoriaB}



\subsection{Puntos de extensión}

\UCExtensionPoint{Solicitar Añadir Oferta} {El actor desea añadir una oferta de trabajo} {Paso 8 de la Trayectoria Principal}{referancia a Caso de uso Añadir Oferta de Trabajo}


\UCExtensionPoint{Modificar Oferta}{El actor desea modificar una oferta de trabajo}{Paso 8 la Trayectoria Principal} {referencia a caso de uso Editar Oferta de trabajo}

\UCExtensionPoint{Eliminar Oferta}{El actor desea eliminar una oferta de trabajo}{Paso 8 la Trayectoria Principal} {referencia a caso de uso Eliminar Oferta de Trabajo}

\UCExtensionPoint{Publicar Boletín}{El actor desea publicar las ofertas de trabajo}{Paso 8 la Trayectoria Principal} {referencia a caso de uso Publicar Boletín de Ofertas de Trabajo}
%-------------------------------------- TERMINA descripción del caso de uso.
%%%%%%%%%%%%%%%%%%%%%%%%%%%%%%%%%%%%%%