%\begin{UseCase}[archivo de imágen]{UCX}{Nombre del Caso de uso}{
	\begin{UseCase}{EM-BolsaWeb-CU1.2}{Eliminar oferta de trabajo.}{
		\noindent
		Este caso de uso permite al actor eliminar una oferta de trabajo previamente registrada en el sistema, pues fue equivocadamente registrada o ya no se desea contar con la información de la misma. 
		\newline
		Es importante mencionar que solo se podrán eliminar aquellas ofertas que se encuentren en estado de ''Sin publicar'' pues una vez publicadas estarán disponibles para consulta desde la app ESCOMobile, imposibilitando así su eliminación. 
		\newline
		}
		\UCitem{Versión}{0.1}
		\UCitem{Actor}{Encargado dpto. de Extensión y servicios educativos de la ESCOM.}
		\UCitem{Propósito}{Proporcionar al actor un mecanismo que le permita eliminar ofertas de trabajo previamente registradas en el sistema.}
		\UCitem{Entradas}{Ninguno.}
		\UCitem{Origen}{No aplica.}
		\UCitem{Salidas}{
			\begin{itemize}
				\item \MSGref{MSG1}{Operación Exitosa}.
				\item \MSGref{MSG7}{Eliminar elemento}.
			\end{itemize}
		}
		\UCitem{Destino}{Pantalla}
		\UCitem{Precondiciones}{
			\begin{itemize}
				\item Que exista al menos una empresa registrada en el sistema.
				\item Que exista al menos una oferta de trabajo en estado de ''Sin publicar'' registrada en el sistema.
			\end{itemize}
		}
		\UCitem{Postcondiciones}{Elimina del sistema la información de la oferta de trabajo seleccionada.}
		\UCitem{Errores}{Ninguno.}
		\UCitem{Tipo}{Caso de uso secundario, viene de \UCref{EM-BolsaWeb-CU1}.}
		\UCitem{Observaciones}{}
		\UCitem{Autores}{Huerta Martínez Jesús Manuel.}
		\UCitem{Reviso}{Fernández Quiñones Isaac.}
	\end{UseCase}
	
	\begin{UCtrayectoria}{Principal}

		% Paso 1.
		\UCpaso[\UCactor] Solicita eliminar una oferta de trabajo en estado ''Sin publicar'' dando clic en el icono \UCicono{eliminar} de la pantalla \IUref{EM-BolsaWeb-UI1}{Gestionar Ofertas de Trabajo}.

		% Paso 2.
		\UCpaso Muestra el mensaje \MSGref{MSG7}{Eliminar elemento}, solicitando la confirmación para eliminar la información de la oferta de trabajo.

		% Paso 3. 
		\UCpaso [\UCactor] Presiona el botón \IUbutton{Aceptar} del mensaje previamente mostrado.

		% Paso 4.
		\UCpaso Elimina del sistema la información de la oferta de trabajo seleccionada, esto es: nombre de la Empresa asociada a la oferta, tipo Horario, horario, nombre del puesto, número de vacantes, requisitos, idiomas, perfil, sueldo, prestaciones.

		% Paso 5. 
		\UCpaso Muestra el mensaje \MSGref{MSG1}{Operación Exitosa} en la pantalla \IUref{EM-BolsaWeb-UI1}{Gestionar Ofertas de Trabajo}, confirmando la correcta eliminación de la oferta de trabajo.

	\end{UCtrayectoria}

	\begin{UCtrayectoriaA}{A}{Cuando el actor no desea eliminar la información de la oferta de trabajo.}

		%Paso A1.
		\UCpaso [\UCactor] Presiona el botón \IUbutton{Cancelar} del mensaje \MSGref{MSG7}{Eliminar elemento}.

		%Paso A2.
		\UCpaso Muestra la pantalla \IUref{EM-BolsaWeb-UI1}{Gestionar Ofertas de Trabajo} y las ofertas en ésta sin ningún cambio. 

	\end{UCtrayectoriaA}
	
%-------------------------------------- TERMINA descripción del caso de uso.
%%%%%%%%%%%%%%%%%%%%%%%%%%%%%%%%%%%%%%
