% Copie este bloque por cada caso de uso:
%-------------------------------------- COMIENZA descripción del caso de uso.
%\begin{UseCase}[archivo de imágen]{UCX}{Nombre del Caso de uso}{
\begin{UseCase}{EM-BolsaWeb-CU4.3}{Editar información de la Empresa}{
		Este caso de uso permite al actor editar la información de una empresa previamente registrada, manteniendo con ello un mayor control de las empresas y ofertas que éstas abren para los alumnos, además de brindar un servicio más seguro a quienes deseen aplicar para una o varias de las ofertas profestas.
	}
	\UCitem{Versión}{0.1}
	\UCitem{Actor}{ Encargado de Bolsa de Trabajo.}
	\UCitem{Propósito}{ Proporcionar al actor un mecanismo que le permita tener siempre actualizada la información de las empresas registradas en el sistema.}
	\UCitem{Entradas}{
		Se requieren los siguientes datos para el registro de una empresa:
		\begin{itemize}
			\item Información de la empresa:
			\begin{itemize}
				\item Nombre.
				\item RFC.
				\item Giro.
				\item Logo.
			\end{itemize}
			\item Contacto de la empresa:
			\begin{itemize}
				\item Nombre del encargado de recursos humanos.
				\item Tipo de medio de contacto. 
				\item Medio de contacto. 
			\end{itemize}
		\end{itemize}
	}
	\UCitem{Origen}{
		Teclado, ratón.
	}
	\UCitem{Salidas}{
		\begin{itemize}
			\item \MSGref{MSG1}{Operación Exitosa}
			\item \MSGref{MSG12}{Editar elemento}
		\end{itemize}
	}
	\UCitem{Destino}{
		Pantalla.
	}
	\UCitem{Precondiciones}{
		Que haya al menos una empresa registrada.
	}
	\UCitem{Postcondiciones}{
		Persiste la información de la empresa.
	}
	\UCitem{Errores}{
		\begin{enumerate}[\hspace*{0.5cm} \bfseries{E}1:]
			%%  E1: Campos obligatorios. 
			\item \label{EM-BolsaWeb-CU4-3-E1} Cuando no se introducen todos los campos obligatorios, muestra el mensaje \MSGref{MSG5}{Falta dato obligatorio} y \textbf{Continúa en el paso \ref{EM-BolsaWeb-CU4-1-DatosObligatorios} de la trayectoria principal}.
			%%  E2: Formato incorrecto. 
			\item \label{EM-BolsaWeb-CU4-3-E2} Cuando alguno de los campos (RFC, correo electrónico, teléfono) es introducido con un formato diferente al establecido, muestra el mensaje \MSGref{MSG6}{Formato de campo Incorrecto} \textbf{Continúa en el paso \ref{EM-BolsaWeb-CU4-1-DatosObligatorios} de la trayectoria principal}.
			%%  E3: Duplicidad de información. 
			\item \label{EM-BolsaWeb-CU4-3-E3} Cuando el nombre de la empresa ya se encuentra registrado en el sistema, muestra el mensaje \MSGref{MSG4}{Información duplicada} y \textbf{termina el caso de uso.}
		\end{enumerate}	
	}
	\UCitem{Tipo}{Secundario}
	\UCitem{Observaciones}{	
	}
	\UCitem{Autor}{Huerta Matínez Jesús Manuel.}
	\UCitem{Revisor}{}
	\UCitem{Estatus}{Sin revisión.}
\end{UseCase}

\begin{UCtrayectoria}{Principal}

	%Paso 1.
	\UCpaso [\UCactor] Solicita editar la información de una empresa presionando el ícono \UCicono{lapiz.png} de la pantalla \IUref{EM-BolsaWeb-UI4}{Gestionar Empresas}. 

	%Paso 2.
	\UCpaso Muestra la pantalla \IUref{EM-BolsaWeb-UI4.3}{Editar información de Empresa} con el formulario necesario para editar la información de una empresa. \label{EM-BolsaWeb-CU4-3-DatosObligatorios}

	%Paso 3.
	\UCpaso [\UCactor] Introduce el nombre y giro de la empresa, opcionalmente introduce RFC e imagen de la empresa. 

	%Paso 4.
	\UCpaso [\UCactor] Introduce el nombre del encargado de recursos humanos.

	%Paso 5.
	\UCpaso [\UCactor] Introduce el medio de contacto del encargado de recursos humanos, compuesto por tipo de medio de contacto y el medio propiamente dicho. \label{EM-BolsaWeb-CU4-3-AgregarContacto}

	%Paso 6.
	\UCpaso [\UCactor] Solicita editar la información de la empresa presionando el botón \IUbutton{Aceptar} de la pantalla \IUref{EM-BolsaWeb-UI4.3}{Editar información de Empresa}. \Trayref{A} \Trayref{B} 

	%Paso 8.
	\UCpaso Muestra el mensaje \MSGref{MSG12}{Editar elemento} confirmando la edición de la empresa. 

	%Paso 9. 
	\UCpaso [\UCactor] Oprime el botón \IUbutton{Sí} del mensaje de confirmación. \Trayref{C} 

	%Paso 10.
	\UCpaso Valida que los campos del formulario no estén vacios, de acuedo a la regla de negocio \BRref{EM-RN-S002}{campos obligatorios}. [Error \ref{EM-BolsaWeb-CU4-3-E1}] 

	%Paso 11.
	\UCpaso Obtiene los datos de la empresa y contacto introducidos en el formulario. 

	%Paso 12.
	\UCpaso Valida que el(los) contacto(s) cumpla(n) con el formato correcto, de acuedo a las reglas de negocio \BRref{EM-RN-S001}{Formato de correo electronico}, \BRref{EM-RN-S001}{Formato de RFC} o \BRref{EM-RN-S001}{Formato de Teléfono}, según sea el caso. [Error \ref{EM-BolsaWeb-CU4-3-E2}] 

	%Paso 13.
	\UCpaso Valida que el nombre de la empresa no se encuentre registrado ya en el sistema, de acuerdo a la regla de negocio \BRref{EM-RN-S007}{Unicidad de elementos.} [Error \ref{EM-BolsaWeb-CU4-3-E3}] 
	
	%Paso 14.
	\UCpaso Persiste la información en el sistema.

	%Paso 15.
	\UCpaso Muestra el mensaje \MSGref{MSG1}{Operación Exitosa} en la pantalla \IUref{EM-BolsaWeb-UI4}{Gestionar Empresas}. 
\end{UCtrayectoria}

%Trayectoria Alternativa A.
\begin{UCtrayectoriaA}{A}{Cuando el actor desea agregar un medio de contacto más.}
	%Paso A1.
	\UCpaso [\UCactor] Presiona el botón \IUbutton{+} de la pantalla \IUref{EM-BolsaWeb-UI4.3}{Editar información de Empresa}. 

	%Paso A2.
	\UCpaso Regresa al paso \ref{EM-BolsaWeb-CU4-3-AgregarContacto} de la trayectoria principal.
\end{UCtrayectoriaA}

%Trayectoria Alternativa B.
\begin{UCtrayectoriaA}{B}{Cuando el actor desea cancelar la operación.}
	%Paso B1.
	\UCpaso [\UCactor] Presiona el botón \IUbutton{Cancelar} de la pantalla \IUref{EM-BolsaWeb-UI4.3}{Editar información de Empresa}. 

	%Paso B2.
	\UCpaso Muestra la pantalla \IUref{EM-BolsaWeb-UI4}{Gestionar Empresas}.
\end{UCtrayectoriaA}

%Trayectoria Alternativa A.
\begin{UCtrayectoriaA}{C}{Cuando el actor no desea editar la información de la empresa.}
	%Paso A1.
	\UCpaso [\UCactor] Presiona el botón \IUbutton{No} del mensaje de confirmación.

	%Paso A2.
	\UCpaso Muestra la pantalla \IUref{EM-BolsaWeb-UI4}{Gestionar Empresas}. 
\end{UCtrayectoriaA}

%-------------------------------------- TERMINA descripción del caso de uso.
%%%%%%%%%%%%%%%%%%%%%%%%%%%%%%%%%%%%%%