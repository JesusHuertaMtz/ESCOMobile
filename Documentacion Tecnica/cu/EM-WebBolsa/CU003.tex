% Copie este bloque por cada caso de uso:
%-------------------------------------- COMIENZA descripción del caso de uso.
%\begin{UseCase}[archivo de imágen]{UCX}{Nombre del Caso de uso}{
\begin{UseCase}{EM-WebBolsa-CU03}{Consultar empresa}{
Este caso de uso permite al actor consultar la información de las empresas registradas en el sistema. Así se podrá determinar si es necesario o no registrar una empresa que quiere publicar una oferta laboral.	}
	\UCitem{Versión}{0.1}
	\UCitem{Actor}{Jefe de Departamento de Extensión y Apoyos Educativos}
	\UCitem{Propósito}{Proporcionar un listado de las empresas registradas en el sistema así como la información de cada una de ellas.}
	\UCitem{Entradas}{Ninguna.}
	\UCitem{Origen}{No aplica.}
	\UCitem{Salidas}{
		\begin{itemize}
			\item Imagen con RFC.
			\item Nombre.
			\item Giro.
			\item Sector.
			\item Correo electrónico.
			\item Imagen.
		\end{itemize}}
	\UCitem{Destino}{Pantalla.}
	\UCitem{Precondiciones}{
		\begin{itemize}
			\item El actor debe haber iniciado sesión.
			\item Debe existir al menos una empresa registrada en el sistema.
		\end{itemize}}
	\UCitem{Postcondiciones}{Ninguna}
	\UCitem{Errores}{
		\begin{enumerate}[\hspace*{0.5cm} \bfseries{E}1:]
			%%  E1: No hay registros. 
			\item \label{EM-WebBolsa-CU003-E1} Cuando no existe ninguna empresa registrada en el sistema muestra el mensaje \MSGref{MSG3}{Elementos No Disponibles} y \textbf{termina el caso de uso.} }
	\UCitem{Tipo}{Primario}
	\UCitem{Observaciones}{Es la primera propuesta de este caso de uso, se espera que se revise para implementar las correciones adecuadas.}
	\UCitem{Autor}{Fernández Quiñones Isaac.}
	\UCitem{Revisor}{}
	\UCitem{Estatus}{Sin revisión.}
\end{UseCase}

\begin{UCtrayectoria}{Principal}
	\UCPaso[\UCActor]




	\UCpaso [\UCactor] Solicita presiona el botón \IUbutton{Registrar empresa} de la pantalla ESCOMobile Bolsa. 
	\UCpaso Muestra la pantalla \IUref{EM-WebBolsa-UI2}{Registrar nueva Empresa} con el formulario necesario para registrar a la empresa. 
	\UCpaso [\UCactor] Introduce los datos requeridos en el formulario. 
	\UCpaso [\UCactor] Solicita registar una nueva empresa presionando el botón \IUbutton{Registrar empresa} de la pantalla \IUref{EM-WebBolsa-UI2}{Registrar nueva Empresa}. \Trayref{A} 
	\UCpaso Valida que los campos del formulario no estén vacios, de acuedo a la regla de negocio \BRref{EM-RN-S002}{campos obligatorios} [Error \ref{EM-WebBolsa-CU02-E1}] 
	\UCpaso Valida que el nombre de la emrpesa no se encuentre registrado ya en el sistema, de acuerdo a la regla de negocio \BRref{EM-RN-S007}{Unicidad de elementos.} [Error \ref{EM-WebBolsa-CU02-E2}] 
	\UCpaso Obtiene los datos de la empresa introducidos en el formulario. 
	\UCpaso Persiste la información en el sistema.
	\UCpaso Muestra el mensaje \MSGref{MSG1}{Operación Exitosa} en la pantalla ESCOMobile Bolsa.  
\end{UCtrayectoria}

%Trayectoria Alternativa A.
\begin{UCtrayectoriaA}{A}{Cuando el actor no desea registrar más una empresa.}
	\UCpaso [\UCactor] Presiona el botón \IUbutton{cancelar} de la pantalla \IUref{EM-WebBolsa-UI2}{Registrar nueva Empresa}. 
	\UCpaso Muestra la pantalla ESCOMobile Bolsa.  
\end{UCtrayectoriaA}


%-------------------------------------- TERMINA descripción del caso de uso.
%%%%%%%%%%%%%%%%%%%%%%%%%%%%%%%%%%%%%%