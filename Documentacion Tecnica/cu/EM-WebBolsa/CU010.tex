	% \IUref{IUAdmPS}{Administrar Planta de Selección}
% \IUref{IUModPS}{Modificar Planta de Selección}
% \IUref{IUEliPS}{Eliminar Planta de Selección}


% Copie este bloque por cada caso de uso:
%-------------------------------------- COMIENZA descripción del caso de uso.

%\begin{UseCase}[archivo de imágen]{UCX}{Nombre del Caso de uso}{
	\begin{UseCase}{CU10}{EM-Web-Bolsa-CU10-Consultar las empresas y contactos}{
		En este caso de uso el actor podrá visualizar la empresas registradas, estas empresas serán las que mandarán las ofertas de trabajo y el actor las registrará para tener un control de que empresas solicitan alumnos o egresados de la ESCOM así como los contactos de cada empresa.}
		\UCitem{Versión}{0.1}
		\UCitem{Actor}{Administrador Bolsa}
		\UCitem{Propósito}{Visualizar las empresas y contactos en el sistema.}

		\UCitem{Entradas}{
		Ninguna
		}

		\UCitem{Origen}{No aplica}
		\UCitem{Salidas}{
		\begin{itemize}
		\item Nombre
		\item Giro
		\item Contacto
		\item RFC
		\item Ofertas Publicadas.
		\item Visualizaciones.
		\end{itemize}
		}
		\UCitem{Destino}{No aplica}
		\UCitem{Precondiciones}{Ninguna.}
		\UCitem{Postcondiciones}{
		Ninguna
		}
		\UCitem{Errores}{
		
		}
		\UCitem{Tipo}{Caso de uso primario}
		\UCitem{Observaciones}{	
			\begin{itemize}
				\item Identar código LATEX.
				\item NO identar texto en los párrafos del documento.
				\item Homologar nombre con el resto.
				\item Descripción completa:

						1) Escribir un resumen más completo.

						2) ''Así mismo'' es un conector que junta dos oraciones diferentes de una misma idea, tomando la segunda como una adicón a la primera. En este caso no es así, no son dos oraciones diferentes, es una misma, usa otro coector. Recomendación: así como.
				\item Atributos importantes.

						1) Escribir algo en los campos ''entradas'' y ''origen'' así sea las palabras ''ninguna'' o ''no aplica''.

						2) ¿No hay salidas en una CONSULTA? Corregir este campo y su correspondiente ''destino''.

						3) No es primario, viene de consultar ofertas.
				\item Trayectoria principal: Las trayectorias, principales y alternativas se escriben para que el programados sepa qué camino seguir exactamente para que el sistema funcione correctamente. Tu trayectoria no le dice nada al programador. 

						1) En el paso 1: el caso de uso hace referencia a las empresas, entonces comienza cuando el actor presiona el botón de empresas, lo demás no nos importa.

						2) Paso 2 ¿Con solo obtener la info de las empresas tenemos mágicamente todo? Debes especificar qué info y de dónde se obtiene. Previamente validado que haya, de lo contrario se habre una TA.

						3) Falta agregar un en el paso después de obtener la info que ésta se mostrará una tabla con tantos y cuales atributos. 

						4) Un paso más para decir que el actor consulta la info y que las empresas tienen acciones, se enlistan y linkean a trayectorias alternativas. 
				\item Trayectoria Alternativa A:

						1) Con base a lo anterior, replantear TA A.
				\item En la descripción de la pantalla WebBolsa-UI5 Consultar empresa:

						1) Describir bien pantalla y llenar todos sus atributos como diseño, salidas, etc. TODOS los que apliquen (todos menos entradas, para este caso).
			\end{itemize}
 		}
		\UCitem{Autores}{Pérez García José David}
		\UCitem{Revisó}{Huerta Martínez Jesús Manuel}
	\end{UseCase}
	\newpage
	
	\begin{UCtrayectoria}{Principal}
	
	\UCpaso [\UCactor] Da click en el botón \IUbutton{Empresas} de la pantalla \IUref {EM-WebBolsa-UI3}{Vista}
	\UCpaso Verifica que exista información de las empresas.\Trayref{A}
	\UCpaso Obtiene la información en el sistema.
	\UCpaso Muestra la pantalla \IUref {EM-WebBolsa-UI5}{Consultar empresa}con una tabla que tiene los siguiente atributos
	\begin{itemize}
		\item Nombre
		\item Giro
		\item Contacto
		\item RFC
		\item Ofertas Publicadas.
		\item Visualizaciones.
		\item Opciones.
	\end{itemize}
	\UCpaso[\UCactor] Consulta las empresas obtenidas. 
	
	\end{UCtrayectoria}
	
\begin{UCtrayectoriaA}{A}{No hay empresas registradas.}
	\UCpaso Muestra la pantalla principal \IUref {EM-WebBolsa-UI5}{Consultar empresa}. 
	\UCpaso Muestra en la tabla MSG:Ningún dato disponible en esta tabla  
	\UCpaso Termina caso de uso.
\end{UCtrayectoriaA}
	


%-------------------------------------- TERMINA descripción del caso de uso.