	% \IUref{IUAdmPS}{Administrar Planta de Selección}
% \IUref{IUModPS}{Modificar Planta de Selección}
% \IUref{IUEliPS}{Eliminar Planta de Selección}


% Copie este bloque por cada caso de uso:
%-------------------------------------- COMIENZA descripción del caso de uso.

%\begin{UseCase}[archivo de imágen]{UCX}{Nombre del Caso de uso}{
	\begin{UseCase}{CU09}{EM-Web-Bolsa-CU08-Consultar número de ofertas por mes.}{
		En este caso de uso el actor podrá visualizar el número de ofertas de trabajo para los Alumnos de ESCOM publicadas en un periodo de 6 meses antes. Estas se mostrarán en una gráfica de lineas. Las cuales en el eje x indica los meses y el eje y el número de ofertas en esos respectivos meses.}
		\UCitem{Versión}{0.2}
		\UCitem{Actor}{Administrador Bolsa}
		\UCitem{Propósito}{Informar al actor el número de  ofertas publicadas por en cada mes .}

		\UCitem{Entradas}{
		No aplica.
		}

		\UCitem{Origen}{No aplica.}
		\UCitem{Salidas}{
         Numero de Ofertas por mes.		
		}
		\UCitem{Destino}{No aplica}
		\UCitem{Precondiciones}{ }
		\UCitem{Postcondiciones}{
		Ninguna
		}
		\UCitem{Errores}{
		
		}
		\UCitem{Tipo}{Caso de uso primario}
		\UCitem{Observaciones}{	
			\begin{itemize}
				\item Identar código LATEX.
				\item NO identar texto en los párrafos del documento.
				\item Homologar nombre con el resto, no es necesario especificar que es hasta 6 meses antes, eso en el resumen.  
				\item Descripción completa:

						1) Escribir un resumen más completo.
				\item Atributos importantes.

						1) Escribir algo en los campos ''entradas'' y ''origen'' así sea las palabras ''ninguna'' o ''no aplica''.

						2) ¿No hay salidas en una CONSULTA? Corregir este campo y su correspondiente ''destino''.
				\item Trayectoria principal: Las trayectorias, principales y alternativas se escriben para que el programados sepa qué camino seguir exactamente para que el sistema funcione correctamente. Tu trayectoria no le dice nada al programador. 

						1) En el paso 1: No puedes decir que ingresa al sistema, pues esta pntalla se obtiene después de eso. 

						2) Paso 2 y 3: ¿Con solo obtener la info de las ofertas tenemos mágicamente todo?

						3) El caso de uso habla sobre consultar el número de ofertas, se liga con la gráfica, en qué momento se hace referencia a las gráficas y la manera en que se construyen.

						4) Como sabemos, una TP, es para que el programador sepa cómo obtener cada dato y cómo generar las salidas, la gráfca y los datos que muestra, son salidas, tienes que explicar como se manejan esos datos, si se promedian, si se suman, si se restan, etc (con base a una regla de negocio que debes escribir), y la manera en que se va a construír y cómo se va a presentar (estructura, ejes, datos mostrados) al usuario. 
				\item En la descripción de la pantalla WebBolsa-UI3 Vista:

						1) Debes de poner nombre alucivo.

						2) No es el único objetivo de la pantalla. 
						
						3) Ya entendí, NO COPIES Y PEGUES. Corrige la descripción de la pantalla.
			\end{itemize}
 		}
		\UCitem{Autores}{Pérez García José David.}
		\UCitem{Reviso}{Huerta Martínez Jesús Manuel}
	\end{UseCase}
	\newpage
	
	\begin{UCtrayectoria}{Principal}
	\UCpaso[\UCactor] Ingresa al sistema .
	\UCpaso Verifica que existan ofertas de trabajo.
	\UCpaso Obtiene el número de ofertas publicadas en el sistema según la regla de Negocio \BRref{EM-RN-N006}.
	\UCpaso Muestra la pantalla principal \IUref {EM-WebBolsa-UI3}{Vista} con una gráfica de lineas, en el eje x el mes y eje y el número de ofertas con los datos obtenidos.
 	\UCpaso Consulta el número de ofertas.
	

	\end{UCtrayectoria}


%-------------------------------------- TERMINA descripción del caso de uso.