	% \IUref{IUAdmPS}{Administrar Planta de Selección}
% \IUref{IUModPS}{Modificar Planta de Selección}
% \IUref{IUEliPS}{Eliminar Planta de Selección}


% Copie este bloque por cada caso de uso:
%-------------------------------------- COMIENZA descripción del caso de uso.

%\begin{UseCase}[archivo de imágen]{UCX}{Nombre del Caso de uso}{
	\begin{UseCase}{CU07}{EM-WebBolsa-CU07-Consultar ofertas registradas}{
		En este caso de uso el actor podrá visualizar las ofertas de trabajo registradas para los Alumnos de la ESCOM las cuales tendrán un tiempo de vigencia no maximo a 6 meses, estas mismas pueden tener dos estatus publicadas o aún no publicadas. Así el actor tendrá un mejor control de las ofertas que ha registrado.}
		\UCitem{Versión}{0.2}
		\UCitem{Actor}{Administrador Bolsa}
		\UCitem{Propósito}{Visualizar las ofertas de trabajo.}

		\UCitem{Entradas}{
		       Ninguno.
		}

		\UCitem{Origen}{No aplica.}
		\UCitem{Salidas}{
		Ofertas Sin Publicar
		\begin{itemize}
         \item No.
		\item Nombre de la empresa.
		\item Horario.
		\item Vacante.
		\item Sueldo.
		\item Contacto.
		\item Fecha Publicación.
		\end{itemize}	
		
		Ofertas Publicadas	
		\begin{itemize}
		\item No.
		\item Nombre de la empresa.
		\item Horario.
		\item Vacante.
		\item Sueldo.
		\item Contacto.
		\item Fecha Publicación.
		\end{itemize}
		}
		\UCitem{Destino}{Pantalla}
		\UCitem{Precondiciones}{Ninguna. }
		\UCitem{Postcondiciones}{
		Ninguna
		}
		\UCitem{Errores}{
		
		}
		\UCitem{Tipo}{Caso de uso primario}
		\UCitem{Observaciones}{	
			\begin{itemize}
				\item Identar código LATEX.
				\item NO identar texto en los párrafos del documento.
				\item Homologar nombre con el resto.  
				\item Descripción completa:

						1) Escribir un resumen más completo.

						2) Faltas de ortografía en párrafo 1: podra => podrá, aun => aún.
				\item Atributos importantes.

						1) Escribir algo en los campos ''entradas'' y ''origen'' así sea las palabras ''ninguna'' o ''no aplica''.

						2) ¿No hay salidas en una CONSULTA? Corregir este campo y su correspondiente ''destino''.
				\item Trayectoria principal: Las trayectorias, principales y alternativas se escriben para que el programados sepa qué camino seguir exactamente para que el sistema funcione correctamente. Tu trayectoria no le dice nada al programador. 

						1) En el paso 1: No puedes decir que ingresa al sistema, pues esta pntalla se obtiene después de eso. 

						2) En el paso 3: Muestra la pantalla X con una tabla que contiene el atributo1, atributo2, atributo3, ... , 
						atributoN de oferta de trabajo obtenidos en el paso de anterior, con las acciones disponibles. [Redactar bien]

						3) Agregar un paso más: Consulta las ofertas de trabajo obtenidas. [Puedes agregar lo de las acciones como puntos de extensión]

						4) En la trayectoria alternativa, regresaría al paso de ''Consulta las ofertas...'', o bien, hacer una nueva pantalla que muestre que no hay ofertas, linkear a esa y que termine el caso de uso.  

				\item En la descripción de la pantalla WebBolsa-UI3 Vista:

						1) Debes de poner nombre alucivo.

						2) No es el único objetivo de la pantalla. 
						
						3) Ya entendí, NO COPIES Y PEGUES. Corrige la descripción de la pantalla.
			\end{itemize}
		
		
 		}
		\UCitem{Autores}{Pérez García José David.}
		\UCitem{Reviso}{Huerta Martínez Jesús Manuel}
	\end{UseCase}
	\newpage
	
	\begin{UCtrayectoria}{Principal}
	\UCpaso[\UCactor] Ingresa al sistema.
	\UCpaso Verifica que existan ofertas registradas.
	\UCpaso Busca los nombres de las empresas. 
	\UCpaso Obtiene la lista de ofertas registradas.
	\UCpaso Separa las ofertas sin publicar.	\Trayref{A}
	\UCpaso Genera una descripción textual del horario con base en la regla de negocio. \BRref{EM-RN-S008}.
	\UCpaso Muestra la pantalla  \IUref {EM-WebBolsa-UI3}{Vista} con una tabla de Ofertas sin publicar la cual  tiene los siguiente atributos
	\begin{itemize}
	\item No.
	\item Nombre de la empresa.
	\item Horario.
	\item Vacante.
	\item Sueldo.
	\item Contacto.
	\end{itemize}
	\UCpaso[\UCactor] Consulta las ofertas de trabajo obtenidas. \Trayref{B}
	\end{UCtrayectoria}

\begin{UCtrayectoriaA}{A}{No hay ofertas registradas.}
	\UCpaso Muestra en la tabla MSG:Ningún dato disponible en esta tabla
	\UCpaso Fin caso de uso.
\end{UCtrayectoriaA}

\begin{UCtrayectoriaB}{B}{Consultar ofertas Publicadas.}
	\UCpaso[\UCactor] Presiona el \IUbutton{Ofertas publicadas}.
	\UCpaso Verifica que existan ofertas registradas.
	\UCpaso Busca los nombres de las empresas. 
	\UCpaso Obtiene la lista de ofertas registradas.
	\UCpaso Separa las ofertas publicadas.	
	\UCpaso Muestra en la pantalla  \IUref {EM-WebBolsa-UI3}{Vista}  una tabla de Ofertas publicadas la cual que tiene los siguiente atributos:
	\begin{itemize}
	\item No.
	\item Nombre de la empresa.
	\item Horario.
	\item Vacante.
	\item Sueldo.
	\item Contacto.
	\item Fecha Publicación.
	\end{itemize}
	\UCpaso[\UCactor] Consulta las ofertas de trabajo obtenidas.
	\UCpaso Fin caso de uso.
\end{UCtrayectoriaB}


	
%-------------------------------------- TERMINA descripción del caso de uso.
%%%%%%%%%%%%%%%%%%%%%%%%%%%%%%%%%%%%%%
