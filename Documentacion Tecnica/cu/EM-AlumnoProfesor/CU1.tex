% Copie este bloque por cada caso de uso:
%-------------------------------------- COMIENZA descripción del caso de uso.
%\begin{UseCase}[archivo de imágen]{UCX}{Nombre del Caso de uso}{
\begin{UseCase}{EM-AlumnoProfesor-CU1}{Buscar Profesor de ESCOM.}{
	\noindent
	Este caso de uso permite al actor realizar la búsqueda de algún profesor de ESCOM de su interés dado
	de alta en la aplicación ESCOMobile, ya sea de manera manual en la lista desplegada de inicio (que
	muestra los nombres de los profesores de ESCOM registrados) o bien, por medio de un filtro y las
	coincidencias que éste arroja con los nombres de profesores en el sistema. 
	Se debe mencionar que se puede realizar la búsqueda de los profesores sin necesidad de que éstos se
	encuentren registrados como usuarios de la app, pues se tiene registro de ellos igualmente en el sistema.
	Sin embargo, de ser el caso, solo se podrá consultar la información de los referidos, mas no se
	podrá agendar cita con ellos. 
	\newline
	}
	\UCitem{Versión}{0.1}
	\UCitem{Actor}{Alumno, Profesor.}
	\UCitem{Propósito}{Proporcionar al actor un mecanismo que le permita buscar dentro de la applicación
	a algún profesor de ESCOM de su interés.}
	\UCitem{Entradas}{
		\begin{itemize}
			\item Nombre del profesor a buscar.
		\end{itemize}
	}
	\UCitem{Origen}{Pantalla.}
	\UCitem{Salidas}{
		\begin{itemize}
			\item Lista que muestra alfabéticamente los nombres de todos los profesores registrados en
			ESCOM, o bien, los coincidentes con la búsqueda. 
		\end{itemize}
	}
	\UCitem{Destino}{Pantalla.}
	\UCitem{Precondiciones}{
		\begin{itemize}
			\item Haber iniciado sesión en el sistema.
			\item Tener profesores dados de alta en el sistema.
		\end{itemize}
	}
	\UCitem{Postcondiciones}{Ninguna.}
	\UCitem{Errores}{
		\begin{enumerate}[\hspace*{0.5cm} \bfseries{E}1:]
			%%  E1: Campos obligatorios. 
			\item \label{EM-AlumnoProfesor-CU1-E1} Cuando no hay profesores cargados en el sistema, muestra el mensaje \MSGref{MSG3}{Elementos No Disponibles} y \textbf{termina el caso de uso}.
		\end{enumerate}. 
	}
	\UCitem{Tipo}{Caso de uso primario.}
	\UCitem{Observaciones}{}
	\UCitem{Autor}{Huerta Martínez Jesús Manuel.}
	\UCitem{Revisor}{Fernández Quiñones Isaac.}
	\UCitem{Estatus}{Corregido.}
\end{UseCase}

\begin{UCtrayectoria}{Principal}

	%Paso 1.
	\UCpaso [\UCactor] Solicita buscar a un profesor de ESCOM seleccionando la opción \textbf{Profesores} de la pantalla \IUref{EM-ESCOMobile-Hamburger}{}.

	%Paso 2.	
	\UCpaso Verifica que haya profesores dados de alta en el sistema. [Error: \ref{EM-AlumnoProfesor-CU1-E1}] 

	%Paso 3.
	\UCpaso Obtiene el nombre de todos los profesores en el sistema. 

	%Paso 4.
	\UCpaso Ordena los nombres obtenidos de manera alfabética. 

	%Paso 6.
	\UCpaso Muestra la pantalla \IUref{EM-AlumnoProfesor-UI1}{Buscar Profesores de ESCOM} con una lista que contiene los nombres de los profesores ya ordenados. \label{EM-AlumnoProfesor-CU1-Busqueda}.

	%Paso 7.
	\UCpaso [\UCactor] Introduce en la barra de búsqueda el nombre de algún profesor de su interés. 

	%Paso 8.
	\UCpaso Obtiene el nombre del profesor introducido.

	%Paso 9.
	\UCpaso Verifica que haya coincidencias con los profesores en el sistema, comparando el nombre introducido con los nombres de profesores dados de alta en la aplicación. \Trayref{A}.

	%Paso 10.
	\UCpaso Obtiene los nombres coincidentes con el nombre introducido.

	%Paso 11.
	\UCpaso Ordena los nombres coincidentes obtenidos de manera alfabética. 

	%Paso 12.
	\UCpaso Muestra la pantalla \IUref{EM-AlumnoProfesor-UI1}{Buscar Profesores de ESCOM} con una lista que contiene los nombres de los profesores que coincidieron ya ordenados. 

\end{UCtrayectoria}

%Trayectoria Alternativa A.
\begin{UCtrayectoriaA}{A}{Cuando no se han encontrado coincidencias de los profesores dados de alta en el sistema y el nombre introducido}
	%Paso A1.
	\UCpaso Muestra el mensaje \MSGref{MSG13}{Ninguna coincidencia encontrada} en la pantalla \IUref{EM-AlumnoProfesor-UI1}{Buscar Profesores de ESCOM}.

	\UCpaso Regresa al paso \ref{EM-AlumnoProfesor-CU1-Busqueda} de la trayectoria principal.
\end{UCtrayectoriaA}



%-------------------------------------- TERMINA descripción del caso de uso.
%%%%%%%%%%%%%%%%%%%%%%%%%%%%%%%%%%%%%%