%\item[$	\textbf{ID1:}$]: EM-Mapa-CU2 
%\item[$\textbf{Nombre:}$]: Mostrar cubículo en mapa.
%\item[$\textbf{Actores:}$]: Alumno.
%\item[$\textbf{Descripción:}$]: El actor podrá consultar la ubicación del cubículo de un profesor. La ubicación se mostrará en el mapa de la ESCOM.

%-------------------------------------- COMIENZA descripción del caso de uso.
%\begin{UseCase}[archivo de imágen]{UCX}{Nombre del Caso de uso}{
\begin{UseCase}{EM-alumnoProfesor-CU1.1.4}{Ubicar profesor en el Mapa.}
	{Este caso de uso brinda al alumno la información de a qué edificio, en qué planta y en qué área se encuentra el
	 cubículo del profesor al mostrar en el mapa la ubicación de su cubiculo. Para que el alumno pueda saber en dónde se encuentra, la mayor parte del 
	 tiempo, el profesor, cuando este último no está impartiendo clase.
	 
	 Debe tenerse en cuenta que la ubicación se mostrará en el mapa de ESCOM en 2D}
	\UCitem{Versión}{0.1}
	\UCitem{Actor}{Alumno.}
	\UCitem{Propósito}{Proporcionar la ubicación y localización del cubículo de un profesor para que 
	el alumno pueda encontrarlo de una manera rápida y sencilla.}
	\UCitem{Entradas}{Ninguna.}
	\UCitem{Origen}{No aplica.}
	\UCitem{Salidas}{
		\begin{itemize}
			\item Muestra en el mapa 2D la ubicación del cubículo.
			\item Nombre del área donde se encuentra el cubículo.
			\item Planta en la que se ubica el cubículo.
			\item Edificio al que pertenece el cubículo.
		\end{itemize}
	}
	\UCitem{Destino}{Pantalla.}
	\UCitem{Precondiciones}{
		\begin{itemize}
			\item Deben estar registradas en el sistema las coordenadas de ESCOM.
			\item Deben estar registradas en el sistema las coordenadas de los cubículos.
			\item Debe estar el profesor asignado a un cubículo o salón.

		\end{itemize}
	}
	\UCitem{Postcondiciones}{Ninguna. Solo se realizan consultas.}
	\UCitem{Errores}{
		\begin{itemize}
			\item Título: A quién ubicamos en el mapa... Piénsalo bien. 
			\item Descripción completa: 
			\begin{itemize}
				\item Título: Acostumbramos a poner ''Este caso de uso'' al inicio de cada CU. 
				\item El caso de uso no te muestra esa información, se redirige al mapa y te pinta el cubículo del profesor, luego si presionas sobre él, entonces sí te muestra más info. 
				\item ''En que planta'', ''en que área'', ''en que edificio'', la palabra ''que'' se acentúa en todos esos casos -> ''qué'', investiga por qué. 
				\item Redactar mejor el CU. 
			\end{itemize}
			\item Precondiciones: 
			\begin{itemize}
				\item Precondición 3: ¿Solo 1?
				\item Precondiciones 4 y 5: Debes definir si realmente se necesita estar registrado y con la sesión activa. 
			\end{itemize}
			\item Errores: ¿No hay errores? 
			\item Tipo: No es primario. 
			\item Trayectoria principal: 
			\begin{itemize}
				\item Paso 1: Cuando hablamos de botones escribimos ''presiona'', cuando son iconos escribimos ''da clic'', entonces en android no hay iconos, por tanto, ''ubicar'' en el mapa es un botón.
				\item Paso 1: No cierras las comillas que abres en ''Ubicar en el mapa
				\item Paso 1: No hay ningún botón llamado ''Ubicar en el mapa'' en la pantalla. 
				\item Paso 6. Actualiza pantallas. 
				\item De la interfaz, corrígela y descríbela. -.-
				\item Contempla errores. 
			\end{itemize}
		\end{itemize}
	}
	\UCitem{Tipo}{Secundario}
	\UCitem{Observaciones}{

	}
	\UCitem{Autor}{José David Pérez García.}
	\UCitem{Revisor}{}
	\UCitem{Estatus}{Corregir revisiones}
\end{UseCase}

\begin{UCtrayectoria}{Principal}
	\UCpaso[\UCactor] Presiona el el botón  \IUbutton{Ubicar en mapa} que se encuentra en la pantalla  \IUref{EM-AlumnoProfesor-UI1-1}{Consultar Perfil del Profesor}.
	\UCpaso Obtiene las coordenadas del cubículo del profesor.
	\UCpaso Dibuja en el mapa las coordenadas del cubículo.
	\UCpaso Añade una marca sobre el cubículo dibujado.
	\UCpaso Añade el nombre del área, planta y edificio del cubículo a la etiqueta.
	\UCpaso Muestra la pantalla \IUref{EM-Mapa-UI2b}{Realizar busqueda sobre el mapa.}
	\UCpaso[\UCactor] Consulta la información desplegada en la pantalla.
\end{UCtrayectoria}

%-------------------------------------- TERMINA descripción del caso de uso.
%%%%%%%%%%%%%%%%%%%%%%%%%%%%%%%%%%%%%%