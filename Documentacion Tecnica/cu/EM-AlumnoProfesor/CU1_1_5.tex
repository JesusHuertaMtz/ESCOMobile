% Copie este bloque por cada caso de uso:
%-------------------------------------- COMIENZA descripción del caso de uso.
%\begin{UseCase}[archivo de imágen]{UCX}{Nombre del Caso de uso}{
\begin{UseCase}{EM-alumnoProfesor-CU1.1.5}{Solicitar Cita con un profesor.}{
	\noindent
	Este caso de uso permite al actor solicitar a algún profesor de la ESCOM el agendar una cita con él, pues requiere apoyo para la resolución de algún asunto académico, por ejemplo, dudas de algún tema de clase, asesorías de una materia en particular, revisiones de TT, entrega de proyectos, entre otros.
	Para solicitar agendar una cita se debe seleccionar al profesor con el cual se desea realizar la cita para, posteriormente ser redireccionado a otra pantalla para establecer los puntos de la misma.
	\newline
	}
	\UCitem{Versión}{0.1}
	\UCitem{Actor}{Alumno.}
	\UCitem{Propósito}{Proporcionar un mecanismo para que el alumno pueda solicitar una cita al profesor para tratar temas académicos.}
	\UCitem{Entradas}{Ninguna.}
	\UCitem{Origen}{No aplica.}
	\UCitem{Salidas}{Ninguna.}
	\UCitem{Destino}{No aplica..}
	\UCitem{Precondiciones}{
		\begin{itemize}
			\item El actor debe estar registrado.
			\item El actor debe haber iniciado sesión.
			\item Debe estar registrado el profesor con el que se requiere agendar la cita.
		\end{itemize}
	}
	\UCitem{Postcondiciones}{
		\begin{itemize}
			\item El sistema tendrá un nuevo registro asociado al alumno con la fecha, hora, asunto,
			salón y nombre del profesor con el que se agendará la cita.
			\item El nuevo registro de la cita estará en estado de no aceptado.
			\item El sistema notificará al profesor que tiene una nueva solicitud de cita.
		\end{itemize}			
	}
	\UCitem{Errores}{
		\begin{enumerate}[\hspace*{0.5cm} \bfseries{E}1:]	
\item \label{EM-alumnoProfesor-CU1-1-5-E1} Cuando no se introducen todos los campos obligatorios, muestra el mensaje \MSGref{MSG5}{Falta dato obligatorio} y \textbf{Continúa en el paso \ref{error} de la trayectoria 
principal}.
\item \label{EM-alumnoProfesor-CU1-1-5-E2} Cuando la fecha o hora se traslapa con cita ya agendada, o con hora de clase del profesor o es dáa inhábil, muestra el mensaje \MSGref{MSG12}{Traslape o fecha y hora no valida} y \textbf{Continúa en el paso \ref{error} de la trayectoria 
principal}.

		\end{enumerate}
	}
	\UCitem{Tipo}{Primario}
	\UCitem{Observaciones}{
		\begin{itemize}
			\item Título: Es correcto, pero puedes agregar ''con un profesor'' y que queda más completo.
			\item Descripción completa: ¿Permite solicitar al alumno? ¿El alumno acepta la cita? Revisar. 
			\item Entradas: 
			\begin{itemize}
				\item El nombre del profesor no es una entrada. Tampoco el salón, aunque se puede tomar por default el cubículo del profesor, y dar la oción a cambiar el salón, así sí sería entrada. 
				\item Origen: No son por teclado. 
			\end{itemize}
			\item Errores: Liga rota en el E1. ¿Qué pasa si el horario o fechas propuestas no son hábiles? ¿Qué pasa si el profesor ya tiene agendada una cita ese día a esa hora? Contemplar esos errores y más. 
			\item Tipo: No es primario. 
			\item Trayectoria principal: 
			\begin{itemize}
				\item Paso 2: Tienes que obtener el horario del profesor y todo lo que éste conlleva para poder comparar con el horario en que solicitas agendar la cita y comparar si s posible. De lo contrario, siempre te dejaría solicitar, incluso fuera de los días y horarios laborales.  
				\item Pasos 6 y 7: Replantear el anaálisis del salón y notificación.
				\item Paso 9: No hay ningún botón llamado ''Agendar cita'' en la pantalla. 
				\item Paso 6. Actualiza pantallas. 
				\item En algún paso debes validar que 1) el profesor no tenga asociada ya una cita los días y horas solictados, y 2) que el día y hora introducidos estén dentro de los días y horarios hábiles del profesor.
				\item Corregir pantalla para la cita y su descripción.
			\end{itemize}
			\item Trayectoria alternativa: En l pantalla no hay ningún botón llamado ''cancelar''.
		\end{itemize}
	}
	\UCitem{Autor}{Huerta Martínez Jesús Manuel}
	\UCitem{Revisor}{Fernández Quiñones Isaac.}
	\UCitem{Estatus}{Sin revisión.}
\end{UseCase}

\begin{UCtrayectoria}{Principal}
	\UCpaso[\UCactor] Solicita una cita presionando el botón \IUbutton{ Solicitar cita  } de la pantalla \IUref{EM-AlumnoProfesor-UI1-1}{Consultar Perfil del Profesor}.

	\UCpaso Obtiene el nombre del profesor.
	\UCpaso Muestra pantalla \IUref{EM-Cita-IU09}{Crear cita II} con el nombre del profesor al que se le pedirá la cita y se mostrará un formulario a llenar. \label{error}
	\UCpaso [\UCactor] Selecciona fecha y hora, en la cual desea la cita.
	\UCpaso  [\UCactor] Selecciona el tipo de cita.
	\UCpaso  [\UCactor] Selecciona tipo de notificación.
	\UCpaso  [\UCactor] Escribe el motivo de la cita.
	\UCpaso  [\UCactor] Presiona el botón \IUbutton{ Solicitar Cita }.
	\UCpaso Valida que los campos del formulario no estén vacios, de acuedo a la regla de negocio \BRref{EM-RN-S002}{campos obligatorios}. [Error \ref{EM-alumnoProfesor-CU1-1-5-E1}]. \Trayref{A}
		\UCpaso Obtiene los datos de la cita introducidos en el formulario. 
		\UCpaso Obtiene el horario del profesor.
		\UCpaso Valida el horario de la cita con el horario del profesor según la regla de negocio \BRref{EM-RN-N007}{Horario citas}.[Error \ref{EM-alumnoProfesor-CU1-1-5-E2}].
		\UCpaso Persiste la información en el sistema.
	\UCpaso Muestra el mensaje \MSGref{MSG1}{Operación Exitosa} en la pantalla \IUref{EM-AlumnoProfesor-UI1-1}{Consultar Perfil del Profesor}.



\end{UCtrayectoria}
		
%========Trayectorias alternativas

\begin{UCtrayectoriaA}{A}{Cuando el actor desea cancelar la operación.}
	\UCpaso [\UCactor] Presiona el botón \IUbutton{Cancelar} de la pantalla \IUref{EM-Cita-IU09}{Crear cita II}
	\UCpaso Muestra la pantalla \IUref{EM-AlumnoProfesor-UI1-1}{Consultar Perfil del Profesor}.
\end{UCtrayectoriaA}

%========

%-------------------------------------- TERMINA descripción del caso de uso.
%%%%%%%%%%%%%%%%%%%%%%%%%%%%%%%%%%%%%%