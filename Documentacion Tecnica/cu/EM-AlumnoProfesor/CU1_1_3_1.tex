% Copie este bloque por cada caso de uso:
%-------------------------------------- COMIENZA descripción del caso de uso.
%\begin{UseCase}[archivo de imágen]{UCX}{Nombre del Caso de uso}{
\begin{UseCase}{EM-alumnoProfesor-CU1.1.3.1}{Consultar comentarios del profesor.}{
	\noindent
	Este caso de uso permite al actor consultar los diferentes comentarios que los alumnos han compartido hacia algún profesor
	en particular. Pues es así que la comunidad de ESCOM tiene mayor conocimiento de los maestros del plantel, de su forma de
	trabajo y diversas actitudes que tiene para con sus grupos y alumnos.
	\newline
	}
	\UCitem{Versión}{0.1}
	\UCitem{Actor}{Alumno, profesor.}
	\UCitem{Propósito}{Proporcionar al actor un mecanismo para conocer las opiniones que la comunidad de ESCOM tiene para los profesores de la misma.}
	\UCitem{Entradas}{Ninguna.}
	\UCitem{Origen}{No aplica.}
	\UCitem{Salidas}{
		\begin{itemize}
			\item Nombre y fotografía del profesor consultado.
			\item Lista con los comentarios asignados al profesor y el nombre de quienes escribieron cada comentario.
			\item \MSGref{MSG3}{Elementos No Disponibles}.
		\end{itemize}
	}
	\UCitem{Destino}{Pantalla.}
	\UCitem{Precondiciones}{
		\begin{itemize}
			\item El actor debe estar registrado.
			\item El actor debe haber iniciado sesión.
		\end{itemize}
	}
	\UCitem{Postcondiciones}{Ninguna.}
	\UCitem{Errores}{
		\begin{enumerate}[\hspace*{0.5cm} \bfseries{E}1:]
			%%  E1: Campos obligatorios. 
			\item \label{EM-AlumnoProfesor-CU1-1-3-1-E1} Cuando no hay comentarios disponibles para consulta, muestra el mensaje \MSGref{MSG3}{Elementos No Disponibles} y \textbf{termina el caso de uso}.
		\end{enumerate} 
	}
	\UCitem{Tipo}{Caso de uso secundario, viene de \UCref{EM-AlumnoProfesor-CU1.1.3}.}
	\UCitem{Observaciones}{}
	\UCitem{Autor}{Huerta Martínez Jesús Manuel}
	\UCitem{Revisor}{Fernández Quiñones Isaac.}
	\UCitem{Estatus}{Corregido.}
\end{UseCase}

\begin{UCtrayectoria}{Principal}

	% Paso 1.
	\UCpaso [\UCactor] Solicita consultar los comentarios asignados al profesor, presionando la opción ''COMENTARIOS'' de la pantalla \IUref{EM-AlumnoProfesor-UI1-1-3}{Consultar estadísticas del profesor}.

	% Paso 2.
	\UCpaso Valida que haya comentarios asociados al profesor consultado. [Error: \ref{EM-AlumnoProfesor-CU1-1-3-1-E1}]

	% Paso 3.
	\UCpaso Obtiene nombre y fotografía del profesor consultado. 

	% Paso 4.
	\UCpaso Obtiene los comentarios asociados al profesor y los alumnos quienes otorgaron cada comentario. 

	% Paso 5.
	\UCpaso Ordena los comentarios y alumnos según la fecha de publicación de los comentarios (más reciente a más antiguo).

	% Paso 6.
	\UCpaso Muestra la pantalla \IUref{EM-AlumnoProfesor-UI1-1-3-1}{Consultar comentarios del profesor} con la información obtenida del profesor y comentarios.

	% Paso 7.
	\UCpaso[\UCactor] Consulta los comentarios asignados al profesor.

\end{UCtrayectoria}
		
