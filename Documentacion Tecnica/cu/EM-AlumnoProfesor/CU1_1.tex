% Copie este bloque por cada caso de uso:
%-------------------------------------- COMIENZA descripción del caso de uso.
%\begin{UseCase}[archivo de imágen]{UCX}{Nombre del Caso de uso}{
\begin{UseCase}{EM-AlumnoProfesor-CU1.1}{Consultar perfil del profesor.}{
	\noindent
	Este caso de uso permite al actor consultar la información de algún
	profesor de su interés por medio de su perfil, esta información es su
	nombre, fotografía, académia, así, también sirve de acceso para consultar
	su horario y estadísticas de desempeño docente, calificarlo, ubicarlo en
	el mapa de ESCOM o solicitarle una cita. Es importante decir que las citas
	solo se podrán solicitar y llevar a cabo en el caso en el que el profesor
	consultado esté regitrado en el sistema, pues aunque se cuenta de igual
	forma con su información básica, es necesario el registro de éste para
	lograr la interración con los alumnos por medio de las citas. En el caso
	en el cual el profesor no se encuentra registrado, se podrá consultar aún
	su información antes mencionada (exceptuando las citas y estadísticas). 
	\newline
	}
	\UCitem{Versión}{0.1}
	\UCitem{Actor}{Alumno.}
	\UCitem{Propósito}{Proporcionar al actor un mecanismo que le permita consultar
	la información de algún profesor de ESCOM de su interés.}
	\UCitem{Entradas}{Ninguna.}
	\UCitem{Origen}{No aplica.}
	\UCitem{Salidas}{
		Se muestra la siguiente información del profesor:
		\begin{itemize}
			\item Nombre.
			\item Academia a la que pertenece
			\item Cúbiculo.
			\item Calificación promedio.
			\item Fotografía. 
		\end{itemize}
	}
	\UCitem{Destino}{Pantalla.}
	\UCitem{Precondiciones}{Ninguna.}
	\UCitem{Postcondiciones}{Ninguna.}
	\UCitem{Errores}{Ninguno.}
	\UCitem{Tipo}{Caso de uso secundario, viene de \UCref{EM-AlumnoProfesor-CU1}.}
	\UCitem{Observaciones}{}
	\UCitem{Autor}{Huerta Matínez Jesús Manuel.}
	\UCitem{Revisor}{Fernández Quiñones Isaac.}
	\UCitem{Estatus}{Corregido.}
\end{UseCase}

\begin{UCtrayectoria}{Principal}

	%Paso 1.
	\UCpaso [\UCactor] Solicita consultar la información de algún profesor de ESCOM seleccionando su nombre en la lista de profesores mostrada en la pantalla \IUref{EM-AlumnoProfesor-UI1}{Buscar Profesores de ESCOM}.

	%Paso 2.
	\UCpaso Obtiene nombre, fotografía (en caso de tener), académia, cubículo y calificación promedio del profesor seleccionado.

	%Paso 3.
	\UCpaso Muestra la pantalla \IUref{EM-AlumnoProfesor-UI1-1}{Consultar Perfil del Profesor} con la información obtenida.

	%Paso 4.
	\UCpaso[\UCactor] Consulta la información del Profesor.

	%Paso 5.
	\UCpaso [\UCactor] Presiona uno de los siguientes botones, según sea la acción que desee realizar: 
	\begin{itemize}
		\item \IUbutton{Horario} para consultar el horario del profesor.
		\item \IUbutton{Estadísticas} para consultar las estadísticas del profesor.
		\item \IUbutton{Ubicar en el mapa} para ubicar el cubículo del profesor en el mapa.
		\item \IUbutton{Calificar} para calificar su desempeño docente
		\item \IUbutton{Solicitar cita} para solicitar generar una cita con el profesor.
		\item \IUbutton{Regresar} para regresar a la pantalla anterior.  
	\end{itemize}

\end{UCtrayectoria}



%-------------------------------------- TERMINA descripción del caso de uso.
%%%%%%%%%%%%%%%%%%%%%%%%%%%%%%%%%%%%%%