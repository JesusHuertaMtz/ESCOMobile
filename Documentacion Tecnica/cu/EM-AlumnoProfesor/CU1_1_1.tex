% Copie este bloque por cada caso de uso:
%-------------------------------------- COMIENZA descripción del caso de uso.
%\begin{UseCase}[archivo de imágen]{UCX}{Nombre del Caso de uso}{
\begin{UseCase}{EM-AlumnoProfesor-CU1.1.1}{Consultar Horario del Profesor.}{
	\noindent
	Este caso de uso permite al actor consultar el horario de algún profesor de su interés en ESCOM, esto es, las materias que imparte a lo largo de la semana, así como la hora, grupo y salón en las que las mismas se ubican. Es importante mencionar los horarios son asignados por la institución a los profesores, y las materias, grupos, salones y horas que éstos contemplan varían semestre a semestre, así mismo, se muestran en ESCOMobile solo los horarios previamente descritos, por lo que horarios extraclase o de actividades personades de los profesores no son considerados. 
	\newline
	}
	\UCitem{Versión}{0.1}
	\UCitem{Actor}{Alumno, Profesor.}
	\UCitem{Propósito}{Proporcionar al actor un mecanismo que le permita consultar el horario de algún profesor de ESCOM de su interés.}
	\UCitem{Entradas}{Ninguna.}
	\UCitem{Origen}{No aplica.}
	\UCitem{Salidas}{
		Se muestra la siguiente información del profesor:
		\begin{itemize}
			\item Nombre.
			\item Fotografía. 
		\end{itemize}
		E información de la materia:
		\begin{itemize}
			\item Nombre.
			\item Hora, grupo y salón en que se imparte. 
		\end{itemize}
	}
	\UCitem{Destino}{Pantalla.}
	\UCitem{Precondiciones}{Ninguna.}
	\UCitem{Postcondiciones}{Ninguna.}
	\UCitem{Errores}{Ninguno.}
	\UCitem{Tipo}{Caso de uso secundario, viene de \UCref{EM-AlumnoProfesor-CU1.1}.}
	\UCitem{Observaciones}{}
	\UCitem{Autor}{Huerta Matínez Jesús Manuel.}
	\UCitem{Revisor}{Fernández Quiñones Isaac.}
	\UCitem{Estatus}{Corregido.}
\end{UseCase}

\begin{UCtrayectoria}{Principal}

	%Paso 1.
	\UCpaso [\UCactor] Solicita consultar el horario del profesor presionando el botón \IUbutton{Horario} de la pantalla \IUref{EM-AlumnoProfesor-UI1-1}{Consultar Perfil del Profesor}.

	%Paso 2.
	\UCpaso Obtiene por día (de lunes a viernes) las materias que el profesor imparte, así como los grupos, salones y horas en que éstas se encuentran. 

	%Paso 3.
	\UCpaso Muestra la pantalla \IUref{EM-AlumnoProfesor-UI1-1-1}{Consultar Horario del Profesor} con la información obtenida organizada en tarjetas despelgables, una por cada día de la semana.

	%Paso 4.
	\UCpaso[\UCactor] Consulta el horario del Profesor seleccionando las tarjetas con los días de las semana.

\end{UCtrayectoria}



%-------------------------------------- TERMINA descripción del caso de uso.
%%%%%%%%%%%%%%%%%%%%%%%%%%%%%%%%%%%%%%