\begin{UseCase}{EM-Mapa-CU2}{Realizar Búsqueda Sobre el Mapa.}{
	\noindent
	Este caso de uso permite al actor realizar una búsqueda de cubículos de profesores, salones, grupos 
	o academias de su interés dentro del mapa de la ESCOM por medio de ESCOMobile. Para ello basta con
	seleccionar el tipo de búsqueda que desea realizar y escribir el nombre del profesor, grupo, salón
	o academia de la cual desea saber la ubicación dentro de la Superior de cómputo, pues, de haber 
	resultados coincidentes, éstos se dibujarán en el propio mapa. 
	\newline 
	}
	\UCitem{Versión}{0.2}
	\UCitem{Actor}{Alumno, Profesor, Invitado.}
	\UCitem{Propósito}{Proporcionar al actor un mecanismo para ubicar en el mapa de ESCOM el cubículo de 
	algún profesor, salón, grupo o academia.}
	\UCitem{Entradas}{
		\begin{itemize}
			\item Nombre del área que desea buscar (academia, salón, grupo) o nombre del profesor
			del cual se desea caber la ubicación del cubículo.
		\end{itemize}
	}
	\UCitem{Origen}{Pantalla.}
	\UCitem{Salidas}{
		\begin{itemize}
			\item Área coincidente en el mapa con la búsqueda realizada. 
		\end{itemize}
	}
	\UCitem{Destino}{Pantalla.}
	\UCitem{Precondiciones}{
		\begin{itemize}
			\item Tener registradas las coordenadas de las diferentes áreas que componen el mapa 
			(edificios, plantas, salones, etc.).
			\item El actor debe haber iniciado sesión.
			\item Tener cargada la imagen del plano de ESCOM.
		\end{itemize}
	}
	\UCitem{Postcondiciones}{Ninguna.}
	\UCitem{Errores}{
		\begin{enumerate}[\hspace*{0.5cm} \bfseries{E}1:]

			\item \label{EM-Mapa-CU02-E1} Cuando no hay resultados coincidentes con la búsqueda realizada, 
			muestra el mensaje \MSGref{MSG15}{Ninguna coincidencia encontrada} y \textbf{termina el caso de uso.}

			\item \label{EM-Mapa-CU02-E2} Cuando no se es posible dibujar las áreas académicas sobre el mapa, pues no se cuentan
			con las coordenadas para ello, muestra el mensaje \MSGref{MSG3}{Elementos No Disponibles} y \textbf{termina el caso de uso.}

		\end{enumerate}
	}
	\UCitem{Tipo}{Primario}
	\UCitem{Observaciones}{}
	\UCitem{Autor}{Huerta Martínez Jesús Manuel.}
	\UCitem{Revisor}{Fernández Quiñones Isaac.}
	\UCitem{Estatus}{Corregido.}
\end{UseCase}

\begin{UCtrayectoria}{Principal}

	% Paso 1. 
	\UCpaso[\UCactor] Presiona sobre la barra de búsqueda de la pantalla \IUref{EM-Mapa-UI1}{Consultar Mapa de ESCOM}.

	% Paso 2. 
	\UCpaso Se muestra sobre la pantalla \IUref{EM-Mapa-UI1}{Consultar Mapa de ESCOM} los filtros de búsqueda (profesores,
	salones, grupos, academias), como se puede observar en la pantalla \IUref{EM-Mapa-UI2-A}{Realizar búsqueda sobre el mapa}.

	% Paso 3.
	\UCpaso [\UCactor] Selecciona alguno de los cuatro filtros disponibles para realizar la búsqueda. 

	% Paso 4.
	\UCpaso [\UCactor] Introduce el nombre del área que desea buscar (en el caso de haber seleccionado el filtro ''profesores'',
	deberá introducir el nombre del profesor del cual desea conocer su cubículo).

	% Paso 5.
	\UCpaso [\UCactor] Presiona sobre el Icono \UCicono{lupa}, solicitando realizar la búsqueda. 

	% Paso 6.
	\UCpaso Verifica que haya coincidencias entre la cadena introducida por el usuario y la información contenida en el sistema,
	aplicando el filtro seleccionado. [Error: \ref{EM-Mapa-CU02-E1}]

	% Paso 7.
	\UCpaso Obtiene y muestra la(s) coincidencia(s) de la búsqueda debajo de los filtros de búsqueda, como se puede ver en la
	pantalla \IUref{EM-Mapa-UI2-A}{Realizar búsqueda sobre el mapa}.

	% Paso 8.
	\UCpaso [\UCactor] Selecciona uno de los resultados coincidentes. 

	% Paso 9.
	\UCpaso Obtiene las coordenadas del área asociada al resultado seleccionado. [Error: \ref{EM-Mapa-CU02-E2}]

	% Paso 10.
	\UCpaso Dibuja el área sobre el mapa de ESCOM. 

	% Paso 11.
	\UCpaso Muestra la pantalla \IUref{EM-Mapa-UI2-B}{Realizar búsqueda sobre el mapa} con el área coincidente con la búsqueda
	ya dibujado en el mapa. 

	% Paso 12.
	\UCpaso [\UCactor] Consulta en el mapa el área coincidente. 


\end{UCtrayectoria}

%========
%\subsection{Puntos de extensión}

%\UCExtensionPoint{EPAcademias}{El actor desea consultar academias por planta}
%{Paso \ref{CU04ConsultarOtros} de la trayectoria principal.}
%{Continua en el caso de uso \UCref{EM-Mapa-CU04}.}

%\UCExtensionPoint{EPAdminstrativa}{El actor desea consultar áreas administrativas por planta}
%{Paso \ref{CU04ConsultarOtros} de la trayectoria principal.}
%{Continua en el caso de uso \UCref{EM-Mapa-CU05}.}

%\UCExtensionPoint{EPSalones}{El actor desea consultar salones por planta}
%{Paso \ref{CU04ConsultarOtros} de la trayectoria principal.}
%{Continua en el caso de uso \UCref{EM-Mapa-CU06}.}

%\begin{UCtrayectoriaA}{A}{El actor selecciona ver los salones.}
%	\UCpaso[\UCactor] Presiona el botón \IUbutton{ Salones } de la \IUref{IU6}{Pantalla de inicio del alumno.}
%\end{UCtrayectoriaA}
%
%\begin{UCtrayectoriaA}{B}{El actor selecciona ver los áreas admnistrativas.}
%	\UCpaso[\UCactor] Presiona el botón \IUbutton{ Salones } de la \IUref{IU6}{Pantalla de inicio del alumno.}
%\end{UCtrayectoriaA}
%
%\begin{UCtrayectoriaA}{C}{El actor selecciona ver los academias.}
%	\UCpaso[\UCactor] Presiona el botón \IUbutton{ Salones } de la \IUref{IU6}{Pantalla de inicio del alumno.}
%\end{UCtrayectoriaA}















