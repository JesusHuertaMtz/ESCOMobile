\begin{UseCase}{EM-Mapa-CU01}{Consultar mapa de ESCOM.}{
	\noindent
	Este caso de uso permite al actor consultar la distribución de las diferentes áreas dentro de la ESCOM
	como lo son lon salones, cubículos, academias, etc. en los diferentes edificos y plantas contenidos en
	la misma institución. Lo anterior se logra por medio de un mapa 2D, en donde se muestran organizados
	las diferentes áreas tal cual se encuentran en la Superior de Cómputo, pues de esta forma se facilitan,
	entre otras cosas, el encontrar una determinada área académica, localizar la estancia de algún profesor
	en específico, conocer los salones en los cuales se imparten clases dado un grupo, etc., agilizando
	así tareas cotidianas en el plantel.
	\newline 
	}
	\UCitem{Versión}{0.2}
	\UCitem{Actor}{Alumno, Profesor, Invitado.}
	\UCitem{Propósito}{Proporcionar al actor un mecanismo para localizar áreas académicas dentro de la ESCOM.}
	\UCitem{Entradas}{Ninguna.}
	\UCitem{Origen}{No aplica.}
	\UCitem{Salidas}{
		\begin{itemize}
			\item Muestra en el mapa 2D la ubicación del área.
			\item Nombre del área.
			\item Planta en la que se ubica el área.
			\item Edificio al que pertenece el área.
		\end{itemize}
	}
	\UCitem{Destino}{Pantalla.}
	\UCitem{Precondiciones}{
		\begin{itemize}
			\item Tener registradas las coordenadas de las diferentes áreas que componen el mapa 
			(edificios, plantas, salones, etc.).
			\item El actor debe haber iniciado sesión.
			\item Tener cargada la imagen del plano de ESCOM.
		\end{itemize}
	}
	\UCitem{Postcondiciones}{Ninguna.}
	\UCitem{Errores}{
		\begin{enumerate}[\hspace*{0.5cm} \bfseries{E}1:]
			\item \label{EM-Mapa-CU01-E1} Cuando no se es posible dibujar las áreas académicas sobre el mapa, pues no se cuentan
			con las coordenadas para ello, muestra el mensaje \MSGref{MSG3}{Elementos No Disponibles} y \textbf{termina el caso de uso.}
		\end{enumerate}
	}
	\UCitem{Tipo}{Primario}
	\UCitem{Observaciones}{}
	\UCitem{Autor}{Fernández Quiñones Isaac.}
	\UCitem{Revisor}{Huerta Martínez Jesús Manuel.}
	\UCitem{Estatus}{Corregido.}
\end{UseCase}

\begin{UCtrayectoria}{Principal}

	% Paso 1. 
	\UCpaso[\UCactor] Inicia sesión en el sistema siguiendo los pasos en el caso de uso \UCref{EM-Acceso-CU2}.

	% Paso 2.
	\UCpaso Verifica que existan las coordenadas de las áreas que componen el mapa de ESCOM. [Error: \ref{EM-Mapa-CU01-E1}]

	% Paso 3.
	\UCpaso Obtiene las coordenadas del mapa de ESCOM.

	% Paso 4.
	\UCpaso Dibuja las coordenadas de los edificios en el mapa de ESCOM sobre una imagen que muestra el plano de la misma.

	% Paso 5.
	\UCpaso Muestra la \IUref{EM-Mapa-UI1}{Consultar Mapa de ESCOM}.

	% Paso 6.
	\UCpaso[\UCactor] Consulta el mapa ESCOM. \label{CU04ConsultarOtros}

\end{UCtrayectoria}

%========
%\subsection{Puntos de extensión}

%\UCExtensionPoint{EPAcademias}{El actor desea consultar academias por planta}
%{Paso \ref{CU04ConsultarOtros} de la trayectoria principal.}
%{Continua en el caso de uso \UCref{EM-Mapa-CU04}.}

%\UCExtensionPoint{EPAdminstrativa}{El actor desea consultar áreas administrativas por planta}
%{Paso \ref{CU04ConsultarOtros} de la trayectoria principal.}
%{Continua en el caso de uso \UCref{EM-Mapa-CU05}.}

%\UCExtensionPoint{EPSalones}{El actor desea consultar salones por planta}
%{Paso \ref{CU04ConsultarOtros} de la trayectoria principal.}
%{Continua en el caso de uso \UCref{EM-Mapa-CU06}.}

%\begin{UCtrayectoriaA}{A}{El actor selecciona ver los salones.}
%	\UCpaso[\UCactor] Presiona el botón \IUbutton{ Salones } de la \IUref{IU6}{Pantalla de inicio del alumno.}
%\end{UCtrayectoriaA}
%
%\begin{UCtrayectoriaA}{B}{El actor selecciona ver los áreas admnistrativas.}
%	\UCpaso[\UCactor] Presiona el botón \IUbutton{ Salones } de la \IUref{IU6}{Pantalla de inicio del alumno.}
%\end{UCtrayectoriaA}
%
%\begin{UCtrayectoriaA}{C}{El actor selecciona ver los academias.}
%	\UCpaso[\UCactor] Presiona el botón \IUbutton{ Salones } de la \IUref{IU6}{Pantalla de inicio del alumno.}
%\end{UCtrayectoriaA}















