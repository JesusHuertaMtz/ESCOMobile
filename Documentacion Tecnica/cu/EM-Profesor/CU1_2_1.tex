% Copie este bloque por cada caso de uso:
%-------------------------------------- COMIENZA descripción del caso de uso.
%\begin{UseCase}[archivo de imágen]{UCX}{Nombre del Caso de uso}{
\begin{UseCase}{EM-Profesor-CU1.2.1}{Consultar Comentarios Asignados.}{
	\noindent
	Este caso de uso permite al actor consultar los diferentes comentarios que los alumnos le han compartido. Así que puede conocer las opiniones de sus alumnos, analizarlas y plantearse nuevas y mejores formas para compartir su conocimiento y experiencia en el salón de calses. Además, es gracias a los comentarios que la comunidad de ESCOM tiene mayor conocimiento de los maestros del plantel, de su forma de trabajo y diversas actitudes que tiene para con sus grupos y alumnos.
	\newline
	}
	\UCitem{Versión}{0.1}
	\UCitem{Actor}{Profesor.}
	\UCitem{Propósito}{Proporcionar al actor un mecanismo para conocer las opiniones que la comunidad de ESCOM tiene sobre su trabajo y desempeño docente.}
	\UCitem{Entradas}{Ninguna.}
	\UCitem{Origen}{No aplica.}
	\UCitem{Salidas}{
		\begin{itemize}
			\item Nombre y fotografía del profesor.
			\item Lista con los comentarios asignados al profesor y el nombre de quienes escribieron cada comentario.
			\item \MSGref{MSG3}{Elementos No Disponibles}.
		\end{itemize}
	}
	\UCitem{Destino}{Pantalla.}
	\UCitem{Precondiciones}{Ninguno.}
	\UCitem{Postcondiciones}{Ninguna. Solo se realizan consultas.}
	\UCitem{Errores}{
		\begin{enumerate}[\hspace*{0.5cm} \bfseries{E}1:]
			%%  E1: Campos obligatorios. 
			\item \label{EM-Profesor-CU1-2-1-E1} Cuando no hay comentarios disponibles para consulta, muestra el mensaje \MSGref{MSG3}{Elementos No Disponibles} y \textbf{termina el caso de uso}.
		\end{enumerate} 
	}
	\UCitem{Tipo}{Caso de secundario, viene \UCref{EM-Profesor-CU1.2}.}
	\UCitem{Observaciones}{}
	\UCitem{Autor}{Huerta Matínez Jesús Manuel.}
	\UCitem{Revisor}{Fernández Quiñones Isaac.}
	\UCitem{Estatus}{Corregido.}
\end{UseCase}

\begin{UCtrayectoria}{Principal}

	% Paso 1.
	\UCpaso [\UCactor] Solicita consultar sus comentarios asignados, presionando la opción ''COMENTARIOS'' de la pantalla \IUref{EM-Profesor-UI1-2}{Consultar estadísticas Asignadas}.

	% Paso 2.
	\UCpaso Valida que haya comentarios asociados al profesor consultado. [Error: \ref{EM-Profesor-CU1-2-1-E1}]

	% Paso 3.
	\UCpaso Obtiene nombre y fotografía del profesor. 

	% Paso 4.
	\UCpaso Obtiene los comentarios asociados al profesor y los alumnos quienes otorgaron cada comentario. 

	% Paso 5.
	\UCpaso Ordena los comentarios y alumnos según la fecha de publicación de los comentarios (más reciente a más antiguo).

	% Paso 6.
	\UCpaso Muestra la pantalla \IUref{EM-Profesor-UI1-2-1}{Consultar comentarios asignados} con la información obtenida del profesor y comentarios.

	% Paso 7.
	\UCpaso[\UCactor] Consulta los comentarios que le han asignado.

\end{UCtrayectoria}
