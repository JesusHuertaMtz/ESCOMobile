% Copie este bloque por cada caso de uso:
%-------------------------------------- COMIENZA descripción del caso de uso.
%\begin{UseCase}[archivo de imágen]{UCX}{Nombre del Caso de uso}{
\begin{UseCase}{EM-Alumno-CU04}{Modificar perfil del Profesor}{
		Este caso de uso permite al profesor modificar su información en el sistema, pues requiere o desea hacerlo por motivos personales o porque había información errónea o desactualizada en el mismo. La información disponible para cambio es: correo y contraseña, mismos que podrán ser visualizados y utilizados en diferentes apartados del sistema por el alumno, de donde, es importante tener la información siempre actualizada.
	}
	\UCitem{Versión}{0.1}
	\UCitem{Actor}{Profesor}
	\UCitem{Propósito}{Proporcionar al profesor un mecanismo, con el cual pueda mantener actualizada su información en el sistema.}
	\UCitem{Entradas}{
		\begin{itemize}
			\item Correo electrónico.
			\item Contraseña.
			\item Nombre.
		\end{itemize}
	}
	\UCitem{Origen}{
		\begin{itemize}
			\item Teclado del smartphone
		\end{itemize}
	}
	\UCitem{Salidas}{
		\begin{itemize}
			\item \MSGref{MSG1}{Operación Exitosa}
		\end{itemize}
	}
	\UCitem{Destino}{Pantalla.}
	\UCitem{Precondiciones}{
		Ninguna. 
	}
	\UCitem{Postcondiciones}{
		\begin{itemize}
			\item Persiste la información sobre el profesor en el sistema.
		\end{itemize}
	}
	\UCitem{Errores}{
		\begin{enumerate}[\hspace*{0.5cm} \bfseries{E}1:]
			\item \label{EM-Profesor-CU02-E1} Cuando no se introdujeron todos los campos marcados como obligatorios. Muestra el mensaje \MSGref{MSG5}{Falta dato obligatorio} y \textbf{continúa en el paso \ref{InfoObligatoria} de la trayectoria Principal.}
			\item \label{EM-Profesor-CU02-E2} Cuando no se cumple con los alguno de los formatos definifdos en la información definida. Muestra el mensaje \MSGref{MSG6}{Formato de campo Incorrecto} y \textbf{continúa en el paso \ref{FotmatoCorrecto} de la trayectoria Principal.}
			\item \label{EM-Profesor-CU02-E3} Cuando no se puede llevar a cabo la operación de manera exitosa. Muestra el mensaje \MSGref{MSG2}{Operación Fallida} y  \textbf{termina el caso de uso.}
		\end{enumerate}	
	}
	\UCitem{Tipo}{Caso de uso primario.}
	\UCitem{Observaciones}{Es la primera propuesta de este caso de uso, se espera que se revise para implementar las correciones adecuadas.}
	\UCitem{Autor}{Huerta Matínez Jesús Manuel.}
	\UCitem{Revisor}{}
	\UCitem{Estatus}{Sin revisión.}
\end{UseCase}

\begin{UCtrayectoria}{Principal}
	%Paso 1.
	\UCpaso [\UCactor] Solicita actualizar su información presionando el icono \UCicono{lapiz.png} de la pantalla \IUref{EM-Menu1}{Menú hamburguer}.

	%Paso 2.
	\UCpaso Muestra la pantalla \IUref{EM-Alumno-UI4a}{Modificar información del alumno}. 

	% Paso 3.
	\UCpaso [\UCactor] Selecciona el (los) campo(s) de información que deseé actualizar por medio de los botones \IUbutton{\UCicono{sobre.png} Correo} y \IUbutton{\UCicono{candado_cerrado.png} Contraseña}.

	%Paso 4.
	\UCpaso [\UCactor] Solicita modificar los campos seleccionados presionando el botón \IUbutton{Siguiente}. 

	%Paso 5.
	\UCpaso [\UCactor] Muestra la pantalla \IUref{EM-Alumno-UI4b}{Modificar información del alumno} con los campos previamente seleccionados para modificar.  

	%Paso 6.
	\UCpaso [\UCactor] Introduce la nueva información en los campos mostrados.

	%Paso 7.
	\UCpaso [\UCactor] Solicita actualizar su información presionando el botón \IUbutton{Aceptar}.

	%Paso 8.
	\UCpaso Muestra mensaje de confirmación sobre el cambio de información.

	%Paso 9. 
	\UCpaso Presiona el botón \IUbutton{Sí}. \Trayref{A} 

	%Paso 10.
	\UCpaso Valida que se hayan introducido todos los campos, de acuero a la regla de negocio \BRref{EM-RN-S002}{campos obligatorios}. [Error \ref{EM-Profesor-CU02-E1}] \label{InfoObligatoria}
	
	%Paso 11.
	\UCpaso Obtiene la información de los campos introducidos.
	
	%Paso 12.
	\UCpaso Valida que los datos cumplan con sus formatos, de acuerdo a las reglas de negocio \BRref{EM-RN-S001}{Formato de correo electronico}, \BRref{EM-RN-S003}{Caracteres aceptados por el sistema} y \BRref{EM-RN-S004}{Contraseña}. [Error \ref{EM-Profesor-CU02-E2}] \label{FotmatoCorrecto}
	
	%Paso 13. 
	\UCpaso Persiste los nuevos datos de los usuarios. [Error \ref{EM-Profesor-CU02-E3}]

	%Paso 14. 
	\UCpaso Muestra el mensaje \MSGref{MSG1}{Operación Exitosa} en la pantalla \IUref{EM-Alumno-UI4b}{Modificar información del alumno}.

	%Paso 15.
	\UCpaso Muestra la pantalla \IUref{EM-Mapa-UI1}{Pantalla de inicio de alumno}.
\end{UCtrayectoria}

\begin{UCtrayectoriaA}{A}{Cuando el actor ha presionado el botón 'No'.}
	\UCpaso Muestra la pantalla \IUref{EM-Mapa-UI1}{Pantalla de inicio de alumno}.
\end{UCtrayectoriaA}


%-------------------------------------- TERMINA descripción del caso de uso.
%%%%%%%%%%%%%%%%%%%%%%%%%%%%%%%%%%%%%%