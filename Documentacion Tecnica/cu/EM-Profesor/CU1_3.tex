% Copie este bloque por cada caso de uso:
%-------------------------------------- COMIENZA descripción del caso de uso.
%\begin{UseCase}[archivo de imágen]{UCX}{Nombre del Caso de uso}{
\begin{UseCase}{EM-Profesor-CU1.3}{Modificar Perfil de Profesor.}{
	\noindent
	Este caso de uso permite al actor modificar su información registrada en el 
	sistema, pues algún dato está erróneo o desactualizado. La información que
	se encuentra disponible para edición es el correo electrónico, la contraseña, su cubículo y la fotografía, siendo los dos primeros ingresados cuando se realizó el registro. Se debe mencionar que la edición de los anteriores no afecta en
	nada el funcionamiento de la cuenta en la aplicación o las citas previamente
	asociadas a su cuenta. Además, la información antes referida puede ser
	modificada cuantas veces sea requerido por el actor dentro de la app.
	\newline
	}
	\UCitem{Versión}{0.1}
	\UCitem{Actor}{Profesor.}
	\UCitem{Propósito}{Proporcionar al actor un mecanismo que le permita
	actualizar su información registrada en el sistema.}
	\UCitem{Entradas}{
		\begin{itemize}
			\item Correo electrónico (nuevo).
			\item Contraseña (nueva).
			\item Duplicado de contraseña (nueva). 
			\item Cubículo (nueva).
			\item Fotografía (nueva). 
		\end{itemize}
	}
	\UCitem{Origen}{Teclado en pantalla.}
	\UCitem{Salidas}{
	Se muestra la siguiente información del profesor:
		Se muestra la siguiente información del Profesor:
		\begin{itemize}
			\item Nombre.
			\item Fotografía (original).
			\item Correo electrónico (original).
			\item Contraseña (original).
			\item Cubículo (original).
		\end{itemize}
		\item \MSGref{MSG1}{Operación Exitosa}.
	}
	\UCitem{Destino}{Pantalla.}
	\UCitem{Precondiciones}{Ninguno.}
	\UCitem{Postcondiciones}{
		\begin{enumerate}[\hspace*{0.5cm} \bfseries{E}1:]
			\item \label{EM-Profesor-CU1-3-E1} Cuando algún campo no comple con el formato valido definido. Muestra el mensaje \MSGref{MSG6}{Formato de campo Incorrecto} y \textbf{continúa en el paso \ref{l_Alumno_CU1_1_E1} de la trayectoria Principal.}

			\item \label{EM-Profesor-CU1-3-E2} Cuando las contraseñas introducidas no coinciden. Muestra el mensaje \MSGref{MSG16}{Contraseñas no coinciden} y \textbf{continúa en el paso \ref{l_Alumno_CU1_1_E1} de la trayectoria Principal.}.
		\end{enumerate}
	}
	\UCitem{Errores}{Ninguno.}
	\UCitem{Tipo}{Caso de secundario, viene \UCref{EM-Profesor-CU1}.}
	\UCitem{Observaciones}{}
	\UCitem{Autor}{Huerta Matínez Jesús Manuel.}
	\UCitem{Revisor}{Fernández Quiñones Isaac.}
	\UCitem{Estatus}{Corregido.}
\end{UseCase}

\begin{UCtrayectoria}{Principal}

	%Paso 1.
	\UCpaso [\UCactor] Solicita consultar sus estadisticas asignadas presionando el botón \IUbutton{Estadisticas y comentarios} de la pantalla \IUref{EM-Profesor-UI1}{Consultar Perfil Propio}.

	% Paso 2.
	\UCpaso Obtiene el nombre del profesor. 

	% Paso 3.
	\UCpaso Verifica que el profesor tenga asignada una calificación promedio, resultado de las calificaciones previas hechas por los alumnos. 

	% Paso 4.
	\UCpaso Obtiene la calificación promedio del profesor según la regla de negocio \BRref{EM-RN-N004}{Calificación Promedio de Profesor}. \label{CalifPromedio} \Trayref{A} 

	% Paso 5. 
	\UCpaso Obtiene el total de citas solicitadas al profesor, el número de citas aceptadas, canceladas y solicitudes pendientes de revisar por el mismo, así como la fotografía del perfil (en caso de haber). 

	% Paso 6. 
	\UCpaso Calcula el porcentaje equivalente para las Citas aceptadas, Citas canceladas y Solicitudes pendientes, con respecto al total de las solicitudes obtenidas. \label{l_EM_Profesor_CU1_2_datosGrafica}

	% Paso 7. 
	\UCpaso Construye una gráfica con los datos obtenidos en el paso \ref{l_EM_Profesor_CU1_2_datosGrafica} de la trayectoria principal.

	%Paso 8.
	\UCpaso Muestra la pantalla \IUref{EM-Profesor-UI1-2}{Consultar estadísticas Asignadas}

	% Paso 9.
	\UCpaso [\UCactor] Consulta sus estadísticas.

\end{UCtrayectoria}

\begin{UCtrayectoriaA}{A}{Cuando no hay calificación promedio asignada al profesor.}
	\UCpaso	Se asigna calificación promedio igual a cero.
	\UCpaso Continúa en el paso \ref{CalifPromedio} de la trayectoria principal.
\end{UCtrayectoriaA}
