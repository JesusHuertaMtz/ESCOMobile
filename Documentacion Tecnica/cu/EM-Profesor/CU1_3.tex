% Copie este bloque por cada caso de uso:
%-------------------------------------- COMIENZA descripción del caso de uso.
%\begin{UseCase}[archivo de imágen]{UCX}{Nombre del Caso de uso}{
\begin{UseCase}{EM-Profesor-CU1.3}{Modificar Perfil de Profesor.}{
	\noindent
	Este caso de uso permite al actor modificar su información registrada en el 
	sistema, pues algún dato está erróneo o desactualizado. La información que
	se encuentra disponible para edición es el correo electrónico, la contraseña, su cubículo y la fotografía, siendo los dos primeros ingresados cuando se realizó el registro. Se debe mencionar que la edición de los anteriores no afecta en
	nada el funcionamiento de la cuenta en la aplicación o las citas previamente
	asociadas a su cuenta. Además, la información antes referida puede ser
	modificada cuantas veces sea requerido por el actor dentro de la app.
	\newline
	}
	\UCitem{Versión}{0.1}
	\UCitem{Actor}{Profesor.}
	\UCitem{Propósito}{Proporcionar al actor un mecanismo que le permita
	actualizar su información registrada en el sistema.}
	\UCitem{Entradas}{
		\begin{itemize}
			\item Correo electrónico (nuevo).
			\item Contraseña (nueva).
			\item Duplicado de contraseña (nueva). 
			\item Cubículo (nuevo).
			\item Fotografía (nueva). 
		\end{itemize}
	}
	\UCitem{Origen}{Teclado en pantalla.}
	\UCitem{Salidas}{
		\begin{itemize}
			\item Se muestra la siguiente información del Profesor:
			\begin{itemize}
				\item Nombre.
				\item Fotografía (original).
				\item Correo electrónico (original).
				\item Contraseña (original).
				\item Cubículo (original).
			\end{itemize}
			\item \MSGref{MSG1}{Operación Exitosa}.
		\end{itemize}
	}
	\UCitem{Destino}{Pantalla.}
	\UCitem{Precondiciones}{Ninguno.}
	\UCitem{Postcondiciones}{
		\begin{enumerate}[\hspace*{0.5cm} \bfseries{E}1:]
			\item \label{EM-Profesor-CU1-3-E1} Cuando algún campo no cumple con el formato valido definido. Muestra el mensaje \MSGref{MSG6}{Formato de campo Incorrecto} y \textbf{continúa en el paso \ref{l_Profesor_CU1_3_E1} de la trayectoria Principal.}

			\item \label{EM-Profesor-CU1-3-E2} Cuando las contraseñas introducidas no coinciden. Muestra el mensaje \MSGref{MSG16}{Contraseñas no coinciden} y \textbf{continúa en el paso \ref{l_Profesor_CU1_3_E1} de la trayectoria Principal.}.
		\end{enumerate}
	}
	\UCitem{Errores}{Ninguno.}
	\UCitem{Tipo}{Caso de secundario, viene \UCref{EM-Profesor-CU1}.}
	\UCitem{Observaciones}{}
	\UCitem{Autor}{Huerta Martínez Jesús Manuel.}
	\UCitem{Revisor}{Fernández Quiñones Isaac.}
	\UCitem{Estatus}{Corregido.}
\end{UseCase}

\begin{UCtrayectoria}{Principal}

	% Paso 1.
	\UCpaso [\UCactor] Solicita modificar su información registrada en el sistema
	presionando en el icono \UCicono{engrane} de la pantalla \IUref{EM-Profesor-UI1}{Consultar Perfil Propio}.

	% Paso 2.
	\UCpaso Valida que el profesor tenga un cubículo asignado. \Trayref{A} 

	% Paso 3.
	\UCpaso Obtiene el cubículo asociado al profesor. 

	% Paso 4.
	\UCpaso Obtiene Nombre, Fotografía, Correo electrónico y contraseña
	asociados a la cuenta del profesor. \label{l_EM_Profesor_CU1_3_SinCubo}

	% Paso 5.
	\UCpaso Muestra la pantalla \IUref{EM-Profesor-UI1-3}{Modificar Perfil del Profesor} con la información obtenida. 

	% Paso 6.
	\UCpaso[\UCactor] Introduce los campos que desea modificar. \label{l_Profesor_CU1_3_E1}

	% Paso 7.
	\UCpaso [\UCactor] Solicita modificar la información de la cuenta
	presionando el botón \IUbutton{Aceptar}.
	
	% Paso 8.
	\UCpaso Valida que el correo electrónico proporcionado cumpla con el formato correcto, según la regla \BRref{EM-RN-S001}{Formato de correo electrónico}. [Error: \ref{EM-Profesor-CU1-3-E1}]

    % Paso 9.
    \UCpaso Valida la contraseña ingresada cumpla con el formato de contraseña definido, según la regla de nogocio \BRref{EM-RN-S004}{Formato de Contraseña}. [Error: \ref{EM-Profesor-CU1-3-E1}]

    % Paso 10.
    \UCpaso Valida que las contraseñas obtenidas de los campos ''Contraseña'' y ''Repite tu contraseña'' coincidan una con la otra. [Error: \ref{EM-Profesor-CU1-3-E2}]

    % Paso 11. 
    \UCpaso Valida que la fotografía seleccionada cumpla con el formato adecuado, según la regla \BRref{EM-RN-N011}{Formato de fotografía}. [Error \ref{EM-Profesor-CU1-3-E1}] \Trayref{B} 

    % Paso 12. 
    \UCpaso Persiste la información introducida del alumno en el sistema. \label{l_EM_Profesor_CU1_3_PersisteInfo}

	% Paso 13. 
    \UCpaso Muestra el mensaje \MSGref{MSG1}{Operación Exitosa} en la pantalla
    \IUref{EM-Profesor-UI1}{Consultar Perfil Propio}. y actualiza la
    información de la misma. 
     
\end{UCtrayectoria}

\begin{UCtrayectoriaA}{A}{Cuando el actor no tiene un cubículo asignado.}

	% A1.
	\UCpaso Muestra vacío el campo de cubículo en la pantalla: \IUref{EM-Profesor-UI1-3}{Modificar Perfil del Profesor}.

	\UCpaso Continúa en el paso \ref{l_EM_Profesor_CU1_3_SinCubo} de la trayectoria principal.

\end{UCtrayectoriaA}

\begin{UCtrayectoriaA}{B}{Cuando el actor no desea agregar o modificar
una foto en su perfil.}

	% B1.
	\UCpaso Continúa en el paso \ref{l_EM_Alumno_CU1_1_PersisteInfo} de la trayectoria principal.

\end{UCtrayectoriaA}
