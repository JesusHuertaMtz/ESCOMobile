% Copie este bloque por cada caso de uso:
%-------------------------------------- COMIENZA descripción del caso de uso.
%\begin{UseCase}[archivo de imágen]{UCX}{Nombre del Caso de uso}{
\begin{UseCase}{EM-Profesor-CU1.1}{Consultar Horario Propio.}{
	\noindent
	Este caso de uso permite al actor consultar su horario de clases en ESCOM, esto es, las materias que imparte a lo largo de la semana, así como la hora, grupo y salón en las que las mismas se ubican. Es importante mencionar los horarios son asignados por la institución a los profesores, y las materias, grupos, salones y horas que éstos contemplan varían semestre a semestre, así mismo, se muestran en ESCOMobile solo los horarios previamente descritos, por lo que horarios extraclase o de actividades personales de los profesores no son considerados. 
	\newline
	}
	\UCitem{Versión}{0.1}
	\UCitem{Actor}{Profesor.}
	\UCitem{Propósito}{Proporcionar al actor un mecanismo que le permita consultar su horario en ESCOM.}
	\UCitem{Entradas}{Ninguna.}
	\UCitem{Origen}{No aplica.}
	\UCitem{Salidas}{
	Muestra la pantalla con la siguiente información del profesor: 
		Se muestra la siguiente información del profesor:
		\begin{itemize}
			\item Nombre.
			\item Fotografía. 
		\end{itemize}
		E información de la materia:
		\begin{itemize}
			\item Nombre.
			\item Hora, grupo y salón en que se imparte. 
		\end{itemize}
		\MSGref{MSG3}{Elementos No Disponibles}
	}
	\UCitem{Destino}{Pantalla.}
	\UCitem{Precondiciones}{Ninguno.}
	\UCitem{Postcondiciones}{Ninguna. Solo se realizan consultas.}
	\UCitem{Errores}{
		\begin{enumerate}[\hspace*{0.5cm} \bfseries{E}1:]
			%%  E1: Campos obligatorios. 
			\item \label{EM-Profesor-CU1-1-E1} Cuando no hay materias asociadas al profesor, muestra el mensaje \MSGref{MSG3}{Elementos No Disponibles} y \textbf{termina el caso de uso}.
		\end{enumerate} 
	}
	\UCitem{Tipo}{Caso de secundario, viene \UCref{EM-Profesor-CU1}.}
	\UCitem{Observaciones}{}
	\UCitem{Autor}{Huerta Martínez Jesús Manuel.}
	\UCitem{Revisor}{Fernández Quiñones Isaac.}
	\UCitem{Estatus}{Corregido.}
\end{UseCase}

\begin{UCtrayectoria}{Principal}
	
	%Paso 1.
	\UCpaso [\UCactor] Solicita consultar su horario presionando el botón \IUbutton{Horario} de la pantalla \IUref{EM-Profesor-UI2}{Consultar Perfil de otro Profesor}.

	% Paso 2.
	\UCpaso Verifica que haya materias para impartir asociadas al profesor. [Error: \ref{EM-Profesor-CU1-1-E1}]

	% Paso 3.
	\UCpaso Obtiene por día (de lunes a viernes) las materias que el profesor imparte, así como los grupos, salones y horas en que éstas se encuentran. 

	% Paso 4.
	\UCpaso Muestra la pantalla \IUref{EM-Profesor-UI1-1}{Consultar Horario Propio} con la información obtenida organizada en tarjetas desplegables, una por cada día de la semana.

	% Paso 5.
	\UCpaso[\UCactor] Consulta su horario seleccionando las tarjetas con los días de la semana.

\end{UCtrayectoria}
