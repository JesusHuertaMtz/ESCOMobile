	% \IUref{IUAdmPS}{Administrar Planta de Selección}
% \IUref{IUModPS}{Modificar Planta de Selección}
% \IUref{IUEliPS}{Eliminar Planta de Selección}


% Copie este bloque por cada caso de uso:
%-------------------------------------- COMIENZA descripción del caso de uso.

%\begin{UseCase}[archivo de imágen]{UCX}{Nombre del Caso de uso}{
	\begin{UseCase}{CU05}{EM-Evento-CU01-Consultar vista  de eventos}{
		En este caso de uso el actor podra visualizar los eventos proximos a realizarse, mostrando solo la imagen del evento y una descripción}
		\UCitem{Versión}{0.1}
		\UCitem{Actor}{Alumno y Profesor}
		\UCitem{Propósito}{Mostrar los eventos de la table de eventos.}

		\UCitem{Entradas}{
		\begin{itemize}
		\item Ninguna
		\end{itemize}
		}

		\UCitem{Origen}{}
		\UCitem{Salidas}{Ninguna}
		\UCitem{Destino}{No aplica}
		\UCitem{Precondiciones}{Tener eventos dados de alta en el sistema . }
		\UCitem{Postcondiciones}{
		Ninguna
		}
		\UCitem{Errores}{
		\begin{enumerate}[\hspace*{0.5cm} \bfseries{E}1:]
			\item \label{EM-Evento-CU01-E1} Cuando no existen eventos en el sistema. Muestra el mensaje \MSGref{MSG11}{Eventos inexistentes} y \textbf{termina el caso de uso.}.
			
		\end{enumerate}	
		}
		\UCitem{Tipo}{Caso de uso primario}
		\UCitem{Observaciones}{	
			
		
		
 		}
		\UCitem{Autores}{ Pérez García José David.}
		\UCitem{Reviso}{sin revisar}
	\end{UseCase}
	\newpage
	
	\begin{UCtrayectoria}{Principal}
	\UCpaso[\UCactor] Pulsa el botón \IUbutton{Eventos}  de la pantalla \IUref{EM-Acceso-IU5}{Menu Hamburguer.}.
	\UCpaso Verifica que existan eventos registrados segun regla \BRref{EM-RN-N003}{Eventos Registrados}.[Error \ref{EM-Evento-CU01-E1}]
	
	\UCpaso Obtiene la lista de eventos registrados en el sistema.
	\UCpaso Muestra pantalla \IUref{EM-Evento-UI1}{Consultar Eventos}.
	
	\end{UCtrayectoria}

%-------------------------------------- TERMINA descripción del caso de uso.