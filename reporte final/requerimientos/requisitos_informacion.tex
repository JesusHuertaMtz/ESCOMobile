\subsection{Requisitos de información}
Los requisitos de información describen que información debe persistir el sistema para poder cumplir sus
objetivos. Además, deben identificar el concepto relevante sobre el que guardar información así como 
qué datos específicos del concepto son importantes para cumplir los objetivos del sistema.
%REFERENCIA http://www.lsi.us.es/docencia/get.php?id=6845

\begin{requisitos}{RI Información de horarios.} 
	\RFitem{Descripción}{El sistema desplegará la siguiente información referente al horario del profesor:
		\begin{itemize}
			\item Día de la semana en el que la unidad de aprendizaje se imparte.
			\item Hora de inicio y termino de la unidad de aprendizaje.
			\item Nombre del profesor que imparte la unidad de aprendizaje.
			\item Nombre de la unidad de aprendizaje.
			\item Salón en el que se imparte la unidad de aprendizaje.
		\end{itemize}
	La información anterior se podrá consultar a través del perfil del profesor del cual se deseé conocer
	su horario.} 
	\RFitem{Propósito}{Proporcionar al alumno la información de cuando un profesor se encuentra 
	impartiendo clase, que unidades de aprendizaje imparte.} 
	\RFitem{Usuario}{Alumno.}
	\RFitem{Tipo}{Móvil.} 
\end{requisitos}

\begin{requisitos}{RI Información en el mapa de ESCOM.}
	\RFitem{Descripción}{Se mostrará en el mapa textos descriptivos como:
		\begin{itemize}
			\item Nombre del salón.
			\item Nombre de cada área.
			\item Divisiones de salones.
			\item Nombre de las áreas administrativas.
		\end{itemize}}
	\RFitem{Propósito}{Ayudar a ubicar con mayor facilidad el lugar que busca el actor.} 
	\RFitem{Usuario}{Alumno, Profesor.}
	\RFitem{Tipo}{Móvil.} 
\end{requisitos}	

\begin{requisitos}{RI Información de club.}
	\RFitem{Descripción}{Habrá un módulo para poder consultar, dentro de la aplicación, los horarios 
	y clubes disponibles en ESCOM. Así como las horas necesarias que debe asistir al club para 
	liberar la unidad de aprendizaje Electiva.}
	\RFitem{Propósito}{Mostrar la variedad de clubes a los que los alumnos de ESCOM podrían unirse.} 
	\RFitem{Usuario}{Alumno.}
	\RFitem{Tipo}{Móvil.} 
\end{requisitos}

\begin{requisitos}{RI Eventos.}
	\RFitem{Descripción}{Se dispondrá de una sección, dentro de la aplicación, para poder consultar 
	los eventos que están próximos a llevarse acabo en las instalaciones de ESCOM. La información 
	que se debe mostrar es la siguiente:
	\begin{itemize}
		\item Cartel del evento.
		\item Fecha y hora de inicio del evento.
		\item Breve descripción del mismo.
		\item Lugar en donde se realizará.
		\item Nombre del ponente.
	\end{itemize}} 
	\RFitem{Propósito}{Difundir los eventos y de esta manera interesar al actor de ESCOM a asistir 
	al evento.}
	\RFitem{Usuario}{Alumno, profesor.} 
	\RFitem{Tipo}{Móvil.} 
\end{requisitos}

\begin{requisitos}{RI Información para el registro del profesor.}
	\RFitem{Descripción}{Los campos necesarios para el registro del profesor son los siguientes: 
		\begin{itemize}
			\item Nombre completo.
			\item Numero de empleado.
			\item Correo electrónico.
			\item Horarios de asesorías.
			\item Contraseña para acceder al sistema.
			\item Fotografía. (opcional)
		\end{itemize}} 
	\RFitem{Propósito}{Identificar a los profesores de ESCOM.} 
	\RFitem{Usuario}{Profesor.} 
	\RFitem{Tipo}{Móvil.}
\end{requisitos}

\begin{requisitos}{RI Información editable del perfil del profesor.} 
	\RFitem{Descripción}{La información que puede editar el profesor en su perfil es la siguiente:
		\begin{itemize}
			\item Correo electrónico.
			\item Contraseña.
			\item Fotografía. (opcional)
		\end{itemize}} 
	\RFitem{Propósito}{Mantener actualizada la información del profesor.} 
	\RFitem{Usuario}{Profesor.} 
	\RFitem{Tipo}{Móvil.} 
\end{requisitos}

\begin{requisitos}{RI Información para el registro de alumnos.}
	\RFitem{Descripción}{La información necesaria para poder registrar a un alumno es la siguiente:
		\begin{itemize}
			\item Nombre completo.
			\item Número de boleta.
			\item Correo electrónico.
			\item Fotografía. (opcional)
		\end{itemize}}
	\RFitem{Propósito}{Saber si el alumno es parte de la comunidad de ESCOM.}
	\RFitem{Usuario}{Alumno.}
	\RFitem{Tipo}{Móvil.} 
\end{requisitos}

\begin{requisitos}{RI Información editable del alumno. } 
	\RFitem{Descripción}{La aplicación permite al usuario editar:
		\begin{itemize}
			\item Nombre completo.
			\item Número de boleta.
			\item Correo electrónico.
			\item Fotografía. (opcional)
		\end{itemize}} 
	\RFitem{Propósito}{El usuario mantiene al día (actualizados) su información e intereses.} 
	\RFitem{Usuario}{Alumno.}
	\RFitem{Tipo}{Web / móvil.} 
\end{requisitos}

\begin{requisitos}{RI Tiempo recordatorio.} 
	\RFitem{Descripción}{El tiempo de notificación podrá ser seleccionado (por el alumno) y podrá ser de 5
	opciones diferentes:
	\begin{itemize}
		\item 5 minutos.
		\item 15 minutos.
		\item 30 minutos.
		\item 1 hora.
		\item 1 día.
	\end{itemize}} 
	\RFitem{Propósito}{El alumno podrá estar notificado y asi no se le olvidara su cita.}
	\RFitem{Usuario}{Alumno.}
	\RFitem{Tipo}{Móvil.} 
\end{requisitos}

%\begin{requisitos}{RI Usuarios no registrados.} 
%	\RFitem{Descripción}{El sistema muestra a los usuarios que no cuenten con una cuenta activa (pues no se
%	han registrado) únicamente el mapa de la ESCOM así como la distribución de la misma.} 
%	\RFitem{Propósito}{Extender la información y el servicio del mapa a quienes no desean o puedan
%	registrar una cuenta.} 
%	\RFitem{Usuario}{Visitante}
%	\RFitem{Tipo}{Web / móvil.} 
%\end{requisitos}
%
%\begin{requisitos}{RI Horarios libres del profesor.} 
%	\RFitem{Descripción}{La app habilita o deshabilita como disponible los espacios en donde el profesor
%	no tiene clase, según sean sus citas ya programadas.} 
%	\RFitem{Propósito}{El usuario puede ver como disponible solo las horas en donde el profesor no tiene
%	ya citas agendadas.}
%	\RFitem{Usuario}{Alumno.}
%	\RFitem{Tipo}{Web / móvil.} 
%\end{requisitos}

\begin{requisitos}{RI Muestra de estadísticas.} 
	\RFitem{Descripción}{La aplicación muestra dichas estadísticas para personas registradas en la app. } 
	\RFitem{Propósito}{Controlar esas estadísticas para solo alumnos.} 
	\RFitem{Usuario}{Alumno.} 
	\RFitem{Tipo}{Móvil.} 
\end{requisitos}