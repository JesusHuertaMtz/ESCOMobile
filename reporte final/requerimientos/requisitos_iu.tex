\subsection{Requisitos de interacción con el usuario}
Son aquellos que restringen las decisiones relativas al diseño de la aplicación: Restricciones de otros estándares, limitaciones del hardware, etc. 

\begin{requisitos}{RIU Información editable del perfil del profesor.} 
	\RFitem{Descripción}{La información que puede editar el profesor en su perfil es la siguiente:
		\begin{itemize}
			\item Correo electrónico.
			\item Contraseña.
			\item Fotografía. (opcional)
		\end{itemize}} 
	\RFitem{Propósito}{Mantener actualizada la información del profesor.} 
	\RFitem{Usuario}{Profesor.} 
	\RFitem{Tipo}{Móvil.} 
\end{requisitos}

\begin{requisitos}{RIU Concepto de GUIs.} 
	\RFitem{Descripción}{El sistema ofrece al usuario interfaces gráficas basadas en Material Design de
	Google.} 
	\RFitem{Propósito}{Ofrecer interfaces frescas y que se puedan adaptar a diferentes tamaños y
	orientaciones de pantalla de los dispositivos.} 
	\RFitem{Usuario}{Alumno, Profesor, Visitante} 
	\RFitem{Tipo}{Web / móvil.} 
\end{requisitos}

\begin{requisitos}{RIU Estilo de GUIs. } 
	\RFitem{Descripción}{El sistema ofrece al usuario interfaces gráficas minimalistas y con el contenido
	necesario.}
	\RFitem{Propósito}{El usuario siente empatía con el sistema. } 
	\RFitem{Usuario}{Alumno, Profesor, Visitante}
	\RFitem{Tipo}{Web / móvil.} 
\end{requisitos}

\begin{requisitos}{RIU Usabilidad de GUIs.} 
	\RFitem{Descripción}{El sistema ofrece al usuario interfaces gráficas intuitivas y fáciles de usar.} 
	\RFitem{Propósito}{El usuario siente aprende a usar fácil y rápidamente la app.} 
	\RFitem{Usuario}{Alumno, Profesor, Visitante} 
	\RFitem{Tipo}{Web / móvil.} 
\end{requisitos}

\begin{requisitos}{RIU Diseño de GUIs.} 
	\RFitem{Descripción}{El sistema ofrece al usuario interfaces gráficas visualmente atractivas y
	llamativas usando la psicología del color y formas.} 
	\RFitem{Propósito}{El usuario siente aprende a usar fácil y rápidamente la app.} 
	\RFitem{Usuario}{Alumno, Profesor, Visitante} 
	\RFitem{Tipo}{Web / móvil.} 
\end{requisitos}