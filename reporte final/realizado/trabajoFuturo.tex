\section{Trabajo futuro}
\noindent
En esta sección se describen los resultados obtenidos a lo largo del desarrollo de la aplicación ESCOMobile, las difucultades que se encontraron hasta el momento para establecer las ideas, el anális, diseño y el desarrollo propiamente dicho del sistema, y qué se decidió realizar para obtener mejores resultados o corregir los previamente encontrados. Además se plantea un panorama general de la aplicación actual y el trabajo que se planea realizar durante los siguientes meses para culminar la creación de la app. Se establecen los puntos importantes requeridos a trabajar en el futuro, además de lo planeado para lograr dichos puntos y posteriormete con ello cumplir los objetivos marcados desde el inicio del desarrollo. Es decir, se delimitan las actividades a seguir y cómo realizarlas. 

\noindent
Así, con lo implementado de la aplicación hasta el momento y tomando en cuenta los problemas que surgieron al momento de estar analizando, diseñando y codificando la aplicación, se ha realizado un análisis sobre los resultados obtenidos hasta el momento, y hemos comparado éstos últimos con lo esperado y ya diseñado de la aplicación, pues como se ha explicado con anterioridad, ESCOMobile se ha divido en módulos para obtener mejores resultados en el tiempo requerido, siendo estos módulos: Acceso, Mapa, Alumno, Profesor, Administración, Cita, Evento, Web Evento, Web Club, Web Bolsa. Bien, de los ya mencionados y su desarrollo, así como sus respectivas implementaciones y pruebas se plantea el trabajo a realizar y a entrgar a futuro. A continuación se detalla el avance planeado en cada uno de los módulos mencionados, mismos que juntos completan la aplicación ESCOMobile. \\

\subsection{Módulo de acceso}
\noindent
El módulo de acceso se ha desarrollado hasta el momento un prototipo, el cual muestra un primer login y registro para la app. Sin embargo, éstos no cuentan aún con la seguridad necesaria y requerida para garantizar que los datos de los usuarios estén resguardados y seguros. Es por ello que se planea mejorar la seguridad este módulo de la app y garantizar la seguridad y buen uso de los datos. Por otro lado, se ha decidido también agregar funcionalidad extra al módulo, como lo son el cambio de contraseña y el cerrar o eiminar una sesión una vez que ésta se inicie. \\

\subsection{Módulo de Mapa}
\noindent
Para este módulo, del cual ya se tiene de igual manera un prototipo, se planea printcipalmente en terminar el mapa, así como organizarlo por plantas y salones, para un mayor entendimiento del usuario y una fácil interacción del mismo con el sistema; esto implica establecer funcionalmente el mapa con sus respectivas plantas y pisos (planta baja, primer piso y segundo piso) así cono las áreas de interés identificadas, como lo pueden ser salones, académias, clubes, etc. Así como una pequeña descripción de cada área registrada y ubicar en el mapa algun cubículo de algún profesor de interés. Es importante decir además que este módulo funciona para usuarios registrados y para usuarios no registrados (invitados), no permitiendo para éstos determinadas funciones. \\

\subsection{Módulo de Alumno}
\noindent
Este módulo es de especial importancio, pues se centra en uno de los usuarios principales para la aplicación, los alumos. Son ellos para quienes se realiza principalmente la aplicación y quienes, junto a los profesores, hacen posible la interacción y funcionamiento de la misma. Así bien, para ellos en este módulo se pretende implementar perfiles para los profesores, mismos que contendrán la información de éstos últimos y que estarán disponibles para los alumnos como consulta, dichos perfiles contendrán también opciones disponibles para los alumnos, con el objetivo de permitir permitir el compartir y conocer imformación directamente de los propios alumnos. Estas funciones son: comentar y califar a un profesor específico; consultar las estadísticas asociadas al mismo, así como su horario. Por otro lado, se plantea que el alumno pueda consultar la bolsa de trabajo disponible para la ESCOM y el mapa curricular de la institución, además de poder modificar y mantener siempre actualizada su información dentro del sistema. \\

\subsection{Módulo de Profesor}
\noindent
Complementando al alumno, el profesor es un usuario que apesar de no ser necesario, puede aportar bastante al sistema. Es por eso que para este módulo se planea integrar lo siguiente al sistema: completar su perfil integrando información que puede servir para informar de mejor manera al alumno y otros usuarios, se puede también consultar y ser consultado por otros profesores y consultar los resultados de sus estadísticas.\\

\subsection{Módulo de Administración}
\noindent
ESCOMobile pretende ser una aplicación donde los alumnos puedan compartir y recibir opiniones e información por igual, siempre con respeto e intentando que la información sea real, ya sea por medio de la bolsa de trabajo, los eventos o las estadísticas de los profesores. Sin embargo, para garantizar que esta información sea lo más cercana a la real y que sea información que no ofenda o hiera a nadie es importante mantener un control sobre ella. Es por eso que se presenta un módulo que sirve como intermedio entre el sistema y los usuario finales, cuyo objetivo es manterner la información clara y apegada a la realidad. Pues, por diferentes razones puede ocurrir el hecho de publicar información errónea o desactualizada. Así, con la implementación del presente se podría, con ayuda de un usuario encargado de gestionarlo, realizar lo siguiente: informar al sistema si un área descrito en el sistema y su información son correctos o no; modificar la información de alguna de las áreas en el mapa, como lo son nombre o ubicación o reportar un problema que sea asociado con éstas y el sistema.  \\ 

\subsection{Módulo de Cita}
\noindent
Uno de los módulos con mayor importancia, pues es en él donde se concentra gran parte de la interracción entre los alumnos y profesores, y los conecta de manera directa, más allá de consultas sobre información. Es aquí donde ambos usuarios se conectan y generan juntos parte de la información que posteriormente será utilizada en otros módulos del sistema, siendo además especialmente importante, porque aquí está una piaza clave para cumplir algunos objetivos del sistema, que se relacionan directamente con los procesos de la escuela y los pasos y tiempos a seguir para cumplirlos. Dicho lo anterior, es necesario decir que para este módulo se presente realizar lo siguiente: permitir a los alumnos solicitar y agendar citas con los profesores de su interés, siempre y cuando éstos utilicen la aplicación también; consultar las citas agendadas y cancelar las mismas. Por otra parte, del lado de los profesores, aquí se podrán ceptar o rechazar citas pedidas por los alumnos, consultar sus citas pendientes, así como eliminar el historial de las citas que ya ha tenido antes. Es relevante comentar que los resultados de las citas sérán parte de las estadísticas que se podrán consultar en los perfiles de los profesores y que el objetivo de las citas es siempre intentar solventar alguna cituación académica por parte del profesor hacia el alumno. \\

\subsection{Módulo de Evento}
\noindent
Otro módulo que se pretende implementar para generar la difución de imformación dentro de la escuela es el módulo de evento, el cual pretende mostrar a los actores, idependientemente de su rol, el consultar información sobre los eventos en la Superior de Cómputo, como lo son conferencias, conciertos, concursos, ferias, exposiciones, entre otros. Eso ayuda a mantener y difundir información oficial, y a promover las actividades sociales, culturales y deportivas dentro del plantel y su comunidad. \\

\subsection{Módulos web}
\noindent
Finalmente se plantea que, al igual que el módulo de administración, se implementen diferentes módulos que ayuden a mantener el control en la aplicación y de la información que en ella se despliega para consulta por los usuario. Tal es el caso de los eventos, la bolsa de trabajo o las actividades culturales y deportivas. Es por ello que se planea crear tres módulos más: Web Evento, Web Club y Web Bolsa, módulos web en donde se podrá gestionar la información presentada en la app sobre los eventos, la bolsa de trabajo y las actividades culturales y deportivas, y se podrían realizar acciones como registros de nuevos eventos, propuestas de trabajo, eliminar los ya existentes y no necesarios, modificar información que necesite ser actualizada sobre éstos, entre otras cosas. \\

\noindent
Así, y dicho todo lo anterior, como trabajo a futuro se planea implementar los módulos anteriormente descritos, llevar un correcto análisis, diseño y codificación de éstos, y en caso de ser requerido ajustarlos buscando siempre el mejor desempeño de la app para los usuarios y el cumplimiento de los objetivos establecidos. Se planea también realizar pruebas al sistema para garantizar su funcionamiento e interacción con el usuario. Por otro lado, se pretende poner especial atención a las correcciones y observaciones generales de los señores sinodales hacia el trabajo presentado durante la evaluación de Trabajo Terminal 1, considerar y analisar si son pertinentes y según sea el caso acoplarlas e integrarlas al sistema y continuar con el crecimiento de ideas sobre la app, anexarlas al presente documento o documentación técnica en caso de ser necesarias y aceptadas para la aplicación. Así bien, todo ello es el trabajo a realizar en los siguientes meses. 


