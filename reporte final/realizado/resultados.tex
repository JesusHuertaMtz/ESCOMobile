\noindent
En este apartado se describen los avances que se tuvieron a lo largo del semestre, se realizaron diferentes propuestas y prototipos de la aplicación, desde la arquitectura, hasta el diseño de las interfaces gráficas de usuario, pasando por la base de datos y diferentes aspectos de la misma app.

\noindent
Sin embargo, pasado el tiempo y experimentando diferentes soluciones para un mismo con nuestra app, se logró dar una única propuesta que involucra el análisis y el diseño de ésta, así como el desarrollo de diferentes prototipos, hasta enfocarlos y comenzar el desarrollo formal de la aplicación. Dichos resultados se presentan a continuación:

\section{Análisis y diseño de la aplicación}
\noindent
En este apartado logramos establecer las ideas generales de la aplicación, como lo son los objetivos, la arquitectura, el público a quien se dirige, los requerimientos, las funcionalidades, las tecnologías a usar y otros aspectos importantes para la app así como las razones por las cuales se decidió realizar de esa manera. Se plantea una propuesta de diseño de las interfaces gráficas de usuario, los colores y distribución de las pantallas. Se establecen propuestas generales para la base de datos y los diagramas de caso de uso de la aplicación. Ejemplos de estos avances se pueden consultar en el apartado: \ref{trabajo_realizado}

\noindent
En la figura \ref{BD} se muestra el diagrama de la base de datos. Y en la figura \ref{CU} se muestra el de diagrama de casos de uso general obtenido mediante el análisis de la aplicación.

\IUfig[1]{baseDeDatos}{BD}{Diagrama de la base de datos de la aplicación ESCOMobile.}
\cfinput{diagramas/paquetes}

\newpage
\section{Desarrollo, prototipo de la aplicación}
\noindent
En cuanto al desarrollo de la aplicación se han implementado diferentes casos de uso previamente analizados, principalmente el acceso a la aplicación, así como registrar un nuevo usuario y algunos prototipo de los salones de la ESCOM que se mostrarán en el mapa, por ejemplo. Pues, aunque nos enfrentamos a varios problemas, principalmente con el análisis de la aplicación, logramos encontrar una salida óptima y aplicarla con éxito en la aplicación. Ejemplos de estos avances se pueden consultar en el apartado: \ref{trabajo_realizado}
