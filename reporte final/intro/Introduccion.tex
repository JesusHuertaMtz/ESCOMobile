%---------------------------------------------------------
\section{Introducción}

% Introducción.
\noindent
Este documento presenta la aplicación ESCOMobile, propuesta por estudiantes de la Escuela Superior de Cómputo (ESCOM) y dirigida a la misma institución. A lo largo del mismo se plantean los aspectos importantes a considerar en torno a la app, como lo pueden ser la problemática que la origina, las soluciones que se proponen, el diseño de la aplicación, la investigación relizada acerca de otras aplicaciones similares, lo requerimientos necesarios para su desarrollo, etcétera. Pues son todos estos factores importantes para que la aplicación funcione correctamente y sea de apoyo para sa solución del problema que más adelante se describirá detalladamente. 
Es en este apartado donde se explican las razones de ser del proyecto ESCOMobile, se describe el análisis que se lleva a cabo para posteriormente encontrar acciones y requerimientos, así como tecnologías necesarias para la creación de la aplicación. Pues, el documento se organiza en diferentes secciones, cada una de ellas explica una parte del proceso requerido para la creación de la app, y ayuda a entender de mejor manera el entorno en el que ésta se ubica, el público al que se dirige, los objetivos que se pretenden alcanzar, las acciones que con ella se puedes realizar, las razones por las cuales es importante y en qué y cómo ayudaría la aplicación en la ESCOM. 

% Justificación.
\noindent
Así bien, se cuenta primeramente con un apartado de justificación, en el cual se exponen las razones y los diferentes problemas por los cuales la aplicación ESCOMobile surge como medida para solucionar dichos problemas. Se detalla aquí el ''por qué'' de nuestro proyecto, los diferentes caminos que podemos seguir para conseguir resultados favorables ante las problemáticas y los resultados que se espera tener cuando la aplicación llegue al usuario final. Pues es bien sabido que debemos enfocarnos en ellos, y en cómo tratar con los problemas para así encontrar la mejor solución, con los mejores beneficios y mayores resultados. 

% Marco teórico. 
\noindent
Se establece un marco teórico en donde se describe el entorno en el cual se desarrolla el proyecto, el público al que va dirigida, las aplicaciones, la interacción de los usuario con las nuevas tecnologías así como la interacción de éstos en los diferentes escenarios en los cuales la aplicación estaría operando, la forma en que ésta lo haría. Además se plantean aquí las ideas y los problemas mismos que orillan a la creación de ESCOMobile para intentar solventarlos. Por otro lado se tiene también un análisis sobre las diferentes tecnologías y plataformas computacionales que nos ayudarán a realizar el proyecto, destacando la importancia de éstos y las concecuencias en la app y en los objetivos de la última. Por último, en este apartado se enlistan las diferentes palabras y términos que a lo largo del documento y en la propia aplicación se utilizan, y una descripción de los mismos, con el objetivo de contextualizar al lector y comprender mejor la aplicación, su estructura, lo que ésta realiza y la interacción que tiene con el usuario final.

% Estado del arte. 
\noindent
En esta sección se muestra principalmente el trasfondo de la aplicación y las herramientas que se utilizan para su desarrollo. Se presentan las aplicaciones ya existentes que realizan tareas similares al proyecto ESCOMobile, el funcionamiento de las mismas, el público al que se dirigen y los propósitos que éstas consideran. Se realizan comparativas para encontrar aquellos puntos diferenciales entre una aplicación y otra, y así implementar de la mejor manera en la aplicación las características con mayor importancia y que nos ayudarán a lograr los objetivos que más adelante se describirán. 

% Propuesta del proyecto. 
\noindent
Se cuenta también con un capítulo dedicado a la aplicación propiamente dicha, en donde se describe la propuesta concreta de la aplicación, las acciones que ésta puede realizar, los usuario que realizan una interacción con ella, así como las características y herramientas que en ella se contemplan para su funcionamiento. Se enlistan los objetivos general y específicos que se pretenden alcanzar con la aplicación además, es en este capítulo donde se tiene un primer gran acercamiento con el sistema, pues es aquí donde se analizan las acciones que se requieren y se empiezan a descubrir y organizar los diferentes requerimientos funcionales y no funcionales que harán que ESCOMobile funcione, logre cumplir sus objetivos y así intentar solventar los problemas que propiiciaron su existencia. 

% Trabajo drealizado y a futuro. 
\noindent
Finalmente se cuenta con un apartado dedicado al desarrollo de la aplicación, al anális y diseño de la misma. En él se muestra el trabajo que se ha realizado desde que nació la idea hasta el presente día. Se detallan las diversas tareas realizadas para la comprensión del problema, las posibles soluciones, las aplicaciones similares, los prototipos de práctica previos al desarrollo de la app. Aquí se concentran en forma de iteraciones toda acción que se realizó con fin de entender el propósito del proyecto, los diferentes conceptos referentes al análisis, diseño y desarrollo de ESCOMobile. Se muestran así, los avances que se tienen hasta hoy de la aplicación, los resultados a los que se ha llegado a lo largo de estos meses de trabajo, los cambios y los problemas que en él se han tenido, así como los logros que se han encontrado. Por último, se dedica una sección en este apartado para enunciar todo aquello que falta realizar y que se implementará en un futuro para que ESCOMobile se realice completa y exitosamente, y con ella los objetivos planteados, y con ellos solucionar aquellos problemas que en su momento fueron quienes dieron pauta a la creación de la aplicación. 

% Conclusión. 
\noindent
Así, como se explicó, en este documento nos encontramos con un proyecto que pretende servir y apoyar en ciertos aspectos de la Superior de Cómputo, así como la forma en que se estructura y las manera en que se fue desarrollando. 

