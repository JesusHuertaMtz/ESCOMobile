Se propone implementar un sistema móvil que ayude a profesores y alumnos a interactuar y llevar una mejor 
comunicación, todo por medio de la difusión de información acerca de profesores, sus horarios, unidades de
aprendizaje que imparte; así como la posibilidad de generar citas para apoyar el correcto aprendizaje de los 
alumnos o simplemente para atender situaciones académicas que se lleguen a presentar a lo largo de los 
cursos.
Se trata de una aplicación en la cual el usuario (alumnos y profesores) puedan gestionar estos
tiempos y horarios dedicados al aprendizaje y a la atención de los alumnos por parte del profesor. La aplicación se centrará, como
ya dijimos, en alumnos y profesores, teniendo funcionalidades o perspectivas diferentes para cada uno. Como
el hecho de generar las citas para los profesores, o la posibilidad que se presenta a los alumnos de
mantenerse al día con la información relevante y referente a la escuela, conocer los próximos eventos a
realizarse, saber más acerca de los diferentes clubes y equipos, áreas recreativas y en general actividades
extra clase. Mostrando además un mapa del plantel en donde se podrán localizar los diferentes espacios de
la escuela, como salones, cubículos, academias, servicios, etc. Y la posibilidad de buscar a
profesores de interés, consultar sus perfiles, horarios y agendar citas, mismas que se espera solucionen 
ciertos problemas y/o dudas que puedan tener los alumnos en las unidades de aprendizaje.
Con lo anterior se pretende apoyar a los estudiantes a reforzar su conocimiento y aprendizaje por medio de
las asesorías y comunicación más cercana con sus profesores, además, de una educación integral, pues se
pretende que con la información cultural, intelectual y deportiva presentada en la escuela y que la app
facilita, los alumnos se motiven a llevar una vida más saludable y productiva, física y mentalmente. Las
competencias anteriores podrán reforzarse con la ayuda de ESCOMobile.