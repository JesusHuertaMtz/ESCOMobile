%---------------------------------------------------------
\section{Justificación}

% Introducción.
\noindent
En este apartado se describen las razones y los problemas por los cuales la aplicación ESCOMobile surge. Se detalla el ''por qué'' del proyecto, los diferentes caminos que podemos seguir para conseguir resultados favorables ante las problemáticas y los resultados que se espera tener cuando la aplicación llegue al usuario final. Así bien, a continuación se presenta la problemática que da origen al sistema. 

% Problemática.
\subsection{Problemática}

\noindent
La educación es uno de los factores que más influye en el avance y progreso de personas y sociedades. 
Además de proveer conocimientos, la educación enriquece la cultura, el espíritu, los valores y todo aquello 
que nos caracteriza como seres humanos.

\noindent
La educación es necesaria en todos los sentidos. Para alcanzar mejores niveles de bienestar social y de 
crecimiento económico; para nivelar las desigualdades económicas y sociales; para propiciar la movilidad 
social de las personas; para acceder a mejores niveles de empleo; para elevar las condiciones culturales de 
la población; para ampliar las oportunidades de los jóvenes; para vigorizar los valores cívicos y laicos 
que fortalecen las relaciones de las sociedades; para el avance democrático y el fortalecimiento del Estado 
de derecho; para el impulso de la ciencia, la tecnología y la innovación \cite{Importancia_Educacion_UNAM}.

\noindent
Sin embargo, la educación, aunque es pieza clave para un buen desarollo, es cambiante de región en región y 
propicia cambios drásticos en las mismas. Europa, por ejemplo, las políticas o estrategias dependen del 
nivel educativo, concretamente en el nivel superior se procura promover la educación, la investigación y la 
innovación; además de que el enfoque cambia y se vuelve más competitivo, favoreciendo siempre la 
excelencia. También se da prioridad a la relación entre la enseñanza y el mundo laboral, para que la 
transición de los estudiantes a este nuevo mundos no sea tan traumática y pueda integrarse de manera 
rápida. Para lograr esto, se proveen datos e información actualizada sobre las tareas laborales actuales y 
se le da la oportunidad de desarrollar conocimientos antes de terminar su enseñanza, para que así pueda 
aclarar dudas y aportar ideas. Por otro lado, el sistema educativo latinoamericano no disfruta de una buena 
reputación. En esta región puede observarse que, en la mayoría de los países, no se le da mucha importancia 
a la educación, sino que se privilegian otras áreas como la economía y la política, sin percatarse de que 
la raíz de gran parte de los problemas que afronta la región provienen de los fallos en la enseñanza. 
Igual que en Europa, la educación en la región cuenta con los mismos tres niveles que en el resto de los 
continentes, pero los números estadísticos muestran un fenómeno interesante: el número de estudiantes va 
decreciendo conforme van avanzando de nivel. 
Se habla entonces de una tasa importante de deserción escolar, lo que sería el principal problema, y no el 
acceso a la educación como se suele creer.
Por lo que se puede observar, la educación en esta región tiene un problema de cultura de fondo, pues no se 
le da la relevancia necesaria a los estudios \cite{Sistemas_Educacion_Mundo}

\noindent
Así, en México, como América Latina, la educación -y específicamente la educación superior-, pese a los 
esfuerzos y avances de las últimas dos décadas, debe persistir en la búsqueda de una mayor equidad y 
calidad educativas. Ambos aspectos concentran las mayores dificultades y representan el mayor reto del 
sistema en el nivel superior. Las principales iniciativas deben concentrarse en ampliar las oportunidades 
educativas para un mayor número de jóvenes, principalmente en las regiones y grupos sociales más 
desfavorecidos, así como en mejorar de forma significativa su oferta educativa \cite{Estado_Educacion_UNAM}

\noindent
Por tanto, sabemos que uno de los factores más importantes para el desarrollo es la educación. Sin embargo, 
no siempre se obtienen  los resultados más idóneos; ésto se debe a muchas y muy variadas causas que van 
desde la falta de experiencia de los profesores hasta la falta de interés de los alumnos, pasando por el 
poco apoyo de las intituciones para brindar mejores oportunidades de estudio. Sin embargo, una de causas 
que nos parece de espcial relevancia es la falta de atención, dedicación y/o interés de los alumnos a sus 
clases (y tiempos de estudio o extra-clase) y viceversa \cite{Importancia_Educacion_UNAM}

\noindent
Para ejemplificar lo anterior se han realizado estudios en diversas universidades como es el caso de la 
Universidad de Salamanca y la Universidad de Sonora, solo por citar un par, los resultados mostraron lo 
siguiente: El fracaso académico (abandono) se concentra en los primeros cursos, acumulándose el 80 por 
ciento de los alumnos desertores entre el primer y segundo semestre de la carrera. Las peores 
calificaciones se dan en los primeros años de la carrera \cite{Sistemas_Educacion_Mundo}.

\noindent
Bien, como se puede ver, la educación, a pesar de ser un factor tan importante para el 
satisfactorio y buen desarrollo de las naciones así como de su gente, hay diversos factores que 
contrarestan el impacto positivo que ésta debería tener, como en México. Además, la importancia y la 
atención que las propias naciones hacia ella es escasa igual. Las universidades son pocas y de difícil 
alcance (ya sea por costo o por demanda), tomando el peso principalmente la Universidad Nacional Autónoma 
de México (UNAM) y el Instituto politécnico Nacional (IPN). Que aunque bien tiene calidad, siguen siendo 
pocas para la gran población que se tiene en el país. A pesar de ello, y haciendo frente a los problemas 
presentes, las grandes casas de estudio postulan siempre nuevas ideas y alternativas que intenten solventar 
o atenuar algunos de los problemas que los países y la población tienen que enfrentar en materia de 
educación. 
Creación de nuevos planteles, planes de estudio a distancia, nuevas carreras y reestructuración de las 
actuales son algunas de las estrategias aplicadas por las intituciones mexicanas hoy en día. Por ejemplo, 
en la Escuela Superior de Cómputo (ESCOM) del IPN, en el año 2009 se implementó en cambios del modelo 
educativo y el rediseño curricular, con el objetivo de mantener siempre actualizados los contenidos y las 
formas de enseñar, para así tener mayores y mejores resultados de aprensizaje por parte de los estudiantes. 
Sin embargo, la implementación mencionada que causó desafíos para el desempeño docente, uno de ellos, 
desarrollar competencias pedagógicas. Miguel Zabalza (2003) propuso un esquema de competencias, solo 
describiremos dos.

\begin{itemize}	
	\item Relacionarse constructivamente con los alumnos: Capacidad que  se relaciona con la  habilidad 
	para entablar relaciones  interpersonales, con la motivación y  el liderazgo del profesor, lo que  
	genera climas propicios para el  aprendizaje.
	\item Tutorar: Capacidad de dirigir el proceso de formación integral de nuestros alumnos y que 
	permite acompañarlos a lo largo de su vida escolar.
\end{itemize}

\noindent
El profesor universitario, en esta nueva perspectiva, deja de ser un mero transmisor de conocimientos 
dedicando una gran parte de su actividad docente a guiar y orientar al estudiante en su itinerario 
formativo, principalmente académico pero también profesional y personal. La formación del estudiante no 
tiene así como único escenario la clase, sino todo el abanico de recursos y espacios curriculares 
sincrónicos y asincrónicos diseñados a cumplir con ese objetivo: bibliotecas, programas informáticos, 
portales digitales, actividades diversas en el aula y en el entorno, etc. La tutoría académica adquiere  
así un papel esencial en este nuevo escenario docente \cite{Tutoria}.
\noindent
Dicho todo lo anterior y teniendo claros algunos de los problemas que se presentan en la educación mexicana 
-y las causas que están tras ellos-, y conociendo la importancia que ésta representa en muchos aspectos y 
la necesidad de soluciones o propuestas que ayuden a la solución de los problemas antes dichos, se propone 
implementar un sistema móvil que ayude a profesores y alumnos a interactuar y llevar una mejor 
comunicación, todo por medio de la difución de información de profesores acerca de sus horarios, materias 
impartidas y demás información relevante; así como la posibilidad de generar citas para apoyar el correcto
aprendizaje de los alumnos o simplemente para atender situaciones académicas que se lleguen a presentar a
lo largo de los cursos.
Se trata de una aplicación en la cual el usuario (alumnos y profesores) puedan gestionar mejor estos 
tiempos horarios dedicados al aprendizaje y a la atención de los alumnos. La aplicación se centrará, como 
ya dijimos, en alumnos y profesores, teniendo funcionalidades o perspectivas diferentes para cada uno. Como 
el hecho de generar las citas para los profesores, o la posibilidad que se presenta a los alumnos de 
mantenerse al día con la información relevante y referente a la escuela, conocer los próximos eventos a 
realizarse, saber más acerca de los diferentes clubes y equipos, áreas recreativas y en general actividades 
extra clase. Mostrando además un mapa del plantel en donde se podrán localizar los diferentes espacios de 
la escuela, como salones, cubículos, académias, áreas verdes, servicios, etc. Y la posibilidad de buscar a 
profesores de interés, consultar sus perfiles, horarios y agendar citas, mismas que se espera solucionen 
ciertos problemas que puedan tener los alumnos. 

\noindent
Con lo anterior se pretende apoyar a los estudiantes a reforzar su conocimiento y aprendizaje por medio de 
las asesoarías y comunicación más cercana con sus profesores, además, de una educación integral, pues se 
pretende que con la información cultural, intelectual y deportiva presentada en la escuela y que la app 
facilita, los alumnos se motiven a llevar una vida más saludable y productiva, física y mentalmente. Las 
competencias anteriores podrán reforzarse con la ayuda de ESCOMobile.