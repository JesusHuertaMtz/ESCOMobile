En el presente capítulo se hablará del Estado del Arte, se dará una pequeña introducción sobre la historia de las tecnologías móviles y en específico de los smartphones, el sistema operativo Android, así mismo se describirá cada una de las apps con una imagen y un tabla con sus características principales.

Se incluirá nuestra propuesta y sus características así mismo una tabla comparativa general de todas las apps y las distintas características que tienen cada una y que no ofrecen.


\section{Historia}
En la actualidad, uno de los aspectos vitales y en crecimiento está constituido por las tecnologías de la información, esta apoya la toma de decisiones gerenciales en empresas para agilizar procesos y lograr una mejor colaboración entre grupos de trabajo.
Si bien lo anterior es un punto importante para las empresas la forma en el que las personas interactúan entre sí y con las organizaciones se ha visto alterada con el surgimiento de Internet y de los smartphones\cite{TI}.

Ahora revisaremos los acontacemientos más importantes que fueron un punto clave en la evolución de los teléfonos a los smartphone.
En 1993 IBM creó el primer smartphone. Lo llamo Simon. Fue el primer smartphone en integrar una pantalla táctil. Simon sería el precursor para los smartphones actuales\cite{Simon}.
En 1996 Nokia crea el primer smartphone, Nokia Communicator 9000, con acceso a la web ya  que integraba un básico navegador web\cite{Nokia}.

En el año 2007, con el lanzamiento del primer iPhone por parte de Apple se revolucionó el mercado 
de la telefonía móvil y por tanto, muy ligado a él, el de las aplicaciones móviles.
En 2008 se lanza al mercado el primer smartphone con sistema operativo Android. Es en este mismo año en el que aparece la primer tienda de aplicaciones de Apple: AppStore. Un año después sale Android 
Market \cite{Rioja}.

Android, en 2017 ha logrando casi un 81.7 por ciento del mercado global, seguido muy distantemente por iOS, 
que casi alcanzó un 12 por ciento en el segundo trimestre del mismo año\cite{IDC}.

Se han desarrollado una gran gama de aplicaciones para smartphone. Hay aplicaciones 
para entretenimento, para conocer personas, para espionaje, para comunicarse a través de mensajes, 
entre muchas otras\cite{playstore}. Existe un ramo de las aplicaciones móviles diseñado especificamente para 
brindar información acerca de como desplazarse y/o consultar sitios de interés ya sea dentro de una ciudad 
o un centro comercial. Algunos ejemplos son Google maps, waze, foursquare, 


Un software de código abierto es gratis y accesible a todo el mundo. Esto es especialmente útil para los 
desarrolladores, quienes pueden experimentar y probar, mientras que cada fabricante puede introducir sus 
particularidades. 


En la actualidad las escuelas estan dando prioridad a laa tecnologia de las app, aprovechando que casi 
todos los estudiantes poseen un telefono inteligente.


La mayoria de las aplicaciones moviles en el ambito de la educación superior se desarrollan para que el 
alumno pueda estar informado y en contacto con su situación academica de su escuela.

Estas apps tienen diferentes caracteristicas cada una las cuales mencionaremos en este apartado.

Esta seccíon permitirá comparar trabajos o sistemas de otras escuelas que actualmente se encuentran en 
funcionamiento y disponibles en tiendas de aplicaciones.

\section{Aplicaciónes }
Una de las aplicaciones encontradas es la siguiente:
\subsection{Red Anáhuac}



\IMGfig[.3]{fotosarte/4}{Anahuac}{Pantalla Principal Red Anáhuac.}

Aplicación NetAnahuac, esta aplicación esta hecha para la Universidad Anáhuac, es una aplicación que conecta a todos los usuarios de las escuelas de la Anahuac en toda la Republica,
Net anahuac App se puede consultar en cualquier momento y en cualquier lugar.
El estudiante puede obtener información precisa y actualizada.
Sólo necesita usar ID y PIN de Intranet Anáhuac, el mismo que se utiliza comúnmente en el sistema institucional.
Ofrecido por: Fomento e Investigación Integral S.C.
Completa suite de Servicios Academicos y Financieros para la red de Universidades Anáhuac en Mexico
Esta aplicación se encuentra disponible para IOS y Android, 
Vista de la pantalla principal \IMGref{Anahuac}
 \begin{table}[]
\centering
\caption{RED ANAHUAC}
\label{my-label1}
\begin{tabular}{@{}|c|c|c|@{}}
\toprule
Caracteristicas  Academicas                                                                                                                                                                                       & Información Financiera                                                                                            & Otros                                                        \\ \midrule
\begin{tabular}[c]{@{}c@{}}-Busqueda de Cursos\\ -Cursos Planeados\\ -Cita de Inscripción\\ -Horario\\ -Perfil\\ -Situación Academica\\ -Calificaciones Parciales\\ -Historia Academica\\ -Mi Avance\end{tabular} & \begin{tabular}[c]{@{}c@{}}-Estado de Cuenta\\ -Credito Educativo\\ -Apoyo Financiero\\ -Retenciones\end{tabular} & \begin{tabular}[c]{@{}c@{}}-Noticias\\ -Eventos\end{tabular} \\ \bottomrule
\end{tabular}
\end{table}



\pagebreak
\subsection{MIT MOBILE}

Aplicación MIT Mobile Experience Lab, esta aplicación está hecha para el MIT,MIT Mobile ofrece muchos servicios esenciales del Instituto para el usuario.
Se pueden consultar Noticias del campus de la Oficina de Noticias del MIT,Seguimiento en vivo del campus,Mapa del campus de búsqueda,Calendario de eventos, exposiciones, vacaciones y el calendario académico.
Ofrecido por: Massachusetts Institute of Technology .
Esta aplicación se encuentra disponible para IOS y Android, 

		
		\IMGfig[.3]{fotosarte/5}{MIT}{Pantalla Principal MIT MOBILE.}
		
 \begin{table}[]
\centering
\caption{MIT MOBILE}
\label{my-label4}
\begin{tabular}{|c|}
\hline
Caracteristicas                                                                                                                                                                                                                                                    \\ \hline
\begin{tabular}[c]{@{}c@{}}-\\ Noticias del Campus,\\ Busqueda del mapa del campus,\\ Calendario de Eventos,\\ Busqueda de Directorio,\\ Informacion sobre el MIT\\ Informacion sobre campus de emergencias\\ Menu de comidas\\ Reportes\\ Scanner QR\end{tabular} \\ \hline
\end{tabular}
\end{table}
	
	\pagebreak
\subsection{CONEXION UVM}	

Solución móvil de la UVM que soporta el proceso de toma y consulta de asistencia, así como la consulta de calificaciones en línea de los alumnos en curso; además de proveer un medio de comunicación directa entre la comunidad estudiantil, docente y administrativa a través del envío de notificaciones, encuestas y mensajes.
Ofrecida por:Moofwd
Disponible para Android y IOS.



		
		\IMGfig[.3]{fotosarte/6}{UVM}{Pantalla Principal UVM.}

 \begin{table}[]
\centering
\caption{CONEXIÓN UVM}
\label{my-label2}
\begin{tabular}{|c|}
\hline
Caracteristicas                                                                                                                                                                                         \\ \hline
\begin{tabular}[c]{@{}c@{}}Consultade Asistencia\\ Consulta de Calificaciones\\ Medio,de comunicación entre la comunidad estudiantil\\ Notificaciones\\ Comunicación,estudiante y profesor\end{tabular} \\ \hline
\end{tabular}
\end{table}

Disponible para IOS Y ANDROID
\pagebreak
\subsection{IBERO MOVIL}

Ibero Móvil es tu aplicación de la Universidad Iberoamericana Ciudad de México, para acceder a tu información académica, de estados de cuenta, y realizar la reinscripción de tus materias. Fue realizada por la Dirección de Informática y Telecomunicaciones.
Ofrecida por: Universidad Iberoamericana, A.C.
Disponible para IOS y Android.

		
		\IMGfig[.3]{fotosarte/ibero}{Ibero}{Pantalla Principal Ibero Movil.}
 \begin{table}[]
\centering
\caption{IBERO MOVIL}
\label{my-label3}
\begin{tabular}{|c|}
\hline
Caracteristicas                                                                                                                                                                                                   \\ \hline
\begin{tabular}[c]{@{}c@{}}-Busqueda de Cursos\\ -Cursos Planeados\\ -Cita de Inscripción\\ -Horario\\ -Perfil\\ -Situación Academica\\ -Calificaciones Parciales\\ -Historia Academica\\ -Mi Avance\end{tabular} \\ \hline
\end{tabular}
\end{table}
	
	
	Tambien en la ESCOM se han desarrolado ciertas aplicaciones para el uso de los estudiantes, aqui mostramos algunas:	
	
	
	\pagebreak
	\subsection{MANIFEST ESCOM}
	
		
		\IMGfig[.3]{fotosarte/manifest}{MANIFEST}{Pantalla Principal Manifest ESCOM.}
	Aplicación para la difusión de la información en la Escuela Superior de Cómputo
	
	Manifest es una aplicación diseñada para mantenerte al tanto de los últimos acontecimientos que ocurren en la Escuela Superior de Cómputo, y asi seguir los temas de interes para la comunidad.Puedes consultar eventos, cursos y convocatorias.
	Se te pide registrarte.
	Actualmente fuera de servicio.
	Ofrecida por: Escuela Superior de Cómputo.
	Disponible para Android.
	
	 \begin{table}[]
\centering
\caption{MANIFEST ESCOM}
\label{my-label5}
\begin{tabular}{|c|}
\hline
Caracteristicas                                                                                            \\ \hline
\begin{tabular}[c]{@{}c@{}}-Eventos\\ -Cursos\\ -Convocatorias\\ -Noticias\\ -Avisos Urgentes\end{tabular} \\ \hline
\end{tabular}
\end{table}
	\pagebreak
	\subsection{ESCOMobile}
	
		\IMGfig[.3]{fotosarte/escomobile}{ESCOMobile}{Pantalla Principal ESCOMobile.}
	ESCOMobile es nuestra propuesta de aplicación para la ESCOM, en esta principalmente se podrá tener acceso al mapa de la escuela, se podrán generar citas entre alumnos y profesores, información de eventos y bolsa de trabajo.
	Ofrecido por:Alumnos de la ESCOM.
	Disponible para Android
	
 \begin{table}[]
\centering
\caption{ESCOMobile}
\label{my-label6}
\begin{tabular}{|c|}
\hline
Caracteristicas                                                                                                                                                                                                           \\ \hline
\begin{tabular}[c]{@{}c@{}}-Horarios de clase\\ -Registro Alumnos y Profesores\\ -Eventos\\ -Citas\\ -Mapa de la ESCOM\\ -Cursos\\ -Perfiles\\ -Busqueda de Profesores\\ -Notificaciones\\ -Acceso Invitados\end{tabular} \\ \hline
\end{tabular}
\end{table}

En la siguiente tabla comparativa tomamos las diferentes caracteristicas de cada app y comparamos con cada una.


\begin{table}[]
\centering
\caption{Tabla comparativa}
\label{my-label}
\begin{tabular}{@{}|c|c|c|c|c|c|c|@{}}
\toprule
Caracteristicas            & \begin{tabular}[c]{@{}c@{}}Red \\ Anahuac\end{tabular} & \begin{tabular}[c]{@{}c@{}}MIT \\ Mobile\end{tabular} & \begin{tabular}[c]{@{}c@{}}Ibero \\ Movil\end{tabular} & \begin{tabular}[c]{@{}c@{}}Conexión \\ UVM\end{tabular} & \begin{tabular}[c]{@{}c@{}}Manifest\\ ESCOM\end{tabular} & ESCOMobile \\ \midrule
Busqueda de Cursos         & x                                                      & x                                                     & x                                                      & x                                                       & x                                                        & x          \\ \midrule
Calificaciones Parciales   & x                                                      & x                                                     & x                                                      & x                                                       & -                                                        & -          \\ \midrule
Cita de Inscripción        & x                                                      & -                                                     & x                                                      & x                                                       & -                                                        & -          \\ \midrule
Estado de Cuenta           & x                                                      & -                                                     & -                                                      & x                                                       & -                                                        & -          \\ \midrule
Eventos                    & x                                                      &                                                       & x                                                      &                                                         &                                                          & x          \\ \midrule
Citas                      & -                                                      & -                                                     & -                                                      & -                                                       & -                                                        & x          \\ \midrule
Mensajes alumno - Profesor & -                                                      & -                                                     & -                                                      & x                                                       & -                                                        & -          \\ \midrule
Acceso invitado            & -                                                      & x                                                     & -                                                      & -                                                       & -                                                        & x          \\ \midrule
Mapa del Campus            & -                                                      & x                                                     & -                                                      & -                                                       & -                                                        & x          \\ \midrule
Registro de usuarios       & -                                                      & -                                                     & -                                                      & -                                                       & -                                                        & x          \\ \bottomrule
\end{tabular}
\end{table}




Como resultado de esto hemos visto que cada escuela tiene su aplicación con las necesidades de cada una de ellas, vemos funcionalidades de pagos especialmente con las apps de escuelas particulares.

Ninguna de las aplicaciones anteriores resuelven el conflicto de la interacción Alumno profesor solamente conexión UVM con un chat entre estos dos.

Al final de comparar cada funcionalidad y checar que problemas resuelven las aplicaciones vistas y que problema es el que queremos resolver podemos ver que algunas características de otras app desarrolladas no son necesarias para nosotros y resolvemos una problemática que ellos no la cubren.
